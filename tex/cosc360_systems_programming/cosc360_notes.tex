\documentclass[code]{amznotes}

\title{\textbf{Systems Programming}\\
\large UT Knoxville, Spring 2023, COSC 360}
\author{Dr. James S. Plank, Alex Zhang}

\begin{document}
\maketitle
\tableofcontents

\addcontentsline{toc}{chapter}{Preface}
\chapter*{Preface}
These notes attempt to give a concise overview of \textbf{Systems Programming} course at the University of Tennessee. The contents of these notes come from Dr. James Plank's online COSC 360 notes\footnote{\url{https://web.eecs.utk.edu/\textasciitilde jplank/plank/classes/cs360/lecture\_notes.html}} as well as select comments from his in-person lectures.

\chapter{Moving from C++ to C}
In moving from C++ to C, we lose a lot of nice features: standard template library (STL), \texttt{iostream}, objects, methods, and operator overloading to name a few. We can think of C as a subset of C++ in terms of features. Only in rare circumstances will C code not compile in C++.

\section{Basic Terminal I/O}
In C, \texttt{stdio.h} provides \texttt{printf()} for terminal output, and \texttt{fgets()} and \texttt{scanf()} for terminal input.

\begin{codebox}{Hello World in C}{cool}
    \amzinputcode{c}{../code_snippets/helloworld.c}
\end{codebox}

\section{Scalar Types}

\begin{dfnbox}{Scalar Type}{}
    A \dfntxt{scalar type} is a type that stores a single value.
\end{dfnbox}

In C, there are seven default scalar types:
\begin{itemize}
    \item \texttt{char} --- 1 byte
    \item \texttt{short} --- 2 bytes
    \item \texttt{int} --- 4 bytes
    \item \texttt{long} --- 4 or 8 bytes (depending on machine)
    \item \texttt{float} --- 4 bytes
    \item \texttt{double} --- 8 bytes
    \item \texttt{pointer} --- 4 or 8 bytes (depending on machine)
\end{itemize}

Unlike C++, \texttt{bool} is not a default scalar type. As of C99, \texttt{bool} is defined in \texttt{stdbool.h}. Despite only storing 1 bit of info, \texttt{bool} occupies 1 byte of memory.


Scalar variables can be declared in one of three scopes:
\begin{itemize}
    \item outside of all functions as \textbf{global} variables
    \item inside a function implementation as a \textbf{local} variable
    \item inside a function prototype as a \textbf{procedure} parameter
\end{itemize}

\textbf{Global} variables are allocated to memory once the program starts and only deallocates once the program terminates. \textbf{Local} variables are only allocated once the program reaches its function and are deallocated when the function ends.

\begin{notebox}
    There are no reference variables (\texttt{\&}) in C. Function parameters are \textbf{always} copied.
\end{notebox}

\section{Aggregate Types}

\begin{dfnbox}{Aggregate Type}{}
    An \dfntxt{aggregate type} stores one or more values.
\end{dfnbox}

In C, there are only two aggregate types: \nameref{dfn:array} and \nameref{dfn:struct}.

\begin{dfnbox}{Array}{array}
    An \dfntxt{array} is a contiguous piece of memory that stores multiple variables of the same scalar type.
\end{dfnbox}

In most programming languages (including C), arrays are represented by the memory address of its first element. Since the array is contiguous, C can easily calculate the memory address of all other elements.

% TODO: add better explanation of dereferencing arrays

In C, arrays can only be declared either globally or locally. When we pass an array to a function, we are only passing a pointer to its first element.

\begin{dfnbox}{Structure}{struct}
    A \dfntxt{structure} or \dfntxt{struct} is a contiguous piece of memory that stores multiple variables which can be different scalar types.
\end{dfnbox}

In C++, we can use structs and classes interchangeably---the only difference being that, by default, structs make members \texttt{public} while classes makes members \texttt{private}. In contrast, C does not have classes, and structs in C don't have the following capabilities:

\begin{itemize}
    \item No \textbf{access modifiers} like \texttt{public}, \texttt{private}, or \texttt{protected}
    \item No \textbf{constructors} nor \textbf{destructors}
    \item No \textbf{method functions}
\end{itemize}

In addition, C structs do not create a type by default. We have to explicitly create one using the \texttt{typedef} keyword.

Member variables of a struct get padded to be aligned to ensure each variable is aligned to 4-bytes \textbf{and} consecutive memory is aligned to 4-bytes.

The assignment operator (\texttt{=}) works as intended with structs but not so with arrays. If we have \texttt{int arr1[20]} and \texttt{int arr2[20]}, we cannot simply write \texttt{arr1 = arr2;}. Instead, we have to iterate through the array and copy each \texttt{int} one at a time.

Generally, we always copy a specific number of bytes using the assignment operator. The only exception is when we copy a struct that has an array as a member. Assignment operators for structs copies \textbf{all data}.

\begin{codebox}{Assignment Operator for Aggregate Types}{}
    \amzinputcode{c}{../code_snippets/equals_arr_struct.c}
\end{codebox}

Similarly, if a function an argument that is a struct with a member array, it will \textbf{copy} the struct as a procedure parameter.

\begin{codebox}{Confusing Struct Argument}{}
    \amzinputcode{c}{../code_snippets/confusing_struct_arg.c}
\end{codebox}

\texttt{s.arr[9]} will still be 10! That's because \texttt{my\_func()} takes a \texttt{MyStruct} parameter, creating a copy of \texttt{MyStruct} for use in the function. Hence, we are only changing the copied \texttt{MyStruct} which gets deleted at the end of the function.
In this case, \texttt{my\_func()} effectively does nothing.

\section{Memory and \texttt{malloc()}}
\texttt{malloc()} returns a pointer to a piece of continuous memory allocated only to our program. When we call \texttt{malloc()}, we specify a certain number of bytes which is then allocated to the program by the operating system. If the operating system can't allocate that much memory, \texttt{malloc} returns \texttt{NULL}. To prevent potential bus errors, \texttt{malloc()} only returns pointers aligned to $8$ bytes.

All \texttt{malloc}'ed memory is deallocated either by calling \texttt{free()} on the pointer or when the program terminates. \texttt{free()} is mostly useful to deallocate data once it's no longer being used. Generally, there is no reason to call \texttt{free()} right before the program terminates.

%Whereas \texttt{new} in C++ specifies an amount of a certain type, \texttt{malloc()} only accepts a number of bytes. It does not know how the memory is being used. As a consequence, we have to keep track of every array's size.

\section{Strings}

\begin{dfnbox}{String, Null Character}{}
In C and C++, a string is a contiguous sequence of characters whose last character is the null character (ASCII value 0).
\end{dfnbox}

While C++ has \texttt{std::string} which handles the memory of the characters for us, we don't have such luxury in C. Instead, we have to explicitly use an array of \texttt{char}, handling it just as we would any other array.

When using a function like \texttt{printf("\%s", str)}, we are only passing a pointer to the first character of \texttt{str}. Then, \texttt{printf} iteratively prints each byte in its ASCII representation, stopping once it reaches the null character.

\section{Libfdr}
To ease the transition between C++ and C, Dr. Plank has created a library for formatting input doubly-linked lists, and red-black trees (hence, fdr).\footnote{Documentation is available on \url{https://web.eecs.utk.edu/\~jplank/plank/classes/cs360/lecture_notes.html}.}
\begin{itemize}
    \item The \dfntxt{fields} library helps parse input from stdin and files
    \item The \dfntxt{dllist} library provides doubly-linked lists
    \item The \dfntxt{JRB} library provides red-black trees, similar to \texttt{std::multimap}
\end{itemize}

\chapter{File System}

\begin{dfnbox}{File System}{}
    A \dfntxt{file system} is a hierarchy of files and directories.
\end{dfnbox}

In the UNIX file system, every user has a home directory, aliased by ``\textasciitilde''.

\begin{dfnbox}{Signal}{}
    A \dfntxt{signal} is an interruption in a running program.
\end{dfnbox}

\begin{dfnbox}{System Call}{}
    \dfntxt{System calls} allow programs to directly interface with the operating system. In C, system calls appear just as regular procedure calls.
\end{dfnbox}

Whereas regular procedures (functions) create a stack frame, system calls incur a signal to the program, interrupting it. Hence, they are very expensive and should be used sparingly. By design, system calls do not produce segmentation faults, allowing for some potential memory errors.

In C, there are five system calls defined in \texttt{fcntl.h} for file I/O:
\begin{itemize}[noitemsep]
    \item \texttt{open()} -- opens a handle to a file for I/O; specifies read and/or write permission
    \item \texttt{close()} -- close a handle to a file
    \item \texttt{read()} -- reads a specified number of bytes; return number of bytes read
    \item \texttt{write()} -- writes specified bytes to the file
    \item \texttt{lseek()} -- repositions the file pointer
\end{itemize}

\begin{notebox}
    We generally avoid using these system calls and instead use the \texttt{stdio} equivalents: \texttt{fopen()}, \texttt{fclose()}, \texttt{fread()}, \texttt{fwrite()}, \texttt{fseek()}. For example, \texttt{fread()} still calls \texttt{read()}, but it will store read bytes into a buffer to sparingly call \texttt{read()}.
\end{notebox}

Every file has three distinct parts:
\begin{enumerate}[noitemsep]
    \item The actual data of the file
    \item The \dfntxt{metadata} (information about the file)
    \item The directory entry
\end{enumerate}

In UNIX, the file has a directory entry and an associated \dfntxt{inode} that stores the metadata and the location of the actual bytes of the file.

% \chapter{Assembler}
% The CPU has several components:
% \begin{itemize}
%     \item The \dfntxt{ALU} - Arithmetic and logic unit
%     \item Instruction decoder
%     \item A fixed number of \dfntxt{registers} that can store one 32/64 bit value
%     \item Clock
% \end{itemize}

% To help us learn, we will consider a make-believe reduced instruction set architecture with 32-bit pointers and 8 general purpose registers:
% \begin{itemize}
%     \item r0--r4
%     \item sp: stack pointer
%     \item fp
%     \item pc: program counter (a pointer to the instruction being executed)
% \end{itemize}

% In normal operation, each clock cycle increments the program counter by four. We will also have three read-only register:
% \begin{itemize}
%     \item g0: contains the value 0
%     \item g1: contains the value 1
%     \item gm1: contains the value -1
% \end{itemize}

% And two special registers:
% \begin{itemize}
%     \item IR: the current instruction
%     \item CSR: the result of the previous instruction
% \end{itemize}

% Our instructions will consist of:
% \begin{itemize}
%     \item \texttt{ld mem reg} ($\text{reg} \coloneq \text{mem}$)
%     \item \texttt{st reg mem} ($\text{mem} \coloneq \text{reg}$)
%     \item
% \end{itemize}

\chapter{Memory}
We can think of memory as just a really long array. In 32-bit systems, we can have up to $2^{32}$ bytes of memory available to us.

For each process, UNIX declares four regions of memory:
\begin{enumerate}[noitemsep]
    \item \dfntxt{Text:} your executable instructions (assembler compiled down to bytes), typically read-only
    \item \dfntxt{Global variables:} these are determined at compile time
    \begin{itemize}[noitemsep]
        \item ``data'' -- initialized globals
        \item ``BSS'' -- uninitalized globals
    \end{itemize}
    \item \dfntxt{Heap:} This memory is retrieved by \texttt{sbrk()} or \texttt{mmap()} -- typically, we see \texttt{malloc()}
    \item \dfntxt{Stack:} Local variables, function call overhead, etc. \todo{more}
\end{enumerate}
This is set up by the operating system in conjunction with the hardware.

\begin{dfnbox}{Memory Page}{}
Memory is partitioned into fixed-size intervals called \dfntxt{pages}. These are typically 4 or 8 kibibytes.
\end{dfnbox}

Whenever we reference some memory, the OS looks at the \dfntxt{page table} to find the page it is stored in. If it is an invalid page, the system incurs a segmentation fault.

C has special variables that mark the end of the memory segments for each region (not POSIX standard).

\begin{notebox}
    Flush before fork
\end{notebox}

\amzindex
\end{document}
