\chapter{Probability}

\section{Introduction}

\begin{dfnbox}{Experiment, event, simple event, sample point}{experiment}
    An \dfntxt{experiment} is the process by which an observation is made. Something can be an experiment if it has a measurable outcome. An \dfntxt{event} is the set of outcome(s). An event is \dfntxt{simple} if it only contains one outcome, in which case it cannot be decomposed.
\end{dfnbox}

\begin{dfnbox}{Sample point, sample space, discrete sample space}{}
    A \dfntxt{sample point} is any single outcome from an experiment. The \dfntxt{sample space} is the set of all possible sample points. A sample space is \dfntxt{discrete} if it contains a countable amount of distinct sample points.
\end{dfnbox}

\begin{dfnbox}{Probability}{}
    Intuitively, the \dfntxt{probability} of an event is the likelihood that the event occurs. Given an event $E$, we write $P(E)$ to denote the probability that event $E$ occurs.
    \tcblower
    Let $S$ be the sample space associated with an experiment. To each event $A \subseteq S$, we assign a number $P(A)$ called the \dfntxt{probability} of $A$, satisfying:
    \begin{enumerate}
        \item $P(A) \geq 0$ for all $A \subseteq S$,
        \item $P(S) = 1$, and
        \item if $A_1, \ldots, A_n$ are disjoint, then we have $P(A_1 \cup \cdots \cup A_n) = P(A_1) + \cdots + P(A_n)$.
    \end{enumerate}

\end{dfnbox}

\begin{exbox}{Betting}{betting}
    \textbf{Question:} Suppose that two people, $B$ and $P$, have placed equal bets on winning the best of 5 fair coin flips. $B$ is betting on heads, and $T$ is betting on tails. They are interrupted after 3 flips and have to stop the game short, with $B$ ahead 2 heads to 1 tails. How should they fairly divide pot?
    \tcblower
    It's clear that $B$ is more likely to win this game than $P$. The question is: how probable is $B$ winning in this scenario? If we replicated this experiment 44 times, we may see that $B$ wins 39 times and $P$ wins 15 times. Denoting $S$ as the set of all possible outcomes, we have:
    \[ S \coloneq \{ \text{HH}, \text{HT}, \text{TH}, \text{TT} \} \]
    From these 4 outcomes, $B$ will win 3 of the 4 outcomes, and $P$ will win 1 of the  outcomes.
\end{exbox}

In this context, we can think of an event as a collection of sample points. Note that a simple event is always a singleton set whereas a sample point does not have to be; we will not worry over these differences and simply use the two terms interchangeably.

For example, let's consider a homemade, six-sided die. Using $S$ to denote our sample space, we have
\[ S \coloneq \{ E_1, E_2, E_3, E_4, E_5, E_6 \} \]
where $E_j$ is the event that $j$ is rolled. Suppose that we had the following probabilities for each of the events:
\begin{align*}
    P(E_1) &= \sfrac13 \\
    P(E_2) &= \sfrac14 \\
    P(E_3) &= \sfrac16 \\
    P(E_4) &= \sfrac{1}{12} \\
    P(E_5) &= \sfrac18 \\
    P(E_6) &= \sfrac{1}{24}
\end{align*}
To find the probability that we roll an even number, we look at the event $\{E_2, E_4, E_6\}$. The probability that this event occurs is $\sfrac{1}{4} + \sfrac{1}{12} + \sfrac{1}{24} = \sfrac{3}{8}$. 

\begin{exbox}{Laptop Refurbish}{}
    Suppose we had a refurbished laptop with the following probabilities:
    \begin{align*}
        P(\text{bad battery}) &= 0.32 \\
        P(\text{bad screen}) &= 0.18 \\
        P(\text{bad battery and bad screen}) &= 0.12
    \end{align*}
    Find the probabilities for:
    \begin{align*}
        P(\text{bad battery OR bad screen}) \\
        P(\text{neither defect}) \\
        P(\text{bad screen but NOT bad battery})
    \end{align*}
    \tcblower
    In this example, we have only two simple events: having a bad battery but not a bad screen, and having a bad screen but not a bad battery.

    TODO: venndiagram

    From this, we have:
    \begin{align*}
        P(\text{bad battery OR bad screen}) &= 0.20 \\
        P(\text{neither defect}) &= 0.62 \\
        P(\text{bad screen but NOT bad battery}) &= 0.06
    \end{align*}
\end{exbox}

\begin{exbox}{Markers}{}
    Five seemingly identical markers are left in a classroom. Only two of have enough ink to write well. The instructor selects two of these markers at random.
    \begin{enumerate}
        \item What is the sample space?
        \item Assign probabilities to each sample point in the sample space.
        \item What is the probability that neither marker has enough ink to write?
    \end{enumerate}
    \tcblower
    To determine the sample space, we consider all the possibilities for the instructor. We denote each marker as:
    \[ W_1, W_2, D_1, D_2, D_3 \]
    If we consider picking $W_1$ then $W_2$ to be the same event as picking $W_2$ then $W_1$, we can simply enumerate all possible pairs of markers for our sample space:
    \[ S \coloneq \left\{ W_1W_2, W_1D_1, W_1D_2, W_1D_3, W_2D_1, W_2D_2, W_2D_3, D_1D_2, D_1D_3, D_2D_3 \right\} \]
    The probability of any sample point occurring is $\sfrac{1}{10}$. From this, we can simply add up the sample points for each of the events. The probability that the instructor selects two dead markers is given by the event $\{D_1D_2, D_1D_3, D_2D_3\}$ whose probability is $\sfrac{3}{10}$.
\end{exbox}

\begin{tecbox}{Calculating Probability: The Sample Point Method}{}
    The sample point method is a very straightforward approach to calculating the probability of an event in an experiment.
    \begin{enumerate}
        \item List all the simple events associated with an experiment. This defines the sample space $S$.
        \item Assign reasonable probabilities to the sample points in $S$.
        \item Define the event of interest $A$ as a subset of $S$.
        \item Find $P(A)$ by summing the probability of each sample point in $A$.
    \end{enumerate}
\end{tecbox}

\section{Combinatorics}

\begin{dfnbox}{Permutation}{}
    A \dfntxt{permutation} is an ordered arrangement of $r$ distinct objects. The number of permutations of size $n$ among $r$ objects is defined as:
    \[ P_n^r \coloneq \frac{n!}{(n-r)!} = n(n-1)(n-2)\cdots(n-r+1) \]
\end{dfnbox}

\begin{thmbox}{Number of partitions}{}
    The number of ways partitioning $n$ distinct objects into $k$ disjoint sets is:
    \[ \binom{n}{n_1\ n_2\ \cdots\ n_k} \coloneq \frac{n!}{n_1! n_2! \cdots n_k!} \]
\end{thmbox}

The terms $\binom{n}{n_1\ n_2\ \cdots\ n_k}$ are often called \dfntxt{multinomial coefficients} because they occur in the expansion of $y_1 + y_2 + \cdots + y_k$ raised to the $n$th power:
\[ (y_1 + y_2 + \cdots + y_k)^n = \sum \binom{n}{n_1\ n_2\ \cdots\ n_k} y_1^{n_1} y_2^{n_2} \cdots y_k^{n_k} \]

\begin{dfnbox}{Combination}{}
    The number of combinations of $n$ objects taken $r$ at a time is given by
    \[ \binom{n}{r} = \frac{n!}{r!(n-r)!} \]
\end{dfnbox}

We can think of the above definition in terms of sets. Given a set $A$ of size $N$, $\binom{n}{r}$ is the number of possible distinct subsets of $A$ that are of size $r$.

\begin{thmbox}{Binomial Theorem}{}
    \[ (x+y)^n = \sum_{i=0}^{n} \binom{n}{i} x^i y^{n-i} \]
    \tcblower
    \textbf{Intuition:} For example, consider $(t+h)^5$, which expands to:
    \[ (t+h)(t+h)(t+h)(t+h)(t+h) \]
    If we were to fully distribute this out, we have:
    \[ t^5 + 5t^4h + \binom{5}{3} t^3 h^2 + \binom{5}{2} t^2 h^3 + 5th^4 + h^5 \]
\end{thmbox}


\begin{exbox}{Coin flips}{}
    Four fair coins are flipped. What is the most likely outcome?
    \begin{enumerate}[label=(\alph*)]
        \item All H or all T
        \item 2H, 2T
        \item 3H 1T or 1H 3T
    \end{enumerate}
    \tcblower
    Let's consider how many ways there are to get each of the results.
    \begin{enumerate}[label=(\alph*)]
        \item There are $2$ ways to get either all heads or all tails.
        \item There are $6$ ways to get 2 heads and 2 tails.
        \item There are $8$ ways to get 3 heads 1 tail, or 3 tails 1 head.
    \end{enumerate}
    (TODO: choose notation and explanations)
\end{exbox}

\begin{exbox}{Poker hands}{}
    A standard deck of cards has 52 cards, with 4 suits and 13 ranks. The number of distinct $5$ card hands (not accounting for order) can be calculated by $\binom{52}{5}$. If order does matter, then we count the number of permutations $P_5^{52}$. Most card games don't care about the order of the hand, so we look at the first option (which is called choices without replacement).
    \tcblower
    What is the probability of having a hand with exactly 2 aces? The total number of 2 ace hands is calculated by:
    \[ \binom{4}{2} \binom{48}{3} \]
    Although there are 50 cards left over after selecting 2 aces, we do not want three or four-tuple aces, which explains 48 instead of 50. (TODO: wording) Thus, the probability of getting a 2 ace hand is:
    \[ \frac{\binom{4}{2} \binom{48}{3}}{\binom{52}{5}} \] 
    (TODO: 48 instead of 50?)

    The probability that we get a hand with two pairs is decided by the following choices:
    \begin{itemize}
        \item Which pairs?
        \item Which cards for those pairs?
        \item What's the left over card?
    \end{itemize}
    As such, we can calculate the number of two pair hands by:
    \[ \underbrace{\binom{13}{2}}_\text{choose two ranks} \cdot \binom{4}{2} \binom{4}{2} \binom{44}{1} \]

    The probability that we get a full house (1 pair, 1 triple) is decided by (TODO). The total number of full house hands is calculated by:
    \[ \underbrace{\binom{13}{1}}_\text{pair rank} \underbrace{\binom{12}{1}}_\text{triple rank} \underbrace{\binom{4}{2}}_\text{which pair} \underbrace{\binom{4}{3}}_\text{triple} \]
    (left to right, pair rank, triple rank, which pair, triple)
\end{exbox}

\begin{exbox}{Yahtzee}{}
    In Yahtzee, we roll 5 six-sided dice. There are $6^5$ total different outcomes. 

    The number of two-pair rolls can be calculated by:
    \[ \binom{6}{2} \binom{4}{1} \binom{5}{2} \binom{3}{2} \]
    (left to right: ranks of the pairs, rank of leftover, location of first pair, location of second pair)

    The number of rolls where all five dice are different can be calculated by:
    \[ \underbrace{6}_\text{which num. missing} \cdot \underbrace{5!}_\text{ways to rearrange} \]
\end{exbox}

TODO :oisdjfojsdoisjf

\section{Conditional Probability and Independence}

\begin{dfnbox}{Conditional probability}{}
    The \dfntxt{conditional probability} of an event $A$, given that $B$ has occurred, is defined as:
    \[ P(A \mid B) \coloneq \frac{P(A \cap B)}{P(B)} \quad \text{if} \quad P(B) > 0. \]
    We read $P(A \mid B)$ as ``probability of $A$ given $B$.''
\end{dfnbox}

For example, if we roll a six-sided die, and we already know that the die landed on an odd number, then the probability that it's $1$ is:

\[ P(1 \mid \text{odd}) = \frac{P(1\ \text{and odd})}{P(\text{odd})} = \frac{\sfrac16}{\sfrac36} \]

\begin{exbox}{Cards}{}
    Cards are dealt one at a time from a standard deck. If the first 2 are spades, what is the probability that the next 3 are also spades?
    \tcblower
    In this example, we consider the first two being spades to be the initial condition $B$ and the next 3 being spades as $A$. We first calculate $P(A \cap B)$:
    \[ P(A \cap B) = \frac{\binom{13}{5}}{\binom{52}{5}} \]
    \[ P(B) = \frac{\binom{13}{2}}{\binom{52}{2}} \]
    Thus:
    \[ P(A \mid B) = \frac{P(A \cap B)}{P(B)} = \frac{\binom{13}{5}\binom{52}{2}}{\binom{52}{5}\binom{13}{2}} = \ldots \]
    (TODO: answer)
    To confirm our answer, we can asdofadsiofj
\end{exbox}

\begin{dfnbox}{Independent, dependent}{}
    Intuitively, two events are called \dfntxt{independent} if the occurrence of one does not affect the probability of occurrence of the other.
    \tcblower
    More formally, two events $A$ and $B$ are \dfntxt{independent} any of the following are true:
    \begin{itemize}
        \item $P(A \mid B) = P(A)$,
        \item $P(B \mid A) = P(B)$, or
        \item $P(A \cap B) = P(A) \cdot P(B)$.
    \end{itemize}
    Otherwise, the events are \dfntxt{dependent}.
\end{dfnbox}

\begin{exbox}{Independent dice rolls}{}
    Roll a six-sided die once. Let:
    \begin{itemize}
        \item $A \coloneq \{ \text{roll is odd} \}$,
        \item $B \coloneq \{ \text{roll is even} \}$,
        \item $C \coloneq \{ \text{roll is 1 or 2} \}$.
    \end{itemize}
    We can see $A$ and $B$ are \textbf{not} independent by the following calculation:
    \[ 0 = P(A \cap B) \neq P(A) \cdot P(B) = \frac{1}{2} \cdot \frac{1}{2} = \frac{1}{4} \]
    However, we can see $A$ and $C$ are independent by the following:
    \[ \frac{1}{6} = P(A \cap C) = P(A) \cdot P(B) = \frac36 \cdot \frac26 = \frac16 \]
\end{exbox}

\begin{exbox}{Independent coffee brands}{}
    Three brands of coffee, $X$, $Y$, and $Z$, are ranked according to taste by a judge.
    \begin{itemize}
        \item $A$: $X$ is better than $Y$
        \item $B$: $X$ is the best
        \item $C$: $X$ is second best
        \item $D$: $X$ is last
    \end{itemize}
    If the ranking is truly random, is $A$ independent of $B$, $C$, and/or $D$?
    \tcblower
    First, we see that there are $6$ possible rankings, which we will denote by $S$:
    \[ S \coloneq \{ XYZ, XZY, YXZ, YZX, ZXY, ZYX \} \]
    We have the following probabilities:
    \begin{itemize}
        \item $P(A) = \sfrac12$
        \item $P(B) = \sfrac13$
        \item $P(C) = \sfrac13$
        \item $P(D) = \sfrac13$
    \end{itemize}
\end{exbox}

\section{Two Laws of Probability}
\begin{thmbox}{Multiplicative Law of Probability}{}
    The probability of the intersection of two events $A$ and $B$ is:
    \[ P(A \cap B) = P(A)P(B \mid A) = P(B) P(A \mid B) \]
    If $A$ and $B$ are independent, then:
    \[ P(A \cap B) = P(A) P(B) \]
\end{thmbox}

\begin{thmbox}{Additive Law of Probability}{}
    The probability of the union of two events $A$ and $B$ is:
    \[ P(A \cup B) = P(A) + P(B) - P(A \cap B) \]
    If $A$ and $B$ are disjoint, then:
    \[ P(A \cap B) = 0 \] 
\end{thmbox}

\begin{thmbox}{Probability of complementary event}{}
    Let $A$ be an event. Then $P(A) = 1 - P(\overline{A})$.
\end{thmbox}

\section{The Law of Total Probability and Bayes' Theorem}

\begin{dfnbox}{Partition}{}
    Let $S$ be a set. If $S = B_1 \cup B_2 \cup \cdots \cup B_k$, and these sets $B_1, \ldots, B_k$ are disjoint, then the set $\{ B_1, \ldots, B_k \}$ is called a \dfntxt{partition} of $S$. Moreover, for any $A \subseteq S$:
    \[ A = (A \cap B_1) \cup (A \cap B_2) \cup \cdots \cup (A \cap B_k) \]
    where $A \cap B_1, \ldots, A \cap B_k$ are disjoint.    
\end{dfnbox}

Partitions are especially useful to us in computing probability. For example:
\[ P(A) = \sum_{i=1}^{n} P(A \cap B_i) = \sum_{i=1}^{k} P(A \mid B_i) P(B_i) \]

\begin{exbox}{Probability using partition}{disease}
    A diagnostic test for a disease is 95\% accurate. Let $E_d$ denote the event where a person has a disease, and let $E_+$ denote a positive test. Then:
    \begin{align*}
        P(E_+ \mid E_d) &= 0.95 \\
        P(\overline{E_+} \mid \overline{E_d}) &= 0.95
    \end{align*}
    If 1\% of the population has the disease, what is the probability that a randomly selected person tests positive?
    \tcblower
    Since $E_d$ and $\overline{E_d}$ are disjoint, we can partition the event $E_+$ into two sets, $E_+ \cap E_d$ and $E_+ \cap \overline{E_d}$. As such, we can calculate $P(E_+)$ by:
    \begin{align*}
        P(E_+)
        &= P(E_+ \cap E_d) + P(E_+ \cap \overline{E_d}) \\
        &= P(E_+ \mid E_d) P(E_d) + P(E_+ \mid \overline{E_d}) P (\overline{E_d}) \\
        &= 0.95 \cdot 0.01 + 0.05 \cdot 0.99 \\
        &= 0.059
    \end{align*}
\end{exbox}

\begin{thmbox}{Bayes' Theorem}{}
    Let $S$ be a set. If $\{B_1, \ldots, B_k\}$ is a partition of $S$, then for any $j \in \{1, \ldots, k\}$:
    \begin{align*}
        P(B_j \mid A) &= \frac{P(B_j \cap A)}{P(A)} \\
        &= \frac{P(A \mid B_j)P(B_j)}{\sum_{i=0}^{k} P(A \mid B_i) P(B_i) }
    \end{align*}
\end{thmbox}

\begin{exbox}{Probability using Baye's theorem}{}
    Revisiting Example \ref{ex:disease}, we have:
    \begin{align*}
        P(E_d \mid E_+)
        &= \frac{P(E_+ \mid E_d) P(E_d)}{P(E_+ \mid E_d)P(E_d) + P(E_+ \mid \overline{E_d}) P(\overline{E_d})} \\
        &= \frac{0.95 \cdot 0.01}{0.95 \cdot 0.01 + 0.05 \cdot 0.99} \\
        &= 0.16
    \end{align*}
\end{exbox}

\begin{exbox}{}{}
    Five identical bowls. Bowl $i$ contains $i$ white marbles and $5-i$ black marbles. A bowl is randomly selected, and two marbles are removed without replacement. Determine the probability of $P(\text{both white})$ and $P(\text{bowl 3} \mid \text{both white})$.
    \tcblower
    \todo[inline]{On paper notes; branching thing!!!}
\end{exbox}
