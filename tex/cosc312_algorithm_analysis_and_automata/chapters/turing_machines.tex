\chapter{Turing Machines}

The \dfntxt{Turing machine} is a much more powerful and accurate model for describing general purpose computers. It can solve any problem that a real computer can do. However, there are still some problems that a Turing machine (and thus a computer) cannot solve.

Turing machines model memory with a sequence of 0s and 1s called a \dfntxt{tape}. The machine has a ``tape head'' which can move across the tape, reading or writing one bit at a time.

\begin{dfnbox}{Turing Machine (TM)}{}
    A \dfntxt{Turing machine (TM)} is an abstract model that can generalize any computational problem or task.
    \tcblower
    More formally, a \dfntxt{Turing machine} is defined as a 7-tuple $(Q, \Sigma, \Gamma, \delta, q_0, q_a, q_r)$ where:
    \begin{itemize}[noitemsep]
        \item $Q$ is a finite set of states,
        \item $\Sigma$ is the input alphabet not containing the \dfntxt{blank symbol} \textvisiblespace,
        \item $\Gamma$ is the \dfntxt{tape alphabet} where $\textvisiblespace \in \Gamma$ and $\Sigma \subseteq \Gamma$,
        \item $\delta : Q \times \Gamma \to Q \times \Gamma \times \{L, R\}$ is the transition function,
        \item  $q_0 \in Q$ is the start state,
        \item $q_a \in Q$ is the accept state, and
        \item $q_r \in Q$ is the reject state where $q_a \neq q_r$.
    \end{itemize}
\end{dfnbox}

A Turing machine $M$ computes as follows:
\begin{enumerate}
    \item $M$ receives an input string $w \coloneq w_1 \ldots w_n \in \Sigma^*$ on the leftmost $n$ squares of the tape, and the rest of the tape is blank.
    \item The tape head starts on the leftmost square of the tape. Since $\Sigma$ does not contain the blank symbol, the first blank the machine sees marks the end of the input.
    \item The machine computes according to the transition function $\delta$. If it tries to move the tape head left off the left-end of the tape, it just stays in the same place for that move.
    \item Computation continues until it reachers either the accept or reject state, at which point it halts.
\end{enumerate}

As a Turing machine computes, three things may change: the current state, current tape contents, and current head location. These are collectively referred to as the \dfntxt{configuration} of the Turing machine.

\begin{dfnbox}{Turing Machine Configuration}{}
    Given a Turing machine, its \dfntxt{configuration} refers to its current state, current tape contents, and current head location. For a state $q$ and two strings $u$ and $v$, we write $u\ q\ v$ to denote the configuration where the current state is $q$, the current tape contents are $uv$, and the current head location is the first symbol of $v$.
\end{dfnbox}

\begin{dfnbox}{Yield}{}
    Given a Turing machine and two configurations $C_1$ and $C_2$, we say $C_1$ \dfntxt{yields} $C_2$ if the machine can legally move from $C_1$ to $C_2$ in a single step.
\end{dfnbox}

\begin{itemize}
    \item The \dfntxt{start configuration} using input $w$ and the initial state is $q_0$ is $\epsilon q_0w$.
    \item A \dfntxt{accepting configuration} is any configuration with state $q_a$.
    \item A \dfntxt{rejecting configuration} is any configuration with state $q_r$.
    \item Accepting and rejecting configurations are called \dfntxt{halting configurations} and cannot yield further configurations.
\end{itemize}

\begin{dfnbox}{Decider}{}
    We say a Turing machine is a \dfntxt{decider} if it halts on all possible inputs.
\end{dfnbox}

\begin{dfnbox}{Recognizes, Decides}{}
    For a language $L$, we say a Turing machine:
    \begin{itemize}
        \item \dfntxt{recognizes} $L$ if it accepts all strings in the language and either rejects or loops on strings not in the language. In this case, $L$ is \dfntxt{Turing-recognizable}.
        \item \dfntxt{decides} $L$ if it accepts all strings in the language and rejects all strings not in the language. In this case, $L$ is \dfntxt{Turing decidable}.
    \end{itemize}
\end{dfnbox}

\begin{thmbox}{CFLs are not Closed Under Complementation}{}
    \begin{proof}
        Suppose for contradiction that all CFLs are closed under complementation. Let $G_1$ and $G_2$ be CFGs. Then $\overline{L(G_1)}$ and $\overline{L(G_2)}$ are context-free. Since CFLs are closed under union, then $\overline{L(G_1)} \cup \overline{L(G_2)}$ is context-free.
        \todo[inline]{Finish proof, apply complement again to get intersectino, nto context fre}
    \end{proof}
\end{thmbox}

% \todo{elaboration on turing recognizable, difference between the two, why decideable is better}

\section{The Halting Problem}

