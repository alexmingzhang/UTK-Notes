\documentclass[12pt]{report}

\usepackage[margin=1in]{geometry}
\usepackage[math]{amznotes}
\usepackage{xfrac}
\usepackage{enumitem}
\usepackage{tikz}
\usetikzlibrary{arrows.meta}

\title{\textbf{Algorithm Analysis and Automata}\\
\large UT Knoxville, Spring 2023, COSC 312}
\author{Dr. Michael W. Berry, Alex Zhang}

\begin{document}
\maketitle
\tableofcontents

\chapter{Introduction}
As computer scientists, we like to think of ourselves as problem solvers. We can abstract problems in three distinct components: unknowns, data, and conditions.

In understanding how we abstract problems, computer scientists have conceptualized the Theory of Computation (TOC) which covers three areas:
\begin{itemize}
    \item \textbf{Automata} -- problem solving devices
    \item \textbf{Computability} -- framework that categorizes devices by computing power
    \item \textbf{Complexity} -- space complexity of tools used to solve them
\end{itemize}

In understanding our approach to solving problems, we think of the data, conditions, and unknowns as such:
\begin{itemize}
    \item \textbf{Data} exists as ``words'' in a given ``alphabet''
    \item \textbf{Conditions} form a set of words called a \textbf{language}
    \item \textbf{Unknowns} are boolean values, true if a word is in the language, false if otherwise
\end{itemize}

We denote an ``alphabet'' using $\Sigma$ and all possible words of finite length with $\Sigma^*$. Any subset $L \subseteq \Sigma^*$ is a \dfntxt{formal language}.

\chapter{Finite Automata}

\begin{dfnbox}{Finite Automaton}{}
    A \dfntxt{finite automaton} (FA) is a simple, idealized machine used to recognize patterns within input taken from some character set.
\end{dfnbox}

Finite automata can be used to generalize many applications ranging from parsers for compilers, pattern recognition, speech processing, and market prediction.

\begin{dfnbox}{Deterministic Finite Automaton (DFA)}{dfa}
    A \dfntxt{deterministic finite automaton} is a finite automaton where each state transition leads to exactly one state.
    \tcblower
    We can formally define a DFA as a 5-tuple $(Q, \Sigma, \delta, q_0, F)$ where:
    \begin{itemize}[noitemsep]
        \item $Q$ is a finite set of all possible states
        \item $\Sigma$ is a finite set of symbols called an \dfntxt{alphabet}
        \item $\delta : Q \times \Sigma \to Q$ is a \dfntxt{transition function}
        \item $q_0 \in Q$ is the \dfntxt{start state}
        \item $F \subseteq Q$ is the set of \dfntxt{accept states}
    \end{itemize}
\end{dfnbox}

Let's consider a simple, arbitrary finite automaton $M_1$.
\begin{itemize}
    \item $M_1$ receives input symbols one at a time
    \item $M_1$ transitions states based on input
    \item When $M_1$ reads the last symbol, it outputs whether or not it ends in an acceptable final state
\end{itemize}

\begin{dfnbox}{Acceptance}{}
    Let $M = (Q, \Sigma, \delta, q_0, F)$ be a finite automaton and $w = a_1 \ldots a_n$ be a string over $\Sigma$. We say $M$ \dfntxt{accepts} $w$ if there exists a sequence of states $r_0 \ldots r_n$ in $Q$ such that:
    \begin{itemize}[noitemsep]
        \item $r_0 = q_0$
        \item $\delta(r_i, a_{i+1}) = r_{i+1}$
        \item $r_n \in F$
    \end{itemize}
\end{dfnbox}

Finite automaton design approach:
\begin{enumerate}
    \item Identify the possible states of the finite automaton
    \item Identify the condition to change from one state to another
    \item Identify initial and final states
    \item Add missing transitions
\end{enumerate}

Given a finite automaton, we can deduce the language of possible inputs.

\begin{exbox}{Simple Finite Automaton}{}
    Let $M_1 \coloneq (Q, \Sigma, \delta, q_1, F)$ where $Q = \{q_1, q_2, q_3\}$, $\Sigma = \{0, 1\}$, and $F = \{q_2\}$. Define a possible transition function $\delta$.
    \tcblower
    The transition function $\delta : Q \times \Sigma \to Q$ must map every ordered pair of a state and letter to another state.

    \begin{center}\begin{tabular}{c | c | c }
        & 0 & 1 \\ \hline
        $q_1$ & $q_1$ & $q_2$ \\
        $q_2$ & $q_3$ & $q_2$ \\
        $q_3$ & $q_2$ & $q_2$
    \end{tabular}\end{center}

    Then, the language is:
    \[ L(M_1) = \left\{ (w \in \Sigma^*: 1 \in w) \land (\text{has even number of 0s following the last 1}) \right\} \]
\end{exbox}

\begin{exbox}{Simple Finite Automaton}{}
    Let $M_2 \coloneq (Q, \Sigma, \delta, q_1, F)$ where $Q = \{q_1, q_2\}$, $\Sigma = \{0, 1\}$, and $F = \{q_2\}$. Consider a possible transition function $\delta$ defined as such:

    \begin{center}\begin{tabular}{c | c | c }
        & 0 & 1 \\ \hline
        $q_1$ & $q_1$ & $q_2$ \\
        $q_2$ & $q_1$ & $q_2$ \\
    \end{tabular}\end{center}

    This time, our language is much simpler:
    \[ L(M_2) = \{ w \in \Sigma^* : w\ \text{ends in a 1} \} \]

    Now imagine if kept everything the same but made $F = \{q_1\}$. Because our finite automaton's initial state is $q_1$, we must now consider the possibility of the empty word. Then our language is:
    \[ L(M_2) = \{ w \in \Sigma^* : w\ \text{ does not end in 1} \} \]
\end{exbox}

\begin{exbox}{Pattern Recognizing Finite Automaton}{}
    Let $M_3 \coloneq (Q, \Sigma, \delta, q_1, F)$ where $Q = \{s, q_1, q_2, r_1, r_2\}$, $\Sigma = \{a, b\}$, and $F = \{q_1, r_1\}$. Consider a possible transition function $\delta$ defined as such:

    \begin{center}\begin{tabular}{c | c | c }
        & 0 & 1 \\ \hline
        $s$ & $q_1$ & $r_1$ \\
        $q_1$ & $q_1$ & $q_2$ \\
        $q_2$ & $q_1$ & $q_2$ \\
        $r_1$ & $r_2$ & $r_1$ \\
        $r_2$ & $r_2$ & $r_1$ \\
    \end{tabular}\end{center}

    Now our machine encompasses two smaller machines. $s$ acts as a branch point where we must lock ourselves into either $q$ states or $r$ states.

    % \begin{center}\begin{tikzpicture}
    %     \begin{scope}[every node/.style={circle,thick,draw}]
    %         \node (s) at (0,0) {s};
    %         \node (q1) at (-2,-3) {q1};
    %         \node (q2) at (-2, -6) {q2};
    %         \node (r1) at (2,-3) {r1};
    %         \node (r2) at (2, -6) {r2};
    %     \end{scope}

    %     \begin{scope}[>={Stealth[black]},
    %           every node/.style={fill=white,circle},
    %           every edge/.style={draw=red,very thick}]
    %         \path [->] (s) edge node {$a$} (q1);
    %         \path [->] (s) edge node {$b$} (r1);
    %         \path [->] (q1) edge node {$b$} (q2);
    %         \path [->] (r1) edge node {$a$} (r2);
    %     \end{scope}
    % \end{tikzpicture}\end{center}

    Then our language is:
    \[ L(M_3) = \{ w \in \Sigma^* : (w\ \text{starts and ends with}\ a) \lor (w\ \text{starts and ends with}\ b) \} \]
\end{exbox}

Now, let's start with a given language, then find an acceptable transition function $\delta$.
\begin{exbox}{Finding $\delta$ from Language}{}
    Suppose $\Sigma = \{a, b\}$. Let $F_1$ be a finite automaton that recognizes the language $A_1 \coloneq \{w : w\ \text{has at exactly two a's} \}$. Let $F_2$ be a finite automaton that recognizes the language $A_2 \coloneq \{w : w\ \text{has at least two b's} \}$.

    % \begin{center}\begin{tikzpicture}
    %     \begin{scope}[every node/.style={circle,thick,draw}]
    %         \node (1) at (0,0) {};
    %         \node (2) at (3,0) {};
    %         \node (3) at (6,0) {};
    %         \node (4) at (9,0) {};
    %     \end{scope}

    %     \begin{scope}[>={Stealth[black]},
    %           every node/.style={fill=white,circle},
    %           every edge/.style={draw=red,very thick}]
    %         \path [->] (1) edge [loop above] node {$b$} (1);
    %         \path [->] (1) edge node {$a$} (2);
    %         \path [->] (2) edge [loop above] node {$b$} (2);
    %         \path [->] (2) edge node {$a$} (3);
    %         \path [->] (3) edge [loop above] node {$b$} (3);
    %         \path [->] (3) edge node {$a$} (4);
    %         \path [->] (4) edge [loop above] node {$a,b$} (4);
    %     \end{scope}
    % \end{tikzpicture}\end{center}
\end{exbox}

\chapter{Regular Languages}

\begin{dfnbox}{Regular Language}{}
    A language is \dfntxt{regular} if there exists a finite automaton that can accept every possible word from the language.
\end{dfnbox}

\begin{dfnbox}{Concatenation}{}
    Let $A$ and $B$ be languages. $A$ \dfntxt{concatenate} $B$ is defined as:
    \[ A \circ B \coloneq \{ xy | x \in A \land y \in B \}  \]
\end{dfnbox}

\begin{dfnbox}{Star}{}
    \[ A^* \coloneq \{ x_1 \ldots x_k : k \geq 0, x_i \in A, 0 \leq i \leq k \} \]
\end{dfnbox}

\begin{thmbox}{Closure of Regular Languages}{}
    Class of regular languages is closed under intersection and closed under complementation.
\end{thmbox}

\section{Nondeterminism}

For complicated languages, it is difficult to create a completely deterministic finite automaton. If we forego determinism, we can have a more generalized finite automaton that allows for branching options.

\begin{dfnbox}{Nondeterministic Computation}{}
    A machine that is \dfntxt{nondeterministic} is allowed to choose its next state.
\end{dfnbox}

To introduce nondeterminism, we need more choices for the next state and allow state change without any input.

\begin{dfnbox}{Nondeterministic Finite Automaton (NFA)}{}
    A \dfntxt{nondeterministic finite automaton} is a finite automaton where each state transition can lead to an arbitrary number of states, chosen by the machine.
    \tcblower
    We can formally define an NFA as a 5-tuple $(Q, \Sigma, \delta, q_0, F)$ where:
    \begin{itemize}[noitemsep]
        \item $Q$ is a finite set of states
        \item $\Sigma$ is a finite set of symbols called an \dfntxt{alphabet}
        \item $\delta : Q \times (\Sigma \cup \{ \epsilon\}) \to \mathcal{P}(Q)$ is a \dfntxt{transition function}
        \item $q_0 \in Q$ is the \dfntxt{start state}
        \item $F \subseteq Q$ is the set of \dfntxt{accept states}
    \end{itemize}
    In this context, $\epsilon$ represents a nondeterministic choice by the machine.
\end{dfnbox}

To account for the nondeterminism, $\delta$ now points to a subset of $Q$. NFA computation is usually more expensive than DFA computation. In theory, every NFA can be converted into an equivalent DFA.

\begin{dfnbox}{Acceptance (NFA)}{}
    Let $N \coloneq (Q, \Sigma, \delta, q_0, F)$ be an NFA and $w \coloneq y_1 \ldots y_n$ be a string over $\Sigma \cup \{\epsilon\}$. We say $N$ \dfntxt{accepts} $w$ if there exists a sequence of states $r_0, \ldots, r_m \in Q$ such that:
    \begin{enumerate}[noitemsep]
        \item $r_0 = q_0$
        \item $\delta(r_i, y_{i+1}) = r_{i+1}$ for $i = 0, \ldots, m-1$
        \item $r_m \in F$.
    \end{enumerate}
\end{dfnbox}

\begin{thmbox}{Closure of Regular Languages}{}
    The class of regular languages is closed under the union operation.
    \tcblower
    \begin{proof}[Proof sketch]
        TODO: draw proof sketch here
    \end{proof}
\end{thmbox}

\section{DFA/NFA Equivalence}

The ambiguity in an NFA lies in the state transitions. We can convert an NFA into an equivalent DFA, shifting the ambiguity to the states.

\begin{tecbox}{Converting NFA to DFA}{}
    To convert an NFA to a DFA, we carry out the following steps:
    \begin{enumerate}
        \item Let $N \coloneq (Q, \Sigma, \delta, q_0, F)$ be an NFA that recognizes the language $A$.
        \item Construct the DFA $M$ that also recognizes $A$. For convenience, define:
        \[ M \coloneq (Q\prime, \Sigma, \delta\prime, q_0\prime, F\prime) \]
        \item For all $R \subseteq Q$, define $E(R)$ to be the collection of all states that can be reached from $R$ by going along $\epsilon$ transitions, including members of $R$ themselves.
        \item Modify $\delta\prime$ to place additional edges on all states that can be reached by going along $\epsilon$ edges after every step.
        \item Set $q_0\prime = E(\{q_0\})$ and $F\prime = \{ R \in Q\prime : R\ \text{contains an accept state of}\ N \}$.
    \end{enumerate}
\end{tecbox}

\chapter{Irregular Languages}
\begin{dfnbox}{Regular Expression (RE)}{}
    We say that $R$ is a \dfntxt{Regular expression} if $R$ is one of the following:
    \begin{itemize}[noitemsep]
        \item $a$ for some $a \in \Sigma$ (a symbol of the alphabet)
        \item $\epsilon$ (language contains only the empty string)
        \item $\emptyset$ (language contains no strings)
        \item $(R_1 \cup R_2)$, where $R_1$ and $R_2$ are regular expressions
        \item $(R_1 \circ R_2)$, where $R_1$ and $R_2$ are regular expressions
        \item $R_1^*$, where $R_1$ is a regular expression
    \end{itemize}
\end{dfnbox}

\begin{thmbox}{}{}
    A language is regular if and only if some regular expression describes it.
    \tcblower
    \begin{proof}
        If a language $R$ is described by a regular expression, then $A$ is recognized by an NFA. Thus, $A$ must be regular. If the language $R$ is regular, then it is recognized by a DFA from which a regular expression can be deduced.
    \end{proof}
\end{thmbox}

\begin{exbox}{Irregular Languages}{}
    Consider the following languages:
    \begin{itemize}[noitemsep]
        \item $B = \{ 0^n 1^n : n \geq 0\}$ is not regular.
        \item $C = \{ w : w\ \text{has an equal number of 0s and 1s} \}$ is not regular. We would need infinite states to account for all possible inputs.
        \item $D = \{ w : w\ \text{has an equal number of 01 and 10 substrings} \}$ is regular. We can think of this as recording transitions between 0s and 1s, and vice versa. In this sense, every 01 must be eventually followed by a 10, and every 10 must be eventually followed by a 01.
    \end{itemize}
\end{exbox}

\section{Pumping Lemma}
How do we prove that a language is irregular? We would have to prove that there does not exist any machine that recognizes that language.

\begin{dfnbox}{Pumping, Pumping Length}{}
    All strings in a language can be \dfntxt{pumped} if they are as long or longer than a specified \dfntxt{pumping length}.
\end{dfnbox}

In other words, every string contains a section that can be repeated any number of times, and the resulting string is still in the language. For example, we can write sqrt(sqrt(sqrt($\ldots$(sqrt($x$))$\ldots$))) in C, and it's still valid.

\begin{thmbox}{Pumping Lemma}{pl}
    Let $A$ be a regular language. There exists a pumping length $p$ such that, for any string $s$ of length $p$ or more, $s$ may be divided into three pieces, $s = x \circ y \circ z$ where all the following hold:
    \begin{enumerate}[noitemsep]
        \item $x \circ y^* \circ z \in A$
        \item $\abs{y} > 0$
        \item $\abs{x \circ y} \leq p$
    \end{enumerate}
    \tcblower
    \begin{proof}[Proof sketch]
        As a guide, the proof follows this general form:
        \begin{itemize}[noitemsep]
            \item Let $M \coloneq (Q, \Sigma, \delta, q_1, F)$ be a DFA that recognizes $A$. Let $p$ be the number of states.
            \item Show that any string $s \in A$ where $\abs{s} \geq p$ can be broken into $x \circ y \circ z$ satisfying the three conditions.
            \item If there are no strings in $A$ of length $p$ or more, then the lemma is true because all three conditions hold for all strings of length $p$ or more.
        \end{itemize}
    \end{proof}

    A formal proof is as follows:
    \begin{proof}
        Let $M \coloneq  (Q, \Sigma, \delta, q_1, F)$ be a DFA that recognizes $A$. Let $p \coloneq \abs{Q}$. Let $s \coloneq s_1 s_2 \ldots s_n$ be a string over $\Sigma$ with $n \geq p$ and $r = r_1 r_2 \ldots r_{n+1}$ be the sequence of states encountered while processing $s$.
        TODO: finish proof
    \end{proof}
\end{thmbox}

\begin{tecbox}{Proving a Language is Irregular}{}
    To prove that a language $A$ is not regular, we:
    \begin{enumerate}
        \item Suppose for contradiction $A$ is regular
        \item Find $s \in A$ where $\abs{s} \geq p$ that cannot be pumped; consider all the ways of dividing $s$ into $x \circ y \circ z$, and show that for each division, at least one of the conditions fail.
    \end{enumerate}
\end{tecbox}

\begin{exbox}{Simple Pumping Lemma Example}{}
    Prove that the language $B \coloneq \{ 0^n 1^n : n \geq 0 \}$ is not regular.
    \tcblower
    We can show that any substring of $B$ cannot be pumped by contradicting the first condition of the \nameref{thm:pl}.
    \begin{proof}
        Suppose for contradiction $B$ is regular. Let $p$ be the pumping length for $B$. Let $s \coloneq 0^p1^p \in B$. Then $\abs{s} = 2p > p$. By the \nameref{thm:pl}, we can partition $s$ into three substrings, say $x,y,z$, such that $x \circ y^* \circ z \in B$. Let $s\prime \coloneq x \circ y \circ y \circ z$. WE will show $s\prime \notin B$. Consider the three possible cases for the contents of $y$:
        \begin{enumerate}
            \item $y \in 0^+$. Then. $s\prime$ has more $0$s than $1$s. Thus, $s\prime \notin B$, contradicting the first condition of the \nameref{thm:pl}.
            \item $y \in 1^+$. Then, following the logic from the first case, this is not possible.
            \item $y$ consists of $0$s and $1$s. Then $y \circ y \notin 0^n1^n$, so $s \prime \notin B$.
        \end{enumerate}
    \end{proof}

    We can simplify the above proof by targeting condition 3 instead:
    \begin{proof}
        By the third condition of the \nameref{thm:pl}, we have $\abs{x \circ y} \leq p$. Hence, $y$ could only contain $0$s.
    \end{proof}
\end{exbox}

\begin{exbox}{Another Pumping Lemma Example}{}
    Prove that the language $E \coloneq \left\{ 0^i 1^j : i > j \right\}$ is not regular.
    \tcblower
    \begin{proof}
        Suppose for contradiction $$E$$ is regular. Let $$p$$ be the pumping length for $$E$$, and let $$s \coloneq 0^{p+1}1^p \in E$$. Then $$\abs{s} = (p+1) + p > p$$. By the \nameref{thm:pl}, we can partition $$s$$ into three substrings, say $$x,y,z$$, such that $$x \circ y^* \circ z \in E$$. Then, by the third condition, we have $$\abs{x \circ y} \leq p$$. Note that there are $$p+1$$ number of $$0$$s at the beginning of $$s$$. Thus, $$x \in 0^*$$ and $$y \in 0^+$$. Consider the case when $$i = 0$$. Then $$x \circ y^+ \circ z \notin E$$, leaving only $$x \circ z$$. However, $$\abs{y} > 0$$ by the second condition, so $$x \circ z$$ would lose at least one $$0$$. Thus, $$x \circ z \notin E$$, so $$x \circ y^* \circ z \notin E$$. Therefore, $$E$$ is not regular.
    \end{proof}
\end{exbox}

\amzindex
\end{document}
