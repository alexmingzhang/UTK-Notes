\chapter{Floating Point Numbers}
Brief: How computers store numbers

\section{Integers}
Every integer can be represented as:
\[ \alpha_n \beta^n + \alpha_{n-1} \beta^{n-1} + \cdots + \alpha_{1} \beta^1 + \alpha_0 \beta^0 \]
where $\beta \in \Z$ represents the base we choose, and $\alpha \in \Z$ must be $0 \leq \alpha < \beta$.

(TODO: binary representation for computers, two's complement for signed integers)

A negative integer $-y$, $1 \leq y \leq 2^{31}$ is stored as the binary representation of $2^{32} - y$.

\section{Real Numbers}
