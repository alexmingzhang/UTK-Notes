\chapter{Introduction}

\begin{dfnbox}{Von Neumann architecture}{}
    The \dfntxt{Von Neumann architecture} describes a computer architecture that fetches, decodes, and executes instructions from memory.
\end{dfnbox}

In the Von Neumann architecture, a program is composed of many instructions. The central computing unit (CPU) executes each instruction one after the other until the program completes.

\begin{dfnbox}{Operating system (OS), virtualization}{}
    An \dfntxt{operating system} is some software that manages a system's resources and makes it easy to run programs. Generally, it creates abstractions of physical resources such as the processor or memory---a technique called \dfntxt{virtualization}.
\end{dfnbox}

Virtualization allows an operating system to efficiently and easily manage resources as well as providing concurrency to processes. Concurrency is one of the most difficult things to get right, especially with the advent of multicore CPUs.
