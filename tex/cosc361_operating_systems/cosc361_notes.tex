\documentclass[code]{amznotes}

\begin{document}

\chapter{Introduction}
Computer science is all about problem-solving, and as computer scientists, we have developed a method of abstracting problems into three key components: unknowns, data, and conditions. To further our understanding of this process, we have developed the Theory of Computation (TOC), which encompasses three main areas: Automata, Computability, and Complexity.

Automata are problem-solving devices that help us model and solve problems. Computability provides a framework for categorizing these devices based on their computing power, while Complexity measures the space complexity of the tools we use to solve problems.

When approaching problems, we think of the data as ``words'' in a given ``alphabet'', while conditions form a set of words known as a language. The unknowns in this equation are boolean values, which are true if a word is in the language and false if it is not.

We can define these terms more formally:

\begin{dfnbox}{Alphabet, Symbols}{}
    An \dfntxt{alphabet} is any nonempty, finite set. Members of the alphabet are called the \dfntxt{symbols} of the alphabet.
\end{dfnbox}

We denote the alphabet using the Greek letter $\Sigma$.

\begin{dfnbox}{String}{}
    A \dfntxt{string} over an alphabet $\Sigma$ is a (finite) sequence of symbols in $\Sigma$.
    \begin{itemize}[noitemsep]
        \item The length of a string $w$ is denoted by $\abs{w}$.
        \item $\epsilon$ represents the \dfntxt{empty string}, the string of length $0$
        \item The \dfntxt{concatenation} of $w_1$ and $w_2$ is the string obtained from appending $w_2$ to the end of $w_1$. We denote concatenation either by juxtaposition $w_1w_2$, or with superscript notation $w_1^n$.
    \end{itemize}
\end{dfnbox}

For example, if $w_1 = \text{hello}$ and $w_2 = \text{world}$, then $w_1w_2 = \text{helloworld}$.

% Overall, the Theory of Computation provides a framework for understanding how we solve problems using computers. By breaking down problems into their component parts, we can use automata to model and solve them, and categorize these solutions based on their computing power and space complexity.

\chapter{Virtualization}

\section{Processes}

\begin{dfnbox}{Process, process state}{}
    A \dfntxt{process} is an OS abstraction of a running program. A process can be described by its \dfntxt{state}, which includes:
    \begin{itemize}
        \item memory or \dfntxt{address space} which contains the program's instructions and data that the program reads from and writes to;
        \item registers which many instructions must read from or write to, as well as specialized registers such as the program counter, stack pointer, and frame pointer\footnote{The program counter specifies the very next instruction for the program to execute. A stack pointer and frame pointer are used to manage the process' stack memory.}; and
        \item I/O information that specifies access to persistent storage devices.
    \end{itemize}
\end{dfnbox}

Nowadays, our computers are often running tens if not hundreds of processes concurrently, despite lacking tens or hundreds of CPU cores. The OS presents the illusion of many CPUs through virtualization.

\begin{dfnbox}{Time sharing, space sharing}{}
    \dfntxt{Time sharing} is a basic technique used by an OS to share a resource. It involves rapidly switching between executing different concurrent processes, allowing for the illusion of concurrency at the cost of performance as more processes are being run. \dfntxt{Space sharing} simply divides a resource among the processes that use it (e.g. disk space).
\end{dfnbox}

The OS does not arbitrarily choose which process it will start and stop running; it has a \dfntxt{scheduling policy} which enforces time sharing decisions, which we discuss later.

\paragraph{Process API}
The operating system makes available an application programming interface (API) to manage processes. The \dfntxt{process API} includes functionality for
\begin{itemize}[noitemsep]
    \item creating new processes,
    \item destroying running processes,
    \item waiting for a process to finish running,
    \item suspending and resuming processes,
    \item obtaining status information about a process, and
    \item other miscellaneous control features.
\end{itemize}

We discuss this in more detail in Section \ref{sec:processapi}.

\paragraph{Process Creation}
To create a process, an OS:
\begin{enumerate}
    \item loads its code and any static data (e.g. initialized variables) into memory as the address space for the process\footnote{Older operating systems tend to load programs eagerly, meaning loading all its code at once. Newer operating systems tend to load code lazily, meaning loading code as needed during execution.};
    \item allocates memory for the process' \dfntxt{stack}, used for local variables, function parameters, and return addresses, as well as arguments such as the \texttt{main()} function's \texttt{argc} and \texttt{argv};
    \item may allocate memory for the process' \dfntxt{heap};
    \item does I/O setup, such as opening file descriptors for stdin, stdout, and stderr; and finally
    \item starts the program execution at the entry point \texttt{main()}, transfering control of the CPU to the newly-created process.
\end{enumerate}

\paragraph{Process States}
A process can be in the following states:
\begin{enumerate}
    \item \dfntxt{Running:} The processor is executing instructions of the program.
    \item \dfntxt{Ready:} A process is ready to run, but the OS has not chosen to run it yet.
    \item \dfntxt{Blocked:} A process has performed some operation that requires it to wait until another thing happens first.
\end{enumerate}

Moving a process from ready to running is \dfntxt{scheduling} a process; moving from running to ready is \dfntxt{descheduling} a process. When a process is blocked, the OS will wait for the desired event to occur, then unblock it by moving it to the ready state.

Deciding whether to move a process to running or ready and vice versa is dictated by an OS scheduler, which will be discussed later.

Some other niche states include:
\begin{itemize}
    \item \dfntxt{Initial/Embryo:} The process is being created by the OS.
    \item \dfntxt{Final/Zombie:} The process exited but has not yet been cleaned up; useful for a parent process to see if the child process executed successfully. The parent calls \texttt{wait()} to wait for the child process to finish and to indicate to the OS that it can clean up the child process.
\end{itemize}

\paragraph{Data Structures}
To keep track of processes, the OS maintains a list of processes (called a \dfntxt{process list}) as well as information to keep track of which processes are running and bookkeeping for blocked processes.

The OS stores a couple pieces of important information for each process in what's called a \dfntxt{process control block (PCB)}. This is composed of:
\begin{itemize}
    \item a \dfntxt{register context} which holds the contents of a register right before the process is moved from running to ready,
    \item information about the process memory, such as where it starts and the size of the memory,
    \item process state and a unique identifier (PID),
    \item a pointer to the PCB of the parent (if applicable),
    \item a list of open files for the process,
    \item the current working directory of the process, and
    \item much much more (depending on the operating system).
\end{itemize}

\todo[inline]{More things from the proc struct?}

\section{Process API} \label{sec:processapi}

\todo[inline]{Fork, exec, wait system calls}

\paragraph{API Design Motivations}
The separation of \texttt{fork()} and \texttt{exec()} is essential to UNIX shells, enabling the shell to run code after forking and before execing. For example, in 


\section{Limited Direct Execution}
The OS gives users the illusion of concurrency by switching execution time between active processes, a kind of virtualization called \dfntxt{time-sharing}. With this technique comes some pertinent design challenges:
\begin{itemize}
    \item \textbf{Performance:} how can we implement virtualization without adding excessive overhead to the system?
    \item \textbf{Control:} how can we run processes efficiently while retaining control over the CPU?
\end{itemize}

\paragraph{Limited Direct Execution}
The simplest approach is simply running processes natively on hardware CPU. In short, when we want to start a process, the OS goes through the following steps:
\begin{enumerate}[noitemsep]
    \item Create entry for process list.
    \item Allocate memory for program.
    \item Load program into memory.
    \item Set up stack with argc/argv.
    \item Clear registers.
    \item Execute call main(), in which the program takes over control and runs main until it returns.
    \item After the program returns, free memory of process.
    \item Remove from process list.
\end{enumerate}
This approach is straightforward and has the advantage of being fast. However, it poses two problems:
\begin{enumerate}
    \item The OS gives full trust to the program to run successfully without doing anything we don't want it to do.
    \item The OS cannot suspend the process while it's running, so we cannot implement any kind of time-sharing technique.
\end{enumerate}

\paragraph{Restricted Operations and System Calls}
To prevent processes from doing things we don't want them to do, we introduce two modes of execution: \dfntxt{user mode} and \dfntxt{kernel mode}.

\begin{dfnbox}{User mode, kernel mode}{}
    \dfntxt{User mode} and \dfntxt{kernel mode} describe modes of operation for processes.
    \begin{itemize}
        \item Code that runs in \dfntxt{user mode} has restricted capabilities. For example, user mode processes cannot issue I/O requests.
        \item Code that runs in \dfntxt{kernel mode} has unrestricted capabilities, allowed to do anything.
    \end{itemize}
\end{dfnbox}

When a user process wants to perform a restricted operation, the kernel offers a set of \dfntxt{system calls} to user processes.
\begin{itemize}
    \item These carefully expose certain key pieces of functionality to user processes, such as I/O operations, creating/killing/communicating with processes, allocating memory, and much more.
    \item  All system calls start with a \dfntxt{trap} instruction which transfers control to the kernel and raises privilege level to kernel mode. Once done, the kernel issues a \dfntxt{return-from-trap} instruction which returns into the calling program and deescalate privilege to user mode.
    \item Before executing the trap instruction, the processor saves a copy of the sensitive registers onto a per-process \dfntxt{kernel stack}\footnote{The kernel stack is used exclusively when invoking kernel-mode instructions, like trap. The user stack is responsible for local variables, function parameters, return addresses, etc. The user stack is generally just called ``stack''.}. When the system call returns, the original registers get restored.
    \item Each system call has a corresponding number, which is handled by the OS in some functionality called the \dfntxt{trap handler} to jump into the right portion of kernel code. This protects the kernel code from being arbitrarily jumped into.
    \item This set of correct system call numbers is established when the OS boots and is only possible in kernel mode.
\end{itemize}

There are two distinct phases in limited direct execution:
\begin{enumerate}
    \item At boot time, the kernel initializes the trap table which gets saved by the CPU for future use. This is done by an instruction only available in kernel mode.
    \item When running a process, sets up some bookkeeping info for a process and issues a return-from-trap to start running the process. If the process issues a system call, it ``traps'' back to the OS which handles it and returns via a return-from-trap instruction. The process ends by returning from \texttt{main()}, which usually returns into some end code such as a trap back into the OS. From here, the OS cleans up the dead process.
\end{enumerate}

\todo[inline]{Figure 6.2 Limited Direct Execution Protocol}

\paragraph{Switching Between Processes}
With all this, we have a way for user processes to invoke kernel functionality. However, how can the kernel voluntarily suspend a process to switch to another? There are a few approaches.

\paragraph{Cooperative Approach}
If the OS is willing to trust processes to behave reasonably, then we could just hope that processes voluntarily relinquish control, whether it be through system calls or trying to invoke an illegal operation (e.g. divide by zero, which generates a trap instruction). From there, the OS can decide to switch to another process. In this approach, we can offer processes a \dfntxt{yield} system call. All it does is relinquish control of the current process, letting the OS decide what to do from there.

Obviously, this is a flawed approach. A process can simply invoke an infinite loop, never giving up control of the processor.

\paragraph{Non-cooperative Approach}
The OS can reclaim control at its own will by interrupting the running process. This obviously cannot be done in software alone, so we introduce a hardware timer that interrupts the processor at a fixed interval. When this interrupt happens, the running process is halted, and the OS runs a pre-configured interrupt handler. From here, the OS can decide to switch to another process. This interrupt handler is configured at boot time, a privileged operation similar to establishing the trap table.

\paragraph{Context Switch}
In practice, the OS regains control through both cooperative and non-cooperative approaches. Once it does so, a scheduler decides whether to keep running the same process or switch to another. If it does choose to switch, it performs what is called a \dfntxt{context switch}.

\begin{dfnbox}{Context switch}{}
    A \dfntxt{context switch} is a low-level piece of code run by the OS when it switches which process it runs. In it, the OS saves important register values for the previous process (e.g. on its kernel stack) and restore a few register values for the next process (from its kernel stack).
\end{dfnbox}

The context switch is crucial in creating smooth transitions when switching between processes. Some registers that get saved include general purpose registers, program counter (PC), and stack pointers of the current process.

There are two kinds of context switches, depending on how they're invoked:
\begin{enumerate}[label=\textbf{\Alph*.}]
    \item When a timer interrupt happens, the hardware implicitly saves the user registers of the running process into the kernel stack of the running process.
    \item When the OS decides to switch to another process, the OS explicitly saves the kernel registers into the process structure of the previous process.
\end{enumerate}

\paragraph{Concurrency}
It's possible that an interrupt can happen while the OS is already handling an interrupt. A simple fix is to just disable interrupts during interrupt processing. A downside is that disabling interrupts for too long can lead to lost interrupts. There are other solutions, such as some locking mechanisms to protect concurrent access. We discuss this more later.

\section{Scheduling}

\begin{dfnbox}{Preemptive, non-preemptive}{}
    A scheduling policy is \dfntxt{preemptive} if it can stop the currently running process in favor of starting/resuming another process (called a context switch). A scheduling policy is \dfntxt{non-preemptive} if it always executes the currently running process to completion, never interrupting it.
\end{dfnbox}

Nearly all modern operating systems use a preemptive scheduling policy. Non-preemptive scheduling policies are prone to \dfntxt{starvation}, where a process can wait an arbitrary amount of time before starting.

\paragraph{Workload Assumptions}
The operating system must enforce a scheduling policy to determine which processes it should execute at any given time. These scheduling policies almost always make choices based on the current active processes or \dfntxt{jobs}, referred to collectively as the \dfntxt{workload}.

To simplify our approach, we will consider a workload in which:

\begin{enumerate}[noitemsep]
    \item Each job runs for the same duration.
    \item All jobs arrive at the same time.
    \item Once started, each job runs to completion.
    \item All jobs only use the CPU and perform no I/O.
    \item The run-time of each job is deterministic.
\end{enumerate}

It will be unrealistic to assume that practical workloads follows these assumptions. 

\paragraph{Scheduling Metrics}
We compare different scheduling policies by \dfntxt{scheduling metrics}. Some common ones include:
\begin{itemize}
    \item \dfntxt{Turnaround time:} the time between when the job first arrives and when the job finishes.
    \item \dfntxt{Response time:} the time between when the job first arrives and when the job first starts running.
    \item \dfntxt{Fairness:} ensuring every job gets a good chance to run.
\end{itemize}

We call turnaround time and response time \dfntxt{performance} metrics.

% A simple scheduling metric is \dfntxt{turnaround time}, which is the elapsed time between when the job arrived in the system and when the job finished. More formally, it is calculated by:
% \[ T_\text{turnaround} = T_\text{completion} - T_\text{arrival} \]
% The turnaround is a performance metric. Other metrics may be more interested in fairness, which is often at odds with performance in scheduling. Another metric by which to compare schedules is \dfntxt{response time}. It's the amount of time between when the job first arrives and when the job gets its first run. It's calculated by:
% \[ T_\text{response} = T_\text{first run} - T_\text{arrival} \]
% This metric is important to users interacting with the computer. The scheduling policies discussed thus far are good for turnaround but are not necessarily optimal for response time.


\paragraph{First In, First Out (FIFO)}
The most basic scheduling policy is \dfntxt{First In, First Out (FIFO)}. As the name suggests, the very first job gets fully executed, and any jobs that arrive must wait in a queue to get executed. As such, it is a non-preemptive scheduling policy.

\begin{tabularx}{\linewidth}{| X | X |}
    \hline
    \textbf{Pros} & \textbf{Cons} \\ \hline
    Incredibly simple to understand and implement.
    &
    \dfntxt{Convoy Effect:} Long-duration jobs will force long waits for future jobs, yielding bad turnaround times and bad response times.
    \\ \hline
\end{tabularx}

\paragraph{Shortest Job First (SJF)}
This non-preemptive scheduling policy solves the convoy effect present in FIFO. It runs the shortest job first, then the next shortest, and so on. This works fine if all the jobs come at the same time, but can go bad if a big job arrives before a small job.

\begin{tabularx}{\linewidth}{| X | X |}
    \hline
    \textbf{Pros} & \textbf{Cons} \\ \hline
    Good turnaround times, so long as jobs all arrive at the same time.
    &
    Bad response times, and bad turnaround times if big jobs arrive first.
    \\ \hline
\end{tabularx}

\paragraph{Shortest Time-to-Completion First (STCF)}
\dfntxt{Shortest Time-to-Completion First} adds preemption to SJF, solving the issues in SJF. That is, if we allow the operating system to interrupt a currently running job, then we can solve the problem in SJF. Whenever a new job arrives, the OS can decide to suspend the currently running job in favor for the new job if the new job will complete faster than the current job. 

\begin{tabularx}{\linewidth}{| X | X |}
    \hline
    \textbf{Pros} & \textbf{Cons} \\ \hline
    Always good turnaround times.
    &
    Bad response times.
    \\ \hline
\end{tabularx}

\paragraph{Round Robin}
\dfntxt{Round-Robin (RR)} scheduling is optimized for response time. Instead of running jobs to completion, RR runs a job for a set \dfntxt{time slice}, then switches to the next job in the queue. RR performs context switches only on timer-interrupt periods, so the duration of each time slice must be a multiple of the timer-interrupt period. Performing context switches often can be costly, so the time slices are often larger multiples of the timer-interrupt period (at the cost of reduced response time).\todo{mention amortization in here}

\begin{tabularx}{\linewidth}{| X | X |}
    \hline
    \textbf{Pros} & \textbf{Cons} \\ \hline
    Absolute best response time.
    &
    Terrible turnaround times.
    \\ \hline
\end{tabularx}

\paragraph{Considering I/O}
When a process access I/O resources, it will be blocked waiting for the I/O operation to finish. As such, it's best for the scheduler to switch to another job while the current process is waiting for I/O to finish.

In the case of preemptive sheduling, an I/O operation can incur a context switch. Read and write can both cause a process to be blocked. Meanwhile, the OS may schedule another process to run while the I/O job finishes. \todo{Check this fact}

\todo[inline]{MORE!!! overlapping stuff}

\paragraph{Shortest CPU Burst First}
In practice, we can't accurately guess the runtime of jobs. We instead think of a process as a series of \dfntxt{CPU bursts}, which are periods of execution without any internal events. A CPU burst usually ends when a context switch is necessary, such as an I/O operation. 


\dfntxt{Shortest CPU Burst First} is the best way to approximate SJF. Although we cannot estimate the runtime of an entire job, we can reasonably estimate the runtime of a single CPU burst. We estimate a CPU burst by how long the current CPU burst lasted. The longer it runs, the more likely \todo{finish this}

\todo{figure out shit}
Process Priority:
\begin{itemize}
    \item Called ``niceness'' in Linux
    \item A single number associated with a process
    \item Simple order ready queue as a priority queue
    \item Calculate effective priority from base priority and wait time in queue (since last time random) called \dfntxt{aging}
\end{itemize}

% Note that the cost of context switching does not arise solely from the
% OS actions of saving and restoring a few registers. When programs run,
% they build up a great deal of state in CPU caches, TLBs, branch predictors,
% and other on-chip hardware. Switching to another job causes this state
% to be flushed and new state relevant to the currently-running job to be
% brought in, which may exact a noticeable performance cost [MB91].


Underlying theme: minimality; shaped infrastructure for years to come

What is a process? There are many ways to interpret this question:
\begin{enumerate}
    \item It can be natively executed alongside the operating system
    \item It can be interpreted code, where only an interpreter is natively executed (e.g. the Java virtual machine)
    \item 
\end{enumerate}

% An OS process is:
% \begin{itemize}
%     \item A collection of states, including a memory image, data register contents, system register contents (todo: process control blocks). Proc-to-OS; subroutine call, OS-to-Proc: return. The OS can choose which process to return to (i.e. running a process).
%     \item 
% \end{itemize}

The pieces of a process are:
\begin{itemize}
    \item Memory image (contents of RAM)
    \item Register contents (both user-visible and OS managed)
    \item Process control block (an OS controlled data structure in kernel mode)
\end{itemize}

Inside the process control block (PCB) are:
\begin{itemize}
    \item A unique integer identifier for the process called a \dfntxt{process ID}
    \item A \dfntxt{state} which denotes the current state of the process, such as running
    \item A pointer to the parent process
\end{itemize}

Process Creation involves:
\begin{itemize}
    \item Allocation of resources within the OS
    \item User identity (UID), a field of the process control block
    \item Group identity (GID), which identifies a set of UIDS
    \item Limits on resources and billing
\end{itemize}

In the POSIX standard, a process creates another process by the \texttt{fork()} system call. This creates a child process that replicates the parent process, allocating a PCB and memory for the child. It populates the child PCB with copies of parent's fields except for identity (PID) and data structures that are allocated and copied. It also copies memory image and register values from parent to child.

Both the child and parent processes commence execution by returning from the \texttt{fork()} system call. In the parent process, \texttt{fork()} returns the PID of the child. In the child process, \texttt{fork()} returns \texttt{0}. This is the definitive way to differentiate between child and parent process, and it should be used to control the flow of logic in your program.

\texttt{fork()} allows for asynchronous computation, perhaps letting the child process execute a subroutine call. To synchronize the two processes after the child has done its task, the parent may call \texttt{wait()}, which waits for the child process to die in order to continue execution. It will return the PID of the child process. If there are more than one dead child processes, \texttt{wait()} returns the PID of one random child process.

The \texttt{init} process is the initial process that creates all other processes. There is a file called an administrative file that gives the init process a list of processes to fork off and start. Init has some special properties:
\begin{itemize}
    \item \texttt{pid} set to 0,
    \item \texttt{uid} set to 0 (which is the root user's \texttt{uid})
\end{itemize}
Every terminal device attached to the machine has an associated \texttt{login} process, which outputs ``login'' to the terminal, and waits for user input to determine the identity of the user. After this, it changes UID to the respective user.

Memory management:
\begin{itemize}
    \item control what a process can see and change
    \item 
\end{itemize}

It's an allocation of resources---a data structure within the operating system. The operating system devotes resources to each process. This is a power that was deemed too sensitive for the user to have control over. Moreover, the creation of processes was often a dangerous task. Because of this, this is something that the operating system takes care of.

The operating system is considered to be out of the hands of the end user. There are several layers that make up the operating system, including:
\begin{itemize}
    \item Hardware
    \item Boot loader, kernel, and OS
    \item Utilities
    \item User programs
\end{itemize}
Anything other than user programs was out of the user's control and are deemed ``protected'' from the user.

In many operating systems before UNIX, there was a supervisory mode of operation which existed between the user and kernel. Nowadays, the POSIX standard has since remove the supervisory mode, giving us user and kernel modes.

OS mechanisms:
\begin{itemize}
    \item Native execution, available to the bootloader and operating system without memory management.
    \item Multiple registers contexts in hardware
    \item Base and limit registers for memory management
    \item 
\end{itemize}

\begin{notebox}
    Virtual memory is completely implemented in hardware and thus has zero overhead for executing processes.
\end{notebox}

User mode is a sandbox. Going from kernel to user mode is completely controlled by the kernel. However, to go from user to kernel, we need a special mechanism. There are generally four user-to-kernel transitions. These are to allow the kernel to regain control from the user:
\begin{enumerate}
    \item Clock ticks: save PC and PCR values to memory, load the PC for a kernel entry point, and load a PCR value for kernel execution. The PC and PCR values are called the \dfntxt{kernel entry vector}. Then, we save user-accessible registers to the process control block (the \texttt{proc} data structure).
    \item Hardware interrupts: these are handled just like clock ticks, but are generated from hardware other than the CPU. These can be disabled separated.
    \item System calls: an instruction \texttt{TRAP} where we load the PC and PCR from entry vector. There is sometimes an argument (syscall number) that indicates the purpose of the system call. The syscall number may index into a list of entry vectors, each corresponding to a different piece of OS code. For example, \texttt{fork()} and \texttt{read()} have a unique system call number as the argument to \texttt{TRAP} which correspond to a specific functionality. It looks like a subroutine call. A system call's arguments are stored on the stack, and the result of the system call is stored in a register.
    \item 
\end{enumerate}

There are four kernel entry events to transition from executing a user program to executing the OS:
\begin{enumerate}
    \item Exceptions
    \item Clock tick
    \item System calls (through a trap instruction)
    \item Hardware interrupts
\end{enumerate}

Note that clock ticks and hardware interrupts are not part of the software, rather controlled by the hardware itself. On the other hand, exceptions and system calls are invoked directly by the software itself.

Sandbox mode includes the following controls:
\begin{itemize}
    \item Create sandbox with memory management and kernel/user mode
    \item Kernel mode is all-powerful
    \item We want to minimize resources and privileges of processes (weak, simple, generic, resource-limited)
    \item Superuser (i.e. root) identity has UID 0. For example, the \texttt{init} process (the process creating during system bootstrap) runs as root. The \texttt{login} process also runs as root.
    \item 
\end{itemize}

\begin{notebox}
    Running persistent processes as root is prone to security vulnerabilities (e.g. finger daemon, Cornell worm).
\end{notebox}

In UNIX, the \texttt{exec()} system call takes a string filename, reads that file, puts the corresponding portions into the currently running process' memory, and starts executing the new file. It reuses the same PCB, with the code and data portions being overwritten by whatever is in the new file. Thus, when a process calls \texttt{exec()}, it ``becomes'' a new process, and \texttt{exec()} never returns.

The fork-exec sequence seems to be wasteful. (TODO: more). This used to be optimized in the following way. If a process calls \texttt{fork()}, it creates a copy of the memory image for the child. We can avoid this copying by just letting the child use the same memory as the parent. This is done through the \texttt{vfork()} call, where the parent is immediately blocked, and the child uses the parent's same memory image without copying. This must only be used when the child does not interfere with the parent's memory image, such as when it performs \texttt{exec()} immediately afterwards. Nowadays, virtual memory takes care of this overhead in a much safer way.

Pipes are a shared storage resource in kernel (OS) memory. It's used as a buffer between a writing and reading process, shared between those two processes. It is created by only a single running process.


(Page table stuff)

The cache is a small, fast piece of memory. It holds a subset of the Page Table Entires (PTE). Initially, it is empty. Each PTE used is inserted into cache. On subsequenct accesses to memory, it looks into cache simultaneously with address translation. If found, abort translation (lookaside buffer). Process switch: clear cache.

\end{document}
