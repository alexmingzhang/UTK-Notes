\chapter{Open and Closed Sets}

We will describe some concepts that generalize open/closed intervals. This chapter also serves as a very light introduction to topology---specifically, the topology of the real number line.

\section{Open Sets}

\begin{dfnbox}{Open Set}{}
    Intuitively, set is \dfntxt{open} if it does not contain any of its ``boundary points'', such as minimum or maximum.
    \tcblower
    More formally, we say $A \subseteq \R$ is \dfntxt{open} if, for all $x \in A$, there exists $r > 0$ such that $(x-r, x+r) \subseteq A$.
    \[ \forall(x \in A) \exists (r > 0) \left( (x-r, x+r) \subseteq A \right) \]
\end{dfnbox}


\begin{exbox}{$[0,1)$ is not open}{}
    The interval $[0,1)$ is not open.
    \tcblower
    \begin{proof}
        $0 \in [0,1)$, but $(0-r, 0+r) \not \subseteq [0,1)$ for any $r > 0$.
    \end{proof}
\end{exbox}

\begin{dfnbox}{Open Ball}{}
    We call the interval $(x-r, x+r)$ the \dfntxt{open ball} of radius $r$ centered at $x$, notated as $B(x,r)$ or $B_r(x)$.
    \tcblower
    \[ B(x,r) = B_r(x) = (x-r, x+r) \]
\end{dfnbox}

This new notation lets us write ideas more succinctly. For example, $\R$ is open. Given any $x \in \R$, then any $r > 0$ will give us $B(x, r) \in \R$. Also, $\emptyset$ is vacuously open.

\begin{lembox}{Open Intervals are Open Sets}{}
    Let $a,b \in \R$ where $a < b$. Then $(a,b)$ is an open set.
    \tcblower
    \begin{proof}
        Let $c \coloneq \frac{a+b}{2}$, and let $R \coloneq \frac{b-a}{2}$. Then $(a,b) = B(c, R)$. Let $x \in B(c, R)$. Then $\abs{x - c} < R$. Let $r \coloneq R - \abs{x - c} > 0$. We now prove $B(x,r) \subseteq B(c,R)$. Let $y \in B(x,r)$. Then $\abs{x-y} < r$, so:
        \[ \abs{y - c} = \abs{y-x+x-c} \leq \abs{y-x} + \abs{x-c} < r + \abs{x - c} = R - \abs{x-c} + \abs{x-c} = R \]
        Hence, $y \in B(c, R) = (a,b)$. Therefore, $(a,b)$ is an open set.
    \end{proof}
\end{lembox}

As we prove below, an arbitrary union of open sets is itself an open set.

\begin{thmbox}{Union of Open Sets is Open}{}
    Suppose $\Lambda$ is a set, and for each $\lambda \in \Lambda$, $O_{\lambda}$ is an open subset of $\R$. Then $\bigcup_{\lambda \in \Lambda} O_\lambda$ is an open set.
    \tcblower
    \begin{proof}
        Let $x \in \bigcup_{\lambda \in \Lambda} O_{\lambda}$. Then there exists some $\lambda_0 \in \Lambda$ such that $x \in O_{\lambda_0}$. Since $O_{\lambda_0}$ is open, there exists $r > 0$ such that:
        \[ (x-r, x+r) \subseteq O_{\lambda_0} \subseteq \bigcup_{\lambda \in \Lambda} O_\lambda \]
    \end{proof}
\end{thmbox}

The intersection of open sets is more troublesome. Countable intersections of open sets may not be open. For example, let $A_n \coloneq \left( -\frac{1}{n}, \frac{1}{n} \right)$ for each $n \in \N$. Then $\bigcap_{n \in \N} A_n = \{0\}$ is not open!

\begin{thmbox}{Finite Intersection of Open Sets is Open}{}
    Let $n \in \N$, and let $O_1, O_2, \ldots, O_n$ be open subsets of $\R$. Then $\bigcap_{k = 1}^n O_k$ is open.
    \tcblower
    \begin{proof}
        Let $x \in \bigcap_{k = 1}^n O_k$. Then $x \in O_k$ for $k = 1,2,\ldots,n$. Then, for each $k \in \{1,2,\ldots,n\}$, there must be some radius $r_k > 0$ such that $B(x, r_k) \subseteq O_k$. Since there are only finitely many open sets, we can take the minimum radius. Let $r \coloneq \min\{r_1, r_2, \ldots, r_n\}$. Then, $r \leq r_k$ for each $k \in \{1,2,\ldots,n\}$. Hence:
        \[ B(x, r) \subseteq B(x, r_k) \subseteq O_k \quad \text{for all}\ k \in \{1,2,\ldots,n\} \]
        Therefore, $B(x,r) \subseteq \bigcap_{k = 1}^n O_k$, so it is open.
    \end{proof}
\end{thmbox}

Note how the above theorem only works by taking the minimum radius of all the open sets. We can only take this minimum radius because there are only a finite number of open sets.

\section{Closed Sets}

\begin{dfnbox}{Closed Set}{}
    Intuitively, a set is \dfntxt{closed} if it contains all of its ``boundary points''.
    \tcblower
    More formally, a set $E \subseteq \R$ is \dfntxt{closed} if every convergent sequence $(s_n)$ where $s_n \in E$ for all $n \in \N$ satisfies $\lim_{n \to \infty} s_n \in E$.
\end{dfnbox}

\begin{exbox}{$(0,1]$ is not closed}{}
    The interval $[0,1)$ is not closed.
    \tcblower
    \begin{proof}
        Consider the sequence $(s_n) \coloneq \sfrac{1}{n}$. Then $(s_n)$ converges to $0$, but $0 \notin (0,1]$.
    \end{proof}
\end{exbox}

Note that this interval $(0, 1]$ is neither open nor closed! It is wrong to think of open/closed as strictly one or the other (i.e. openness and closedness are not mutually exclusive). Moreover, a set can be both open and closed (or \dfntxt{clopen}), going against the intuition of open and closed sets. There are only two clopen sets in the real numbers: $\R$ itself, and $\emptyset$.

\begin{lembox}{Closed Intervals are Closed Sets}{}
    Let $a,b \in \R$ with $a<b$. Then $[a,b]$ is a closed set.
    \tcblower
    \begin{proof}
        Let $(s_n)$ be an arbitrary convergent sequence of real numbers where $a \leq s_n \leq b$ for all $n \in \N$. Since $(s_n)$ is convergent, then $\lim_{n \to \infty} s_n$ exists. By the properties of limits, we have:
        \[ \lim_{n\to\infty} a \leq \lim_{n\to\infty} s_n \leq \lim_{n\to\infty} b \]
        Hence, $\lim_{n\to\infty} s_n \in [a,b]$. Therefore, $[a,b]$ is a closed set.
    \end{proof}
\end{lembox}

\begin{thmbox}{Intersection of Closed Sets is Closed}{}
    Let $\Lambda$ be a set, and let $E_\lambda \subseteq \R$ be closed for all $\lambda \in \Lambda$. Then $\bigcap_{\lambda \in \Lambda} E_\lambda$ is a closed set.
    \tcblower
    \begin{proof}
        Let $(s_n)$ be an arbitrary convergent sequence of real numbers entirely contained within $\bigcap_{\lambda \in \Lambda} E_\lambda$. Since $(s_n)$ is convergent, then $\lim_{n \to \infty} s_n$ exists. Let $l$ denote that limit. Let $\lambda \in \Lambda$ be arbitrary. Then $s_n \in E_\lambda$ for all $n \in \N$. Since $E_\lambda$ is closed, then any convergent sequence contained in $E_\lambda$ has its limit in $E_\lambda$. Thus, $\lim_{n \to \infty} s_n \in E_\lambda$. Since $\lambda \in \Lambda$ is arbitrary, then $\lim_{n \to \infty} s_n \in E_\lambda$ for all $\lambda \in \Lambda$. Therefore, $s \in \bigcap_{\lambda \in \Lambda} E_\lambda$, so this set is closed.
    \end{proof}
\end{thmbox}

Similar to the intersection of open sets, the union of closed sets is guaranteed to be closed if it is a finite union. For example, the union $\left( \bigcup_{n \in \N} [\sfrac{1}{n}, 1]\right) = (0,1]$ is not closed!

\begin{thmbox}{Finite Union of Closed Sets is Closed}{finite-union-closed-sets}
    Let $n \in \N$, and let $E_1, E_2, \ldots, E_n$ be closed subsets of $\R$. Then $\bigcup_{k=1}^n E_k$ is a closed set.
    \tcblower
    A direct proof of this theorem can be found in the textbook.
\end{thmbox}

The direct proof here is rather wordy and awkward. We will first establish a concrete relationship between open and closed sets, then leverage that to prove this theorem ``indirectly''.

\begin{thmbox}{Complement of an Open Set is Closed}{}
    Let $O \subseteq \R$ be open. Then $\R \setminus O$ is closed.
    \tcblower
    \begin{proof}
        Let $(x_n)$ be an arbitrary convergent sequence entirely contained within $R \setminus O$. Let $l_x \coloneq \lim_{n\to\infty} x_n$. Suppose for contradiction that $l_x \notin \R \setminus O$. Then $l_x \in O$. Since $O$ is open, there exists some radius $r > 0$ such that $B(l_x, r) \in O$. Since $(x_n)$ converges to $l_x$, then there exists $N \in \N$ such that $\abs{x_n - l_x} < r$ for all $n > N$. That is, $x_n \in B(l_x, r) \subseteq O$ for all $n > N$. This contradicts $x_n \in \R \setminus O$. Thus, $l_x \in \R \setminus O$, so $\R \setminus O$ is closed.
    \end{proof}
\end{thmbox}

\begin{thmbox}{Complement of a Closed Set is Open}{}
    Let $C \subseteq \R$ be closed. Then $\R \setminus C$ is open.
    \tcblower
    \begin{proof}
        Let $x \in \R \setminus C$. We must prove the following statement:
        \[ \exists (n \in \N) \left( B \left(x, \sfrac{1}{n} \right) \subseteq \R \setminus C \right) \]
        Suppose for contradiction the negation of the previous statement holds. That is:
        \[ \forall (n \in \N) \left( B \left(x, \sfrac{1}{n} \right) \not \subseteq \R \setminus C \right) \]

        In other words, for all $n \in \N$, there exists $x_n \in B(x, \sfrac{1}{n})$ such that $x_n \in C$. Hence, the sequence $(x_n)$ satisfies $x_n \in C$ for all $n \in \N$ and $\abs{x_n - x} < \sfrac{1}{n}$. Thus, $(x_n)$ converges to $x$. However, $C$ is closed, and $(x_n)$ is a sequence in $C$, so $x \in C$. This contradicts $x \in \R \setminus C$. Therefore, our original statement holds, so $\R \setminus C$ is open.
    \end{proof}
\end{thmbox}

Combining the two above theorems, we can infer a pretty useful relationship between open and closed sets.

\[ A\ \text{is open} \iff \R \setminus A\ \text{is closed} \]
\[ B\ \text{is closed} \iff \R \setminus B\ \text{is open} \]

We can apply this relationship to directly prove that the finite union of closed sets is closed (Theorem \ref{thm:finite-union-closed-sets}).

\begin{proof}[Proof of Theorem \ref{thm:finite-union-closed-sets}]
    By De Morgan's Laws, we have:
    \[ \R \setminus \left( \bigcup_{k=1}^n E_k \right) = \bigcap_{k=1}^{n} \left( \R \setminus E_k \right) \]
    Since each $\R \setminus E_k$ is open, the finite intersection $\bigcap_{k=1}^n (\R \setminus E_k)$ is also open. Hence, $\R \setminus \left( \bigcup_{k=1}^n E_k \right)$ is open, so $\bigcup_{k=1}^n E_k$ is closed.
\end{proof}

\section{Closure}

\begin{dfnbox}{Closure of a Set}{}
    Intuitively, the \dfntxt{closure} of a set is the set containing its elements and boundary points.
    \tcblower
    Formally, for $A \subseteq \R$, the \dfntxt{closure} of $A$ is the set:
    \[ \overline{A} \coloneq \{ x \in \R : \exists\ (\text{sequence}\ (x_n)) \forall(n \in \N)(x_n \in A)\ \text{and}\ x_n \to x \} \]
\end{dfnbox}

For example, let's consider the closure the open interval $ A \coloneq (0,1)$.
\begin{itemize}
    \item We can take the constant sequence of $\sfrac{1}{2}$ contained in $(0,1)$ which converges to $\sfrac{1}{2}$, so $\sfrac{1}{2} \in \overline{A}$.
    \item We can take the sequence $\sfrac{1}{n+1}$ contained in $(0,1)$ which converges to $0$, so $0 \in \overline{A}$.
    \item We can take the sequence $1 - \sfrac{1}{n+1}$ contained in $(0,1)$ which converges to $1$, so $1 \in \overline{A}$.
\end{itemize}

\begin{thmbox}{Properties of Closures of Sets}{}
    Let $A \subseteq \R$. Then:
    \begin{enumerate}[label=(\roman*)]
        \item $A \subseteq \overline{A}$,
        \item $\overline{A}$ is closed,
        \item $A = \overline{A}$ if and only if $A$ is closed,
        \item $\overline{A} = \overline{\overline{A}}$,
        \item if $F \subseteq \R$ is closed and $A \subseteq F$, then $\overline{A} \subseteq F$, and
        \item $\overline{A} = \bigcap \left\{ F \subseteq \R : F\ \text{is closed, and}\ A \subseteq F \right\}$
    \end{enumerate}
    \tcblower
    \begin{proof}[Proof] The proofs for i through vi are as follows:
        \begin{enumerate}[label=(\roman*)]
            \item Let $x \in A$, and let $(x_n) \coloneq (x, x, x, \ldots)$. Then $x_n \in A$ for all $n \in \N$, and $\lim_{n \to \infty} x_n = x$. Therefore, $x \in \overline{A}$.
            \item Let $(x_n)$ be a sequence contained within $\overline{A}$ that converges to some $x \in \R$. For each $x_n \in \overline{A}$, there exists $y_n \in A$ such that $\abs{y_n - x_n} < \sfrac{1}{n}$. Then:
            \[ \abs{y_n - x} = \abs{y_n - x_n + x_n - x} \leq \abs{y_n - x_n} + \abs{x_n - x} < \sfrac{1}{n} + \abs{x_n - x} \]
            Thus, $y_n - x \to 0$, so $y_n \to x$. Therefore, $x \in \overline{A}$.
            \item First, suppose $A = \overline{A}$. Then by (ii), $\overline{A}$ is closed, so $A$ is closed. Conversely, suppose $A$ is closed. Then by (i), $A \subseteq \overline{A}$. Now we show $\overline{A} \subseteq A$. Let $x \in \overline{A}$. By definition, there exists a sequence $(x_n)$ contained in $A$ that converges to $x$. Since $A$ is closed, then $x \in A$, so $\overline{A} \subseteq A$. Thus, $\overline{A} = A$.
            \item By (ii), $\overline{A}$ is closed. By (iii), $\overline{A} = \overline{\overline{A}}$.
            \item Let $x \in \overline{A}$. By definition, there exists a sequence $(x_n)$ contained in $A$ that converges to $x$. Since $A \subseteq F$, then $(x_n)$ is also contained in $F$. Since $F$ is closed, then $x \in F$.
            \item By (v), if $F$ is closed and $A \subseteq F$, then $\overline{A} \subseteq F$. Therefore, $\overline{A} \subseteq \bigcap\{ F \subseteq \R : F\ \text{is closed, and}\ A \subseteq F \}$. By (ii), $\overline{A}$ is closed, and by (i), $A \subseteq \overline{A}$. Thus, we have:
            \[ \bigcap \{ F \subseteq \R : F\ \text{is closed, and}\ A \subseteq F \} \subseteq \overline{A} \]
        \end{enumerate}
    \end{proof}
\end{thmbox}

These properties can make it easier to prove statements about closures.

\begin{exbox}{Using Properties of Closure}{}
    If $A \subseteq B$, then $\overline{A} \subseteq \overline{B}$.
    \tcblower
    \begin{proof}
        By (i), $A \subseteq B \subseteq \overline{B}$, and by (ii), $\overline{B}$ is closed. Thus, by (v), $\overline{A} \subseteq \overline{B}$.
    \end{proof}
\end{exbox}

\begin{notebox}
    The corresponding idea for open sets is the \dfntxt{interior} of a set.
\end{notebox}
