\chapter{Number Systems}
Our goal is to create an axiomatic basis for the real numbers $\R$. We need to establish axioms for $\R$ and then derive all further properties from the axioms. We would like these axioms to be as minimal and agreeable as possible; however, finding axioms that characterize $\R$ is not easy. Instead, we'll start from the natural numbers $\N$ and expand from there.

\section{Natural Numbers $\N$ and Induction}
How do we define the natural numbers? Listing every natural number is definitely not an option. We could try to define the natural numbers as $\N \coloneq \{ 1, 2, \ldots \}$. However, the ``$\ldots$'' is ambiguous. Instead, we can define $\N$ in terms of its properties.

\begin{dfnbox}{Peano Axioms for $\N$}{}
    The \dfntxt{Peano axioms} are five axioms that can be used to define the natural numbers $\N$.
    \begin{enumerate}[noitemsep]
        \item $1 \in \N$
        \item Every $n \in \N$ has a successor called $n+1$.
        \item $1$ is \textbf{not} the successor of any $n \in \N$.
        \item If $n,m \in \N$ have the same successor, then $n = m$.
        \item If $1 \in S$ and every $n \in S$ has a successor, then $\N \subseteq S$.
    \end{enumerate}
\end{dfnbox}

\begin{notebox}
    Note that there is not one ``prescribed'' way to do define the natural numbers. This is just the most popular approach.
\end{notebox}

From the fifth Peano axiom, we can derive a new proof technique for proving statements about consecutive natural numbers.

\begin{thmbox}{Principle of Induction (by the Peano Axioms)}{induction}
    Let $P(n)$ be a statement for each $n \in \N$. Suppose that:
    \begin{enumerate}[noitemsep]
        \item $P(1)$ is true, and
        \item if $P(n)$ is true, then $P(n+1)$ is true.
    \end{enumerate}
    Then $P(n)$ is true for all $n \in \N$.
    \tcblower
    \begin{proof}
        Let $S \coloneq \{ n \in \N : P(n) \}$. Then $1 \in S$ because $P(1)$ is true. Note that if $n \in S$, then $P(n)$ is true. Hence, $P(n+1)$ is true by assumption, so $n+1 \in S$. By the fifth Peano axiom, we have $\N \subseteq S$. Since $S$ was defined as a subset of $\N$, we have $\N = S$. Therefore, $P(n)$ is true for all $n \in \N$.
    \end{proof}
\end{thmbox}

A proof by induction has a ``domino effect''. Imagine a domino for each natural number $1$, $2$, $3$, and so on, arranged in an infinite row. Knocking the 1st domino will knock them all down.

%We set up the dominoes by proving $P(n) \implies P(n+1)$ and knock over the first domino by proving $P(1)$. The result is that all the dominoes will topple each other, leaving no domino standing.

\[ \underbracket{P(1)}_{\text{by 1.}} \implies \underbracket{P(2)}_{\text{by 2.}} \implies \underbracket{P(3)}_{\text{by 2.}} \implies \cdots \]

\begin{tecbox}{Proof by Induction}{induction}
    To prove a statement $P(n)$ for all $n \in \N$, we need to prove two statements:
    \begin{enumerate}
        \item \dfntxt{Base Case:} Prove $P(1)$.
        \item \dfntxt{Induction Step:} Assume $P(n)$ is true from some $n \in \N$, then prove $P(n) \implies P(n+1)$.
    \end{enumerate}
\end{tecbox}

It is crucial that we actually use our assumption that $P(n)$ is true in the induction step. Otherwise, our proof is most likely wrong.

\begin{exbox}{Simple Proof by Induction}{}
    Prove that $1+2+\cdots + n = \frac{n(n+1)}{2}$ for all $n \in \N$.
    \tcblower
    \begin{proof}
        Let $P(n)$ be the statement $1 + 2 + \cdots + n = \frac{n(n+1)}{2}$.

        \textbf{Base Case:} When $n=1$, $\text{LHS} = 1$ and $\text{RHS} = \frac{1(1+1)}{2} = 1$, so $P(1)$ is true.

        \textbf{Induction Step:} Assume that $P(n)$ is true for some $n \in \N$. Then:
        \begin{align*}
            1 + 2 + \cdots + n + (n+1)
            &= \frac{n(n+1)}{2} + (n+1) \\
            &= (n+1) \left( \frac{n}{2} + 1 \right) \\
            &= \frac{(n+1)(n+2)}{2}
        \end{align*}
        That is, $P(n+1)$ is true. By the \nameref{thm:induction}, $P(n)$ is true for all $n \in \N$.
    \end{proof}
\end{exbox}

\section{Integers $\Z$}
From the natural numbers, we can easily construct the integers. First, we assume the existence an operation, addition ($+$) and multiplication ($\cdot$). On $\N$, we assume addition and multiplication satisfy the following properties for all $a,b,c \in \N$:

\begin{center}\begin{tabular}{l l l}
    \tabitem\dfntxt{Commutativity} & $a+b = b+a$ & $a \cdot b = b \cdot a$ \\
    \tabitem\dfntxt{Associativity} & $(a+b)+c = a+(b+c)$ & $(a \cdot b) \cdot c = a \cdot (b \cdot c)$ \\
    \tabitem\dfntxt{Distributivity} & $a \cdot (b+c) = a \cdot b + a \cdot c$ \\
    \tabitem\dfntxt{Identity} & $1 \cdot n = n$
\end{tabular}\end{center}
\todo{Sort out this wonky formatting}

We can expand this number system by including:
\begin{enumerate}
    \item an \dfntxt{additive identity} ($n+0 = n$ for all $n \in \N$)
    \item \dfntxt{additive inverses} (for all $n \in \N$, add $-n$ so $-n + n = 0$)
\end{enumerate}
From this, we can construct the set of integers.

\begin{dfnbox}{Integers $\Z$}{}
    The set of \dfntxt{integers} is defined as:
    \[ \Z \coloneq \N \cup \{0\} \cup \{ -n : n \in \N \} \]
\end{dfnbox}

\begin{dfnbox}{Even, Odd, Parity}{}
    Let $a \in \Z$.
    \begin{itemize}[noitemsep]
        \item $a$ is \dfntxt{even} if there exists $k \in \Z$ where $a = 2k$.
        \item $a$ is \dfntxt{odd} if there exists $k \in \Z$ where $a = 2k+1$.
        \item \dfntxt{Parity} describes whether an integer is even or odd.
    \end{itemize}
\end{dfnbox}

\begin{thmbox}{Parity Exclusivity}{}
    Every integer is either even or odd, never both.
    \tcblower
    \todo[inline]{TODO: prove this}
\end{thmbox}

\begin{exbox}{Parity of Square}{square-parity}
    For $n \in \Z$, if $n^2$ is even, then $n$ is even.
    \tcblower
    \begin{proof}
        We proceed by contraposition. Suppose $n$ is not even. Then $n$ is odd, and thus can be expressed as $n = 2k+1$ for some $k \in \Z$. Then:
        \begin{align*}
            n^2 &= (2k+1)(2k+1) \\
            &= 4k^2 + 4k + 1
        \end{align*}
        Since the integers are closed under addition and multiplication, then $4k^2 + 4k \in \Z$. Thus, $n^2$ is odd.
    \end{proof}
\end{exbox}

\section{Rational Numbers $\Q$}

We can further expand this number system by the following:
\begin{enumerate}
    \item Include \dfntxt{multiplicative inverses} (for all $n \in \Z \setminus \{0\}$, define $\sfrac{1}{n}$ such that $n \cdot \sfrac{1}{n} = 1$)
    \item Define $m \cdot \sfrac{1}{n} \coloneq \sfrac{m}{n}$ when $n \neq 0$.
\end{enumerate}

From this, we can construct the set of rational numbers.

\begin{dfnbox}{Rational Numbers $\Q$}{}
    The set of \dfntxt{rational numbers} is defined as:
    \[ \Q \coloneq \left\{ \frac{m}{n} : m,n \in Z \land n \neq 0 \right\} \]
\end{dfnbox}

To ensure multiplication works as intended, we also define $\frac{m}{n} \cdot \frac{k}{l} \coloneq \frac{m \cdot k}{n \cdot l}$.

We say $\frac{m_1}{n_1} = \frac{m_2}{n_2}$ if and only if $m_1n_1 = m_2n_2$ where $n_1, n_2 \neq 0$. In other words, $\frac{m_1}{n_1} \sim \frac{m_2}{n_2} \iff m_1n_2 = m_2n_1$. Thus, $\Q$ is the set of equivalence classes for this relation.

If $n = kp$ and $m = kq$, where $k,p,q \in \Z$, $k \neq 0$, $q \neq 0$, then:
\[ \frac{n}{m} = \frac{kp}{kq} = \frac{k}{p}, \quad \text{because}\ kpq = kqp \]
If $n$ and $m$ have no common factor (except $\pm 1$), then we say that $\sfrac{n}{m} \in \Q$ is in the ``lowest terms'' or ``reduced terms''. The set $(Q, +, \cdot)$ forms a field. However, we cannot write $x = \sfrac{n}{m}$ where $x^2 = 2$. \todo{Fix this whole section up. Very confusing.}

\begin{thmbox}{$\sqrt{2}$ is not a Rational Number}{root-2-irrational}
    $\sqrt{2} \notin \Q$
    \tcblower
    \begin{proof}
        Suppose for contradiction $\sqrt{2}$ is a rational number. Then, there exist $n,m \in \Z$ such that $(\sfrac{n}{m})^2 = 2$. If $n = kp$ and $m = kq$, then we can ``cancel'' the common factor $k$ to write $\sfrac{n}{m} = \sfrac{p}{q}$. That is, we can assume that $n$ and $m$ have no (non-trivial) common factors.
        % The above explanation feels kind of insubstantial.
        Now, $\sfrac{n^2}{m^2} = 2$, so by multiplying both sides by $m^2$, we get $n^2 = 2m^2$. Thus, $n^2$ is an even number, so $n$ is also even (Example \ref{ex:square-parity}). Then, we can write $n = 2k$ where $k \in \Z$. Then:
            \begin{alignat*}{2}
                & \implies \quad & (2k)^2 &= 2m^2 \\
                & \implies \quad & 4k^2 &= 2m^2 \\
                & \implies \quad & 2k^2 &= m^2
            \end{alignat*}
            Then $m^2$ is even, so $m$ is even. Thus, $m$ and $n$ are both even, so they are multiples of $2$. This contradicts the fact that we defined $\sfrac{n}{m}$ in the lowest terms.
    \end{proof}
\end{thmbox}

Does there exist $r \in \Q$ such that $r^2 = 3$?

\begin{dfnbox}{Divides}{}
    For $a,b, \in \Z$, we say $a$ \dfntxt{divides} $b$ if $b$ is a multiple of $a$.
    \tcblower
    \[ a \mid b \iff \exists(c \in \Z)(b=ac) \]
\end{dfnbox}

\begin{thmbox}{Division Algorithm}{}
    Suppose $a,b \in \Z$. Then $a = kb + r$ where $k \in \Z$ and $r \in \Z$ where $0 \leq r < a$.
\end{thmbox}
\todo{Need proof here}

\begin{exbox}{}{}
    If $p \in \N$ and $3 \mid p^2$, then $3 \mid p$.
    \tcblower
    \begin{proof}
        By the division algorithm, $p = 3k+j$ where $k \in \Z$ and $j \in \Z$ where $0 \leq j < 3$. Then, $p^2 = (3k+j)^2 = 9k^2 + 6kj + j^2$. Suppose that $3 \mid p^2$. Then, $p^2 = 3l = 9k^2 + 6kj + j^2$. Thus:
        \[ j^2 = 3l-9k^2-6kj = 3(l-3k^2-2kj) \]
        We have $3 \mid j^2$. Hence, $j \neq 1, j \neq 2$, leaving only $j = 0$. Therefore, $p = 3k + 0$, so $3 \mid p$.
    \end{proof}
\end{exbox}

\begin{exbox}{$\sqrt{3}$ is not a Rational Number}{}
    \begin{proof}
        Suppose for contradiction $\sqrt{3}$ is a rational number. Then, there exist $n, m \in \Z$ such that $(\sfrac{n}{m})^2$. If $n$ and $m$ share a common factor, then we can ``cancel'' the common factor to where $\sfrac{n}{m} = \sfrac{kp}{kq} = \sfrac{p}{q}$. Thus, we may assume that $n$ and $m$ have no nontrivial common factor.
        \begin{alignat*}{2}
            && \left(\frac{n}{m}\right)^2 &= 3 \\
            & \implies \quad & \frac{n^2}{m^2} &= 3 \\
            & \implies \quad & n^2 &= 3m^2
        \end{alignat*}
        Thus, $3 \mid n^2$, so $3 \mid n$ by the previous lemma. Writing $n = 3k$ for some $k \in \Z$, we have:
        \begin{alignat*}{2}
            && (3k)^2 &= 3m^2 \\
            & \implies \quad & 9k^2 &= 3m^2 \\
            & \implies \quad & 3k^2 &= m^2
        \end{alignat*}
        That is, $3 \mid m^2$ so $3 \mid m$. Thus, $3$ divides both $n$ and $m$. This contradicts the fact that we defined $\sfrac{n}{m}$ in the lowest terms.
    \end{proof}
\end{exbox}

\section{Fields}

\begin{dfnbox}{Field}{field}
    A \dfntxt{field} is a set $F$ with two defined operations, addition and multiplication, satisfying the following for all $a,b,c \in F$:

    \begin{center}\begin{tabular}{l l l}
		Axiom & \text{Addition} & \text{Multiplication} \\ \hline
		\dfntxt{Associativity} & $(a+b)+c = a+(b+c)$ & $(ab)c = a(bc)$ \\
		\dfntxt{Commutativity} & $a+b = b+a$ & $ab=ba$ \\
		\dfntxt{Distributivity} & $a(b+c) = ab+ac$ & $(a+b)c = ac + bc$ \\
		\dfntxt{Identities} & $\exists(0 \in \F)(a+0 = a)$ & $\exists(1 \in \F)(1 \neq 0 \land 1 a  = a)$ \\
		\dfntxt{Inverses} & $\exists(-a \in \F)(a + (-a) = 0)$ & $(a \neq 0) \iff \exists(a^{-1} \in \F)(a a^{-1} = 1)$
	\end{tabular}\end{center}

\end{dfnbox}

All the ``standard facts'' of arithmetic and algebra in $\R$ follows from these axioms.

$\Q$, $\R$, and $\C$ are infinite fields, but $\Z_p$ (arithmetic modulo $p$) is a finite field if $p$ is prime.

More generally, $F_q$ where $q = p^k$ is a finite field.

\begin{thmbox}{Facts about Fields}{}
    Let $F$ be a field. For all $a,b,c \in F$:
    \begin{enumerate}[noitemsep,label=(\alph*)]
        \item if $a+c = b+c$, then $a = b$
        \item $a \cdot 0 = 0$
        \item $(-a) \cdot b = -(a \cdot b)$
        \item $(-a) \cdot (-b) = a \cdot b$
        \item if $a \cdot c = b \cdot c$ and $c \neq 0$, then $a = b$
        \item if $a \cdot b = 0$, then $a = 0$ or $b = 0$
        \item $-(-a) = a$
        \item $-0 = 0$
    \end{enumerate}
    \tcblower
    \begin{proof}[Proof of (g)]
        \begin{align*}
            -(-a)
            &= -(-a) + 0 \\
            &= -(-a) + (a + (-a)) \\
            &= -(-a) + (-a + a) \\
            &= \left(-(-a) + (-a) \right) + a \\
            &= \left( (-a) + -(-a) \right) + a \\
            &= 0 + a \\
            &= a + 0 \\
            &= a
        \end{align*}
    \end{proof}
\end{thmbox}

\section{Ordered Fields}

\begin{dfnbox}{Ordered Field}{}
    An \dfntxt{ordered field} is a field with a relation $<$ such that for all $a,b,c \in F$:
    \begin{center}\begin{tabular}{l l}
        Axiom & Description \\ \hline
        \dfntxt{Trichotomy} &  Only one is true: $a<b$, $a=b$, or $b<a$ \\
        \dfntxt{Transitivity} & if $a<b$ and $b<c$ then $a<c$ \\
        \dfntxt{Additive Property} & if $b < c$, then $a+b < a+c$ \\
        \dfntxt{Multiplicative Property} & if $b<c$ and $0<a$, then $a \cdot b < a \cdot c$
    \end{tabular}\end{center}
\end{dfnbox}

We then define $>$ as the inverse relation of $<$.

\begin{thmbox}{Facts about Ordered Fields}{}
    \begin{itemize}[noitemsep]
        \item if $a < b$ then $-b < -a$
        \item if $a < b$ and $c < 0$, then $cb < ca$
        \item if $a \neq 0$, then $a^2 = a \cdot a > 0$
        \item $0 < 1$
        \item if $0<a<b$ then $0 < \sfrac{1}{b} < \sfrac{1}{a}$
    \end{itemize}
\end{thmbox}

Although $\C$ is a field, it is not an ordered field. We can certainly define some kind of ``order'' on $\C$, but there is no way to make it satisfy the four axioms of an ordered field. For example, $i^2 = -1 < 0$, contradicting the fact that any nonzero number's square is greater than $0$ in an ordered field.

$\R$ and $\Q$ are ordered fields.

\begin{dfnbox}{Absolute Value}{}
    Let $F$ be an ordered field. For $a \in F$, we define the \dfntxt{absolute value} of $a$ as:
    \[ \abs{a} \coloneq \begin{cases} a, & a \geq 0 \\ -a, & a < 0 \end{cases} \]
\end{dfnbox}

We can think of $\abs{a-b}$ as the distance between $a$ and $b$. More generally, $\abs{a-b} = d(a,b)$ is the metric we will be using throughout real analysis.

\begin{thmbox}{Properties of Absolute Value}{}
    \begin{itemize}
        \item $\abs{a} \geq 0$, $a \leq \abs{a}$, and $-a \leq \abs{a}$
        \item $\abs{ab} = \abs{a}\abs{b}$
    \end{itemize}
\end{thmbox}

\begin{thmbox}{Triangle Inequality}{triangle-inequality}
    Let $F$ be an ordered field. For any $a,b \in F$, $\abs{a+b} \leq \abs{a} + \abs{b}$.
    \tcblower
    \begin{proof}
        There are two cases to consider. If $a+b \geq 0$, then:
        \begin{align*}
            \abs{a+b}
            &= a+b \\
            &\leq \abs{a} + b \\
            &\leq \abs{a} + \abs{b}
        \end{align*}
        If $a+b < 0$, then:
        \begin{align*}
            \abs{a+b}
            &= -(a+b) \\
            &= -a-b \\
            &\leq \abs{a} - b \\
            &\leq \abs{a} + \abs{b}
        \end{align*}
    \end{proof}
\end{thmbox}

\section{Completeness}

\begin{dfnbox}{Bounded Above, Bounded Below, Bounded}{}
    Let $F$ be an ordered field, and let $A \subseteq F$.
    \begin{itemize}[noitemsep]
        \item $A$ is \dfntxt{bounded above} if there exists $b \in F$ such that $a \leq b$ for all $a \in A$. In this context, $b$ is an \dfntxt{upper bound} for $A$.
        \item $A$ is \dfntxt{bounded below} if there exists $c \in F$ such that $c \leq a$ for all $a \in A$. In this context, $c$ is a \dfntxt{lower bound} for $A$.
        \item $A$ is \dfntxt{bounded} if $A$ is bounded above and bounded below.
    \end{itemize}
\end{dfnbox}

\begin{exbox}{Upper and Lower Bounds}{}
    Consider the set $ (0,1) \coloneq \{ x \in \R : 0<x<1\}$.
    \begin{itemize}[noitemsep]
        \item $(0,1)$ is bounded above by $1$ and any number greater than $1$.
        \item $(0,1)$ is bounded below by $0$ and any negative number.
    \end{itemize}
    Consider the set $[3, \infty) \coloneq \{ x \in \R : 3 \leq x \}$.
    \begin{itemize}[noitemsep]
        \item $[3, \infty)$ is not bounded above.
        \item $[3, \infty)$ is bounded below by $3$ and any number less than $3$.
    \end{itemize}
\end{exbox}

\begin{dfnbox}{Maximum, Minimum}{}
    Let $F$ be an ordered field, and let $A \subseteq F$.
    \begin{itemize}[noitemsep]
        \item If there exists $M \in A$ such that $M$ is an upper bound for $A$, then $M$ is the \dfntxt{maximum} of $A$, denoted $M = \max A$
        \item If there exists $m \in A$ such that $m$ is a lower bound for $A$, then $m$ is the \dfntxt{minimum} of $A$, denoted $m = \min A$.
    \end{itemize}
\end{dfnbox}

Note that from the above example, $(0,1)$ has neither a maximum nor a minimum. However, 3 is the minimum of $[3,\infty)$.

\begin{dfnbox}{Supremum}{}
    Let $F$ be an ordered field, and let $A \subseteq F$. $s \in F$ is a \dfntxt{supremum} of $A$ if:
    \begin{enumerate}
        \item $s$ is an upper bound for $A$, and
        \item if $t$ is an upper bound for $A$, then $s \leq t$.
    \end{enumerate}
\end{dfnbox}

In other words, the supremum is the least upper bound for $A$. If $A$ has a supremum, then that supremum is unique. \todo{Prove this}

\begin{thmbox}{Maximum is the Supremum}{}
    Let $F$ be an ordered field, and let $A \subseteq F$. If $A$ has a maximum $M$, then $M = \sup A$.
    \tcblower
    \begin{proof}
        Since $M = \max A$, we know $M$ is an upper bound for $A$. Let $t$ be an upper bound for $A$. Since $M \in A$, then $t \geq M$. Thus, $M$ is less than or equal to any upper bound $t$, so $M = \sup A$.
    \end{proof}
\end{thmbox}

\begin{exbox}{Supremum of $(0,1)$}{}
    Prove that $\sup (0,1) = 1$.
    \tcblower
    \begin{proof}
        First, note that $1$ is an upper bound for $(0,1)$. Next, suppose that $t \in \Q$ is an upper bound for $(0,1)$. Since $0 < \sfrac{1}{2} < 1$, then $0 < \sfrac{1}{2} \leq t$. By transitivity, $t > 0$. Suppose for contradiction $t < 1$. Because $0 < t < 1$, we have $1 < 1 + t < 2$. Dividing across by $2$, we have $\sfrac{1}{2} < \sfrac{1+t}{2} < 1$. That is, $\sfrac{1+t}{2} \in (0,1)$. But $t < 1$, so $2t < 1+t$. Thus, $t < \sfrac{1+t}{2}$. This contradicts our assumption that $t$ is an upper bound for $(0,1)$. Therefore, $t \geq 1$, so $\sup (0,1) = 1$.
    \end{proof}
\end{exbox}

\begin{dfnbox}{Completeness}{}
    An ordered field $F$ is \dfntxt{complete} if every nonempty subset of $F$ that is bounded above has a supremum in $F$.
\end{dfnbox}

\begin{thmbox}{$\Q$ is not complete}{}
    \begin{proof}[Proof sketch]
        Let $A \coloneq \left\{ x \in \Q : x^2 < 2 \right\}$. In other words, $A = \left(-\sqrt{2}, \sqrt{2}\right) \subseteq \Q$. Then $A$ is nonempty and bounded above. Suppose for contradiction that $\Q$ is complete. Then $A$ has a supremum, say $s = \sup(A)$. Consider the following cases:
        \begin{enumerate}
            \item If $s^2 < 2$, let $n \in \N$ such that $\left( s + \sfrac{1}{n} \right)^2 < 2$. Then $s + \sfrac{1}{n} \in A$, contradicting $s$ being an upper bound for $A$.
            \item If $s^2 > 2$, let $n \in \N$ such that $\left( s - \sfrac{1}{n} \right)^2 > 2$. Then $s - \sfrac{1}{n}$ is an upper bound smaller than $s$, contradicting $s$ being the least upper bound (supremum).
            \item If $s^2 = 2$, then $s \notin \Q$ (Theorem \ref{thm:root-2-irrational}).
        \end{enumerate}
        Thus, $A \subseteq \Q$ does not have a supremum. Therefore, $\Q$ is not complete.
    \end{proof}
\end{thmbox}

% TODO: elaborate on notation of mathematical structures

\begin{dfnbox}{Real Numbers $\R$}{}
    The \dfntxt{real numbers} are a set $\R$ with two operations, $+$ and $\cdot$, and order relation $<$ such that:
    \begin{enumerate}[noitemsep]
        \item $(R, +, \cdot)$ is a field,
        \item $(\R, +, \cdot, <)$ is an ordered field, and
        \item $(\R, +, \cdot, <)$ is complete.
    \end{enumerate}
\end{dfnbox}

Alternatively, $\R$ can be constructed explicitly using ``Dedekind cuts''. Either way, $\R$ is the \textbf{only} unique complete ordered field up to isomorphism. That is, if there is some other imposter complete ordered field $\R\prime$, we can map every element of $\R$ to $\R\prime$ such that we preserve all the operations and relations between things in $\R$. More formally, there exists an isomorphism $T : \R \to \R\prime$ where $T$ is bijective, and:
\begin{itemize}[noitemsep]
    \item $T(x+y) = T(x) + T(y)$
    \item $T(xy) = T(x)T(y)$
    \item $x < y \iff T(x) < T(y)$
\end{itemize}

Additionally, $\N \subseteq \R$ where $\N$ satisfies the Peano axioms.

\begin{thmbox}{$\sqrt{2}$ is a Real Number}{}
    \begin{proof}[Proof sketch]
        Let $A \coloneq \left\{ x \in \R : x^2 < 2 \right\}$.
        \begin{itemize}[noitemsep]
            \item Show $A \neq \emptyset$ and $A$ is bounded above
            \item Completeness says $s \coloneq \sup A$ exists
            \item Show $s^2 = 2 \implies s = \sqrt{2} \in \R$.
        \end{itemize}
        More generally, if $n,m \in \N$, then $\sqrt[n]{m} \in \R$.
    \end{proof}
\end{thmbox}
