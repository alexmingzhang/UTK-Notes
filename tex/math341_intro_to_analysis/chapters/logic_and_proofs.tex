\chapter{Logic and Proofs}
% Logic is the backbone of all formal mathematics. When building a logically sound model of mathematics, we start with a small collection of axioms. We then work with those axioms to deduce other logically sound statements, reaffirming what we already knew and discovering new ideas along the way.

Formal logic is the foundation of mathematics. It allows us to construct a logically consistent framework for mathematics, starting with a small set of assumed statements, then systematically deducing new statements from them. This process not only helps us to prove known results, but it also helps us find and prove new ideas.

\section{Basic Logic}
\begin{dfnbox}{Statement}{}
    A \dfntxt{statement} is a claim that is either true or false.
    \tcblower
    \[ p : \text{The sky is blue.} \]
\end{dfnbox}

We usually denote statements with a letter like $p$. For example, we can write ``$p: x > 2$'', which means $p$ represents the statement ``$x$ is greater than $2$''. Throughout this chapter, we will use $p$ and $q$ to represent arbitrary statements.

\begin{dfnbox}{Conjunction}{conjunction}
    Logical \dfntxt{conjunction} is an operation that takes two statements and produces a new statement that is true only when both input statements are true.
    \tcblower
    \[ p \land q : p\ \text{is true \textbf{and}}\ q\ \text{is true} \]
\end{dfnbox}

\begin{dfnbox}{Disjunction}{disjunction}
    Logical \dfntxt{disjunction} is an operation that takes two statements and produces a new statement that is true when at least one of the input statements is true.    \tcblower
    \[ p \lor q : p\ \text{is true \textbf{or}}\ q\ \text{is true} \]
\end{dfnbox}

\nameref{dfn:conjunction} and \nameref{dfn:disjunction} follow our intuition of ``and'' and inclusive ``or'', respectively. We can visualize the two logical connectives using \dfntxt{truth tables}.

\begin{exbox}{Truth Table of Conjunction}{conjunction}
    \begin{center}\begin{tabular}{c | c || c}
        $p$ & $q$ & $p \implies q$ \\ \hline
        T & T & T \\
        T & F & F \\
        F & T & F \\
        F & F & F
    \end{tabular}\end{center}
\end{exbox}

\begin{exbox}{Truth Table of Disjunction}{disjunction}
    \begin{center}\begin{tabular}{c | c || c}
        $p$ & $q$ & $p \implies q$ \\ \hline
        T & T & T \\
        T & F & T \\
        F & T & T \\
        F & F & F
    \end{tabular}\end{center}
\end{exbox}


% TODO: add truth tables

\begin{dfnbox}{Negation}{}
    The \dfntxt{negation} of a statement is a statement with opposite truth values.
    \tcblower
    \[ \neg p \]
\end{dfnbox}

For example, if $p$ denotes the statement ``the sky is blue,'' then $\neg p$ denotes the statement ``the sky is \textbf{not} blue.'' Notice that $\neg p$ doesn't say that the sky is any specific color like red or green; it only says that the color is not blue.

\begin{dfnbox}{Implication, Hypothesis, Conclusion}{}
    An \dfntxt{implication} ``$p$ implies $q$'' states ``if $p$ is true, then $q$ is true.'' We call $p$ the \dfntxt{hypothesis} and $q$ the \dfntxt{conclusion}.
    \tcblower
    \[ p \implies q \]
\end{dfnbox}

If the hypothesis of an implication is false to begin with, then the implication is not really meaningful. Instead of assigning those kinds of implications no truth value, we simply consider them true by convention. These kinds of truths are called \dfntxt{vacuous truths}.

\begin{notebox}
    Does this mean a false statement can imply any other statement, regardless of its truth value? The answer is yes, and it is not as problematic as one may think. This is because the concept of implication in logic isn't about a causal or chronological connection, but rather about the consistency of statements. In the realm of logic, if a false statement is said to imply another statement, it doesn't create any inconsistency, since the initial condition is never met. This principle is known as ``ex falso quodlibet'' or ``from falsehood, anything follows.'' Understanding this allows us to maintain a consistent framework in logic, despite the seemingly counterintuitive nature of vacuous truths.
\end{notebox}

To reiterate, let's consider the truth table for logical implications and some simple examples:

\begin{exbox}{Truth Table of Implication}{}
    \begin{center}\begin{tabular}{c | c || c}
        $p$ & $q$ & $p \implies q$ \\ \hline
        T & T & T \\
        T & F & F \\
        F & T & T \\
        F & F & T
    \end{tabular}\end{center}
\end{exbox}

\begin{exbox}{Simple Logical Implications}{}
    Let $p : x > 2$ and $q : x^2 > 1$. Let's figure out (without proof) whether the following statements are true or false:
    \begin{itemize}
        \item ``For every real number $x$, $p \implies q$.''

        In English, this statement reads, ``for every real number $x$, if $x > 2$, then $x^2 > 1$.'' This statement is \textbf{true}. Since $x > 2$, then $x^2$ can only be bigger than $2^2$ which equals $4$.

        \item ``For all real numbers $x$, $q \implies p$''

        In English, this statement reads, ``for every real number $x$, if $x^2 > 1$, then $x > 2$.'' This statement is \textbf{false}. What if $x = 1.1$? Then $x^2 = 1.21 > 1$, but $x = 1.1 < 2$.
    \end{itemize}
\end{exbox}

\begin{dfnbox}{Logical Equivalence}{equiv}
    Two statements $p$ and $q$ are \dfntxt{logically equivalent} if $p \implies q$ and $q \implies p$.
    \tcblower
    \[ p \iff q \]
\end{dfnbox}

In other words, $p \iff q$ means that $p$ and $q$ share the same truth value. Either $p$ and $q$ are \textbf{both true}, or $p$ and $q$ are \textbf{both false}. Logical equivalence says nothing about the individual truth values of $p$ nor $q$.

In English, logical equivalence can be expressed as ``$p$ if and only if $q$.'' Some people abbreviate this as ``$p$ iff $q$.''

\begin{exbox}{Truth Table of Logical Equivalence}{}
    \begin{center}\begin{tabular}{c | c || c}
        $p$ & $q$ & $p \iff q$ \\ \hline
        T & T & T \\
        T & F & F \\
        F & T & F \\
        F & F & T
    \end{tabular}\end{center}
\end{exbox}

\begin{dfnbox}{Converse}{converse}
    Given an implication $p \implies q$, its \dfntxt{converse} statement is $q \implies p$.
\end{dfnbox}

It's important to note that an implication and its converse have no intrinsic equivalence. That is, if $p$ implies $q$, it's not generally true to say that $q$ also implies $p$---unless $p$ and $q$ are logically equivalent.


\begin{exbox}{Truth Table of Converse}{}
    \begin{center}\begin{tabular}{c | c || c | c}
        $p$ & $q$ & $p \implies q$ & $q \implies p$ \\ \hline
        T & T & T & T \\
        T & F & F & T \\
        F & T & T & F \\
        F & F & T & T
    \end{tabular}\end{center}
\end{exbox}

\begin{dfnbox}{Contrapositive}{contrapositive}
    Given the implication $p \implies q$, its \dfntxt{contrapositive} statement is $\neg q \implies \neg p$.
\end{dfnbox}

Unlike the converse, an implication and its contrapositive are logically equivalent. To help our intuition, we can construct a truth table.

\begin{exbox}{Truth Table of Contrapositive}{}
    \begin{center}\begin{tabular}{c | c || c | c | c | c}
        $p$ & $q$ & $\neg p$ & $\neg q$ & $p \implies q$ & $\neg q \implies \neg p$ \\ \hline
        T & T & F & F & T & T \\
        T & F & F & T & F & F \\
        F & T & T & F & T & T \\
        F & F & T & T & T & T
    \end{tabular}\end{center}
\end{exbox}

As we can see, no matter what the truth values of the hypothesis and conclusion are, an implication and its contrapositive always have the same truth values.

\begin{notebox}
    It's important to note that when we construct a truth table, it's best practice to include all the intermediate statements, not just the final statement.
\end{notebox}

% TODO: add truth table for contrapositive
\todo[inline]{TODO: Logical quantifiers $\forall$ and $\exists$}

\section{Proofs and Proof Techniques}
While truth tables are a useful tool for evaluating simple statements, they quickly become impractical when dealing with more complex propositions. Moreover, they do not offer insights into the reasoning behind such statements. In contrast, proofs can provide us with a deeper understanding of logical relationships and help us reason about complex statements.

We often need to prove implications of the form $p \implies q$, where the truth of $p$ guarantees the truth of $q$. In this chapter, we will break down three main techniques for proving such an implication. Here's a quick overview:
\begin{enumerate}
    \item \dfntxt{Direct Proof:} Assume $p$ is true, then reason that $q$ must be true as well.
    \item \dfntxt{Proof by Contradiction:} Assume both $p$ and $\neg q$ are true, then logically derive some contradiction.
    \item \dfntxt{Proof by Contrapositive:} Assume $\neg q$ is true, then reason that $\neg p$ must be true as well.
\end{enumerate}

In mathematical proofs, there are two main types of reasoning: direct and indirect. A direct proof shows a clear path from the premises to the conclusion, providing valuable insights into the underlying mathematics. In contrast, indirect proofs rely on a contradictory hypothesis to establish the truth of the conclusion. While indirect proofs can be useful when a direct proof is not readily available, they may be less insightful since they do not provide much context surrounding the premises.

However, it is worth noting that an indirect proof may be easier to find than a direct proof in certain cases. While a direct proof requires identifying the correct path that leads to the conclusion, an indirect proof only needs to deduce any contradictory statement. Despite this advantage, indirect proofs should be used sparingly and only when a direct proof is not feasible.

% In a direct proof, the reasoning to get from $p$ to $q$ provides a lot of insight about the context of $p$ and the surrounding mathematics. Similarly, the direct reasoning in proof by contrapositive provides context surrounding $\neg q$. However, the reasoning done in a proof by contradiction is based on a contradictory hypothesis. Thus, it is often less insightful and is typically avoided when a direct proof is readily available.\footnote{\url{https://math.stackexchange.com/a/1688}}

% That being said, it is sometimes easier to find a proof by contradiction than a direct proof. Whereas direct proof needs to deduce the correct path that leads to the conclusion, a proof by contradiction only needs to deduce any contradictory statement.

%It's hard to decide which proof technique is easiest for any given problem. Direct proofs are often more ``enlightening'', but it can be difficult to find the appropriate logic to reach the conclusion. It may be easier to try proof by contradiction or contrapositive.

\subsection*{Proof by Contradiction}

\begin{tecbox}{Proof by Contradiction}{contradiction}
    To prove $p \implies q$ by contradiction, we carry out the following steps:
    \begin{enumerate}
        \item Assume $p$ is true, and suppose for the sake of contradiction $\neg q$ is true.
        \item Logically derive a statement that contradicts something we know to be true.
        \item Ultimately conclude that $q$ must be true.
    \end{enumerate}
\end{tecbox}

For a contrived example (courtesy of GPT-4), let's prove the fact that pigs cannot fly. This is a rather simplified example, but it effectively illustrates the process of proof by contradiction.

\begin{exbox}{Pigs Can't Fly (Proof by Contradiction)}{}
    Prove that if an animal is a pig, then that animal cannot fly.
    \tcblower
    We have two statements connected by logical implication. For convenience, we will:
    \begin{itemize}
        \item use $p$ to denote the statement ``an animal is a pig,'' and
        \item use $q$ to denote the statement ``that animal cannot fly.''
    \end{itemize}
    \begin{proof}
        We assume that $p$ is true (an animal is a pig), and for the sake of contradiction, suppose $\neg q$ is also true (it can fly). This leads us to a situation where we have a flying pig. However, we know from established biological and physiological facts that pigs, as a species, do not have the capability to fly. This contradicts our known reality.
        Does this mean that the world we know has been a contradiction, and our whole lives heretofore have been lies? The answer is no; we must have made an erroneous assumption.
        Thus, we conclude that our supposition that $\neg q$ was true is actually wrong! Therefore, $q$ must be true when $p$ is true. That is, if an animal is a pig, then that animal cannot fly.
    \end{proof}
\end{exbox}

In terms of logic notation, proof by contradiction can be expressed as such:
\[ \left[ \left( p \land (\neg q) \right) \implies \text{Contradiction} \right] \implies \left[ p \implies q \right]\]

\begin{exbox}{Truth Table of \nameref{tec:contradiction}}{}
    \begin{center}\begin{tabular}{c | c || c | c | c | c }
        $p$ & $q$ & $p \implies q$ & $\neg q$ & $p \land (\neg q)$ & $\neg \left[ p \land (\neg q) \right]$ \\ \hline
        T & T & T & F & F & T \\
        T & F & F & T & T & F \\
        F & T & T & F & F & T \\
        F & F & T & T & F & T
    \end{tabular}\end{center}
\end{exbox}

By the above truth table, we can safely assume the following logical equivalence:
\[ (p \implies q) \iff \neg \left[ p \land (\neg q) \right] \]

\subsection*{Proof by Contrapositive}

\begin{tecbox}{Proof by Contrapositive}{}
    To prove $p \implies q$ by contrapositive, we carry out the following steps:
    \begin{enumerate}
        \item Assume $\neg q$ is true.
        \item Directly reason that $\neg p$ is true.
    \end{enumerate}
\end{tecbox}

Let's revisit the idea that pigs can't fly, this time proving it by contrapositive.

\begin{exbox}{Pigs Can't Fly (Proof by Contrapositive)}{}
    Prove that if an animal is a pig, then that animal cannot fly, denoted as $p \implies q$.
    \tcblower
    \begin{proof}
        We assume that $\neg q$ holds (an animal can fly). Since we know that pigs are incapable of flight due to their biology and physiology, any animal capable of flight must belong to a different species. Therefore, we can conclude $\neg p$ holds (it isn't a pig).
    \end{proof}
\end{exbox}


In terms of logic notation, proof by contrapositive can be expressed as:
\[ (\neg q \implies \neg p) \iff (p \implies q) \]
Although we can simply use a truth table to reveal the logical equivalence here, let's instead give a formal proof of the equivalence. Notice that, since this is a logical equivalence, there are really two implications we must prove: one going the ``forward'' direction $(\implies)$, and one going the ``backwards'' or converse direction $(\impliedby)$. 
\begin{lembox}{Logical Equivalence of Contrapositive}{}
    Given statements $p$ and $q$, $p \implies q$ if and only if $\neg q \implies \neg p$.
    \tcblower
    \begin{proof}
        First, suppose that $p \implies q$. To prove $\neg q \implies \neg p$, we can suppose for contradiction that $\neg q$ and $p$ are both true. But $p \implies q$, so $q$ is true which contradicts $\neg q$. Hence, the assumption that $p$ is true was incorrect. Thus, $\neg q \implies \neg p$.

        Conversely, suppose that $\neg q \implies \neg p$. From above, we have $\neg ( \neg p ) \implies \neg (\neg q)$, so $p \implies q$.
    \end{proof}
\end{lembox}

\begin{exbox}{Proving Simple Logic Statements}{}
    Let $p$, $q$, and $r$ be arbitrary statements. Prove that $\left[ p \implies (q \lor r) \right] \iff \left[ (p \land \neg q) \implies r \right]$.
    \tcblower
    \begin{proof}
        Assume $p \implies (q \lor r)$. Suppose $p \land \neg q$. Then $p$ is true, so $q \lor r$ is true by assumption. Also, $\neg q$ is true, so $r$ must be true from $q \lor r$.

        Assume $(p \land \neg q) \implies r$. Suppose $p$ is true. There are two possibilities:
        \begin{enumerate}
            \item If $q$ is true, then $q \lor r$ is true.
            \item If $\neg q$ is true, then $p \land \neg q$ is true. Thus, $r$ is true by assumption. Hence, $q \lor r$ is true.
        \end{enumerate}
    \end{proof}
\end{exbox}

% When we prove an implication $p \implies q$ directly, we assume $p$, and then make some intermediate conclusions, before finally deducing $q$. Those intermediate conclusions provide insight about the context of $p$ and the surrounding mathematics. Similarly, when we prove the contrapositive, we assume $\neg q$, and make some valuable intermediate conclusions, before finally deducing $\neg p$.

% In contrast, a proof by contradiction carries little of this extra value. We make intermediate conclusions under a contradictory hypothesis where $p$ and $\neg q$ hold. Since those intermediate conclusions are built on false logic, they provide less insight about logically sound mathematics.
