\chapter{Naive Set Theory}
Instead of forming a rigorous, axiomatic basis for sets, we will simply take an informal approach to sets guided by our intuition. Ultimately, our introduction to real analysis does not fiddle with the fine details of set theory, so it's safe to take a naive approach.

\section{Sets}

\begin{dfnbox}{Set}{set}
    A \dfntxt{set} is a collection of distinct objects.
\end{dfnbox}

For example, $\N = \{1,2,3\ldots\}$ is the set of all natural numbers, and $\Z = \{ \ldots, -2, -1, 0, 1, 2, \ldots\}$ is the set of all integers. These aren't restricted to just sets of numbers; for example, we may have a set of letters $\{ \texttt{a}, \texttt{b}, \texttt{c} \} $ or a set of shapes $\{ \star, \circ, \sqcap \}$. It's conventional to use capital letters to denote sets and use lowercase letters to denote elements of sets. Throughout this chapter, we will use $A$ and $B$ to represent arbitrary sets.

\begin{dfnbox}{Set Membership ($\in$)}{}
    We write $a \in A$ to mean ``$a$ is in $A$''.
\end{dfnbox}

\begin{dfnbox}{Subset ($\subseteq$)}{}
    $A$ is a \dfntxt{subset} of $B$ if everything in $A$ is also in $B$.
    \tcblower
    \[ A \subseteq B \iff \forall(x \in A)(x \in B) \]
\end{dfnbox}

\begin{dfnbox}{Set Equality ($=$)}{}
    $A$ \dfntxt{equals} $B$ if $A$ is a subset of $B$ and $B$ is a subset of $A$.
    \tcblower
    \[ A = B \iff (A \subseteq B \land B \subseteq A) \]
\end{dfnbox}

\begin{dfnbox}{Proper Subset ($\subsetneq$)}{}
    $A$ is a \dfntxt{proper subset} of $B$ if $A$ is a subset of $B$ but $B$ is not a subset of $A$.
    \tcblower
    \[ A \subsetneq B \iff (A \subseteq B \land B \not\subseteq A) \]
\end{dfnbox}

In other words, $A$ is a proper subset of $B$ if everything in $A$ is also in $B$, but $B$ has something that $A$ does not.

\begin{notebox}
    Among mathematical texts, the generic subset symbol $\subset$ has no standardized definition. Some use it to represent subset or equal; others use it to represent proper subset. We will simply not use $\subset$ to avoid any ambiguity.
\end{notebox}

\begin{dfnbox}{Empty Set ($\emptyset$)}{}
    The \dfntxt{empty set} is the set that contains no elements.
    \tcblower
    \[ \emptyset \coloneq \{ \} \]
\end{dfnbox}

As convention, we assume that $\emptyset$ is a subset of every set, including itself.

% A useful way of visualizing set relations is with venn diagrams.

\begin{tecbox}{Proving a Subset Relation}{}
    To prove that $A \subseteq B$:
    \begin{enumerate}
        \item Let $x$ be an arbitrary element of $A$.
        \item Show that $x \in B$.
    \end{enumerate}
    \tcblower
    To prove that $A \not\subseteq B$, choose a specific $x \in A$ and show $x \notin B$.
\end{tecbox}

\begin{exbox}{Proving Simple Subset Relation}{}
    Suppose that $A \subseteq B$ and $B \subseteq C$. Prove that $A \subseteq C$.
    \tcblower
    \begin{proof}
        Let $x \in A$ be arbitrary. Since $A \subseteq B$, then $x \in B$. Similarly, since $B \subseteq C$, then $x \in C$. Therefore, $A \subseteq C$.
    \end{proof}
\end{exbox}

When defining the contents of a set, we often use a special notation called \dfntxt{set builder notation}. It specifies conditions that members of the set must reach. For example, we can notate the set of all non-negative integers $\Z^+$ as:

\[ \Z^+ \coloneq \{ x \in \Z : x \geq 0 \} \]

In English, this could read as ``the set of every integer that is greater than or equal to $0$.'' Also note the use of the ``$\coloneq$'' symbol, which is used to emphasize definition/assignment rather than express a relationship about equality.

\begin{dfnbox}{Union ($\cup$)}{}
    The \dfntxt{union} of two sets is the set of all things that are in one or the other set.
    \tcblower
    \[ A \cup B \coloneq \left\{ x : x \in A \lor x \in B \right\} \]
\end{dfnbox}

\begin{dfnbox}{Intersection ($\cap$)}{}
    The \dfntxt{intersection} of two sets is the set of all things that are in both sets.
    \tcblower
    \[ A \cap B \coloneq \left\{ x : x \in A \land x \in B \right\} \]
\end{dfnbox}

More generally, we can apply union and intersection to an arbitrary number of sets, finite or infinite. We use a notation similar to summation using $\sum$. Let $\Lambda$ be an indexing set, and for each $\lambda \in \Lambda$, let $A_\lambda$ be a set.
\begin{align*}
    \bigcup_{\lambda \in \Lambda} A_\lambda &\coloneq \left\{ x : x \in A_\lambda\ \text{for some}\ \lambda \in \Lambda \right\} \\
    \bigcap_{\lambda \in \Lambda} A_\lambda &\coloneq \left\{ x : x \in A_\lambda\ \text{for all}\ \lambda \in \Lambda \right\}
\end{align*}
\begin{exbox}{Indexed Sets}{}
    For $n \in \N$, let $A_n = \left[ \frac{1}{n}, 1 \right] = \left\{ x \in \R : \frac{1}{n} \leq x \leq 1 \right\}$. Prove that:
    \begin{enumerate}[label=(\alph*)]
        \item $\bigcup_{n=1}^\infty = (0,1]$
        \item $\bigcap_{n=1}^\infty = \{1\}$
    \end{enumerate}
    \tcblower
    \begin{proof}[Proof of (a)]
        Suppose $x \in \bigcup_{n=1}^\infty A_n$. Then there exists $n \in \N$ such that $x \in A_n = \left[ \frac{1}{n}, 1 \right]$. That is, $0 < \frac{1}{n} \leq x \leq 1$. Therefore, $x \in (0, 1]$.

        Suppose $x \in (0, 1]$. Then $x > 0$, so there exists $n_0 \in \N$ such that $\frac{1}{n_0} < x$. Then $\frac{1}{n_0} \leq x \leq 1$, so $x \in A_{n_0}$. Therefore, $x \in \bigcup_{n=1}^\infty A_n$.
    \end{proof}

    \begin{proof}[Proof of (b)]
        Suppose $x \in \bigcap_{n=1}^\infty A_n$. Then $x \in A_1 = \{1\}$.

        Suppose $x \in \{1\}$. Then $x = 1 \in \left[ \frac{1}{n}, 1 \right]$ for all $n \in \N$. Therefore, $x \in \bigcap_{n=1}^\infty A_n$.
    \end{proof}
\end{exbox}

\begin{dfnbox}{Set Minus ($\setminus$)}{}
    The \dfntxt{set difference} of two sets is the set of things that are in the first but not the second set.
    \tcblower
    \[ A \setminus B \coloneq \{ x : x \in A \land x \notin B \} \]
\end{dfnbox}

\begin{dfnbox}{Complement ($A^c$)}{}
    Let $X$ be a set called the \dfntxt{universal set}. The \dfntxt{complement} of $A$ in $X$ is defined as $X \setminus A$.
    \tcblower
    \[ A^c \coloneq X \setminus A = \{ x \in X : x \notin A \} \]
\end{dfnbox}

\begin{thmbox}{De Morgan's Laws for Sets}{}
    Suppose $X$ is a set, and for any subset $S$ of $X$, let $S^c = X \setminus S$. Suppose that $A_\lambda \subseteq X$ for every $\lambda$ belonging to some index set $\Lambda$. Prove that:
    \begin{enumerate}[label=(\alph*)]
        \item \( \left( \bigcup_{\lambda \in \Lambda} A_\lambda \right)^c = \bigcap_{\lambda \in \Lambda} A_\lambda^c \);
        \item \( \left( \bigcap_{\lambda \in \Lambda} A_\lambda \right)^c = \bigcup_{\lambda \in \Lambda}A_\lambda^c \).
    \end{enumerate}
    \tcblower
    \begin{proof}[Proof of (a)]
        First, let $a \in \left( \bigcup_{\lambda \in \Lambda} A_\lambda \right)^c$. Then, $a \in X \setminus \left( \bigcup_{\lambda \in \Lambda} A_\lambda \right)$, so $a \in X$ but $a \notin \left( \bigcup_{\lambda \in \Lambda} A_\lambda \right)$. Thus, $a \notin A_\lambda$ for any $\lambda \in \Lambda$, so $a \in X \setminus A_\lambda$ for all $\lambda \in \Lambda$. In other words, $a \in \bigcap_{\lambda \in \Lambda} A_\lambda^c$.

        Next, let $a \in \bigcap_{\lambda \in \Lambda} A_\lambda^c$. Then $a \in A_\lambda^c$ for all $\lambda \in \Lambda$, so $a \in X$ but $a \notin A_\lambda$ for all $\lambda \in \Lambda$. That is, $a \notin \left( \bigcup_{\lambda\in\Lambda} A_\lambda \right)$. In other words, $a \in \left( \bigcup_{\lambda\in\Lambda} A_\lambda \right) ^ c$.
    \end{proof}

    \begin{proof}[Proof of (b)]
        First, let $a \in \left( \bigcap_{\lambda \in \Lambda} A_\lambda \right)^c$. Then, $a \in X \setminus  \bigcap_{\lambda \in \Lambda} A_\lambda$, so $a \in X$ but $a \notin \bigcap_{\lambda \in \Lambda} A_\lambda$. That is, $a \notin A_\lambda$ for some $\lambda \in \Lambda$. Thus, $a \in X \setminus A_\lambda$ for some $\lambda \in \Lambda$. Therefore, $a \in \bigcup_{\lambda \in \Lambda} A_\lambda^c$.

        Next, let $a \in \bigcup_{\lambda \in \Lambda} A_\lambda^c$. Then $a \in A_\lambda^c$ for some $\lambda \in \Lambda$, so $a \in X$ but $a \notin A_\lambda$ for some $\lambda \in \Lambda$. That is, $a \notin \left( \bigcap_{\lambda \in \Lambda} A_\lambda \right)$. Therefore, $a \in  \left( \bigcap_{\lambda \in \Lambda} A_\lambda \right)^c$.
    \end{proof}
\end{thmbox}

\section{Functions}
We generally think of functions as a ``map'' or ``rule'' that assigns numbers to other numbers. For example, $f(x) = 2x$ maps $1 \mapsto 2$, $2 \mapsto 4$, etc. More formally, we define functions in terms of sets. 


\begin{dfnbox}{Cartesian Product ($\times$)}{}
    Let $X$ and $Y$ be sets. The \dfntxt{Cartesian product} of $X$ and $Y$ is the set of all ordered pairs $(x,y)$ where $x \in X$ and $y \in Y$.
    \tcblower
    \[ X \times Y \coloneq \left\{ (x,y) : x \in X \land y \in Y \right\} \]
\end{dfnbox}

\begin{dfnbox}{Relation}{relation}
    Let $X$ and $Y$ be sets. A \dfntxt{relation} between $X$ and $Y$ is a subset of the Cartesian product $X \times Y$.
\end{dfnbox}

\begin{dfnbox}{Function, $f : X \to Y$}{function}
    Let $X$ and $Y$ be sets. A \dfntxt{function} from $X$ to $Y$ is a relation from $X$ to $Y$ such that for every $x \in X$, there exists a unique $y \in Y$ where $(x,y) \in f$.
    \tcblower
    More formally, a \dfntxt{function} $f : X \to Y$ is a subset of $X \times Y$ satisfying:
    \begin{enumerate}[noitemsep]
        \item $\forall (x \in X) \left[ \exists (y \in Y)((x,y) \in f) \right]$
        \item $(x,y_1),(x,y_2) \in f \implies y_1 = y_2$
    \end{enumerate}
\end{dfnbox}

Given a function $f : X \to Y$, we call $X$ the \dfntxt{domain} of $f$ and $Y$ the \dfntxt{codomain} of $f$. Given $x \in X$, we write $f(x)$ to denote the unique element of $Y$ such that $(x,y) \in f$.
\[ f(x) = y \iff (x,y) \in f \]

\begin{dfnbox}{Function Image}{image}
    Let $f : X \to Y$ be a function and $A \subseteq X$. The \dfntxt{image} of $A$ under $f$ is the set containing all possible function outputs from all inputs in $A$.
    \tcblower
    \[ f[A] \coloneq \{ f(a) : a \in A \} \]
\end{dfnbox}

Given $f : X \to Y$, we call $f[X]$ the \dfntxt{range} of $f$.

\begin{exbox}{Function Images}{}
    Suppose $f : X \to Y$ is a function, and $A_\lambda \subseteq X$ for each $\lambda \in \Lambda$. Then:
    \begin{enumerate}[label=(\alph*)]
        \item $f \left[ \bigcup_{\lambda \in \Lambda} A_\lambda \right] = \bigcup_{\lambda \in \Lambda} f \left[ A_\lambda \right]$
        \item $f \left[ \bigcap_{\lambda \in \Lambda} A_\lambda \right] \subseteq \bigcap_{\lambda \in \Lambda} f \left[ A_\lambda \right]$
    \end{enumerate}
    \tcblower
    In this example, we will only prove the ``forward'' direction. That is, we want to show that $f \left[ \bigcup_{\lambda \in \Lambda} A_\lambda \right] \subseteq \bigcup_{\lambda \in \Lambda} f \left[ A_\lambda \right]$.
    \begin{proof}[Proof of (a)]
        Let $y \in f \left[ \bigcup_{\lambda \in \Lambda} A_\lambda \right]$. By definition of \nameref{dfn:image}, there exists $x \in \bigcup_{\lambda \in \Lambda} A_\lambda$ such that $y = f(x)$. Thus, there exists $\lambda_0 \in \Lambda$ such that $x \in \lambda_0$. That is, $y \in f \left[ A_{\lambda_0} \right]$. Therefore, $y \in \bigcup_{\lambda \in \Lambda} f \left[ A_\lambda \right]$.
    \end{proof}
\end{exbox}

\begin{dfnbox}{Function Inverse Image}{inverse-image}
    Let $f : X \to Y$ be a function and $B \subseteq Y$. The \dfntxt{inverse image} of $B$ under $f$ is the set containing all possible function inputs whose output is in $B$.
    \tcblower
    \[ f^{-1}[B] \coloneq \{ x \in X: f(x) \in B \} \]
\end{dfnbox}

Note the following logical equivalence:
\[ x \in f^{-1} [B] \iff f(x) \in B \]

\begin{exbox}{Function Inverse Images}{}
    Suppose $f : X \to Y$ is a function, and $B_\lambda \subseteq Y$ for each $\lambda \in \Lambda$. Then:
    \[ f^{-1} \left[ \bigcup_{\lambda \in \Lambda} B_\lambda \right] = \bigcup_{\lambda \in \Lambda} f^{-1} \left[ B_\lambda \right] \]
    \tcblower
    Again, we will only prove the ``forward direction''.
    \begin{proof}
        Let $x \in f^{-1} \left[ \bigcup_{\lambda \in \Lambda} B_\lambda \right]$. Then, $f(x) \in \bigcup_{\lambda \in \Lambda} B_\lambda$. That is, $f(x) \in B_{\lambda_0}$ for some $\lambda_0 \in \Lambda$. Thus, $x \in f^{-1} \left[ B_{\lambda_0} \right]$, so $x \in \bigcup_{\lambda \in \Lambda} f^{-1} \left[ B_\lambda \right]$.
    \end{proof}
\end{exbox}

\section{Injectivity and Surjectivity}

\begin{dfnbox}{Injective, One-to-one}{}
    A function $f : X \to Y$ is \dfntxt{injective} or \dfntxt{one-to-one} if no two inputs in $X$ have the same output in $Y$.
    \tcblower
    \[ \forall (x_1, x_2 \in X) \left[ x_1 \neq x_2 \implies f\left(x_1\right) \neq f\left(x_2\right) \right] \]
\end{dfnbox}

We can also think of injectivity as, ``if two inputs have the same output, then the two inputs must be the same''. It's really just the contrapositive of our initial definition, which we know must be logically equivalent.
\[ \forall (x_1, x_2 \in x) \left[ f(x_1) = f(x_2) \implies x_1 = x_2 \right] \]

For example, the function $f(x) = x^2$ is not injective, because $f(-1) = 1$ and $f(1) = 1$. We have two distinct inputs that map to the same output.

\begin{tecbox}{Proving a Function is Injective}{}
    To prove a function $f : X \to Y$ is injective:
    \begin{enumerate}
        \item Let $x_1, x_2 \in X$ where $f(x_1) = f(x_2)$.
        \item Reason that $x_1 = x_2$.
    \end{enumerate}
\end{tecbox}

\begin{exbox}{Proving Injectivity}{}
    $f(x) = -3x-7$ is injective.
    \tcblower
    \begin{proof}
        Suppose $f(x_1) = f(x_2)$. Then $-3x_1+7 = -3x_2+7$, so $-3x_1 = -3x_2$. Thus, $x_1 = x_2$, so $f$ is injective.
    \end{proof}
\end{exbox}

\begin{exbox}{Disproving Injectivity}{}
    Prove that $f(x)=x^2$ is not injective.
    \tcblower
    \begin{proof}
        $f(-1) = 1$ and $f(1) = 1$, but $-1 \neq 1$. Thus, $f$ is not injective.
    \end{proof}
\end{exbox}

\begin{dfnbox}{Surjective, Onto}{}
    A function $f : X \to Y$ is \dfntxt{surjective} or \dfntxt{onto} if everything in $Y$ has a corresponding input in $X$.
    \tcblower
    \[ \forall (y \in Y) \left[ \exists (x \in X) (f(x) = y) \right] \]
\end{dfnbox}

Note that $f : X \to f[X]$ is \textbf{always} surjective.

\begin{tecbox}{Proving a Function is Surjective}{}
    To prove a function $f : X \to Y$ is surjective:
    \begin{enumerate}
        \item Let $y \in Y$ be arbitrary.
        \item ``Undo'' the function $f$ to obtain $x \in X$ where $f(x)=y$.
    \end{enumerate}
\end{tecbox}

\begin{exbox}{Proving Surjectivity}{}
    Prove that $f : \R \to \R$ defined by $f(x) = -3x+7$ is surjective.
    \tcblower
    \begin{proof}
        Let $y \in Y$ be arbitrary. Let $x \coloneq \frac{y-7}{-3}$. Then $x \in \R$, and:
        \begin{align*}
            f(x)
            &= -3 \left( \frac{y-7}{-3} \right) + 7 \\
            &= (y-7) + 7 \\
            &= y
        \end{align*}
        Therefore, $f$ is surjective.
    \end{proof}
\end{exbox}

\begin{dfnbox}{Bijective}{}
    A function $f : X \to Y$ is \dfntxt{bijective} if it is both injective and surjective.
\end{dfnbox}

\begin{dfnbox}{Function Composition}{}
    Let $f : X \to Y$ and $g : Y \to Z$ be functions. The \dfntxt{composition} of $f$ and $g$ is a function $g \circ f : X \to Z$ defined by:
    \[ (g \circ f) (x) \coloneq g(f(x)) \]
\end{dfnbox}

\begin{thmbox}{Composition Preserves Injectivity and Surjectivity}{}
    Suppose $f : X \to Y$ and $g : Y \to Z$ are functions.
    \begin{enumerate}[label=(\alph*)]
        \item If $f$ and $g$ are injective, then $g \circ f$ is injective.
        \item If $f$ and $g$ are surjective, then $g \circ f$ is surjective.
        \item If $f$ and $g$ are bijective, then $g \circ f$ is bijective.
    \end{enumerate}
    \tcblower
    \begin{proof}[Proof of (a)]
        Let $x_1, x_2 \in X$. Suppose that $(g \circ f)(x_1) = (g \circ f)(x_2)$. Then, $g(f(x_1)) = g(f(x_2))$. Because $g$ is injective, we have $f(x_1) = f(x_2)$. Because $f$ is injective, we have $x_1 = x_2$. Therefore, $g \circ f$ is injective.
    \end{proof}

    \begin{proof}[Proof of (b)]
        Let $z \in Z$. Because $g$ is surjective, there exists an element $y \in Y$ such that $g(y) = z$. Because $f$ is surjective, there exists an element $x \in X$ such that $f(x) = y$. Thus, $(g \circ f)(x) = g(f(x)) = g(y) = z$. Therefore, $g \circ f$ is surjective.
    \end{proof}

    \begin{proof}[Proof of (c)]
        We know that from (a) and (b) composition preserves injectivity and surjectivity. Thus, composition must also preserve bijectivity.
    \end{proof}
\end{thmbox}

\begin{dfnbox}{Inverse Function}{}
    Let $f : X \to Y$ be a bijection. The \dfntxt{inverse function} of $f$ is a function $f^{-1} : Y \to X$ defined by:
    \[ f^{-1} \coloneq \{ (y,x) \in Y \times X : (x,y) \in f \} \]
\end{dfnbox}

The notation for inverse functions conflicts may be confused with the notation for inverse images. A key distinction to make is that the inverse image is applied to a subset of the range, whereas the inverse function is applied to specific members of range. Furthermore, only bijections can have an inverse function, but we can apply the inverse image to any function. Thus, given a bijection $f : X \to Y$, we know $f^{-1}(f(x)) = x$ for all $x \in X$, and $f(f^{-1}(y)) = y$ for all $y \in Y$.

\begin{exbox}{}{}
    Let $f : X \to Y$ and $g : Y \to X$ be functions such that $(g \circ f) = x$ for all $x \in X$, and $(f \circ g)(y) = y$ for all $y \in Y$. $f^{-1} = g$.
    \tcblower
    \begin{proof}
        todo: finish proof
    \end{proof}
\end{exbox}
