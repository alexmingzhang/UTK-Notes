\chapter{Sequences and Convergence}

\begin{dfnbox}{Sequence}{}
    A \dfntxt{sequence} is an ordered list of real numbers.
    \[ s = (s_1, s_2, s_3, s_4, \ldots) \]
    \tcblower
    Formally, a \dfntxt{sequence} is a function $s : \N \to \R$. We write $s_n$ to denote $s(n)$.
\end{dfnbox}

We can define a sequence using an expression, like $s_n \coloneq n^2$. Then $s = (1,4,9,16,\ldots)$. Also, we can informally define a sequence in terms of its elements, like $s = (3,1,4,1,5,9,\ldots)$. We could just have a random sequence like $s \coloneq (12.3, e^2, 1 - \pi, 10000. \ldots)$.

Let's consider how we can formalize the definitions of limits and convergence. Consider the sequence $s_n \coloneq \sfrac{1}{n}$, then $(s_n) = \left( 1, \sfrac12, \sfrac13, \sfrac14, \ldots \right)$. We have an intuitive idea that, as $n$ gets bigger, then $\sfrac{1}{n}$ gets closer to $0$. We can say that this sequence ``converges'' to $0$.

Now consider the sequence $s \coloneq (1,0,1,0,0,1,0,0,0,1,0,0,0,0,\ldots)$. Does this sequence converge? This really depends on our definition of convergence. We might define this as, ``$s_n$ gets close to $l$ as $n$ gets large''. It certainly matches our intuition, but what exactly does ``close to $l$'' mean? Maybe we could say, ``$\abs{s_n - l}$ gets small as $n$ gets large''. More precisely, this might be ``for all $\epsilon > 0$, $\abs{s_n -l} < \epsilon$ when $n$ is large''. That ``$n$ is large'' is still imprecise. Fixing that part, we get the formal definition for convergence:
%Instead, we can write ``for all $\epsilon > 0$, there exists $N \in \N$ such that $\abs{s_n - l} < \epsilon$ for all $n > N$''.

\begin{dfnbox}{Convergence}{}
    Let $s \coloneq (s_n)_{n\in\N}$ be a sequence of real numbers, and let $l \in \R$. We say $s_n$ \dfntxt{converges} to $l$ if, for all $\epsilon > 0$, there exists $N \in \N$ such that $\abs{s_n - l} < \epsilon$ for all $n > N$.
    \tcblower
    \[ \forall(\epsilon > 0) \exists(N \in \N) \forall (n > N) (\abs{s_n - l} < \epsilon) \]
\end{dfnbox}

Like in the approximation property, we use $\epsilon$ to denote some arbitrarily tiny value that's really really close to $0$, but not actually $0$. We can also write $\lim_{n \to \infty} s_n = l$ or $s_n \to l$ to mean $s_n$ converges to $l$.

\begin{tecbox}{Proving Convergence}{}
    To prove that a sequence $s$ converges to $l$, we carry out the following steps:
    \begin{enumerate}
        \item As some scratch work, solve the inequality $\abs{s_n - l} < \epsilon$ for $n$.
        \item In the formal proof, let $\epsilon > 0$, and let $N$ be greater than the solved thing. Let $n > N$, then work towards $\abs{s_n - l} < \epsilon$.
    \end{enumerate}
\end{tecbox}
\todo{Make this explanation better}

\begin{exbox}{$\sfrac{1}{n}$ converges to $0$}{}
    Prove that $\lim_{n \to \infty} \frac{1}{n} = 0$.
    \tcblower
    \textbf{Intuition:} Since we're proving something for all $\epsilon > 0$, let's start by choosing some arbitrary $\epsilon > 0$. Next, we need to choose some $N \in \N$ where $\abs{s_n - l} < \epsilon$ for all $n > N$. Thus:
    \begin{align*}
        \abs{s_n - l} &< \epsilon \\
        \abs*{\frac{1}{n} - 0} &< \epsilon \\
        \frac{1}{n} &< \epsilon \\
        n &> \frac{1}{\epsilon}
    \end{align*}
    So we choose $N > \frac{1}{\epsilon}$.

    \begin{proof}
        Let $\epsilon > 0$. Let $N \in \N$ where $N > \sfrac{1}{\epsilon}$. If $n > N > \sfrac{1}{\epsilon}$, then $\sfrac{1}{n} < \epsilon$. Thus:
        \[ \abs{s_n - l} = \abs{\sfrac{1}{n} - 0} = \sfrac{1}{n} < \epsilon \]
        Therefore, $s$ converges to $0$.
    \end{proof}
\end{exbox}

\begin{exbox}{}{}
    Prove that $\lim_{n \to \infty} \frac{2n+3}{3n+7} = \frac{2}{3}$.
    \tcblower
    \textbf{Intuition:} This time, we want to choose some $N \in \N$ such that $\abs{s_n - l} < \epsilon$. Thus:
    \begin{align*}
        \abs*{\frac{2n+3}{3n+7} - \frac{2}{3}} &< \epsilon \\
        \abs*{\frac{6n+9-6n-14}{9n+21}} &< \epsilon \\
        \frac{5}{9n+21} &< \epsilon \\
        \frac{5}{\epsilon} &< 9n + 21 \\
        \frac{1}{9}\left( \frac{5}{\epsilon} - 21 \right) &< n
    \end{align*}
    Thus, we want to choose $N >  \sfrac{1}{9}\left( \sfrac{5}{\epsilon} - 21 \right)$.
    \begin{proof}
        Let $\epsilon > 0$. Let $N \in \N$ such that $N > \sfrac{1}{9}\left( \sfrac{5}{\epsilon} - 21 \right)$. If $n > N > \sfrac{1}{9}\left( \sfrac{5}{\epsilon} - 21 \right)$, then:
        \begin{align*}
            9n &> \sfrac{5}{\epsilon} - 21 \\
            9n &> \frac{5}{\epsilon} - 21 \\
            9n + 21 &> \frac{5}{\epsilon} \\
            \frac{5}{9n+21} < \epsilon
        \end{align*}
        Thus:
        \begin{align*}
            \abs{s_n-l} &= \abs*{\frac{2n+3}{3n+7} - \frac{2}{3}} \\
            &= \abs*{\frac{6n+9-6n-14}{9n+21}} \\
            &= \frac{5}{9n+21} \\
            &< \epsilon
        \end{align*}
    \end{proof}

    The above proof chooses a sort of ``optimal'' or ``best possible'' $N$. We could have thrown away the $21$ in the denominator, and the inequality we're aiming for will still be the same.
    \begin{proof}[Alternate proof]
        Let $\epsilon > 0$. Let $N \in \N$ such that $N > \frac{5}{9\epsilon}$. If $n > N > \frac{5}{9\epsilon}$, then $\frac{5}{9n} < \epsilon$, so $\frac{5}{9n+21} < \frac{5}{9n} < \epsilon$. Then:
        \[ \abs{s_n -l } = \abs*{\frac{2n+3}{3n+7} - \frac{2}{3}} = \frac{5}{9n+21} < \epsilon \]
    \end{proof}
\end{exbox}

\begin{exbox}{}{}
    Prove that $\lim_{n \to \infty} \frac{2n+3}{3n-7} = \frac{2}{3}$.
    \tcblower
    \textbf{Intuition:} Here, we have to be careful about throwing away terms.
    \begin{align*}
        \abs{s_n - l} &< \epsilon \\
        \abs*{\frac{2n+3}{3n-7} - \frac{2}{3}} &< \epsilon \\
        \abs*{\frac{6n+9-6n+14}{9n-21}} &< \epsilon \\
        \frac{23}{\abs{9n-21}} &< \epsilon
    \end{align*}
    We want $9n-21 > 0$, so we must have $n \geq 3$. We can apply this restriction on $n$ to get rid of the absolute value:
    \begin{align*}
        \frac{23}{9n-21} &< \epsilon \\
        \frac{23}{\epsilon} &< 9n - 21 \\
        \frac{1}{9} \left( \frac{23}{\epsilon} + 21 \right) &< n
    \end{align*}
    Thus, we want to choose some $N > \frac{1}{9} \left( \frac{23}{\epsilon} + 21 \right) $ and $N \geq 3$.
    \begin{proof}
        Let $\epsilon > 0$. Let $N \in \N$ such that $N > \frac{1}{9} \left( \frac{23}{\epsilon} + 21 \right)$. Then $N > \frac{21}{9}$, and since $N$ is a natural number, then $N \geq 3$. Let $n \in \N$ where $n > N$. Then:
        \begin{align*}
            9n &> \frac{23}{\epsilon} + 21 \\
            9n - 21 &> \frac{23}{\epsilon} \\
            \epsilon &> \frac{23}{9n - 21}
        \end{align*}
        Thus:
        \[ \abs{s_n - l} = \abs*{\frac{2n+3}{3n-7} - \frac{2}{3}} = \abs*{\frac{23}{9n-21}} = \frac{23}{9n-21} < \epsilon \]

    \end{proof}
\end{exbox}

\begin{dfnbox}{Divergence}{}
    A sequence \dfntxt{diverges} if it does not converge.
    \tcblower
    \[ \exists(\epsilon > 0) \forall(N \in \N) \exists(n > N) (\abs{s_n -l} \geq \epsilon) \]
\end{dfnbox}

\begin{exbox}{Diverging Sequence}{}
    Prove that $s = (1,0,1,0,0,1,0,0,0,\ldots)$ does not converge to $0$.
    \tcblower
    \begin{proof}
        Let $\epsilon = \sfrac{1}{2}$. Then for all $N \in \N$, there exists $n > N$ such that $s_n = 1$. Then:
        \[ \abs{s_n - 0} = \abs{1-0} > \epsilon \]
        Therefore, $s$ does not converge.
    \end{proof}
\end{exbox}

\section{Properties of Limits}

A sequence can only converge to one value, not more. That is, if a sequence has a limit, then that limit is unique.

\begin{lembox}{Approximating Zero}{approximating-zero}
    Let $x \in \R$. If $x < \epsilon$ for all $\epsilon > 0$, then $x \leq 0$.
    \tcblower
    \begin{proof}
        We proceed by contraposition. Suppose $x > 0$. Let $\epsilon \coloneq \sfrac{x}{2} > 0$. Then $x \geq \epsilon = \sfrac{x}{2}$.
    \end{proof}
\end{lembox}

\begin{thmbox}{Uniqueness of Limits}{}
    Let $s_n$ be a sequence of real numbers. If $s_n$ converges to $l_1$ and converges to $l_2$, then $l_1 = l_2$.
    \tcblower
    \begin{proof}
        Let $\epsilon > 0$. Since $s_n$ converges to $l_1$, then there exists $N_1 \in \N$ such that $\abs{s_n - l_1} < \sfrac{\epsilon}{2}$ for all $n > N_1$. Similarly, since $s_n$ converges to $l_2$, then there exists $N_2 \in \N$ such that $\abs{s_n - l_2} < \sfrac{\epsilon}{2}$ for all $n > N_2$.

        Let $n \in \N$ where $n > N_1$ and $n > N_2$. Then:
        \[ \abs{l_1 - l_2} = \abs{l_1 - s_n + s_n - l_2} \leq \underbrace{\abs{l_1 - s_n} + \abs{s_n - l_2}}_\text{\nameref{thm:triangle-inequality}} < {\frac{\epsilon}{2} + \frac{\epsilon}{2}} = \epsilon \]
        Hence, $\abs{l_1 - l_2} < \epsilon$ for all $\epsilon > 0$. Thus, by Lemma \ref{lem:approximating-zero}, $\abs{l1-l2} \leq 0$. However, we know that $\abs{l1-l2} \geq 0$ since it's an absolute value. Thus, we have $\abs{l1-l2} = 0$, so $l1 = l2$.
    \end{proof}
\end{thmbox}

\todo[inline]{Definitions of bounds for sequences, show that convergent implies boundedness}

\todo[inline]{Split the theorem below into four separate theorems?}

\begin{thmbox}{Properties of Limits}{}
    Let $(s_n)$ and $(t_n)$ be convergent sequences of real numbers, and let $s,t \in \R$ such that $s_n$ converges to $s$ and $t_n$ converges to $t$. Then:
    \begin{enumerate}
        \item For any $c \in \R$, $cs_n$ converges to $cs$,
        \item $s_n + t_n$ converges to $s+t$,
        \item $s_nt_n$ converges to $st$, and
        \item if $t_n \neq 0$, then for all $n$ and $t \neq 0$, $\frac{s_n}{t_n}$ converges to $\frac{s}{t}$.
    \end{enumerate}
    \tcblower
    \begin{proof}[Proof of 1]
        Let $\epsilon > 0$. Since $(s_n)$ converges to $s$, then there exists $N \in \N$ such that $\abs{s_n - s} < \frac{\epsilon}{1 + \abs{c}}$ for all $n > N$. Then, for all $n > N$, we have:
        \[ \abs{cs_n - cs} = \abs{c(s_n-s)} = \abs{c}\abs{s_n-s} < \abs{c} \frac{\epsilon}{1+\abs{c}} = \frac{\abs{c}}{1+\abs{c}} \epsilon < \epsilon \]
    \end{proof}

    \begin{proof}[Proof of 2]
        Let $\epsilon > 0$. Since $(s_n)$ converges to $s$, then there exists $N_1 \in \N$ such that $\abs{s_n - s} < \sfrac{\epsilon}{2}$ for all $n > N$. Similarly, since $t_n$ converges to $t$, then there exists $N_2 \in \N$ such that $\abs{t_n - t} < \sfrac{\epsilon}{2}$. Let $N \in \N$ where $N \geq N_1$ and $N \geq N_2$. Then:
        \[ \abs{(s_n + t_n) - (s + t)} = \abs{s_n - s + t_n - t} \leq \abs{s_n - s} + \abs{t_n - t} < \frac{\epsilon}{2} + \frac{\epsilon}{2} = \epsilon \]
        That is, $s_n + t_n$ converges to $s + t$.
    \end{proof}

    \begin{proof}[Proof of 3]
        Let $\epsilon > 0$. Since $s_n$ converges to $s$, then there exists $N_1 \in \N$ such that $\abs{s_n - s} < \sfrac{\epsilon}{2(\abs{t} + 1)}$ for all $n > N$. Also, $(s_n)$ converges, so $(s_n)$ is bounded. That is, there exists $M \in \R$ such that $\abs{s_n} \leq M$ for all $n \in \N$. Since $t_n$ converges to $t$, there exists $N_2 \in \N$ such that $\abs{t_n - t} < \frac{\epsilon}{2(M+1)}$ for all $n > N$. Let $N \in \N$ such that $N \geq N_1$ and $N \geq N_2$. If $n > N$, then:
        \begin{align*}
            \abs{s_nt_n - st}
            &= \abs{s_nt_n - s_nt + s_nt - st} \\
            &= \abs{s_n(t_n-t) + (s_n-s)t} \\
            &\leq \abs{s_n(t_n-t)} + \abs{(s_n-s)t} \\
            &= \abs{s_n} \abs{t_n-t} + \abs{s_n-s}\abs{t} \\
            &< M \frac{\epsilon}{2(1+M)} + \frac{\epsilon}{2(1+\abs{t})} \abs{t} \\
            &= \frac{M}{1+M} \frac{\epsilon}{2} + \frac{\epsilon}{2} \frac{abs{t}}{1 + \abs{t}} \\
            &< \frac{\epsilon}{2} + \frac{\epsilon}{2} \\
            &= \epsilon
        \end{align*}
    \end{proof}

    \begin{proof}[Proof of 4]
        We will prove that $\frac{1}{t_n}$ converges to $\frac{1}{t}$. Let $\epsilon > 0$. Since $t_n$ converges to $t$ and $t \neq 0$, then there exists $N_1 \in \N$ such that $\abs{t_n - t} < \frac{\epsilon t^2}{2}$. By \nameref{lem:14.6}, there exists $N_2 \in \N$ such that $\abs{t_n} > \frac{\abs{t}}{2}$ for all $n > N_2$. Let $N \in \N$ such that $N > N_1$ and $N > N_2$. Let $n \in \N$ be arbitrary. Then:
        \begin{align*}
            \abs{\frac{1}{t_n} - \frac{1}{t}} &= \abs{\frac{t-t_n}{t_nt}} \\
            &= \frac{1}{\abs{t_n}} \frac{1}{\abs{t}} \abs{t-t_n} \\
            &< \frac{2}{\abs{t}} \frac{1}{\abs{t}} \frac{\epsilon t^2}{2} \\
            &= \epsilon
        \end{align*}
        By 3, if $s_n$ converges to $s$, then $\frac{s_n}{t_n} = s_n \left( \frac{1}{t_n} \right) $ converges to $s \left( \frac{1}{t} \right) = \frac{s}{t}$.
    \end{proof}
\end{thmbox}
\todo{lemma 14.6 for bounding in proof of 4.}
\todo{Explain new notation}

\begin{lembox}{Limit of a Constant Sequence}{}
    If $s_n$ is a constant sequence $(l, l, l, \ldots)$, then $s_n$ converges to $l$.
    \tcblower
    \begin{proof}
        Let $\epsilon > 0$. For all $n \in \N$, $\abs{s_n - l} = 0 < \epsilon$.
    \end{proof}
\end{lembox}
\todo{better}

\begin{exbox}{Using the Properties}{}
    Prove $\lim_{n \to \infty} \frac{5n^3 - 8n^2 + 15}{7n^3 + 19n + 4} = \frac{5}{7}$.
    \tcblower
    \begin{proof}
        \begin{align*}
            \lim_{n \to \infty} \frac{5n^3 - 8n^2 + 15}{7n^3 + 19n + 4}
            &= \lim_{n \to \infty} \frac{5 - \sfrac{8}{n} + \sfrac{15}{n^3}}{7 + \sfrac{19}{n^2} + \sfrac{4}{n^3}} \\
            &= \frac{\lim_{n \to \infty} 5 - \sfrac{8}{n} + \sfrac{15}{n^3} }{\lim_{n \to \infty} 7 + \sfrac{19}{n^2} + \sfrac{4}{n^3}} \\
            &= \frac{ \lim_{n \to \infty} 5 - \lim_{n \to \infty} \sfrac{8}{n} + \lim_{n \to \infty} \sfrac{15}{n^3} }{ \lim_{n \to \infty} 7 + \lim_{n \to \infty} \sfrac{19}{n^2} + \lim_{n \to \infty} \sfrac{4}{n^3} }
        \end{align*}
        Now we can work with each limit independently. Note that $\lim_{n \to \infty} \frac{1}{n^2} = \left( \lim_{n \to \infty} \frac{1}{n} \right) \left( \lim_{n \to \infty} \frac{1}{n} \right)$, so:
        \[ \lim_{n \to \infty} \frac{5n^3 - 8n^2 + 15}{7n^3 + 19n + 4} = \frac{5}{7} \]
    \end{proof}
\end{exbox}

\begin{dfnbox}{Increasing, Decreasing, Monotonic}{}
    A sequence $(s_n)$ is:
    \begin{itemize}[noitemsep]
        \item \dfntxt{increasing} if $s_n \leq s_{n+1}$ for all $n \in \N$.
        \item \dfntxt{strictly increasing} if $s_n < s_{n+1}$ for all $n \in \N$.
        \item \dfntxt{decreasing} if $s_n \geq s_{n+1}$ for all $n \in \N$.
        \item \dfntxt{strictly decreasing} if $s_n > s_{n+1}$ for all $n \in \N$.
    \end{itemize}
    If $(s_n)$ satisfies any of these properties, then we say $(s_n)$ is \dfntxt{monotonic}.
\end{dfnbox}

For example, $(s_n) = \left( \frac{1}{n} \right) = \left(1, \frac12, \frac13, \frac14, \ldots \right)$ is strictly decreasing and thus monotonic.

\begin{thmbox}{Monotone Sequence Theorem}{monotone}
    Let $(s_n)$ be a sequence of real numbers.
    \begin{enumerate}
        \item If $(s_n)$ is increasing and bounded above, then $(s_n)$ converges to $\sup \{ s_n : n \in \N \}$.
        \item If $(s_n)$ is decreasing and bounded below, then $(s_n)$ converges to $\inf\{s_n : n \in \N \}$.
    \end{enumerate}
    \tcblower
    \textbf{Idea:} Assuming $s$ is our limit, we want to find $N \in \N$ such that $\abs{s - s_n} < \epsilon$, or $s - \epsilon < s_n$ for all $n > N$. Then $s - \epsilon < s_n \leq s$ for all $n > N$.
    \begin{proof}[Proof of 1.]
        Let $\epsilon > 0$. Because $\{s_n : n \in \N\}$ is non-empty and bounded above, then it has a supremum. Let $s \coloneq \sup\{s_n : n \in \N \}$. Thus, there exists $N \in \N$ such that $s_N > s - \epsilon$ (by the approximation property). Since $(s_n)$ is increasing, we have:
        \[ \forall (n > N) \left(s - \epsilon < s_N \leq s_n \leq s \quad \right) \]
        Hence, $\epsilon < s_n - s \leq 0$, so $\abs{s_n - s} < \epsilon$.
    \end{proof}

    \begin{proof}[Proof of 2]
        Suppose $(s_n)$ is decreasing and bounded below. Then $s_{n+1} \leq s_n$ for all $n \in \N$. Moreover, there exists $m \in \R$ such that $s_n \geq m$ for all $n \in \N$. That is, $- s_{n+1} \geq - s_n$ for all $n \in \N$, and $- s_n \leq -m$ for all $n \in \N$. Therefore, $(-s_n)$ is increasing and bounded above. By the first part, we know $(-s_n)$ converges to $\sup\{ -s_n : n \in \N\} = - \inf\{s_n : n \in \N\}$. Hence, $(s_n)$ converges to $\inf\{s_n : n \in \N\}$.
    \end{proof}
\end{thmbox}

\section{Subsequences}

So far, we've only looked at well-behaving sequences that converge. What about sequences that don't converge? Can will still find some nice properties that describe their behavior?
\[ (s_n) = (0,1,0,1,0,1,\ldots) \]
Consider a sequence $(t_n)$ where $t_n = s_{2n}$. That is:
\[ (t_n) = (s_2, s_4, s_6, \ldots) = (1, 1, 1, \ldots) \]
Inside this diverging sequence, we can find a convergent \dfntxt{subsequence}! Intuitively, we can make a subsequence by ``throwing away'' terms but keeping the same order. We can formally define a subsequence as follows:

\begin{dfnbox}{Subsequence}{}
    Given a sequence $(s_n)$, a \dfntxt{subsequence} is any sequence of the form $(t_k)_{k \in \N}$ where $t_k = s_{n_k}$ for all $k \in \N$, $n_k \in \N$ for all $k \in \N$, and $n_k < n_{k+1}$ for all $k \in \N$.
\end{dfnbox}

For example, if we had $s = (s_1, s_2, s_3, s_4, s_5, s_6, s_7, \ldots, s_{213}, s_{214}, s_{215}, \ldots)$, we can have a subsequence like:
\[ (t_n) = (s_3, s_5, s_{213}, \ldots) \]
Here, we would have $n_1 = 3$, $n_2 = 5$, $n_3 = 213$, and so on.

\begin{exbox}{Subsequences}{}
    Let $(s_n) \coloneq \left( 1, \sfrac12, \sfrac13, \sfrac14, \sfrac15, \sfrac16, \ldots \right)$
    \begin{enumerate}
        \item $(t_n) \coloneq \left( 1, \sfrac14, \sfrac19, \sfrac{1}{16}, \sfrac{1}{25} \right)$ is a subsequence of $(s_n)$ where $t_n = \sfrac{1}{n^2}$, or $t_n = s_{n^2}$.
        \item $(t_n) \coloneq \left( \sfrac{1}{5}, \sfrac{1}{25}, \sfrac{1}{125}, \ldots \right)$ is also a subsequence of $(s_n)$ with $t_n = \frac{1}{5^n}$ or $t_n = s_{5^n}$.
        \item $(t_n) \coloneq \left( \sfrac17, \sfrac12, \sfrac{1}{12}, \sfrac16 \right)$ is \textbf{not} a subsequence of $(s_n)$ because the indices in $s_n$ are not strictly increasing. We have $n_1 = 7$, but $n_2 = 2$.
    \end{enumerate}
\end{exbox}

\begin{notebox}
    \textbf{In general:}
    \[ (s_n) = (s_1, s_2, s_3, \ldots) \]
    \[ (t_n) = (s_{n_k}) = (s_{n_1}, s_{n_2}, s_{n_3}) \]
\end{notebox}

\begin{lembox}{Indices of Subsequences}{indices-of-subsequences}
    If $(s_{n_k})_{k \in \N}$ is a subsequence of $(s_n)_{n \in \N}$, then $n_k \geq k$ for all $k \in \N$.
    \tcblower
    We will use induction.

    \textbf{Base Case:} Since $n_1 \in \N$, then $n_1 \geq 1$.

    \textbf{Induction Step:} Suppose $n_k \geq k$ for some $k \in \N$. Since $n_{k+1} > n_k$, we have $n_{k+1} \geq n_k + 1 \geq k + 1$.

    Hence, $n_k \geq k$ for all $k \in \N$.
\end{lembox}

\begin{thmbox}{Limits of Subsequences}{}
    Suppose $(s_n)$ is a sequence of real numbers, and $s_n$ converges to $s$ for some $s \in \R$. If $(s_{n_k})$ is a subsequence of $(s_n)$, then $s_{n_k}$ converges to $s$.
    \tcblower
    \begin{proof}
        Let $\epsilon > 0$. Since $s_n$ converges to $s$, then there exists $N \in \N$ such that $\abs{s_n - s} < \epsilon$ for all $n > N$. Suppose $k > N$. By lemma \ref{lem:indices-of-subsequences}, $n_k \geq k > N$, so $\abs{s_{n_k} - s} < \epsilon$.
    \end{proof}
\end{thmbox}

\section{Limit Superior and Inferior}

Suppose $(s_n)$ is a bounded sequence. Then there exists $M \in \R$ such that $-M \leq s_n \leq M$ for all $n \in \N$. Let:
\begin{align*}
    t_1 &\coloneq \sup\{s_1, s_2, s_3, \ldots\} = \sup\{ s_k : k \geq 1 \} \\
    t_2 &\coloneq \sup\{s_2, s_3, s_4, \ldots\} = \sup\{ s_k : k \geq 2\} \\
    t_3 &\coloneq \sup\{s_3, s_4, s_5, \ldots\} = \sup\{ s_k : k \geq 3\} \\
    &\vdots \\
    t_n &\coloneq \sup\{s_n, s_{n+1}, s_{n+2}, \ldots\} = \sup\{s_k : k \geq n\} \\
    t_{n+1} &\coloneq \sup\{s_{n+1}, s_{n+2}, s_{n+3}, \ldots\} = \sup\{s_k : k \geq n + 1\} \\
\end{align*}

Then:
\[ -M \leq s_n \leq t_n \]
and:
\[ t_{n+1} \leq t_n \]
so $(t_n)$ is bounded below and decreasing. Hence, $(t_n)$ converges by the \nameref{thm:monotone}.

\begin{dfnbox}{Limit Superior, Limit Inferior}{}
    Let $(s_n)$ be a bounded sequence of real numbers. The \dfntxt{limit superior} is defined as:
    \[ \limsup s_n \coloneq \lim_{n \to \infty} \sup\{s_k : k \geq n \} \]
    Similarly, the \dfntxt{limit inferior} is defined as:
    \[ \liminf s_n \coloneq \lim_{n \to \infty} \inf\{s_k : k \geq n \} \]
\end{dfnbox}

\begin{exbox}{}{}
    Define $s_n \coloneq \begin{cases}
        3+\frac1n, & n\ \text{is even} \\
        1 - \frac1n & n\ \text{is odd}
    \end{cases}$

    $(s_n) = (0, 3 + \sfrac12, \sfrac23, 3 + \sfrac14, \sfrac45, 3 + \sfrac16)$

    Let's try to calculate the limit superior of $s_n$. Define $(t_n)$ as follows:
    \begin{align*}
        t_1 &\coloneq \sup\{s_1, s_2, s_3, \ldots\} = 3 + \frac12 \\
        t_2 &\coloneq \sup\{s_2, s_3, s_4, \ldots\} = 3 + \frac12 \\
        t_3 &\coloneq \sup\{s_3, s_4, s_5, \ldots\} = 3 + \sfrac14 \\
        t_4 &\coloneq \sup\{s_4, s_5, s_6, \ldots\} = 3 + \sfrac14 \\
        t_5 &\coloneq \sup\{s_5, s_6, s_7, \ldots\} = 3 + \sfrac16 \\
            &\vdots
    \end{align*}
    We can see that $\limsup s_n = \lim_{n \to \infty} \sup\{s_k : k \geq n\} = 3$. We might refer to $3$ as the ``largest limit point''.

    Now let's try to calculate the limit inferior of $s_n$. Define $(r_n)$  as follows:
    \begin{align*}
        r_1 &\coloneq \inf\{s_1, s_2, s_3, \ldots\} = 0 \\
        r_2 &\coloneq \inf\{s_2, s_3, s_4, \ldots\} = \frac23 \\
        r_3 &\coloneq \inf\{s_3, s_4, s_5, \ldots\} = \frac23 \\
        r_4 &\coloneq \inf\{s_4, s_5, s_6, \ldots\} = \frac45 \\
        &\vdots
    \end{align*}
    We can see that $\liminf s_n = \lim_{n \to \infty} \inf\{s_k : k \geq n\} = 1$. We might refer to $1$ as the ``smallest limit point''.
\end{exbox}

\begin{thmbox}{}{}
    Suppose $(s_n)$ is a bounded sequence of real numbers, and suppose that $(s_{n_k})$ is a convergent subsequence of $(s_n)$. Then $\liminf s_n \leq \lim_{k \to \inf} s_{n_k} \leq \limsup s_n$.
    \tcblower
    \begin{proof}
        Let $r_n \coloneq \inf\{s_k : k \geq n\}$ and $t_n \coloneq \sup\{s_k : k \geq n\}$. Then $r_n \leq s_n \leq t_n$ for all $n \in \N$. In particular, $r_{n_k} \leq s_{n_k} \leq t_{n_k}$ for all $k \in \N$. By (todo: theroem), $\lim_{k \to \infty} r_{n_k} = \lim_{n \to \infty} r_n$ . Note that $\lim_{n \to \infty} r_n = \liminf s_n$, and $\lim_{k \to \infty} t_{n_k} = \lim_{n \to \infty} t_n = \limsup s_n$. By the (todo: problem set squeeze theorem), we have:
        \[ \liminf s_n = \lim_{k \to \infty} r_{n_k} \leq lim_{k\ to \infty} s_{n_k} \leq \lim_{k \to \infty} t_{n_k} = \limsup s_n \]
    \end{proof}
\end{thmbox}

\begin{thmbox}{Bolzano-Weierstrass Theorem}{bw}
    Suppose $(s_n)$ is a bounded sequence of real numbers. The $(s_n)$ has a subsequence that coverges to $\limsup s_n$, and $(s_n)$ has a subsequence that converges to $\liminf s_n$.
    \tcblower
    \textbf{Intuition:}
    \begin{itemize}[noitemsep]
        \item Let $t_k \coloneq \sup \{ s_k, s_{k+1}, s_{k+2}, \ldots\}$, so $\limsup s_n = \lim_{k \to \infty} t_k$.
        \item For each $k \in \N$ we can find some $n_k \geq k$ such that $t_k - \sfrac{1}{k} < s_{n_k}$.
        \item Thus, $-\sfrac{1}{k} < s_{n_k} - t_k \leq 0$, so $\abs{s_{n_k} - t_k} < \sfrac{1}{k}$
        \item By (todo: problem set), $s_{n_k} - t_k \to 0$, so $s_{n_k} = s_{n_k} - t_k + t_k \to \limsup s_n$.
        \item But: we need $n_1 < n_2 < \cdots < n_k < n_{k+1} < \cdots$. So we need to choose $n_k$ inductively!
    \end{itemize}
    \begin{proof}[Proof for limsup.]
        We will choose a subsequence of $(s_n)$ that converges to $\limsup s_n$. For each $k \in \N$, let $t_k \coloneq \sup \{ s_k, s_{k+1}, s_{k+2}, \ldots\}$ For convenience, let $P(n)$ be the statement ``there exists $n_k \in \N$ such that $n_k > n_{k-1}$ and $\abs{s_{n_k} - t_{1 + n_{k-1}}} < \frac{1}{k}$.'' We define $n_0 \coloneq 0$.

        \textbf{Base Case:} Let $t_1 \coloneq \sup \{s_1, s_2, \ldots\}$. By the approximation property (todo ref), there exists $n_1 \in \N$ such that $t_1 - 1 < s_{n_1} \leq t_1$. Subtracting across by $t_1$, we have $-1 < s_{n_1} - t_1 \leq 0$. Thus, $\abs{s_{n_1} - t_1} < 1$.

        \textbf{Induction Step:} Now we aim to prove $P(k-1) \implies P(k)$. There exists $n_k \in \N$ such that $n_k > n_{k-1}$, and:

        \begin{alignat*}{2}
            && t_{1 + n_{k-1}} - \frac{1}{k} &< s_{n_k} \leq t_{1+n_{k-1}} \\
            & \implies \quad & -\frac{1}{k} &< s_{n_k} - t_{1 + n_{k-1}} \leq 0 \\
            & \implies \quad & \abs{s_{n_k} - t_{1 + n_{k-1}}} &< \frac{1}{k}
        \end{alignat*}

        That is, $\lim_{k \to \infty} \left( s_{n_k} - t_{1+ n_{k-1}} \right) = 0$. Since $n_k > n_{k-1}$ for all $k \in \N$, $(s_{n_k})$ is a subsequence of $(s_n)$. But $(t_{1 + n_{k-1}})$ is a subsequence of $(t_k)$, so:
        \[ \lim_{k \to \infty} t_{1 + n_{k-1}} = \lim_{k \to \infty} t_k = \limsup s_n \]
        Thus:
        \[ s_{n_k} = s_{n_k} - t_{1 + n_{k-1}} + t_{1 + n_{k-1}} \]
        so:
        \[ \lim_{k \to \infty} s_{n_k} = \lim_{k \to\infty} \left( s_{n_k} - t_{1 + n_{k-1}} \right) + \lim_{k \to \infty} t_{1 + n_{k-1}} = 0 + \limsup s_n \]
        Therefore, $(s_{n_k})$ is a subsequence of $(s_n)$ that converges to $\limsup s_n$.
    \end{proof}
\end{thmbox}

\begin{thmbox}{Convergence iff $\limsup = \liminf$}{}
    Let $(s_n)$ be a bounded sequence of real numbers. Then $(s_n)$ converges if and only if $\liminf s_n = \limsup s_n$
    \tcblower
    \begin{proof}
        First, suppose $s_n$ converges to some $s \in \R$. By the \nameref{thm:bw}, there exists a subsequence $(s_{n_k})$ of $(s_n)$ such that $\lim_{k \to \infty} s_{n_k} = \limsup s_n$. But $s_n$ converges to $s$, so $s_{n_k}$ also converges to $s$. That is, $s = \lim_{k \to \infty} s_{n_k} = \limsup s_n$. By the same reasoning, we have $s = \liminf s_n$. Hence, $\liminf s_n = \lim s_n = \limsup s_n$.

        Conversely, suppose $\liminf s_n = \limsup s_n$. Let $r_n \coloneq \inf\{s_k : k \geq n\}$ and $t_n \coloneq \sup\{s_k : k \geq n\}$. Then $r_n \leq s_n \leq _n$ for all $n \in \N$. Then $\lim r_n = \liminf s_n = \limsup s_n = \lim t_n$. Therefore, by the Squeeze Theorem (todo: ref), $s_n$ converges to $\liminf s_n$.
    \end{proof}
\end{thmbox}

\section{Cauchy Sequences}

To show that a sequence $(s_n)$ converges using the definition of limit, we need to know what limit is beforehand. Consider the following limit:
\[ \lim_{n\to\infty} \sum_{k=1}^{n} \frac{1}{k^3} \]
This sequence of partial sums converges, but its limit is unknown. Certainly we can get a decimal approximation for this value, but there is no known ``closed'' form of this value.

\begin{dfnbox}{Cauchy Sequence}{}
    We say a sequence is \dfntxt{Cauchy} if, for all $\epsilon > 0$, there exists $N \in \N$ such that $\abs{s_n - s_m} < \epsilon$ for all $n > N$ and $m > M$.
    \tcblower
    \[ \forall(\epsilon > 0) \exists(N \in \N) \forall(n > N, m > N) \left( \abs{s_n - s_m} < \epsilon \right) \]
\end{dfnbox}

In other words, a sequence is Cauchy if \textbf{all} terms in the tail can be made arbitrarily close to each other. Or, for any arbitrarily small distance, there exists some ``tail'' of the sequence that exists entirely within that distance. This definition circumvents any mention of a specific ``limit''. But we can prove that any Cauchy sequence of real numbers is convergent, and vice versa.

\begin{lembox}{Convergent Sequences are Cauchy}{convergent-implies-cauchy}
    If a sequence of real numbers converges, then that sequence is Cauchy.
    \tcblower
    \begin{proof}
        Let $(s_n)$ be a convergent sequence of real numbers. Let $\epsilon > 0$. Since $(s_n)$ converges to some $s \in \R$, there exists $N \in \N$ such that $\abs{s_n - s} < \sfrac{\epsilon}{2}$ for all $n > N$. If $n > N$ and $m > N$, then:
        \[ \abs{s_n - s_m} = \abs{s_n - s + s - s_m} \leq \abs{s_n - s} + \abs{s - s_m} < \frac{\epsilon}{2} + \frac{\epsilon}{2} = \epsilon \]
        % \begin{align*}
        %     \abs{s_n - s_m}
        %     &= \abs{s_n - s + s - s_m} \\
        %     &\leq \abs{s_n - s} + \abs{s - s_m} \\
        %     &= \abs{s_n - s} + \abs{s_m - s} \\
        %     &< \frac{\epsilon}{2} + \frac{\epsilon}{2} \\
        %     &= \epsilon
        % \end{align*}
    \end{proof}
\end{lembox}

\begin{lembox}{}{cauchy-convergent-subsequence}
    Suppose $(s_n)$ is a Cauchy sequence, and that $(s_{n_k})$ is a convergent subsequence of $(s_n)$ where $s_{n_k}$ converges to some $s \in \R$. Then $(s_n)$ converges, and $\lim s_n = s$.
    \tcblower
    \begin{proof}
        Let $\epsilon > 0$. Since $(s_n)$ is Cauchy, then there exists some $N \in \N$ such that $\abs{s_n - s_m} < \sfrac{\epsilon}{2}$ for all $n > N$ and $m > N$. Since $(s_{n_k})$ converges to $s$, there exists $N_1 \in \N$ such that $\abs{s_{n_k} - s} < \frac{\epsilon}{2}$ for all $k > N_1$. Let $k \in \N$ where $k > N$ and $k > N_1$. Since $n_k \geq k$, then $n_k > N$ and $n_k > N_1$. For all $n > k$, we have:
        \[ \abs{s_n - s} = \abs{s_n - s_{n_k} + s_{n_k} - s} \leq \underbrace{\abs{s_n - s_{n_k}}}_{n_1, n_k > N} + \underbrace{\abs{s_{n_k} - s}}_{n_k > N_1} < \frac{\epsilon}{2} + \frac{\epsilon}{2} = \epsilon\]
    \end{proof}
\end{lembox}

\begin{lembox}{Cauchy Sequences are Bounded}{cauchy-implies-bounded}
    If a sequence of real numbers is Cauchy, then that sequence is bounded.
    \tcblower
    \begin{proof}
        Let $(s_n)$ be a Cauchy sequence of real numbers. Then there exists $N \in \N$ such that $\abs{s_n - s_m} < 1$ for all $n > N$ and $m > N$. Let $m \coloneq N+1$. Then, for all $n > N$, we have:
        \[ \abs{s_n} = \abs{s_n - s_m + s_m} \leq \abs{s_n - s_m} + \abs{s_m} < 1 + \abs{s_m} = 1 + \abs{s_{N+1}} \]
        Thus, for all $n \in \N$, we have:
        \[ \abs{s_n} \leq \max\{\abs{s_1}, \abs{s_2}, \ldots, \abs{s_N}, 1 + \abs{s_{N+1}} \} \]
        Therefore, $(s_n)$ is bounded.
    \end{proof}
\end{lembox}

\begin{thmbox}{Cauchy Criterion}{}
    A sequence of real numbers converges if and only if it is Cauchy.
    \tcblower
    \begin{proof}
        By Lemma \ref{lem:convergent-implies-cauchy}, we know that convergence implies Cauchy.

        If $(s_n)$ is Cauchy, then by Lemma \ref{lem:cauchy-implies-bounded}, $(s_n)$ is bounded. By the \nameref{thm:bw}, $(s_n)$ has a convergent subsequence. By Lemma \ref{lem:cauchy-convergent-subsequence}, $(s_n)$ converges.
    \end{proof}
\end{thmbox}

Our definition of completeness in $\R$ predicates on a notion of order between the elements. Specifically, we said $\R$ is complete because every subset of $\R$ that is bounded above has a supremum. What does it mean to say $\R^2$ is complete?

\begin{dfnbox}{Completeness (in terms of Cauchy Sequences)}{}
    A (metric) space is \dfntxt{complete} if every Cauchy sequence converges to a point in the space.
\end{dfnbox}

The intuition is the same: there are no points ``missing'' from the space.
