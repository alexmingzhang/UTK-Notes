\chapter{Consequences of Completeness}
\begin{thmbox}{$\N$ is not Bounded Above}{}
    \begin{proof}
        Suppose for contradiction $\N$ is bounded above. Since $\N$ is not empty, then $\N$ has a supremum in $\R$. Let $s \coloneq \sup \N \in \R$. Then $n \leq s$ for all $n \in \N$. By the Peano axioms, $n$ has a successor $n+1 \in \N$, so $n+1 \leq s$ for all $n \in \N$. Therefore, $n \leq s - 1$ for all $n \in \N$. This contradicts $s$ being the least upper bound for $\N$.
    \end{proof}
\end{thmbox}

\begin{thmbox}{Archimedean Principle}{archimedean}
    Suppose $x,y \in \R$ where $x > 0$. Then, there exists $n \in \N$ such that $nx > y$.
    \tcblower
    \textbf{Intuition:} This is basically an extension of the fact that $\N$ is not bounded above.
    \begin{proof}
        Since $\sfrac{y}{x}$ is not an upper bound for $\N$, then there exists $n \in \N$ such that $n > \sfrac{y}{x}$. Since $x > 0$, then $nx > y$.
    \end{proof}
\end{thmbox}

\begin{thmbox}{Density of $\Q$ in $\R$}{}
    Suppose $x,y \in \R$ where $x < y$. Then there exists $r \in Q$ such that $x < r < y$.
    \tcblower
    \textbf{Intuition:} Given any two different real numbers, there's some rational number between them.
    \begin{proof}
        We will consider three cases:
        \begin{enumerate}
            \item If $x \geq 0$, then $0 \leq x < y$. Since $y - x > 0$, then by the \nameref{thm:archimedean}, there exists $n \in \N$ such that $n(y-x) > 1$. We want to show there is a natural number between $nx$ and $ny$. Let $A \coloneq \{ k \in \N : k > nx \}$. Since $\N$ isn't bounded above, then $A$ is not empty. By the \nameref{thm:wop}, $A$ has a minimum. Let $m \coloneq \min A$. Then $m > nx$, and $m-1 \leq nx$. Thus, $m \leq nx+1$, so:
            \[ nx < m \leq nx+1 < ny \]
            Dividing across by $n$ yields $x < \sfrac{m}{n} < y$. Note that $m,n \in \N \subseteq \Z$, so $\sfrac{m}{n} \in \Q$.
            \item If $x < 0$ and $y > 0$, then $x < 0 < y$ where $0 \in \Q$.
            \item If $x < 0$ and $y \leq 0$, then $x < y \leq 0$. Multiplying across by $-1$, we have $-x > -y \geq 0$. By the first case, there must exist $t \in \Q$ where $-y < t < -x$. Multiply across by $-1$ again to attain $y > -t > x$ where $-t \in \Q$.
        \end{enumerate}
        This completes the proof.
    \end{proof}
\end{thmbox}

\begin{thmbox}{$\sqrt{2}$ is a Real Number}{}
    There exists $s \in \R$ such that $s^2 = 2$.
    \tcblower
    \begin{proof}
        Let $A \coloneq \left\{ x \in \R : x^2 < 2 \right\}$. Since $0^2 < 2$, then $0 \in A$, so $A$ is not empty. Also, $A$ is bounded above, for example by $2$. By completeness, $A$ must have a supremum in $\R$. Let $s \coloneq \sup A$. We will use trichotomy to show that $s^2 = 2$.
        \begin{enumerate}
            \item If $s^2 > 2$, then$\ldots$
            \begin{notebox}
                \textbf{Scratchwork:} We need to show that this is not possible, i.e. show there is some $s - \sfrac{1}{n}$ that is less than $s$ but is still an upper bound for $A$. We want $(s - \sfrac{1}{n})^2 > 2$. Then, $s^2 - \sfrac{2s}{n} + \sfrac{1}{n^2} > 2$. We can chop off the $\sfrac{1}{n^2}$, reducing the inequality to $s^2 - \sfrac{2s}{n} > 2$. Thus, we need to choose $n > \frac{2s}{s^2-2}$.
            \end{notebox}
            $\ldots$ let $n \in \N$ such that $n > \frac{2s}{s^2-2}$. Then:
            \begin{alignat*}{2}
                && n &> \frac{2s}{s^2-2} \\
                & \implies \quad & s^2 - \frac{2s}{n} &> 2 \\
                & \implies &  s^2 - \frac{2s}{n} + \frac{1}{n^2} &> 2 \\
                & \implies & \left( s - \frac{1}{n} \right)^2 &> 2
            \end{alignat*}
            Thus, $s - \sfrac{1}{n}$ is an upper bound for $A$ that is less than $s$. This contradicts $s$ being the supremum for $A$.
            \item If $s^2 < 2$, then$\ldots$
            \begin{notebox}
                \textbf{Scratchwork:} Again, we need to show that this is not possible. We know that in this case, $s \in A$, so we need to find another thing in $A$ that is bigger than $s$. In other words, we want some $(s + \sfrac{1}{n})^2 < 2$. Then, $s^2 + \sfrac{2s}{n} + \sfrac{1}{n^2} < 2$. Choose $n > \sfrac{1}{2s}$ and $n > \frac{4s}{2-s^2}$.
                \begin{align*}
                    \left( s + \frac{1}{n} \right)^2 &= s^2 + \frac{2s}{n} + \frac{1}{n^2}
                \end{align*}
            \end{notebox}
            $\ldots$ let $n \in \N$ such that $n > \max\left\{ \frac{1}{2s}, \frac{4s}{2-s^2} \right\}$. Then $\frac{1}{n} < 2s$ and $s^2 + \frac{4s}{n} < 2$. So:
            \begin{align*}
                \left( s + \frac{1}{n} \right)^2
                &= s^2 + \frac{2s}{n} + \frac{1}{n^2} \\
                &< s^2 + \frac{2s}{n} + \frac{2s}{n} \\
                &= s^2 + \frac{4s}{n} < 2
            \end{align*}
            That is, $s + \frac{1}{n} \in A$. This contradicts $s$ being an upper bound for $A$.
        \end{enumerate}
        By trichotomy, $s^2 = 2$.
    \end{proof}
\end{thmbox}

\begin{thmbox}{Nested Interval Property}{nested-interval-property}
    Suppose that for each $n \in \N$, $a_n, b_n \in \R$ with $a_n \leq b_n$, and $a_n \leq a_{n+1} \leq b_{n+1} \leq b_n$ for all $n \in \N$. Then $\bigcap_{n=1}^\infty [ a_n, b_n ] \neq \emptyset$.
    \tcblower
    \textbf{Intuition:} We can move the two borders of an open interval closer and closer to each other, and it won't be empty.
    \begin{proof}
        Note that $a_n \leq a_{n+1} \leq a_{n+2} \leq \ldots$ and $\ldots \leq b_{n+2} \leq b_{n+1} \leq b_n$. If $k \leq n$, then $a_k \leq a_n \leq b_n$.
        \begin{itemize}[noitemsep]
            \item If $k \leq n$, then $a_k \leq a_n \leq b_n$.
            \item If $k \geq n$, then $a_k \leq b_k \leq b_n$.
        \end{itemize}
        That is, $a_k \leq b_n$ for all $k_n \in \N$. Let $A \coloneq \{ a_k : k \in \N\}$. Then $A$ is bounded above, for example by $b_1$. Also, $A$ is not empty. By completeness, $A$ has a supremum. Let $s \coloneq \sup A$. Note that since $s$ is an upper bound for $A$, then $a_n \leq \sup A$ for all $n \in \N$. Also note that $\sup A$ is the least upper bound for $A$, so $\sup A \leq b_n$ for all $n \in \N$. Thus, $a_n \leq \sup A \leq b_n$ for all $n \in \N$, so $\sup A \in [ a_n, b_n ]$ for all $n \in \N$. Thus, $\sup A \in \bigcap_{n=1}^\infty [ a_n, b_n ]$, so it is not empty.
    \end{proof}
\end{thmbox}

The nested interval property is actually false for open intervals!
\[ \forall (x \in (0,1)) \exists (n \in \N) (\sfrac{1}{n} < x \implies x \notin (0, \sfrac{1}{n}) ) \]
