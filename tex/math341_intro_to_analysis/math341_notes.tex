\documentclass[12pt]{report}

\usepackage[margin=1in]{geometry}
\usepackage[math]{amznotes}
\usepackage{enumitem}
\usepackage{xfrac}
\usepackage[textsize=scriptsize]{todonotes}

\title{\textbf{Introduction to Analysis}\\
\large UT Knoxville, Spring 2023, MATH 341}
\author{Mike Frazier, Peter Humphries, Alex Zhang}

\begin{document}
\maketitle
\tableofcontents

\addcontentsline{toc}{chapter}{Preface}
\chapter*{Preface}
These are my notes for the \textbf{Introduction to Analysis} course at the University of Tennessee (MATH 341). It is compiled from several sources including lecture notes by Dr. Michael Frazier and Dr. Peter Humphries, as well as online resources such as the Mathematics Stack Exchange.

% These notes attempt to strike a good balance between brevity and depth. As such, they are intended to provide a comprehensive and accessible resource for those looking to understand the fundamental concepts of real analysis.

The first few weeks of the course are spent reviewing the material taught in \textbf{Introduction to Abstract Mathematics} (MATH 300): logic, set theory, number systems, and cardinality. They serve as a ``primer'' for the following material on real analysis.

\chapter{Introduction}
Our goal is to understand the theory of real functions in one variable. Specifically, we will deal with functions, limits, sequences, convergence, continuity, differentiation, and integration. The same ideas, concepts, and techniques are used to study more complicated mathematics.

We will primarily focus on the idea of \textbf{convergence}. Many computational techniques and algorithms rely on iteration---successive approximations getting closer to an actual solution. In order for those algorithms to work, they need to converge towards an actual solution.

To motivate our quest to learn about convergence, let's look at some classic iterative methods.

\begin{exbox}{Newton's Method}{}
    Given $c > 0$, suppose we want to calculate $\sqrt{c}$. Start with some initial guess $x_1 > 0$.
    \begin{alignat*}{2}
        & \text{Let}\quad & x_2 &\coloneq \frac{1}{2} \left( x_1 + \frac{c}{x_1} \right) \\
        & \text{Let}\quad & x_3 &\coloneq \frac{1}{2} \left( x_2 + \frac{c}{x_2} \right) \\
        && &\vdots \\
        & \text{Let}\quad & x_{n+1} &\coloneq \frac{1}{2} \left( x_n + \frac{c}{x_n} \right)
    \end{alignat*}
    We find that $\lim_{n\to\infty} x_n = \sqrt{c}$.
\end{exbox}

Does this method work for all $c > 0$ and $x_1 > 0$? Assuming $\lim_{n\to\infty} x_n = x$ converges, then:
\begin{alignat*}{2}
    && x_{n+1} &= \frac{1}{2} \left( x_n + \frac{c}{x_n} \right) \\
    &\implies \quad &x &= \frac{1}{2} \left( x + \frac{c}{x} \right) \\
    &\implies &2x &= x + \frac{c}{x} \\
    &\implies &x &= \frac{c}{x} \\
    &\implies &x^2 &= c \\
    &\implies &x &= \sqrt{c}
\end{alignat*}

The above calculation only makes sense if we know the sequence converges. Consider the sequence $x_{n+1} = 6 - x_n$ where $x_1 = 4$. Then:
\[ x_1 = 4,\quad x_2 = 2,\quad x_3 = 4,\quad x_4 = 2,\quad \ldots \]

% TODO: add reference for monotone convergence theorem
% TODO: explain what a bounded monotone sequence even is

Since this sequence does not converge, there is no limit when $n \to \infty$! In chapter 14, we will cover the Monotone Convergence Theorem, which states that any bounded monotone sequence converges.

\begin{exbox}{Monotone Convergence Theorem}{}
    Suppose that $c > 0$ and $x_1 > 0$. Then, for $n \geq 2$, the sequence $x_{n+1} = \frac{1}{2} \left( x-N + \frac{c}{x_n} \right)$ is:
    \begin{itemize}
        \item \textbf{bounded below} because $x_n > \sqrt{c}$ when $n \geq 2$, and
        \item \textbf{decreasing} because $x_n+1 < x_n$ for $n \geq 2$.
    \end{itemize}
    Therefore, $x_n$ converges by the Monotone Convergence Theorem, and $\lim_{n\to\infty} x_n = \sqrt{c}$.
\end{exbox}

Let's look at a more complicated iterative method.

\begin{exbox}{Picard's Method}{}
    Suppose we had to solve $y\prime = f(x,y)$ where $y(x_0) = y_0$ (i.e. find a function $y$ that satisfies our two conditions). As it turns out, we can use an iterated method to solve this as well.
    \begin{itemize}
        \item Start with an initial guess $y_1(x)$
        \item Define $\displaystyle y_{n+1}(x) \coloneq y_0 + \int_{x_0}^{x} f(t, y_n)\ dt$.
    \end{itemize}
    Provided that $f$ and $y_0$ are ``well-behaving'', then the sequence of functions $y_n(x)$ converges to the solution $y(x)$.
\end{exbox}

The idea that an infinite sequence of functions can converge suggests some notion of ``distance'' between functions. We can use a number of metrics for distance, some possibilities including:
\begin{itemize}
    \item \( \displaystyle \int_a^b \abs{f(x) - g(x)}\ dx \quad \) (total area between the two functions)
    \item \( \displaystyle \sup \left\{ x : x = \abs{f(x) - g(x)} \right\} \quad \) (max possible ``vertical'' distance between the two curves)
\end{itemize}


\chapter{Logic and Proofs}
% Logic is the backbone of all formal mathematics. When building a logically sound model of mathematics, we start with a small collection of axioms. We then work with those axioms to deduce other logically sound statements, reaffirming what we already knew and discovering new ideas along the way.

Formal logic is the foundation of mathematics. It enables us to construct logically consistent models by starting with a set of axioms and systematically deducing new statements from them. This process not only helps us to prove known results but also to uncover new mathematical concepts and theorems.

\section{Basic Logic}
\begin{dfnbox}{Statement}{}
    A \dfntxt{statement} is a claim that is either true or false.
    \tcblower
    \[ p : \text{some claim} \]
\end{dfnbox}

We usually denote statements with a letter like $p$. For example, we can write ``$p: x > 2$'', which means $p$ represents the statement ``$x$ is greater than $2$''. Throughout this chapter, we will use $p$ and $q$ to represent arbitrary statements.

\begin{dfnbox}{Conjunction}{conjunction}
    Logical \dfntxt{conjunction} is an operation that takes two statements and produces a new statement that is true only when both input statements are true.
    \tcblower
    \[ p \land q : p\ \text{is true \textbf{and}}\ q\ \text{is true} \]
\end{dfnbox}

\begin{dfnbox}{Disjunction}{disjunction}
    Logical \dfntxt{disjunction} is an operation that takes two statements and produces a new statement that is true when at least one of the input statements is true.    \tcblower
    \[ p \lor q : p\ \text{is true \textbf{or}}\ q\ \text{is true} \]
\end{dfnbox}

\nameref{dfn:conjunction} and \nameref{dfn:disjunction} follow our intuition of ``and'' and inclusive ``or'', respectively. We can visualize the two logical connectives using \dfntxt{truth tables}.

\begin{exbox}{Truth Table of Conjunction}{conjunction}
    \begin{center}\begin{tabular}{c | c || c}
        $p$ & $q$ & $p \implies q$ \\ \hline
        T & T & T \\
        T & F & F \\
        F & T & F \\
        F & F & F
    \end{tabular}\end{center}
\end{exbox}

\begin{exbox}{Truth Table of Disjunction}{disjunction}
    \begin{center}\begin{tabular}{c | c || c}
        $p$ & $q$ & $p \implies q$ \\ \hline
        T & T & T \\
        T & F & T \\
        F & T & T \\
        F & F & F
    \end{tabular}\end{center}
\end{exbox}


% TODO: add truth tables

\begin{dfnbox}{Negation}{}
    The \dfntxt{negation} of a statement is a statement with opposite truth values.
    \tcblower
    \[ \neg p \]
\end{dfnbox}

\begin{dfnbox}{Implication}{}
    An \dfntxt{implication} ``$p$ implies $q$'' states ``if $p$ is true, then $q$ is true''.
    \tcblower
    \[ p \implies q \]
\end{dfnbox}

In the implication $p \implies q$, we call $p$ the \dfntxt{hypothesis} and $q$ the \dfntxt{conclusion}. If the hypothesis is false to begin with, then the implication is not really meaningful. Instead of assigning those kinds of implications no truth value, we simply consider them true by convention. These kinds of truths are called \dfntxt{vacuous truths}.

\begin{exbox}{Truth Table of Implication}{}
    \begin{center}\begin{tabular}{c | c || c}
        $p$ & $q$ & $p \implies q$ \\ \hline
        T & T & T \\
        T & F & F \\
        F & T & T \\
        F & F & T
    \end{tabular}\end{center}
\end{exbox}

\begin{exbox}{Simple Statements}{}
    Let $p : x > 2$ and $q : x^2 > 1$. Consider the following statements:
    \begin{itemize}
        \item ``For all real numbers $x$, $p \implies q$''

        \textbf{True.} If $x > 2$, then $x^2 > 1$.\footnote{This is normally where we would rigorously prove such a statement, but we will omit this for now.}

        \item ``For all real numbers $x$, $q \implies p$''

        \textbf{False.} Consider $x = 1.1$. Then $x^2 = 1.21 > 1$, but $x = 1.1 < 2$.
    \end{itemize}
\end{exbox}

\begin{dfnbox}{Logical Equivalence}{equiv}
    $p$ and $q$ are \dfntxt{logically equivalent} if $p \implies q$ and $q \implies p$.
    \tcblower
    \[ p \iff q \]
\end{dfnbox}

In other words, $p \iff q$ means that $p$ and $q$ share the same truth value. Either $p$ and $q$ are \textbf{always both true}, or $p$ and $q$ are \textbf{always both false}. Logical equivalence says nothing about the truth of $p$ and $q$ themselves.

We can also say ``$p$ if and only if $q$'' or ``$p$ iff $q$'' to denote logical equivalence.

\begin{exbox}{Truth Table of Logical Equivalence}{}
    \begin{center}\begin{tabular}{c | c || c}
        $p$ & $q$ & $p \iff q$ \\ \hline
        T & T & T \\
        T & F & F \\
        F & T & F \\
        F & F & T
    \end{tabular}\end{center}
\end{exbox}

\begin{dfnbox}{Converse}{converse}
    Given the implication $p \implies q$, its \dfntxt{converse} statement is $q \implies p$.
\end{dfnbox}

It's important to note that an implication and its converse have no intrinsic equivalence.


\begin{exbox}{Truth Table of Converse}{}
    \begin{center}\begin{tabular}{c | c || c | c}
        $p$ & $q$ & $p \implies q$ & $q \implies p$ \\ \hline
        T & T & T & T \\
        T & F & F & T \\
        F & T & T & F \\
        F & F & T & T
    \end{tabular}\end{center}
\end{exbox}

\begin{dfnbox}{Contrapositive}{contrapositive}
    Given the implication $p \implies q$, its \dfntxt{contrapositive} statement is $\neg q \implies \neg p$.
\end{dfnbox}

Unlike the converse, an implication and its contrapositive are logically equivalent. To help our intuition, we can construct a truth table.

\begin{exbox}{Truth Table of Contrapositive}{}
    \begin{center}\begin{tabular}{c | c || c | c | c | c}
        $p$ & $q$ & $\neg p$ & $\neg q$ & $p \implies q$ & $\neg q \implies \neg p$ \\ \hline
        T & T & F & F & T & T \\
        T & F & F & T & F & F \\
        F & T & T & F & T & T \\
        F & F & T & T & T & T
    \end{tabular}\end{center}
\end{exbox}

As we can see, no matter what the truth values of the hypothesis and conclusion are, an implication and its contrapositive always have the same truth values.

\begin{notebox}
    When constructing a truth table, we must include \textbf{all} intermediate statements, not just the final statement.
\end{notebox}

% TODO: add truth table for contrapositive


\section{Proofs and Proof Techniques}
While truth tables are a useful tool for evaluating simple statements, they quickly become impractical when dealing with more complex propositions. Moreover, they do not offer insights into the reasoning behind such statements. In contrast, proofs can provide us with a deeper understanding of logical relationships and help us reason about complex statements.

In particular, we often need to prove implications of the form $p \implies q$, where the truth of $p$ guarantees the truth of $q$. To do so, we can use a variety of proof techniques:
\begin{enumerate}
    \item \dfntxt{Direct Proof:} Assume $p$ is true, then reason that $q$ must be true as well.
    \item \dfntxt{Proof by Contradiction:} Assume both $p$ and $\neg q$ are true, then logically derive some contradiction.
    \item \dfntxt{Proof by Contrapositive:} Assume $\neg q$ is true, then reason that $\neg p$ must be true as well.
\end{enumerate}

In mathematical proofs, there are two main types of reasoning: direct and indirect. A direct proof shows a clear path from the premises to the conclusion, providing valuable insights into the underlying mathematics. In contrast, indirect proofs rely on a contradictory hypothesis to establish the truth of the conclusion. While indirect proofs can be useful when a direct proof is not readily available, they may be less insightful since they do not provide much context surrounding the premises.

However, it is worth noting that an indirect proof may be easier to find than a direct proof in certain cases. While a direct proof requires identifying the correct path that leads to the conclusion, an indirect proof only needs to deduce any contradictory statement. Despite this advantage, indirect proofs should be used sparingly and only when a direct proof is not feasible.

% In a direct proof, the reasoning to get from $p$ to $q$ provides a lot of insight about the context of $p$ and the surrounding mathematics. Similarly, the direct reasoning in proof by contrapositive provides context surrounding $\neg q$. However, the reasoning done in a proof by contradiction is based on a contradictory hypothesis. Thus, it is often less insightful and is typically avoided when a direct proof is readily available.\footnote{\url{https://math.stackexchange.com/a/1688}}

% That being said, it is sometimes easier to find a proof by contradiction than a direct proof. Whereas direct proof needs to deduce the correct path that leads to the conclusion, a proof by contradiction only needs to deduce any contradictory statement.

%It's hard to decide which proof technique is easiest for any given problem. Direct proofs are often more ``enlightening'', but it can be difficult to find the appropriate logic to reach the conclusion. It may be easier to try proof by contradiction or contrapositive.

\begin{tecbox}{Proof by Contradiction}{contradiction}
    To prove $p \implies q$ by contradiction, we carry out the following steps:
    \begin{enumerate}
        \item Assume $p$ is true, and suppose for the sake of contradiction $\neg q$ is true.
        \item Logically derive a statement that contradicts something we know to be true.
        \item Ultimately conclude that $q$ must be true.
    \end{enumerate}
\end{tecbox}

In terms of logic notation, proof by contradiction follows:
\[ \left[ \left( p \land (\neg q) \right) \implies \text{Contradiction} \right] \implies \left[ p \implies q \right]\]

\begin{exbox}{Truth Table of \nameref{tec:contradiction}}{}
    \begin{center}\begin{tabular}{c | c || c | c | c | c }
        $p$ & $q$ & $p \implies q$ & $\neg q$ & $p \land (\neg q)$ & $\neg \left[ p \land (\neg q) \right]$ \\ \hline
        T & T & T & F & F & T \\
        T & F & F & T & T & F \\
        F & T & T & F & F & T \\
        F & F & T & T & F & T
    \end{tabular}\end{center}
\end{exbox}

By the above truth table, we can safely assume the following logical equivalence:
\[ (p \implies q) \iff \neg \left[ p \land (\neg q) \right] \]

\begin{tecbox}{Proof by Contrapositive}{}
    To prove $p \implies q$ by contrapositive, we carry out the following steps:
    \begin{enumerate}
        \item Assume $\neg q$ is true.
        \item Directly prove that $\neg p$ is true.
    \end{enumerate}
\end{tecbox}

In terms of logic notation, proof by contrapositive follows:
\[ (\neg q \implies \neg p) \iff (p \implies q) \]
We can actually prove this using proof by contradiction!
\begin{exbox}{Logical Equivalence of Contrapositive}{}
    Given statements $p$ and $q$, $p \implies q$ and $\neg q \implies \neg p$ are equivalent.
    \tcblower
    \begin{proof}
        Assume $p \implies q$. To prove $\neg q \implies \neg p$, we can suppose for contradiction that $\neg q$ and $p$ are both true. But $p \implies q$, so $q$ is true which contradicts $\neg q$. Hence, the assumption that $p$ is true was incorrect. Thus, $\neg q \implies \neg p$.

        Assume $\neg q \implies \neg p$. From above, we have $\neg ( \neg p ) \implies \neg (\neg q)$, so $p \implies q$.
    \end{proof}
\end{exbox}

\begin{exbox}{Proving Simple Logic Statements}{}
    Let $p$, $q$, and $r$ be arbitrary statements. Prove that $\left[ p \implies (q \lor r) \right] \iff \left[ (p \land \neg q) \implies r \right]$.
    \tcblower
    \begin{proof}
        Assume $p \implies (q \lor r)$. Suppose $p \land \neg q$. Then $p$ is true, so $q \lor r$ is true by assumption. Also, $\neg q$ is true, so $r$ must be true from $q \lor r$.

        Assume $(p \land \neg q) \implies r$. Suppose $p$ is true. There are two possibilities:
        \begin{enumerate}
            \item If $q$ is true, then $q \lor r$ is true.
            \item If $\neg q$ is true, then $p \land \neg q$ is true. Thus, $r$ is true by assumption. Hence, $q \lor r$ is true.
        \end{enumerate}
    \end{proof}
\end{exbox}

% When we prove an implication $p \implies q$ directly, we assume $p$, and then make some intermediate conclusions, before finally deducing $q$. Those intermediate conclusions provide insight about the context of $p$ and the surrounding mathematics. Similarly, when we prove the contrapositive, we assume $\neg q$, and make some valuable intermediate conclusions, before finally deducing $\neg p$.

% In contrast, a proof by contradiction carries little of this extra value. We make intermediate conclusions under a contradictory hypothesis where $p$ and $\neg q$ hold. Since those intermediate conclusions are built on false logic, they provide less insight about logically sound mathematics.


\chapter{Naive Set Theory}
Instead of forming a rigorous, axiomatic basis for sets, we will simply take an informal approach to sets guided by our intuition. Ultimately, our introduction to real analysis does not fiddle with the fine details of set theory, so it's safe to take a naive approach to set theory.

\section{Sets}

\begin{dfnbox}{Set}{set}
    A \dfntxt{set} is a collection of distinct objects.
\end{dfnbox}

For example, $\N \coloneq \{1,2,3\ldots\}$ is the set of all natural numbers, and $\Z \coloneq \{ \ldots, 1, 2, 3, \ldots\}$ is the set of all integers. It's conventional to use capital letters to denote sets and use lowercase letters to denote elements of sets. Throughout this
chapter, we will use $A$ and $B$ to represent arbitrary sets.

\begin{dfnbox}{Membership, $\in$}{}
    We write $a \in A$ to mean ``$a$ is in $A$''.
\end{dfnbox}

\begin{dfnbox}{Subset, $\subseteq$}{}
    $A$ is a \dfntxt{subset} of $B$ if everything in $A$ is also in $B$.
    \tcblower
    \[ A \subseteq B \iff \forall(x \in A)(x \in B) \]
\end{dfnbox}

\begin{dfnbox}{Set Equality, $=$}{}
    $A$ \dfntxt{equals} $B$ if $A$ is a subset of $B$ and $B$ is a subset of $A$.
    \tcblower
    \[ A = B \iff (A \subseteq B \land B \subseteq A) \]
\end{dfnbox}

\begin{dfnbox}{Proper Subset, $\subsetneq$}{}
    $A$ is a \dfntxt{proper subset} of $B$ if $A$ is a subset of $B$ but $B$ is not a subset of $A$.
    \tcblower
    \[ A \subsetneq B \iff (A \subseteq B \land B \not\subseteq A) \]
\end{dfnbox}

In other words, $A$ is a proper subset of $B$ if everything in $A$ is also in $B$, but $B$ has something that $A$ does not.

\begin{notebox}
    Among mathematical texts, the generic subset symbol $\subset$ has no standardized definition. Some use it to represent subset or equal; others use it to represent proper subset. We will simply not use $\subset$ to avoid any ambiguity.
\end{notebox}

\begin{dfnbox}{Empty Set ($\emptyset$)}{}
    The \dfntxt{empty set} is the set that contains no elements.
    \tcblower
    \[ \emptyset \coloneq \{ \} \]
\end{dfnbox}

As convention, we assume that $\emptyset$ is a subset of every set, including itself.

% A useful way of visualizing set relations is with venn diagrams.

\begin{tecbox}{Proving a Subset Relation}{}
    To prove that $A \subseteq B$:
    \begin{enumerate}
        \item Let $x$ be an arbitrary element of $A$.
        \item Show that $x \in B$.
    \end{enumerate}
    \tcblower
    To prove that $A \not\subseteq B$, choose a specific $x \in A$ and show $x \notin B$.
\end{tecbox}

\begin{exbox}{Proving Simple Subset Relation}{}
    Suppose that $A \subseteq B$ and $B \subseteq C$. Prove that $A \subseteq C$.
    \tcblower
    \begin{proof}
        Let $x \in A$ be arbitrary. Since $A \subseteq B$, then $x \in B$. Similarly, since $B \subseteq C$, then $x \in C$. Therefore, $A \subseteq C$.
    \end{proof}
\end{exbox}

\begin{dfnbox}{Union}{}
    The \dfntxt{union} of two sets is the set of all things that are in one or the other set.
    \tcblower
    \[ A \cup B \coloneq \left\{ x : x \in A \lor x \in B \right\} \]
\end{dfnbox}

\begin{dfnbox}{Intersection}{}
    The \dfntxt{intersection} of two sets is the set of all things that are in both sets.
    \tcblower
    \[ A \cap B \coloneq \left\{ x : x \in A \land x \in B \right\} \]
\end{dfnbox}

More generally, we can apply union and intersection to an arbitrary number of sets, finite or infinite. We use a notation similar to summation using $\sum$. Let $\Lambda$ be an indexing set, and for each $\lambda \in \Lambda$, let $A_\lambda$ be a set.
\begin{align*}
    \bigcup_{\lambda \in \Lambda} A_\lambda &= \left\{ x : x \in A_\lambda\ \text{for some}\ \lambda \in \Lambda \right\} \\
    \bigcap_{\lambda \in \Lambda} A_\lambda &= \left\{ x : x \in A_\lambda\ \text{for all}\ \lambda \in \Lambda \right\}
\end{align*}
\begin{exbox}{Indexed Sets}{}
    For $n \in \N$, let $A_n = \left[ \frac{1}{n}, 1 \right] = \left\{ x \in \R : \frac{1}{n} \leq x \leq 1 \right\}$. Prove that:
    \begin{enumerate}[label=(\alph*)]
        \item $\bigcup_{n=1}^\infty = (0,1]$
        \item $\bigcap_{n=1}^\infty = \{1\}$
    \end{enumerate}
    \tcblower
    \begin{proof}[Proof of (a)]
        Suppose $x \in \bigcup_{n=1}^\infty A_n$. Then there exists $n \in \N$ such that $x \in A_n = \left[ \frac{1}{n}, 1 \right]$. That is, $0 < \frac{1}{n} \leq x \leq 1$. Therefore, $x \in (0, 1]$.

        Suppose $x \in (0, 1]$. Then $x > 0$, so there exists $n_0 \in \N$ such that $\frac{1}{n_0} < x$. Then $\frac{1}{n_0} \leq x \leq 1$, so $x \in A_{n_0}$. Therefore, $x \in \bigcup_{n=1}^\infty A_n$.
    \end{proof}

    \begin{proof}[Proof of (b)]
        Suppose $x \in \bigcap_{n=1}^\infty A_n$. Then $x \in A_1 = \{1\}$.

        Suppose $x \in \{1\}$. Then $x = 1 \in \left[ \frac{1}{n}, 1 \right]$ for all $n \in \N$. Therefore, $x \in \bigcap_{n=1}^\infty A_n$.
    \end{proof}
\end{exbox}

\begin{dfnbox}{Set Minus}{}
    The \dfntxt{set difference} of two sets is the set of all things that are in first set but not the second set.
    \tcblower
    \[ A \setminus B = \{ x \in A : x \notin B \} \]
\end{dfnbox}

\begin{dfnbox}{Complement}{}
    Let $X$ be a set called the \dfntxt{universal set}. The \dfntxt{complement} of $A$ in $X$ is defined as $X \setminus A$.
    \tcblower
    \[ A^c = X \setminus A = \{ x \in X : x \notin A \} \]
\end{dfnbox}

\begin{thmbox}{De Morgan's Laws for Sets}{}
    Suppose $X$ is a set, and for any subset $S$ of $X$, let $S^c = X \setminus S$. Suppose that $A_\lambda \subseteq X$ for every $\lambda$ belonging to some index set $\Lambda$. Prove that:
    \begin{enumerate}[label=(\alph*)]
        \item \( \left( \bigcup_{\lambda \in \Lambda} A_\lambda \right)^c = \bigcap_{\lambda \in \Lambda} A_\lambda^c \);
        \item \( \left( \bigcap_{\lambda \in \Lambda} A_\lambda \right)^c = \bigcup_{\lambda \in \Lambda}A_\lambda^c \).
    \end{enumerate}
    \tcblower
    \begin{proof}[Proof of (a)]
        First, let $a \in \left( \bigcup_{\lambda \in \Lambda} A_\lambda \right)^c$. Then, $a \in X \setminus \left( \bigcup_{\lambda \in \Lambda} A_\lambda \right)$, so $a \in X$ but $a \notin \left( \bigcup_{\lambda \in \Lambda} A_\lambda \right)$. Thus, $a \notin A_\lambda$ for any $\lambda \in \Lambda$, so $a \in X \setminus A_\lambda$ for all $\lambda \in \Lambda$. In other words, $a \in \bigcap_{\lambda \in \Lambda} A_\lambda^c$.

        Next, let $a \in \bigcap_{\lambda \in \Lambda} A_\lambda^c$. Then $a \in A_\lambda^c$ for all $\lambda \in \Lambda$, so $a \in X$ but $a \notin A_\lambda$ for all $\lambda \in \Lambda$. That is, $a \notin \left( \bigcup_{\lambda\in\Lambda} A_\lambda \right)$. In other words, $a \in \left( \bigcup_{\lambda\in\Lambda} A_\lambda \right) ^ c$.
    \end{proof}

    \begin{proof}[Proof of (b)]
        First, let $a \in \left( \bigcap_{\lambda \in \Lambda} A_\lambda \right)^c$. Then, $a \in X \setminus  \bigcap_{\lambda \in \Lambda} A_\lambda$, so $a \in X$ but $a \notin \bigcap_{\lambda \in \Lambda} A_\lambda$. That is, $a \notin A_\lambda$ for some $\lambda \in \Lambda$. Thus, $a \in X \setminus A_\lambda$ for some $\lambda \in \Lambda$. Therefore, $a \in \bigcup_{\lambda \in \Lambda} A_\lambda^c$.

        Next, let $a \in \bigcup_{\lambda \in \Lambda} A_\lambda^c$. Then $a \in A_\lambda^c$ for some $\lambda \in \Lambda$, so $a \in X$ but $a \notin A_\lambda$ for some $\lambda \in \Lambda$. That is, $a \notin \left( \bigcap_{\lambda \in \Lambda} A_\lambda \right)$. Therefore, $a \in  \left( \bigcap_{\lambda \in \Lambda} A_\lambda \right)^c$.
    \end{proof}
\end{thmbox}

\section{Functions}
We generally think of functions as a ``map'' or ``rule'' that assigns numbers to other numbers. For example, $f(x) = 2x$ maps $1 \mapsto 2$, $2 \mapsto 4$, etc. More formally, we define functions in terms of sets.


\begin{dfnbox}{Cartesian Product}{}
    Let $X$ and $Y$ be sets. The \dfntxt{Cartesian product} of $X$ and $Y$ is the set of all ordered pairs $(x,y)$ where $x \in X$ and $y \in Y$.
    \tcblower
    \[ X \times Y \coloneq \left\{ (x,y) : x \in X \land y \in Y \right\} \]
\end{dfnbox}

\begin{dfnbox}{Relation}{relation}
    Let $X$ and $Y$ be sets. A \dfntxt{relation} between $X$ and $Y$ is a subset of the Cartesian product $X \times Y$.
\end{dfnbox}

\begin{dfnbox}{Function}{function}
    Let $X$ and $Y$ be sets. A \dfntxt{function} from $X$ to $Y$ is a relation from $X$ to $Y$ such that for every $x \in X$, there exists a unique $y \in Y$ where $(x,y) \in f$.
    \tcblower
    More formally, a \dfntxt{function} $f : X \to Y$ is a subset of $X \times Y$ satisfying:
    \begin{enumerate}[noitemsep]
        \item $\forall (x \in X) \left[ \exists (y \in Y)((x,y) \in f) \right]$
        \item $(x,y_1),(x,y_2) \in f \implies y_1 = y_2$
    \end{enumerate}
\end{dfnbox}


Given $f : X \to Y$, we call $X$ the \dfntxt{domain} of $f$ and $Y$ the \dfntxt{codomain} of $f$. Given $x \in X$, we write $f(x)$ to denote the unique element of $Y$ such that $(x,y) \in f$.
\[ f(x) = y \iff (x,y) \in f \]

\begin{dfnbox}{Function Image}{image}
    Let $f : X \to Y$ be a function and $A \subseteq X$. The \dfntxt{image} of $A$ under $f$ is the set containing all possible function outputs from all inputs in $A$.
    \tcblower
    \[ f[A] \coloneq \{ f(a) : a \in A \} \]
\end{dfnbox}

Given $f : X \to Y$, we call $f[X]$ the \dfntxt{range} of $f$.

\begin{exbox}{Function Images}{}
    Suppose $f : X \to Y$ is a function, and $A_\lambda \subseteq X$ for each $\lambda \in \Lambda$. Then:
    \begin{enumerate}[label=(\alph*)]
        \item $f \left[ \bigcup_{\lambda \in \Lambda} A_\lambda \right] = \bigcup_{\lambda \in \Lambda} f \left[ A_\lambda \right]$
        \item $f \left[ \bigcap_{\lambda \in \Lambda} A_\lambda \right] \subseteq \bigcap_{\lambda \in \Lambda} f \left[ A_\lambda \right]$
    \end{enumerate}
    \tcblower
    In this example, we will only prove the ``forward'' direction. That is, we want to show that $f \left[ \bigcup_{\lambda \in \Lambda} A_\lambda \right] \subseteq \bigcup_{\lambda \in \Lambda} f \left[ A_\lambda \right]$.
    \begin{proof}[Proof of (a)]
        Let $y \in f \left[ \bigcup_{\lambda \in \Lambda} A_\lambda \right]$. By definition of \nameref{dfn:image}, there exists $x \in \bigcup_{\lambda \in \Lambda} A_\lambda$ such that $y = f(x)$. Thus, there exists $\lambda_0 \in \Lambda$ such that $x \in \lambda_0$. That is, $y \in f \left[ A_{\lambda_0} \right]$. Therefore, $y \in \bigcup_{\lambda \in \Lambda} f \left[ A_\lambda \right]$.
    \end{proof}
\end{exbox}

\begin{dfnbox}{Function Inverse Image}{inverse-image}
    Let $f : X \to Y$ be a function and $B \subseteq Y$. The \dfntxt{inverse image} of $B$ under $f$ is the set containing all possible function inputs whose output is in $B$.
    \tcblower
    \[ f^{-1}[B] \coloneq \{ x \in X: f(x) \in B \} \]
\end{dfnbox}

Note the following logical equivalence:
\[ x \in f^{-1} [B] \iff f(x) \in B \]

\begin{exbox}{Function Inverse Images}{}
    Suppose $f : X \to Y$ is a function, and $B_\lambda \subseteq Y$ for each $\lambda \in \Lambda$. Then:
    \[ f^{-1} \left[ \bigcup_{\lambda \in \Lambda} B_\lambda \right] = \bigcup_{\lambda \in \Lambda} f^{-1} \left[ B_\lambda \right] \]
    \tcblower
    Again, we will only prove the ``forward direction''.
    \begin{proof}
        Let $x \in f^{-1} \left[ \bigcup_{\lambda \in \Lambda} B_\lambda \right]$. Then, $f(x) \in \bigcup_{\lambda \in \Lambda} B_\lambda$. That is, $f(x) \in B_{\lambda_0}$ for some $\lambda_0 \in \Lambda$. Thus, $x \in f^{-1} \left[ B_{\lambda_0} \right]$, so $x \in \bigcup_{\lambda \in \Lambda} f^{-1} \left[ B_\lambda \right]$.
    \end{proof}
\end{exbox}

\section{Injective and Surjective}

\begin{dfnbox}{Injective, One-to-one}{}
    A function $f : X \to Y$ is \dfntxt{injective} or \dfntxt{one-to-one} if no two inputs in $X$ have the same output in $Y$.
    \tcblower
    \[ \forall (x_1, x_2 \in X) \left[ x_1 \neq x_2 \implies f\left(x_1\right) \neq f\left(x_2\right) \right] \]
\end{dfnbox}

\begin{tecbox}{Proving a Function is Injective}{}
    To prove a function $f : X \to Y$ is injective:
    \begin{enumerate}
        \item Let $x_1, x_2 \in X$ where $f(x_1) = f(x_2)$.
        \item Reason that $x_1 = x_2$.
    \end{enumerate}
\end{tecbox}

\begin{exbox}{Proving Injectivity}{}
    $f(x) = -3x-7$ is injective.
    \tcblower
    \begin{proof}
        Suppose $f(x_1) = f(x_2)$. Then $-3x_1+7 = -3x_2+7$, so $-3x_1 = -3x_2$. Thus, $x_1 = x_2$, so $f$ is injective.
    \end{proof}
\end{exbox}

\begin{exbox}{Disproving Injectivity}{}
    Prove that $f(x)=x^2$ is not injective.
    \tcblower
    \begin{proof}
        $f(-1) = 1$ and $f(1) = 1$, but $-1 \neq 1$. Thus, $f$ is not injective.
    \end{proof}
\end{exbox}

\begin{dfnbox}{Surjective, Onto}{}
    A function $f : X \to Y$ is \dfntxt{surjective} or \dfntxt{onto} if everything in $Y$ has a corresponding input in $X$.
    \tcblower
    \[ \forall (y \in Y) \left[ \exists (x \in X) (f(x) = y) \right] \]
\end{dfnbox}

Note that $f : X \to f[X]$ is \textbf{always} surjective.

\begin{tecbox}{Proving a Function is Surjective}{}
    To prove a function $f : X \to Y$ is surjective:
    \begin{enumerate}
        \item Let $y \in Y$ be arbitrary.
        \item ``Undo'' the function $f$ to obtain $x \in X$ where $f(x)=y$.
    \end{enumerate}
\end{tecbox}

\begin{exbox}{Proving Surjectivity}{}
    Prove that $f : \R \to \R$ defined by $f(x) = -3x+7$ is surjective.
    \tcblower
    \begin{proof}
        Let $y \in Y$ be arbitrary. Let $x \coloneq \frac{y-7}{-3}$. Then $x \in \R$, and:
        \begin{align*}
            f(x)
            &= -3 \left( \frac{y-7}{-3} \right) + 7 \\
            &= (y-7) + 7 \\
            &= y
        \end{align*}
        Therefore, $f$ is surjective.
    \end{proof}
\end{exbox}

\begin{dfnbox}{Bijective}{}
    A function $f : X \to Y$ is \dfntxt{bijective} if it is both injective and surjective.
\end{dfnbox}

\begin{dfnbox}{Function Composition}{}
    Let $f : X \to Y$ and $g : Y \to Z$ be functions. The \dfntxt{composition} of $f$ and $g$ is a function $g \circ f : X \to Z$ defined by:
    \[ (g \circ f) (x) \coloneq g(f(x)) \]
\end{dfnbox}

\begin{thmbox}{Composition Preserves Injectivity and Surjectivity}{}
    Suppose $f : X \to Y$ and $g : Y \to Z$ are functions.
    \begin{enumerate}[label=(\alph*)]
        \item If $f$ and $g$ are injective, then $g \circ f$ is injective.
        \item If $f$ and $g$ are surjective, then $g \circ f$ is surjective.
        \item If $f$ and $g$ are bijective, then $g \circ f$ is bijective.
    \end{enumerate}
    \tcblower
    \begin{proof}[Proof of (a)]
        Let $x_1, x_2 \in X$. Suppose that $(g \circ f)(x_1) = (g \circ f)(x_2)$. Then, $g(f(x_1)) = g(f(x_2))$. Because $g$ is injective, we have $f(x_1) = f(x_2)$. Because $f$ is injective, we have $x_1 = x_2$. Therefore, $g \circ f$ is injective.
    \end{proof}

    \begin{proof}[Proof of (b)]
        Let $z \in Z$. Because $g$ is surjective, there exists an element $y \in Y$ such that $g(y) = z$. Because $f$ is surjective, there exists an element $x \in X$ such that $f(x) = y$. Thus, $(g \circ f)(x) = g(f(x)) = g(y) = z$. Therefore, $g \circ f$ is surjective.
    \end{proof}

    \begin{proof}[Proof of (c)]
        We know that from (a) and (b) composition preserves injectivity and surjectivity. Thus, composition must also preserve bijectivity.
    \end{proof}
\end{thmbox}

\begin{dfnbox}{Inverse Function}{}
    Let $f : X \to Y$ be a bijection. The \dfntxt{inverse function} of $f$ is a function $f^{-1} : Y \to X$ defined by:
    \[ f^{-1} \coloneq \{ (y,x) \in Y \times X : (x,y) \in f \} \]
\end{dfnbox}

The notation for inverse functions conflicts with the notation for inverse images. A key distinction to make it that only bijections can have an inverse function, but we can apply the inverse image to any function. Thus, given a bijection $f : X \to Y$, we know $f^{-1}(f(x)) = x$ for all $x \in X$, and $f(f^{-1}(y)) = y$ for all $y \in Y$.

\begin{exbox}{}{}
    Let $f : X \to Y$ and $g : Y \to X$ be functions such that $(g \circ f) = x$ for all $x \in X$, and $(f \circ g)(y) = y$ for all $y \in Y$. $f^{-1} = g$.
    \tcblower
    \begin{proof}
        todo: finish proof
    \end{proof}
\end{exbox}


\chapter{Number Systems}
\chapter{Number Systems}
Our goal is to create an axiomatic basis for the real numbers $\R$. We need to establish axioms for $\R$ and then derive all further properties from the axioms. We would like these axioms to be as minimal and agreeable as possible; however, finding axioms that characterize $\R$ is not easy. Instead, we'll start from the natural numbers $\N$ and expand from there.

\section{Natural Numbers $\N$ and Induction}
How do we define the natural numbers? Listing every natural number is definitely not an option. We could try to define the natural numbers as $\N \coloneq \{ 1, 2, \ldots \}$. However, the ``$\ldots$'' is ambiguous. Instead, we can define $\N$ in terms of its properties.

\begin{dfnbox}{Peano Axioms for $\N$}{}
    The \dfntxt{Peano axioms} are five axioms that can be used to define the natural numbers $\N$.
    \begin{enumerate}[noitemsep]
        \item $1 \in \N$
        \item Every $n \in \N$ has a successor called $n+1$.
        \item $1$ is \textbf{not} the successor of any $n \in \N$.
        \item If $n,m \in \N$ have the same successor, then $n = m$.
        \item If $1 \in S$ and every $n \in S$ has a successor, then $\N \subseteq S$.
    \end{enumerate}
\end{dfnbox}

\begin{notebox}
    Note that there is not one ``prescribed'' way to do define the natural numbers. This is just the most popular approach.
\end{notebox}

From the fifth Peano axiom, we can derive a new proof technique for proving statements about consecutive natural numbers.

\begin{thmbox}{Principle of Induction (by the Peano Axioms)}{induction}
    Let $P(n)$ be a statement for each $n \in \N$. Suppose that:
    \begin{enumerate}[noitemsep]
        \item $P(1)$ is true, and
        \item if $P(n)$ is true, then $P(n+1)$ is true.
    \end{enumerate}
    Then $P(n)$ is true for all $n \in \N$.
    \tcblower
    \begin{proof}
        Let $S \coloneq \{ n \in \N : P(n) \}$. Then $1 \in S$ because $P(1)$ is true. Note that if $n \in S$, then $P(n)$ is true. Hence, $P(n+1)$ is true by assumption, so $n+1 \in S$. By the fifth Peano axiom, we have $\N \subseteq S$. Since $S$ was defined as a subset of $\N$, we have $\N = S$. Therefore, $P(n)$ is true for all $n \in \N$.
    \end{proof}
\end{thmbox}

A proof by induction has a ``domino effect''. Imagine a domino for each natural number $1$, $2$, $3$, and so on, arranged in an infinite row. Knocking the 1st domino will knock them all down.

%We set up the dominoes by proving $P(n) \implies P(n+1)$ and knock over the first domino by proving $P(1)$. The result is that all the dominoes will topple each other, leaving no domino standing.

\[ \underbracket{P(1)}_{\text{by 1.}} \implies \underbracket{P(2)}_{\text{by 2.}} \implies \underbracket{P(3)}_{\text{by 2.}} \implies \cdots \]

\begin{tecbox}{Proof by Induction}{induction}
    To prove a statement $P(n)$ for all $n \in \N$, we need to prove two statements:
    \begin{enumerate}
        \item \dfntxt{Base Case:} Prove $P(1)$.
        \item \dfntxt{Induction Step:} Assume $P(n)$ is true from some $n \in \N$, then prove $P(n) \implies P(n+1)$.
    \end{enumerate}
\end{tecbox}

It is crucial that we actually use our assumption that $P(n)$ is true in the induction step. Otherwise, our proof is most likely wrong.

\begin{exbox}{Simple Proof by Induction}{}
    Prove that $1+2+\cdots + n = \frac{n(n+1)}{2}$ for all $n \in \N$.
    \tcblower
    \begin{proof}
        Let $P(n)$ be the statement $1 + 2 + \cdots + n = \frac{n(n+1)}{2}$.

        \textbf{Base Case:} When $n=1$, $\text{LHS} = 1$ and $\text{RHS} = \frac{1(1+1)}{2} = 1$, so $P(1)$ is true.

        \textbf{Induction Step:} Assume that $P(n)$ is true for some $n \in \N$. Then:
        \begin{align*}
            1 + 2 + \cdots + n + (n+1)
            &= \frac{n(n+1)}{2} + (n+1) \\
            &= (n+1) \left( \frac{n}{2} + 1 \right) \\
            &= \frac{(n+1)(n+2)}{2}
        \end{align*}
        That is, $P(n+1)$ is true. By the \nameref{thm:induction}, $P(n)$ is true for all $n \in \N$.
    \end{proof}
\end{exbox}

\section{Integers $\Z$}
From the natural numbers, we can easily construct the integers. First, we assume the existence an operation, addition ($+$) and multiplication ($\cdot$). On $\N$, we assume addition and multiplication satisfy the following properties for all $a,b,c \in \N$:

\begin{center}\begin{tabular}{l l l}
    \tabitem\dfntxt{Commutativity} & $a+b = b+a$ & $a \cdot b = b \cdot a$ \\
    \tabitem\dfntxt{Associativity} & $(a+b)+c = a+(b+c)$ & $(a \cdot b) \cdot c = a \cdot (b \cdot c)$ \\
    \tabitem\dfntxt{Distributivity} & $a \cdot (b+c) = a \cdot b + a \cdot c$ \\
    \tabitem\dfntxt{Identity} & $1 \cdot n = n$
\end{tabular}\end{center}
\todo{Sort out this wonky formatting}

We can expand this number system by including:
\begin{enumerate}
    \item an \dfntxt{additive identity} ($n+0 = n$ for all $n \in \N$)
    \item \dfntxt{additive inverses} (for all $n \in \N$, add $-n$ so $-n + n = 0$)
\end{enumerate}
From this, we can construct the set of integers.

\begin{dfnbox}{Integers $\Z$}{}
    The set of \dfntxt{integers} is defined as:
    \[ \Z \coloneq \N \cup \{0\} \cup \{ -n : n \in \N \} \]
\end{dfnbox}

\begin{dfnbox}{Even, Odd, Parity}{}
    Let $a \in \Z$.
    \begin{itemize}[noitemsep]
        \item $a$ is \dfntxt{even} if there exists $k \in \Z$ where $a = 2k$.
        \item $a$ is \dfntxt{odd} if there exists $k \in \Z$ where $a = 2k+1$.
        \item \dfntxt{Parity} describes whether an integer is even or odd.
    \end{itemize}
\end{dfnbox}

\begin{thmbox}{Parity Exclusivity}{}
    Every integer is either even or odd, never both.
    \tcblower
    \todo[inline]{TODO: prove this}
\end{thmbox}

\begin{exbox}{Parity of Square}{square-parity}
    For $n \in \Z$, if $n^2$ is even, then $n$ is even.
    \tcblower
    \begin{proof}
        We proceed by contraposition. Suppose $n$ is not even. Then $n$ is odd, and thus can be expressed as $n = 2k+1$ for some $k \in \Z$. Then:
        \begin{align*}
            n^2 &= (2k+1)(2k+1) \\
            &= 4k^2 + 4k + 1
        \end{align*}
        Since the integers are closed under addition and multiplication, then $4k^2 + 4k \in \Z$. Thus, $n^2$ is odd.
    \end{proof}
\end{exbox}

\section{Rational Numbers $\Q$}

We can further expand this number system by the following:
\begin{enumerate}
    \item Include \dfntxt{multiplicative inverses} (for all $n \in \Z \setminus \{0\}$, define $\sfrac{1}{n}$ such that $n \cdot \sfrac{1}{n} = 1$)
    \item Define $m \cdot \sfrac{1}{n} \coloneq \sfrac{m}{n}$ when $n \neq 0$.
\end{enumerate}

From this, we can construct the set of rational numbers.

\begin{dfnbox}{Rational Numbers $\Q$}{}
    The set of \dfntxt{rational numbers} is defined as:
    \[ \Q \coloneq \left\{ \frac{m}{n} : m,n \in Z \land n \neq 0 \right\} \]
\end{dfnbox}

To ensure multiplication works as intended, we also define $\frac{m}{n} \cdot \frac{k}{l} \coloneq \frac{m \cdot k}{n \cdot l}$.

We say $\frac{m_1}{n_1} = \frac{m_2}{n_2}$ if and only if $m_1n_1 = m_2n_2$ where $n_1, n_2 \neq 0$. In other words, $\frac{m_1}{n_1} \sim \frac{m_2}{n_2} \iff m_1n_2 = m_2n_1$. Thus, $\Q$ is the set of equivalence classes for this relation.

If $n = kp$ and $m = kq$, where $k,p,q \in \Z$, $k \neq 0$, $q \neq 0$, then:
\[ \frac{n}{m} = \frac{kp}{kq} = \frac{k}{p}, \quad \text{because}\ kpq = kqp \]
If $n$ and $m$ have no common factor (except $\pm 1$), then we say that $\sfrac{n}{m} \in \Q$ is in the ``lowest terms'' or ``reduced terms''. The set $(Q, +, \cdot)$ forms a field. However, we cannot write $x = \sfrac{n}{m}$ where $x^2 = 2$. \todo{Fix this whole section up. Very confusing.}

\begin{thmbox}{$\sqrt{2}$ is not a Rational Number}{root-2-irrational}
    $\sqrt{2} \notin \Q$
    \tcblower
    \begin{proof}
        Suppose for contradiction $\sqrt{2}$ is a rational number. Then, there exist $n,m \in \Z$ such that $(\sfrac{n}{m})^2 = 2$. If $n = kp$ and $m = kq$, then we can ``cancel'' the common factor $k$ to write $\sfrac{n}{m} = \sfrac{p}{q}$. That is, we can assume that $n$ and $m$ have no (non-trivial) common factors.
        % The above explanation feels kind of insubstantial.
        Now, $\sfrac{n^2}{m^2} = 2$, so by multiplying both sides by $m^2$, we get $n^2 = 2m^2$. Thus, $n^2$ is an even number, so $n$ is also even (Example \ref{ex:square-parity}). Then, we can write $n = 2k$ where $k \in \Z$. Then:
            \begin{alignat*}{2}
                & \implies \quad & (2k)^2 &= 2m^2 \\
                & \implies \quad & 4k^2 &= 2m^2 \\
                & \implies \quad & 2k^2 &= m^2
            \end{alignat*}
            Then $m^2$ is even, so $m$ is even. Thus, $m$ and $n$ are both even, so they are multiples of $2$. This contradicts the fact that we defined $\sfrac{n}{m}$ in the lowest terms.
    \end{proof}
\end{thmbox}

Does there exist $r \in \Q$ such that $r^2 = 3$?

\begin{dfnbox}{Divides}{}
    For $a,b, \in \Z$, we say $a$ \dfntxt{divides} $b$ if $b$ is a multiple of $a$.
    \tcblower
    \[ a \mid b \iff \exists(c \in \Z)(b=ac) \]
\end{dfnbox}

\begin{thmbox}{Division Algorithm}{}
    Suppose $a,b \in \Z$. Then $a = kb + r$ where $k \in \Z$ and $r \in \Z$ where $0 \leq r < a$.
\end{thmbox}
\todo{Need proof here}

\begin{exbox}{}{}
    If $p \in \N$ and $3 \mid p^2$, then $3 \mid p$.
    \tcblower
    \begin{proof}
        By the division algorithm, $p = 3k+j$ where $k \in \Z$ and $j \in \Z$ where $0 \leq j < 3$. Then, $p^2 = (3k+j)^2 = 9k^2 + 6kj + j^2$. Suppose that $3 \mid p^2$. Then, $p^2 = 3l = 9k^2 + 6kj + j^2$. Thus:
        \[ j^2 = 3l-9k^2-6kj = 3(l-3k^2-2kj) \]
        We have $3 \mid j^2$. Hence, $j \neq 1, j \neq 2$, leaving only $j = 0$. Therefore, $p = 3k + 0$, so $3 \mid p$.
    \end{proof}
\end{exbox}

\begin{exbox}{$\sqrt{3}$ is not a Rational Number}{}
    \begin{proof}
        Suppose for contradiction $\sqrt{3}$ is a rational number. Then, there exist $n, m \in \Z$ such that $(\sfrac{n}{m})^2$. If $n$ and $m$ share a common factor, then we can ``cancel'' the common factor to where $\sfrac{n}{m} = \sfrac{kp}{kq} = \sfrac{p}{q}$. Thus, we may assume that $n$ and $m$ have no nontrivial common factor.
        \begin{alignat*}{2}
            && \left(\frac{n}{m}\right)^2 &= 3 \\
            & \implies \quad & \frac{n^2}{m^2} &= 3 \\
            & \implies \quad & n^2 &= 3m^2
        \end{alignat*}
        Thus, $3 \mid n^2$, so $3 \mid n$ by the previous lemma. Writing $n = 3k$ for some $k \in \Z$, we have:
        \begin{alignat*}{2}
            && (3k)^2 &= 3m^2 \\
            & \implies \quad & 9k^2 &= 3m^2 \\
            & \implies \quad & 3k^2 &= m^2
        \end{alignat*}
        That is, $3 \mid m^2$ so $3 \mid m$. Thus, $3$ divides both $n$ and $m$. This contradicts the fact that we defined $\sfrac{n}{m}$ in the lowest terms.
    \end{proof}
\end{exbox}

\section{Fields}

\begin{dfnbox}{Field}{field}
    A \dfntxt{field} is a set $F$ with two defined operations, addition and multiplication, satisfying the following for all $a,b,c \in F$:

    \begin{center}\begin{tabular}{l l l}
		Axiom & \text{Addition} & \text{Multiplication} \\ \hline
		\dfntxt{Associativity} & $(a+b)+c = a+(b+c)$ & $(ab)c = a(bc)$ \\
		\dfntxt{Commutativity} & $a+b = b+a$ & $ab=ba$ \\
		\dfntxt{Distributivity} & $a(b+c) = ab+ac$ & $(a+b)c = ac + bc$ \\
		\dfntxt{Identities} & $\exists(0 \in \F)(a+0 = a)$ & $\exists(1 \in \F)(1 \neq 0 \land 1 a  = a)$ \\
		\dfntxt{Inverses} & $\exists(-a \in \F)(a + (-a) = 0)$ & $(a \neq 0) \iff \exists(a^{-1} \in \F)(a a^{-1} = 1)$
	\end{tabular}\end{center}

\end{dfnbox}

All the ``standard facts'' of arithmetic and algebra in $\R$ follows from these axioms.

$\Q$, $\R$, and $\C$ are infinite fields, but $\Z_p$ (arithmetic modulo $p$) is a finite field if $p$ is prime.

More generally, $F_q$ where $q = p^k$ is a finite field.

\begin{thmbox}{Facts about Fields}{}
    Let $F$ be a field. For all $a,b,c \in F$:
    \begin{enumerate}[noitemsep,label=(\alph*)]
        \item if $a+c = b+c$, then $a = b$
        \item $a \cdot 0 = 0$
        \item $(-a) \cdot b = -(a \cdot b)$
        \item $(-a) \cdot (-b) = a \cdot b$
        \item if $a \cdot c = b \cdot c$ and $c \neq 0$, then $a = b$
        \item if $a \cdot b = 0$, then $a = 0$ or $b = 0$
        \item $-(-a) = a$
        \item $-0 = 0$
    \end{enumerate}
    \tcblower
    \begin{proof}[Proof of (g)]
        \begin{align*}
            -(-a)
            &= -(-a) + 0 \\
            &= -(-a) + (a + (-a)) \\
            &= -(-a) + (-a + a) \\
            &= \left(-(-a) + (-a) \right) + a \\
            &= \left( (-a) + -(-a) \right) + a \\
            &= 0 + a \\
            &= a + 0 \\
            &= a
        \end{align*}
    \end{proof}
\end{thmbox}

\section{Ordered Fields}

\begin{dfnbox}{Ordered Field}{}
    An \dfntxt{ordered field} is a field with a relation $<$ such that for all $a,b,c \in F$:
    \begin{center}\begin{tabular}{l l}
        Axiom & Description \\ \hline
        \dfntxt{Trichotomy} &  Only one is true: $a<b$, $a=b$, or $b<a$ \\
        \dfntxt{Transitivity} & if $a<b$ and $b<c$ then $a<c$ \\
        \dfntxt{Additive Property} & if $b < c$, then $a+b < a+c$ \\
        \dfntxt{Multiplicative Property} & if $b<c$ and $0<a$, then $a \cdot b < a \cdot c$
    \end{tabular}\end{center}
\end{dfnbox}

We then define $>$ as the inverse relation of $<$.

\begin{thmbox}{Facts about Ordered Fields}{}
    \begin{itemize}[noitemsep]
        \item if $a < b$ then $-b < -a$
        \item if $a < b$ and $c < 0$, then $cb < ca$
        \item if $a \neq 0$, then $a^2 = a \cdot a > 0$
        \item $0 < 1$
        \item if $0<a<b$ then $0 < \sfrac{1}{b} < \sfrac{1}{a}$
    \end{itemize}
\end{thmbox}

Although $\C$ is a field, it is not an ordered field. We can certainly define some kind of ``order'' on $\C$, but there is no way to make it satisfy the four axioms of an ordered field. For example, $i^2 = -1 < 0$, contradicting the fact that any nonzero number's square is greater than $0$ in an ordered field.

$\R$ and $\Q$ are ordered fields.

\begin{dfnbox}{Absolute Value}{}
    Let $F$ be an ordered field. For $a \in F$, we define the \dfntxt{absolute value} of $a$ as:
    \[ \abs{a} \coloneq \begin{cases} a, & a \geq 0 \\ -a, & a < 0 \end{cases} \]
\end{dfnbox}

We can think of $\abs{a-b}$ as the distance between $a$ and $b$.
%More generally, $\abs{a-b} = d(a,b)$ is the metric we will be using throughout real analysis.
It is a common metric which we will use repeatedly throughout real analysis.

\begin{thmbox}{Properties of Absolute Value}{}
    \begin{itemize}
        \item $\abs{a} \geq 0$, $a \leq \abs{a}$, and $-a \leq \abs{a}$
        \item $\abs{ab} = \abs{a}\abs{b}$
    \end{itemize}
\end{thmbox}

\begin{thmbox}{Triangle Inequality}{triangle-inequality}
    Let $F$ be an ordered field. For any $a,b \in F$, $\abs{a+b} \leq \abs{a} + \abs{b}$.
    \tcblower
    \begin{proof}
        There are two cases to consider. If $a+b \geq 0$, then:
        \begin{align*}
            \abs{a+b}
            &= a+b \\
            &\leq \abs{a} + b \\
            &\leq \abs{a} + \abs{b}
        \end{align*}
        If $a+b < 0$, then:
        \begin{align*}
            \abs{a+b}
            &= -(a+b) \\
            &= -a-b \\
            &\leq \abs{a} - b \\
            &\leq \abs{a} + \abs{b}
        \end{align*}
    \end{proof}
\end{thmbox}

\section{Completeness and Suprema}

\begin{dfnbox}{Bounded Above, Bounded Below, Bounded}{}
    Let $F$ be an ordered field, and let $A \subseteq F$.
    \begin{itemize}[noitemsep]
        \item $A$ is \dfntxt{bounded above} if there exists $b \in F$ such that $a \leq b$ for all $a \in A$. In this context, $b$ is an \dfntxt{upper bound} for $A$.
        \item $A$ is \dfntxt{bounded below} if there exists $c \in F$ such that $c \leq a$ for all $a \in A$. In this context, $c$ is a \dfntxt{lower bound} for $A$.
        \item $A$ is \dfntxt{bounded} if $A$ is bounded above and bounded below.
    \end{itemize}
\end{dfnbox}

\begin{exbox}{Upper and Lower Bounds}{}
    Consider the set $ (0,1) \coloneq \{ x \in \R : 0<x<1\}$.
    \begin{itemize}[noitemsep]
        \item $(0,1)$ is bounded above by $1$ and any number greater than $1$.
        \item $(0,1)$ is bounded below by $0$ and any negative number.
    \end{itemize}
    Consider the set $[3, \infty) \coloneq \{ x \in \R : 3 \leq x \}$.
    \begin{itemize}[noitemsep]
        \item $[3, \infty)$ is not bounded above.
        \item $[3, \infty)$ is bounded below by $3$ and any number less than $3$.
    \end{itemize}
\end{exbox}

\begin{dfnbox}{Maximum, Minimum}{}
    Let $F$ be an ordered field, and let $A \subseteq F$.
    \begin{itemize}[noitemsep]
        \item If there exists $M \in A$ such that $M$ is an upper bound for $A$, then $M$ is the \dfntxt{maximum} of $A$, denoted $M = \max A$
        \item If there exists $m \in A$ such that $m$ is a lower bound for $A$, then $m$ is the \dfntxt{minimum} of $A$, denoted $m = \min A$.
    \end{itemize}
\end{dfnbox}

Note that from the above example, $(0,1)$ has neither a maximum nor a minimum. However, 3 is the minimum of $[3,\infty)$.

\begin{dfnbox}{Supremum}{}
    Let $F$ be an ordered field, and let $A \subseteq F$. $s \in F$ is a \dfntxt{supremum} of $A$ if:
    \begin{enumerate}
        \item $s$ is an upper bound for $A$, and
        \item if $t$ is an upper bound for $A$, then $s \leq t$.
    \end{enumerate}
\end{dfnbox}

In other words, the supremum is the least upper bound for $A$. If $A$ has a supremum, then that supremum is unique. \todo{Prove this}

\begin{thmbox}{Maximum is the Supremum}{}
    Let $F$ be an ordered field, and let $A \subseteq F$. If $A$ has a maximum $M$, then $M = \sup A$.
    \tcblower
    \begin{proof}
        Since $M = \max A$, we know $M$ is an upper bound for $A$. Let $t$ be an upper bound for $A$. Since $M \in A$, then $t \geq M$. Thus, $M$ is less than or equal to any upper bound $t$, so $M = \sup A$.
    \end{proof}
\end{thmbox}

\begin{exbox}{Supremum of $(0,1)$}{}
    Prove that $\sup (0,1) = 1$.
    \tcblower
    \begin{proof}
        First, note that $1$ is an upper bound for $(0,1)$. Next, suppose that $t \in \Q$ is an upper bound for $(0,1)$. Since $0 < \sfrac{1}{2} < 1$, then $0 < \sfrac{1}{2} \leq t$. By transitivity, $t > 0$. Suppose for contradiction $t < 1$. Because $0 < t < 1$, we have $1 < 1 + t < 2$. Dividing across by $2$, we have $\sfrac{1}{2} < \sfrac{1+t}{2} < 1$. That is, $\sfrac{1+t}{2} \in (0,1)$. But $t < 1$, so $2t < 1+t$. Thus, $t < \sfrac{1+t}{2}$. This contradicts our assumption that $t$ is an upper bound for $(0,1)$. Therefore, $t \geq 1$, so $\sup (0,1) = 1$.
    \end{proof}
\end{exbox}

\begin{dfnbox}{Completeness}{}
    An ordered field $F$ is \dfntxt{complete} if every nonempty subset of $F$ that is bounded above has a supremum in $F$.
\end{dfnbox}

\begin{thmbox}{$\Q$ is not complete}{}
    \begin{proof}[Proof sketch]
        Let $A \coloneq \left\{ x \in \Q : x^2 < 2 \right\}$. In other words, $A = \left(-\sqrt{2}, \sqrt{2}\right) \subseteq \Q$. Then $A$ is nonempty and bounded above. Suppose for contradiction that $\Q$ is complete. Then $A$ has a supremum, say $s = \sup(A)$. Consider the following cases:
        \begin{enumerate}
            \item If $s^2 < 2$, let $n \in \N$ such that $\left( s + \sfrac{1}{n} \right)^2 < 2$. Then $s + \sfrac{1}{n} \in A$, contradicting $s$ being an upper bound for $A$.
            \item If $s^2 > 2$, let $n \in \N$ such that $\left( s - \sfrac{1}{n} \right)^2 > 2$. Then $s - \sfrac{1}{n}$ is an upper bound smaller than $s$, contradicting $s$ being the least upper bound (supremum).
            \item If $s^2 = 2$, then $s \notin \Q$ (Theorem \ref{thm:root-2-irrational}).
        \end{enumerate}
        Thus, $A \subseteq \Q$ does not have a supremum. Therefore, $\Q$ is not complete.
    \end{proof}
\end{thmbox}

% TODO: elaborate on notation of mathematical structures

\begin{dfnbox}{Real Numbers $\R$}{}
    The \dfntxt{real numbers} are a set $\R$ with two operations, $+$ and $\cdot$, and order relation $<$ such that:
    \begin{enumerate}[noitemsep]
        \item $(R, +, \cdot)$ is a field,
        \item $(\R, +, \cdot, <)$ is an ordered field, and
        \item $(\R, +, \cdot, <)$ is complete.
    \end{enumerate}
\end{dfnbox}

Alternatively, $\R$ can be constructed explicitly using ``Dedekind cuts''. Either way, $\R$ is the \textbf{only} unique complete ordered field up to isomorphism. That is, if there is some other imposter complete ordered field $\R\prime$, we can map every element of $\R$ to $\R\prime$ such that we preserve all the operations and relations between things in $\R$. More formally, there exists an isomorphism $T : \R \to \R\prime$ where $T$ is bijective, and:
\begin{itemize}[noitemsep]
    \item $T(x+y) = T(x) + T(y)$
    \item $T(xy) = T(x)T(y)$
    \item $x < y \iff T(x) < T(y)$
\end{itemize}

Additionally, $\N \subseteq \R$ where $\N$ satisfies the Peano axioms.

\begin{thmbox}{$\sqrt{2}$ is a Real Number}{}
    \begin{proof}[Proof sketch]
        Let $A \coloneq \left\{ x \in \R : x^2 < 2 \right\}$.
        \begin{itemize}[noitemsep]
            \item Show $A \neq \emptyset$ and $A$ is bounded above
            \item Completeness says $s \coloneq \sup A$ exists
            \item Show $s^2 = 2 \implies s = \sqrt{2} \in \R$.
        \end{itemize}
        More generally, if $n,m \in \N$, then $\sqrt[n]{m} \in \R$.
    \end{proof}
\end{thmbox}

\section{Infima}
\begin{dfnbox}{Infimum}{}
    Let $F$ be an ordered field, and let $A \subseteq F$. $s$ is the \dfntxt{infimum} of $A$ if:
    \begin{enumerate}[noitemsep]
        \item $s$ is a lower bound for $A$, and
        \item $s$ is greater than every other lower bound for $A$.
    \end{enumerate}
\end{dfnbox}

We can prove that the existence of infima is already implied by completeness.

\begin{thmbox}{Existence of Infima in $\R$}{}
    Let $A \subseteq \R$ be nonempty and bounded below. Then $A$ has an infimum in $\R$.
    \tcblower
    \begin{proof}
        Let $A \subseteq \R$ be nonempty and bounded below. Let $B$ be the set of all lower bounds for $A$. In other words, $B \coloneq \left\{ b \in \R : \forall(a \in A)(b < a) \right\}$. Since $A$ is bounded below, then $B$ is nonempty. Note also that $B$ is bounded above by element of $A$. By completeness, $s \coloneq \sup B$ exists. Now, we need to show that $\sup B = \inf A$.
        \begin{enumerate}
            \item Every $a \in A$ is an upper bound for $B$, and $\sup B$ is the least upper bound for $B$. Then, $\sup B \leq a$. That is, $\sup B$ is a lower bound for $A$.
            \item Let $t$ be a lower bound for $A$. Then, by definition of $B$, it follows that $t \in B$. Then $t \leq \sup B$ as required.
        \end{enumerate}
        Therefore, $\sup B = \inf A$ in $\R$.
    \end{proof}
\end{thmbox}

\begin{thmbox}{Well-Ordering Principle}{wop}
    Every non-empty subset of $\N$ has a minimum.
    \tcblower
    \begin{proof}
        We will use induction. For convenience, let $P(n)$ represent the following statement: ``If $A \subseteq \N$ and $A \cap \{1,2,\ldots,n \} \neq \emptyset$, then $A$ has a minimum.''

        \textbf{Base Case:} First, we will prove $P(1)$. If $A \subseteq \N$ and $A \cap \{1\} \neq \emptyset$, then $1 \in A$, so $A$ has a minimum.

        \textbf{Induction Step:} Assume that $P(n)$ holds for some $n \in N$. Suppose $A \subseteq \N$ and $A \cap \{1,2,\ldots,n+1\} \neq \emptyset$.
        \begin{enumerate}
            \item If $A \cap \{ 1,2,\ldots,n \} \neq \emptyset$, then $A$ has a minimum by $P(n)$.
            \item If $A \cap \{1,2,\ldots,n\} = \emptyset$, then $n+1 \in A$, so $\min A = n+1$.
        \end{enumerate}
        By induction, $P(n)$ holds for all $n \in \N$. If $A \subseteq \N$ and $A \neq \emptyset$, then there exists $m \in A$ such that $m \in \N$. By $P(m)$ (which is true by induction), the set $A$ has a minimum.
    \end{proof}
\end{thmbox}

\begin{thmbox}{Pushing Supremum}{push}
    Let $A$ be a nonempty subset of $\R$, and let $b,c$ be real numbers.
    \begin{enumerate}[noitemsep, label=(\alph*)]
        \item If $a \leq b$ for all $a \in A$, then $\sup A \leq b$.
        \item If $c \leq a$ for all $a \in A$, then $c \leq \inf A$.
    \end{enumerate}
    \tcblower
    \textbf{Intuition:} Consider the interval $A \coloneq (0,1)$. Because $a \leq 1$ for all $a \in (0,1)$, we have $\sup A \leq 1$. Because $0 \leq a$ for all $a \in (0,1)$,we have $0 \leq \inf A$.
    \begin{proof}[Proof of (a)]
        Since $a \leq b$ for all $a \in A$, then $b$ is an upper bound for $A$. By completeness, $A$ has a supremum, and $s \coloneq \sup A$ is the least upper bound for $A$. Thus, $s \leq b$.
    \end{proof}
    \begin{proof}[Proof of (b)]

    \end{proof}
\end{thmbox}

\begin{exbox}{}{}
    Suppose $A,B \subseteq \R$, $A \neq \emptyset$, $A \subseteq B$, and $B$ is bounded above. Prove that $A$ is bounded above and $\sup A \leq \sup B$.
    \tcblower
    \begin{proof}
        Since $A \subseteq B$ and $A \neq \emptyset$, then $B \neq \emptyset$. Also, $B$ is bounded above, so $B$ has a supremum (by completeness). Let $a \in A$ be arbitrary. Then $a \in B$, so $a \leq \sup B$. Thus, $A$ is bounded above, so $A$ has a supremum (by completeness). By \nameref{thm:push}, $\sup A \leq \sup B$.
    \end{proof}
\end{exbox}

\begin{thmbox}{Approximation Property of Suprema and Infima}{approx}
    Suppose $A$ is a nonempty subset of $\R$, and $s,r \in \R$. Then:
    \begin{enumerate}[label=(\alph*)]
        \item $s = \sup A$ if and only if (i) $s$ is an upper bound for $A$, and (ii) for all $\epsilon > 0$, there exists $a \in A$ such that $s - \epsilon < a$.
        \item $r = \inf A$ if and only if (i) $r$ is a lower bound for $A$, and (ii) for all $\epsilon > 0$, there exists $a \in A$ such that $a < r + \epsilon$.
    \end{enumerate}
    \tcblower
    \textbf{Intuition:} If we nudge the supremum ever so slightly to the left, then we must have moved past something in $A$.
    \begin{proof}[Proof of (a)]
        Let $s \coloneq \sup A$. Then (i) holds by definition of suprema. To prove (ii), let $\epsilon > 0$. Since $s - \epsilon < s$, then $s - \epsilon$ is not an upper bound for $A$. Therefore, there exists $a \in A$ such that $s - \epsilon < a$.

        Conversely, suppose that (i) and (ii) hold. We need to show $s = \sup A$. From (i), we know that $s$ is an upper bound for $A$. Now, we need to show that $s$ is the least upper bound. Let $t$ be an upper bound for $A$. Suppose for contradiction that $t < s$. Let $\epsilon \coloneq s - t > 0$. Then $t = s - \epsilon$. By (ii), there exists $a \in A$ such that $a > s - \epsilon = t$. This contradicts $t$ being an upper bound for $A$. Thus, there is no upper bound less than $s$. Therefore, $s = \sup A$.
    \end{proof}
\end{thmbox}

\section{Consequences of Completeness}
\begin{thmbox}{$\N$ is not Bounded Above}{}
    \begin{proof}
        Suppose for contradiction $\N$ is bounded above. Since $\N$ is not empty, then $\N$ has a supremum in $\R$. Let $s \coloneq \sup \N \in \R$. Then $n \leq s$ for all $n \in \N$. By the Peano axioms, $n$ has a successor $n+1 \in \N$, so $n+1 \leq s$ for all $n \in \N$. Therefore, $n \leq s - 1$ for all $n \in \N$. This contradicts $s$ being the least upper bound for $\N$.
    \end{proof}
\end{thmbox}

\begin{thmbox}{Archimedean Principle}{archimedean}
    Suppose $x,y \in \R$ where $x > 0$. Then, there exists $n \in \N$ such that $nx > y$.
    \tcblower
    \textbf{Intuition:} This is basically an extension of the fact that $\N$ is not bounded above.
    \begin{proof}
        Since $\sfrac{y}{x}$ is not an upper bound for $\N$, then there exists $n \in \N$ such that $n > \sfrac{y}{x}$. Since $x > 0$, then $nx > y$.
    \end{proof}
\end{thmbox}

\begin{thmbox}{Density of $\Q$ in $\R$}{}
    Suppose $x,y \in \R$ where $x < y$. Then there exists $r \in Q$ such that $x < r < y$.
    \tcblower
    \textbf{Intuition:} Given any two different real numbers, there's some rational number between them.
    \begin{proof}
        We will consider three cases:
        \begin{enumerate}
            \item If $x \geq 0$, then $0 \leq x < y$. Since $y - x > 0$, then by the \nameref{thm:archimedean}, there exists $n \in \N$ such that $n(y-x) > 1$. We want to show there is a natural number between $nx$ and $ny$. Let $A \coloneq \{ k \in \N : k > nx \}$. Since $\N$ isn't bounded above, then $A$ is not empty. By the \nameref{thm:wop}, $A$ has a minimum. Let $m \coloneq \min A$. Then $m > nx$, and $m-1 \leq nx$. Thus, $m \leq nx+1$, so:
            \[ nx < m \leq nx+1 < ny \]
            Dividing across by $n$ yields $x < \sfrac{m}{n} < y$. Note that $m,n \in \N \subseteq \Z$, so $\sfrac{m}{n} \in \Q$.
            \item If $x < 0$ and $y > 0$, then $x < 0 < y$ where $0 \in \Q$.
            \item If $x < 0$ and $y \leq 0$, then $x < y \leq 0$. Multiplying across by $-1$, we have $-x > -y \geq 0$. By the first case, there must exist $t \in \Q$ where $-y < t < -x$. Multiply across by $-1$ again to attain $y > -t > x$ where $-t \in \Q$.
        \end{enumerate}
        This completes the proof.
    \end{proof}
\end{thmbox}

\begin{thmbox}{$\sqrt{2}$ is a Real Number}{}
    There exists $s \in \R$ such that $s^2 = 2$.
    \tcblower
    \begin{proof}
        Let $A \coloneq \left\{ x \in \R : x^2 < 2 \right\}$. Since $0^2 < 2$, then $0 \in A$, so $A$ is not empty. Also, $A$ is bounded above, for example by $2$. By completeness, $A$ must have a supremum in $\R$. Let $s \coloneq \sup A$. We will use trichotomy to show that $s^2 = 2$.
        \begin{enumerate}
            \item If $s^2 > 2$, then$\ldots$
            \begin{notebox}
                \textbf{Scratchwork:} We need to show that this is not possible, i.e. show there is some $s - \sfrac{1}{n}$ that is less than $s$ but is still an upper bound for $A$. We want $(s - \sfrac{1}{n})^2 > 2$. Then, $s^2 - \sfrac{2s}{n} + \sfrac{1}{n^2} > 2$. We can chop off the $\sfrac{1}{n^2}$, reducing the inequality to $s^2 - \sfrac{2s}{n} > 2$. Thus, we need to choose $n > \frac{2s}{s^2-2}$.
            \end{notebox}
            $\ldots$ let $n \in \N$ such that $n > \frac{2s}{s^2-2}$. Then:
            \begin{alignat*}{2}
                && n &> \frac{2s}{s^2-2} \\
                & \implies \quad & s^2 - \frac{2s}{n} &> 2 \\
                & \implies &  s^2 - \frac{2s}{n} + \frac{1}{n^2} &> 2 \\
                & \implies & \left( s - \frac{1}{n} \right)^2 &> 2
            \end{alignat*}
            Thus, $s - \sfrac{1}{n}$ is an upper bound for $A$ that is less than $s$. This contradicts $s$ being the supremum for $A$.
            \item If $s^2 < 2$, then$\ldots$
            \begin{notebox}
                \textbf{Scratchwork:} Again, we need to show that this is not possible. We know that in this case, $s \in A$, so we need to find another thing in $A$ that is bigger than $s$. In other words, we want some $(s + \sfrac{1}{n})^2 < 2$. Then, $s^2 + \sfrac{2s}{n} + \sfrac{1}{n^2} < 2$. Choose $n > \sfrac{1}{2s}$ and $n > \frac{4s}{2-s^2}$.
                \begin{align*}
                    \left( s + \frac{1}{n} \right)^2 &= s^2 + \frac{2s}{n} + \frac{1}{n^2}
                \end{align*}
            \end{notebox}
            $\ldots$ let $n \in \N$ such that $n > \max\left\{ \frac{1}{2s}, \frac{4s}{2-s^2} \right\}$. Then $\frac{1}{n} < 2s$ and $s^2 + \frac{4s}{n} < 2$. So:
            \begin{align*}
                \left( s + \frac{1}{n} \right)^2
                &= s^2 + \frac{2s}{n} + \frac{1}{n^2} \\
                &< s^2 + \frac{2s}{n} + \frac{2s}{n} \\
                &= s^2 + \frac{4s}{n} < 2
            \end{align*}
            That is, $s + \frac{1}{n} \in A$. This contradicts $s$ being an upper bound for $A$.
        \end{enumerate}
        By trichotomy, $s^2 = 2$.
    \end{proof}
\end{thmbox}

\begin{thmbox}{Nested Interval Property}{nested-interval-property}
    Suppose that for each $n \in \N$, $a_n, b_n \in \R$ with $a_n \leq b_n$, and $a_n \leq a_{n+1} \leq b_{n+1} \leq b_n$ for all $n \in \N$. Then $\bigcap_{n=1}^\infty [ a_n, b_n ] \neq \emptyset$.
    \tcblower
    \textbf{Intuition:} We can move the two borders of an open interval closer and closer to each other, and it won't be empty.
    \begin{proof}
        Note that $a_n \leq a_{n+1} \leq a_{n+2} \leq \ldots$ and $\ldots \leq b_{n+2} \leq b_{n+1} \leq b_n$. If $k \leq n$, then $a_k \leq a_n \leq b_n$.
        \begin{itemize}[noitemsep]
            \item If $k \leq n$, then $a_k \leq a_n \leq b_n$.
            \item If $k \geq n$, then $a_k \leq b_k \leq b_n$.
        \end{itemize}
        That is, $a_k \leq b_n$ for all $k_n \in \N$. Let $A \coloneq \{ a_k : k \in \N\}$. Then $A$ is bounded above, for example by $b_1$. Also, $A$ is not empty. By completeness, $A$ has a supremum. Let $s \coloneq \sup A$. Note that since $s$ is an upper bound for $A$, then $a_n \leq \sup A$ for all $n \in \N$. Also note that $\sup A$ is the least upper bound for $A$, so $\sup A \leq b_n$ for all $n \in \N$. Thus, $a_n \leq \sup A \leq b_n$ for all $n \in \N$, so $\sup A \in [ a_n, b_n ]$ for all $n \in \N$. Thus, $\sup A \in \bigcap_{n=1}^\infty [ a_n, b_n ]$, so it is not empty.
    \end{proof}
\end{thmbox}

The nested interval property is actually false for open intervals!
\[ \forall (x \in (0,1)) \exists (n \in \N) (\sfrac{1}{n} < x \implies x \notin (0, \sfrac{1}{n}) ) \]


\chapter{Suprema and Infima}
\begin{dfnbox}{Infimum}{}
    Let $F$ be an ordered field, and let $A \subseteq F$. $s$ is the \dfntxt{infimum} of $A$ if:
    \begin{enumerate}[noitemsep]
        \item $s$ is a lower bound for $A$, and
        \item $s$ is greater than every other lower bound for $A$.
    \end{enumerate}
\end{dfnbox}

We can prove that the existence of infima is already implied by completeness.

\begin{thmbox}{Existence of Infima in $\R$}{}
    Let $A \subseteq \R$ be nonempty and bounded below. Then $A$ has an infimum in $\R$.
    \tcblower
    \begin{proof}
        Let $A \subseteq \R$ be nonempty and bounded below. Let $B$ be the set of all lower bounds for $A$. In other words, $B \coloneq \left\{ b \in \R : \forall(a \in A)(b < a) \right\}$. Since $A$ is bounded below, then $B$ is nonempty. Note also that $B$ is bounded above by element of $A$. By completeness, $s \coloneq \sup B$ exists. Now, we need to show that $\sup B = \inf A$.
        \begin{enumerate}
            \item Every $a \in A$ is an upper bound for $B$, and $\sup B$ is the least upper bound for $B$. Then, $\sup B \leq a$. That is, $\sup B$ is a lower bound for $A$.
            \item Let $t$ be a lower bound for $A$. Then, by definition of $B$, it follows that $t \in B$. Then $t \leq \sup B$ as required.
        \end{enumerate}
        Therefore, $\sup B = \inf A$ in $\R$.
    \end{proof}
\end{thmbox}

\begin{thmbox}{Well-Ordering Principle}{}
    Every non-empty subset of $\N$ has a minimum.
    \tcblower
    \begin{proof}
        We will use induction. For convenience, let $P(n)$ represent the following statement: ``If $A \subseteq \N$ and $A \cap \{1,2,\ldots,n \} \neq \emptyset$, then $A$ has a minimum.''

        \textbf{Base Case:} First, we will prove $P(1)$. If $A \subseteq \N$ and $A \cap \{1\} \neq \emptyset$, then $1 \in A$, so $A$ has a minimum.

        \textbf{Induction Step:} Assume that $P(n)$ holds for some $n \in N$. Suppose $A \subseteq \N$ and $A \cap \{1,2,\ldots,n+1\} \neq \emptyset$.
        \begin{enumerate}
            \item If $A \cap \{ 1,2,\ldots,n \} \neq \emptyset$, then $A$ has a minimum by $P(n)$.
            \item If $A \cap \{1,2,\ldots,n\} = \emptyset$, then $n+1 \in A$, so $\min A = n+1$.
        \end{enumerate}
        By induction, $P(n)$ holds for all $n \in \N$. If $A \subseteq \N$ and $A \neq \emptyset$, then there exists $m \in A$ such that $m \in \N$. By $P(m)$ (which is true by induction), the set $A$ has a minimum.
    \end{proof}
\end{thmbox}

\begin{thmbox}{Pushing Supremum}{push}
    Let $A$ be a nonempty subset of $\R$, and let $b,c$ be real numbers.
    \begin{enumerate}[noitemsep, label=(\alph*)]
        \item If $a \leq b$ for all $a \in A$, then $\sup A \leq b$.
        \item If $c \leq a$ for all $a \in A$, then $c \leq \inf A$.
    \end{enumerate}
    \tcblower
    \textbf{Intuition:} Consider the interval $A \coloneq (0,1)$. Because $a \leq 1$ for all $a \in (0,1)$, we have $\sup A \leq 1$. Because $0 \leq a$ for all $a \in (0,1)$,we have $0 \leq \inf A$.
    \begin{proof}[Proof of (a)]
        Since $a \leq b$ for all $a \in A$, then $b$ is an upper bound for $A$. By completeness, $A$ has a supremum, and $s \coloneq \sup A$ is the least upper bound for $A$. Thus, $s \leq b$.
    \end{proof}
    \begin{proof}[Proof of (b)]

    \end{proof}
\end{thmbox}

\begin{exbox}{}{}
    Suppose $A,B \subseteq \R$, $A \neq \emptyset$, $A \subseteq B$, and $B$ is bounded above. Prove that $A$ is bounded above and $\sup A \leq \sup B$.
    \tcblower
    \begin{proof}
        Since $A \subseteq B$ and $A \neq \emptyset$, then $B \neq \emptyset$. Also, $B$ is bounded above, so $B$ has a supremum (by completeness). Let $a \in A$ be arbitrary. Then $a \in B$, so $a \leq \sup B$. Thus, $A$ is bounded above, so $A$ has a supremum (by completeness). By \nameref{thm:push}, $\sup A \leq \sup B$.
    \end{proof}
\end{exbox}

\begin{thmbox}{Approximation Property of Suprema and Infima}{approx}
    Suppose $A$ is a nonempty subset of $\R$, and $s,r \in \R$. Then:
    \begin{enumerate}[label=(\alph*)]
        \item $s = \sup A$ if and only if (i) $s$ is an upper bound for $A$, and (ii) for all $\epsilon > 0$, there exists $a \in A$ such that $s - \epsilon < a$.
        \item $r = \inf A$ if and only if (i) $r$ is a lower bound for $A$, and (ii) for all $\epsilon > 0$, there exists $a \in A$ such that $a < r + \epsilon$.
    \end{enumerate}
    \tcblower
    \textbf{Intuition:} If we nudge the supremum ever so slightly to the left, then we must have moved past something in $A$.
    \begin{proof}[Proof of (a)]
        Let $s \coloneq \sup A$. Then (i) holds by definition of suprema. To prove (ii), let $\epsilon > 0$. Since $s - \epsilon < s$, then $s - \epsilon$ is not an upper bound for $A$. Therefore, there exists $a \in A$ such that $s - \epsilon < a$.

        Conversely, suppose that (i) and (ii) hold. We need to show $s = \sup A$. From (i), we know that $s$ is an upper bound for $A$. Now, we need to show that $s$ is the least upper bound. Let $t$ be an upper bound for $A$. Suppose for contradiction that $t < s$. Let $\epsilon \coloneq s - t > 0$. Then $t = s - \epsilon$. By (ii), there exists $a \in A$ such that $a > s - \epsilon = t$. This contradicts $t$ being an upper bound for $A$. Thus, there is no upper bound less than $s$. Therefore, $s = \sup A$.
    \end{proof}
\end{thmbox}

\section{Consequences of Completeness}
\begin{thmbox}{$\N$ is not Bounded Above}{}
    \begin{proof}
        Suppose for contradiction $\N$ is bounded above. Since $\N$ is not empty, then $\N$ has a supremum in $\R$. Let $s \coloneq \sup \N \in \R$. Then $n \leq s$ for all $n \in \N$. By the Peano axioms, $n$ has a successor $n+1 \in \N$, so $n+1 \leq s$ for all $n \in \N$. Therefore, $n \leq s - 1$ for all $n \in \N$. This contradicts $s$ being the least upper bound for $\N$.
    \end{proof}
\end{thmbox}

\begin{thmbox}{Archimedean Principle}{archimedean}
    Suppose $x,y \in \R$ where $x > 0$. Then, there exists $n \in \N$ such that $nx > y$.
    \tcblower
    \textbf{Intuition:} This is basically an extension of the fact that $\N$ is not bounded above.
    \begin{proof}
        Since $\sfrac{y}{x}$ is not an upper bound for $\N$, then there exists $n \in \N$ such that $n > \sfrac{y}{x}$. Since $x > 0$, then $nx > y$.
    \end{proof}
\end{thmbox}

\begin{thmbox}{Density of $\Q$ in $\R$}{}
    Suppose $x,y \in \R$ where $x < y$. Then there exists $r \in Q$ such that $x < r < y$.
    \tcblower
    \textbf{Intuition:} Given any two different real numbers, there's some rational number between them.
    \begin{proof}
        We will consider three cases:
        \begin{enumerate}
            \item If $x \geq 0$, then $0 \leq x < y$. Since $y - x > 0$, then by the \nameref{thm:archimedean}, there exists $n \in \N$ such that $n(y-x) > 1$. We want to show there is a natural number between $nx$ and $ny$. Let $A \coloneq \{ k \in \N : k > nx \}$. Since $\N$ isn't bounded above, then $A$ is not empty. By the \nameref{thm:wop}, $A$ has a minimum. Let $m \coloneq \min A$. Then $m > nx$, and $m-1 \leq nx$. Thus, $m \leq nx+1$, so:
            \[ nx < m \leq nx+1 < ny \]
            Dividing across by $n$ yields $x < \sfrac{m}{n} < y$. Note that $m,n \in \N \subseteq \Z$, so $\sfrac{m}{n} \in \Q$.
            \item If $x < 0$ and $y > 0$, then $x < 0 < y$ where $0 \in \Q$.
            \item If $x < 0$ and $y \leq 0$, then $x < y \leq 0$. Multiplying across by $-1$, we have $-x > -y \geq 0$. By the first case, there must exist $t \in \Q$ where $-y < t < -x$. Multiply across by $-1$ again to attain $y > -t > x$ where $-t \in \Q$.
        \end{enumerate}
        This completes the proof.
    \end{proof}
\end{thmbox}

\begin{thmbox}{$\sqrt{2}$ is a Real Number}{}
    There exists $s \in \R$ such that $s^2 = 2$.
    \tcblower
    \begin{proof}
        Let $A \coloneq \left\{ x \in \R : x^2 < 2 \right\}$. Since $0^2 < 2$, then $0 \in A$, so $A$ is not empty. Also, $A$ is bounded above, for example by $2$. By completeness, $A$ must have a supremum in $\R$. Let $s \coloneq \sup A$. We will use trichotomy to show that $s^2 = 2$.
        \begin{enumerate}
            \item If $s^2 > 2$, then$\ldots$
            \begin{notebox}
                \textbf{Scratchwork:} We need to show that this is not possible, i.e. show there is some $s - \sfrac{1}{n}$ that is less than $s$ but is still an upper bound for $A$. We want $(s - \sfrac{1}{n})^2 > 2$. Then, $s^2 - \sfrac{2s}{n} + \sfrac{1}{n^2} > 2$. We can chop off the $\sfrac{1}{n^2}$, reducing the inequality to $s^2 - \sfrac{2s}{n} > 2$. Thus, we need to choose $n > \frac{2s}{s^2-2}$.
            \end{notebox}
            $\ldots$ let $n \in \N$ such that $n > \frac{2s}{s^2-2}$. Then:
            \begin{alignat*}{2}
                && n &> \frac{2s}{s^2-2} \\
                & \implies \quad & s^2 - \frac{2s}{n} &> 2 \\
                & \implies &  s^2 - \frac{2s}{n} + \frac{1}{n^2} &> 2 \\
                & \implies & \left( s - \frac{1}{n} \right)^2 &> 2
            \end{alignat*}
            Thus, $s - \sfrac{1}{n}$ is an upper bound for $A$ that is less than $s$. This contradicts $s$ being the supremum for $A$.
            \item If $s^2 < 2$, then$\ldots$
            \begin{notebox}
                \textbf{Scratchwork:} Again, we need to show that this is not possible. We know that in this case, $s \in A$, so we need to find another thing in $A$ that is bigger than $s$. In other words, we want some $(s + \sfrac{1}{n})^2 < 2$. Then, $s^2 + \sfrac{2s}{n} + \sfrac{1}{n^2} < 2$. Choose $n > \sfrac{1}{2s}$ and $n > \frac{4s}{2-s^2}$.
                \begin{align*}
                    \left( s + \frac{1}{n} \right)^2 &= s^2 + \frac{2s}{n} + \frac{1}{n^2}
                \end{align*}
            \end{notebox}
            $\ldots$ let $n \in \N$ such that $n > \max\left\{ \frac{1}{2s}, \frac{4s}{2-s^2} \right\}$. Then $\frac{1}{n} < 2s$ and $s^2 + \frac{4s}{n} < 2$. So:
            \begin{align*}
                \left( s + \frac{1}{n} \right)^2
                &= s^2 + \frac{2s}{n} + \frac{1}{n^2} \\
                &< s^2 + \frac{2s}{n} + \frac{2s}{n} \\
                &= s^2 + \frac{4s}{n} < 2
            \end{align*}
            That is, $s + \frac{1}{n} \in A$. This contradicts $s$ being an upper bound for $A$.
        \end{enumerate}
        By trichotomy, $s^2 = 2$.
    \end{proof}
\end{thmbox}

\begin{thmbox}{Nested Interval Property}{}
    Suppose that for each $n \in \N$, $a_n, b_n \in \R$ with $a_n \leq b_n$, and $a_n \leq a_{n+1} \leq b_{n+1} \leq b_n$ for all $n \in \N$. Then $\bigcap_{n=1}^\infty [ a_n, b_n ] \neq \emptyset$.
    \tcblower
    \textbf{Intuition:} We can move the two borders of an open interval closer and closer to each other, and it won't be empty.
    \begin{proof}
        Note that $a_n \leq a_{n+1} \leq a_{n+2} \leq \ldots$ and $\ldots \leq b_{n+2} \leq b_{n+1} \leq b_n$. If $k \leq n$, then $a_k \leq a_n \leq b_n$.
        \begin{itemize}[noitemsep]
            \item If $k \leq n$, then $a_k \leq a_n \leq b_n$.
            \item If $k \geq n$, then $a_k \leq b_k \leq b_n$.
        \end{itemize}
        That is, $a_k \leq b_n$ for all $k_n \in \N$. Let $A \coloneq \{ a_k : k \in \N\}$. Then $A$ is bounded above, for example by $b_1$. Also, $A$ is not empty. By completeness, $A$ has a supremum. Let $s \coloneq \sup A$. Note that since $s$ is an upper bound for $A$, then $a_n \leq \sup A$ for all $n \in \N$. Also note that $\sup A$ is the least upper bound for $A$, so $\sup A \leq b_n$ for all $n \in \N$. Thus, $a_n \leq \sup A \leq b_n$ for all $n \in \N$, so $\sup A \in [ a_n, b_n ]$ for all $n \in \N$. Thus, $\sup A \in \bigcap_{n=1}^\infty [ a_n, b_n ]$, so it is not empty.
    \end{proof}
\end{thmbox}

The nested interval property is actually false for open intervals!
\[ \forall (x \in (0,1)) \exists (n \in \N) (\sfrac{1}{n} < x \implies x \notin (0, \sfrac{1}{n}) ) \]


\chapter{Cardinality}
\chapter{Cardinality}
\begin{dfnbox}{Cardinality}{cardinality}
    \dfntxt{Cardinality} is a measure of the amount of elements in a set, denoted $\abs{A}$. We say two sets have the same cardinality if there exists a bijection between them.
\end{dfnbox}

For finite sets, we can think of cardinality as the number of elements in that set. For infinite sets, cardinality can sometimes go against our intuition. For any sets $A,B,C$:
\begin{enumerate}[noitemsep]
    \item $\abs{A} = \abs{A}$
    \item if $\abs{A} = \abs{B}$, then $\abs{B} = \abs{A}$
    \item if $\abs{A} = \abs{B}$ and $\abs{B} = \abs{C}$, then $\abs{A} = \abs{C}$.
\end{enumerate}
Hence, equality of cardinalities is an equivalence relation.
% Above is kinda self evident? maybe delete this

\begin{exbox}{Cardinality of $\N$ and $2\N$}{}
    Let $2\N \coloneq \{ 2n : n \in \N \}$ (i.e. the set of even natural numbers). Then $\abs{\N} = \abs{2\N}$.
    \tcblower
    \begin{proof}
        To show that these two sets have the same cardinality, we need to find some bijection between the sets. Let $f : \N \to 2\N$ be a function defined by $f(n) = 2n$. Note that $f$ is well-defined (i.e. is actually a function) because $f(n) \in 2 \N$ for all $n \in \N$. To prove that $f$ is a bijection, we need to prove it is both injective and surjective.
        \begin{enumerate}
            \item Let $n_1, n_2 \in \N$ such that $f(n_1) = f(n_2)$. Then $2n_1 = 2n_2$, so $n_1 = n_2$. Thus, $f$ is injective.
            \item Let $m \in 2\N$. Then $m = 2k$ for some $k \in \N$, so $m = 2k = f(k)$ for some $k \in \N$. Thus, $f$ is surjective.
        \end{enumerate}
        Therefore, $f$ is a bijection, so $\abs{\N} = \abs{2\N}$.
    \end{proof}
\end{exbox}

\begin{exbox}{Cardinality of Intervals}{}
    Let $a,b \in \R$ where $a < b$. Then $\abs{(0,1)} = \abs{(a,b)}$.
    \tcblower
    \begin{proof}
        We need to find a bijection from $(0,1)$ to $(a,b)$. We need to ``scale'' the interval $(0,1)$ to the width of $(a,b)$, then translate it to match $(a,b)$. Define $f : (0,1) \to (a,b)$ by $f(x) = a + (b-a)x$. (We need to check $f$ is well-defined). Let $x \in (0,1)$. Then $0<x<1$, so multiplying by $(b-a)$ which is positive gives $0 < (b-a)x < b-a$. Adding $a$, we get $a < a+(b-a)x < b$. Now we need to show $f$ is a bijection:
        \begin{enumerate}
            \item Let $x_1, x_2 \in (0,1)$ such that $f(x_1) = f(x_2)$. Then $a + (b-a)x_1 = a+ (b-a)x_2$. Subtracting $a$ from both sides, we get $(b-a)x_1 = (b-a)x_2$. Since $(b-a) \neq 0$, we can divide both side by $(b-a)$ to get $x_1 = x_2$.
            \item Let $y \in (a,b)$.
            \begin{notebox}
                \textbf{Scratchwork:} We want to find some $x \in (0,1)$ where $y = f(x) = a + (b-a)x$. Using some algebra to solve for $x$, we have $x = \frac{y-a}{b-a}$
            \end{notebox}
            Let $x = \frac{y-a}{b-a}$. First, we show $x \in (0,1)$:
            \begin{alignat*}{2}
                && a < y < b \\
                & \implies \quad & 0 < y-a < b-a \\
                & \implies & 0 < \frac{y-a}{b-a} < 1
            \end{alignat*}
            Thus, $x \in (0,1)$. Also:
            \[ f(x) = a + (b-a) \left( \frac{y-a}{b-a} \right) = a + (y-a) = y \]
            Thus, $f$ is surjective.
        \end{enumerate}
        Therefore, $f$ is a bijective, so $\abs{(0,1)} = \abs{(a,b)}$.
    \end{proof}
\end{exbox}

\begin{dfnbox}{Power Set}{}
    Let $A$ be a set. The \dfntxt{power set} of $A$ is the set of all subsets of $A$.
    \tcblower
    \[ \mathcal{P}(A) = \{ B : B \subseteq A \} \]
\end{dfnbox}

For example, the power set of $\{1,2,3\}$ is $\left\{\emptyset, \{1\}, \{2\}, \{3\}, \{1,2\}, \{1,3\}, \{2,3\}, \{1,2,3\} \right\}$. For any finite set with $n$ elements in it, its power set has $2^n$ elements in it.

\begin{exbox}{Cardinality of $\N$ and $\mathcal{P}(N)$}{powerset-card}
    $\abs{\N} \neq \abs{\mathcal{P}(\N)}$
    \tcblower
    \begin{proof}
        We will show that any function $f : \N \to \mathcal{P}(\N)$ cannot be surjective, and thus not bijective. Let $f : \N \to \mathcal{P}(\N)$ be any function defined by $f(n) = A_n$. Note $A_n \subseteq \N$, so $A_n \in \mathcal{P}(A)$. Now we will define a set that isn't in $f[\N]$. For each $n \in \N$, if $n \in A_n$, then $n \notin A$, and if $n \notin A_n$, then $n \in A$. More formally, $A \coloneq \{ n \in \N : n \notin A_n \}$. For all $k \in \N$, note that:
        \begin{itemize}[noitemsep]
            \item if $k \in A_k$, then $k \notin A$, so $A \neq A_k$, and
            \item if $k \notin A_k$, then $k \in A_k$, so $A \neq A_k$.
        \end{itemize}
        Hence, $A \subseteq \N$, but $f(k) \neq A$ for any $k \in \N$. Thus, $f$ is not surjective.
    \end{proof}
\end{exbox}

\begin{dfnbox}{Finite, Countably Infinite, Countable, Uncountable}{}
    Let $A$ be a set. We say $A$ is:
    \begin{itemize}[noitemsep]
        \item \dfntxt{finite} if $A \neq \emptyset$ or $\abs{A} = \abs{\{1, 2, \ldots, n\}}$ for some $n \in \N$.
        \item \dfntxt{countably infinite} if $\abs{A} = \abs{N}$.
        \item \dfntxt{countable} if $A$ is finite or countably infinite
        \item \dfntxt{uncountable} if $A$ is not countable
    \end{itemize}
\end{dfnbox}

\begin{thmbox}{$\mathcal{P}(\N)$ is uncountable.}{}
    \begin{proof}
        We know from Example \ref{ex:powerset-card} that $\mathcal{P}(\N)$ is not countably infinite. We need to show that $\mathcal{P}(\N)$ is not finite. Since $\{1\} \in \mathcal{P}(\N)$, then it cannot be empty. Suppose for contradiction $\abs{\{1,2,\ldots,n\}} = \abs{\mathcal{P}(\N)}$ for some $n \in \N$, then there exists a bijection $f : \{1,2,\ldots,n\} \to \mathcal{P}(\N)$. Define $g : \N \to \{1,2,\ldots,n\}$ by:
        \[ g(k) = \begin{cases} k, & 1\leq k \leq n \\ 1, & k > n \end{cases} \]
        Then $g$ is surjective, so $f \circ g : \N \to \mathcal{P}(\N)$ is surjective. This contradicts the fact that no such function exists (by Example \ref{ex:powerset-card}).
    \end{proof}
\end{thmbox}

Generally, there is never a bijection from a set to its power set.

\begin{notebox}
    \textbf{Intuition:} A set is countable if its elements can be ``listed'' or ``counted''. That is, for finite sets:
    \[ X = \{x_1, x_2, \ldots, x_n\} = \{x_k\}_{k=1}^{n}\]
    For infinitely countable sets:
    \[ X = \{x_1, x_2, \ldots \} = \{x_k\}_{k=1}^\infty \]
    If $X$ is finite, then there exists a bijection $f : \{1,2,\ldots,n\} \to X$ . Thus, $X = \{f(1), f(2), \ldots, f(n) \}$. If $X$ is countably infinite, then there exists a bijection $f : \N \to X$. Thus, $X = \{f(1), f(2), \ldots\}$.
\end{notebox}

\begin{thmbox}{Subsets of Countable Sets are Countable}{card-subset}
    The subset of a countable set is still countable. (i.e. a countable set cannot contain an uncountable subset).
    \tcblower
    \begin{proof}
        Let $X$ be a countable set, and let $A \subseteq X$. We will consider two cases. First, if $A$ is finite, then $A$ is countable, and we are done. Otherwise, $A$ is infinite, and hence $X$ is infinite. Then $X$ is countably infinite, so $X = \{x_1, x_2, \ldots\} = \{x_k\}_{k=1}^\infty$.
        \begin{notebox}
            \textbf{Idea:} Our set $A$ might look something like $\{x_3, x_4, x_6, \ldots\}$. We need to align these indices to $1$, $2$, $3$, and so on. We'll let $k_1 = \min\{3,4,6,\ldots\}$, let $k_2 = \min\{4,6,\ldots\}$, and so on.
        \end{notebox}
        Let $k_1 \coloneq \min\{k \in \N : x_k \in A\}$. Let $a_1 \coloneq x_{k_1}$. For all $j \in \N$ such that $j > 1$, we define $k_j \coloneq \min\{k \in \N : (x_k \in A) \land (k > k_{j-1}) \}$. Let $a_j \coloneq x_{k_j}$. Then $1 \leq k_1 < k_2 < k_3 < \ldots$, so $k_j$ approaches infinity. Let $g : \N \to A$ be a function defined by $g(j) = a_j$. We need to show that $g$ is both injective and surjective, and thus a bijection.
        \begin{itemize}
            \item Suppose that $g(j_1) = g(j_2)$ for some $j_1, j_2 \in \N$. Then $a_{j_1} = a_{j_2}$, so $x_{k_{j_1}} = x_{k_{j_2}}$. Then $k_{j_1} = k_{j_2}$, so $j_1 = j_2$. Thus, $g$ is injective.
            \item Let $a \in A$ Since $A \subseteq X$, then $a \in X$. Thus, $a = x_l$ for some $l \in \N$. Let $m \coloneq \min\{ j \in \N : k_j \geq l \}$. Since $m \in \{ j \in \N : j_k \geq l \}$, then $k_m \geq l$. Also, $m-1 \notin \{j \in \N : k_j \geq l \}$, so $k_{m-1} < l$. Now, $k_m = \min\{k \in \N : (x_k \in A) \land (k > k_{m-1})\}$. But $x_l \in A$, and $l > k_{m-1}$, so $l \in \{k \in \N : (x_k \in A) \land (k > k_{m-1}) \}$. Thus, $k_m \leq l$, because $k_m$ is the minimum of the set containing $l$. By trichotomy, $k_m = l$. Therefore:
            \[ g(m) = a_m = x_{k_m} = x_l = a \]
            So $g$ is surjective.
        \end{itemize}
        Since $g$ is a bijection, then $\abs{\N} = \abs{A}$, so $\abs{A}$ is countable.
    \end{proof}
\end{thmbox}

\begin{thmbox}{Injectivity and Cardinality}{injectivity-cardinality}
    A set $A$ is countable if and only if there exists an injective function $f : A \to \N$.
    \tcblower
    \begin{proof}
        First, suppose $A$ is a countable set. We consider two cases:
        \begin{itemize}
            \item If $A$ is countably infinite, then there exists a bijection $f : A \to \N$.
            \item If $A$ is finite, then there exists a bijection $f : A \to \{1,2,\ldots,n\}$ for some $n \in \N$. Let $g : \{1,2,\ldots,n\} \to \N$ be a function defined by $g(x) = x$ (i.e. an inclusion mapping). Then $f$ and $g$ are both injective, so $g \circ f : A \to \N $ is injective.
        \end{itemize}

        Conversely, suppose $f : A \to \N$ is an injection. Then $f[A] \subseteq \N$, so $f[A]$ is countable by Theorem \ref{thm:card-subset}. Define $g : A \to f[A]$ by $g(a) = f(a)$. Then $g$ is injective because $f$ is injective, and $g$ is surjective because $g[A] = f[A]$. Thus, $g$ is a bijection, so $\abs{A} = \abs{f[A]}$. Therefore, $A$ is countable.
    \end{proof}
\end{thmbox}

\begin{thmbox}{$\abs{\N \times \N} = \abs{\N}$}{nxn-countable}
    $\N \times \N$ is countable.
    \tcblower
    \begin{proof}
        Let $f : \N \times \N \to \N$ be a function defined by $f(n,m) = 2^n 3^m$. We now show that $f$ is bijective. To prove $f$ is injective, suppose $f(n_1, m_1) = f(n_2, n_2)$. Then $2^{n_1} 3^{m_1} = 2^{n_2} 3^{m_2}$.
        \begin{itemize}
            \item If $n_1 > n_2$, then $2^{n_1 - n_2} = 3^{m_2 - m_1}$. Since $n_1 > n_2$, we have $n_1 - n_2 > 0$, so $2^{n_1 - n_2} \in \N$. Then also $3^{m_2 - m_1} \in \N$. But $2^{n_1 - n_2}$ is even, and $3^{m_2 - m_1}$ is odd. This contradicts the fact that $2^{n_1 - n_2} = 3^{m_2 - m_1}$.
            \item If $n_2 > n_1$, then $3^{m_1 - m_2} = 2^{n_2 - n_1}$. By a similar argument, $2^{n_2 - n_1}$ is even and $3^{m_1 - m_2}$ is odd, producing the same contradiction.
            \item If $n_1 = n_2$, then $2^{n_1} = 2^{n_2}$, so cancelling gives $3^{m_1} = 3^{m_2}$. Thus, $m_1 = m_2$.
        \end{itemize}
        Hence, $(n_1, m_1) = (n_2, m_2)$, so $f$ is injective.
        By Theorem \ref{thm:injectivity-cardinality}, $\N \times \N$ is countable. Also, $\N \times \N$ is infinite, so $\abs{\N \times \N} = \abs{\N}$.
    \end{proof}
\end{thmbox}

\begin{thmbox}{Collection of Countable Sets}{}
    Suppose that for all $k \in \N$, $A_k$ is a countable set. Then $\cup_k A_k \in \N$ is countable. (i.e. a countable union of countable sets is countable)
    \tcblower
    \begin{proof}
        Let $k \in \N$. $A_k$ is countable, so $A_k$ can be listed as such:
        \begin{align*}
            A_1 &= \{a_{11}, a_{12}, a_{13}, a_{14}, \ldots\} = \{a_j\}_{j \in \N} \\
            A_2 &= \{a_{21}, a_{22}, a_{23}, a_{24}, \ldots\} \\
            &\vdots \\
            A_k &= \{a_{k1}, a_{k2}, a_{k3}, a_{k4}, \ldots\}
        \end{align*}

        \begin{notebox}
            We want to define some function $f : \bigcup_{k \in \N}A_k \to \N \times \N$ where $f(a_{kj}) = (k,j) \in \N \times \N$.
            However, we need to consider the possibility that the sets $A_k$ are not disjoint. If $a_{12} = a_{34}$, then $a(_{12}) = (1,2)$ and $f(a_{34}) = (3,4)$.
        \end{notebox}
        Given $a \in \bigcup_{k \in \N}A_k$, let $k(a) \coloneq \min\{ k \in \N : a \in A_k \}$. If $a \in A_{k(a)}$, then there is a unique $j(a) \in \N$ such that $a = a_{k(a)j(a)}$. Now define $f : \bigcup_{k \in \N} A_k \to \N \times \N$ by $f(a) = (k(a), j(a))$. We must show that $f$ is injective. Let $x,y \in \bigcup_{k \in \N}A_k$ such that $f(x) = f(y)$. That is, $(k(x),j(x)) = (k(y),j(y))$. Then $x = a_{k(x)j(x)} = a_{k(y)j(y)} = y$. By Theorem $\ref{thm:nxn-countable}$, there exists some injection $g : \N \times \N \to \N$. Hence, $g \circ f : \bigcup_{k \in \N}A_k \to \N$ is injective. By Theorem \ref{thm:injectivity-cardinality}, $\bigcup_{k \in \N} A_k$ is countable.
    \end{proof}
\end{thmbox}

This theorem shows that, in order to prove a countable union of countable sets is countable, we just need to show that each set in the union is countable. We'll use this in our proof that the set of rational numbers is a countable set.

\begin{thmbox}{$\Q$ is Countable}{}
    \begin{proof}
        Let $\Q^+ \coloneq \{r \in \Q : r > 0\}$, and let $\Q^- \coloneq \{ r \in \Q : r < 0 \}$. First, we'll prove that $\Q^+$ is countable. Let $f : \Q^+ \to \N \times \N$ be a function defined as $f(r) = (p,q)$ such that $r = \sfrac{p}{q}$ where $p,q \in \N$ and $p$ shares no common factors with $q$. To show $f$ is injective, let $r_1, r_2 \in \Q$ where  $f(r_1) = f(r_2)$. Then $r_1 = \sfrac{p_1}{q_1}, r_2 = \sfrac{p_2}{q_2}$ where $p_1, q_1 \in \N$ with no common factors, and $p_2, q_2 \in \N$ with no common factors. Thus, $(p_1,q_1) = (p_2, q_2)$, so $p_1 = p_2$ and $q_1 = q_2$. Thus, $r_1 = \sfrac{p_1}{q_1} = \sfrac{p_2}{q_2} = r_2$, so $f$ is injective. Since there exists an injection $g \in \N \times \N \to \N$, then $g \circ f : \Q^+ \to \N$ is injective. Thus, $\Q^+$ is countable.

        Next, we'll prove that $\Q^-$ is countable. Let $h : \Q^- \to \Q^+$ by $h(r) = -r$. We show $h$ is injective. If $h(r_1) = h(r_2)$ where $r_1, r_2 \in \Q^-$, then $-r_1 = -r_2$, so $r_1 = r_2$. Thus, $h$ is injective. From above, there exist an injection $\phi : \Q^+ \to \N$. Hence, $h \circ \phi : \Q^{-1} \to \N$ is injective.

        Finally, $\{0\}$ is countable because it is finite. Since $\Q = \Q^+ \cup \{0\} \cup \Q^-$ is a countable union of countable sets, then $\Q$ is countable.
    \end{proof}
\end{thmbox}

This means we can ``list'' the rational numbers (disregarding order) as $\Q = \{r_1, r_2, \ldots\} = \{r_n\}_{n \in \N}$.

\begin{thmbox}{$\R$ is Uncountable}{}
    \begin{proof}
        Suppose for contradiction $\R$ is countable. Then $\R$ can be ``listed'' as $\R = \{x_1, x_2, \ldots\} = \{x_n\}_{n \in \N}$. We will define a sequence of non-empty closed intervals $\{I_k\}_{k \in \N}$ such that $I_{k+1} \subseteq I_k$ and $x_k \notin I_k$ for all $k \in \N$. Let $I_0 \coloneq [0,1]$. Divide $I_0$ into three equal closed intervals. Then, at least one of these three intervals does not contain $x_1$. Choose such an interval and call it $I_1$. Divide $I_1$ into there equal closed intervals. Then, at least one of these three intervals does not contain $x_2$. Choose such an interval and call it $I_2$. Given $I_k$ for some $k \in \N$, divide $I_k$ into three equal closed intervals, then choose the interval that does not contain $x_{k+1}$ and call it $I_{k+1}$. By induction, we have $\{I_k\}_{k=1}^\infty$ where each $I_k$ is a (nonempty) closed interval, and $I_{k+1} \subseteq I_k$ for each $k \in \N$. By the \nameref{thm:nested-interval-property}, $\bigcap_{k \in \N} I_k$ is not empty, so there exists $x \in \R$ such that $x \in \bigcap_{k \in \N} I_k$. Since $x \in \R$, we have $x = x_n$ for some $n \in \N$ (by our supposition that $\R$ is countable). However, we constructed $I_n$ such that $x_n \notin I_n$, so $x_n \notin \bigcap_{k \in \N} I_k$. This contradiction renders our initial supposition false. Therefore, $\R$ is uncountable.
    \end{proof}
\end{thmbox}

\begin{thmbox}{Irrational Numbers are Uncountable}{}
    \begin{proof}
        Suppose for contradiction $\R \setminus \Q$ is countable. Then $\R = (\R \setminus \Q) \cup \Q$, so $\R$ is a countable union of countable sets, making $\R$ countable. This contradicts the fact that $\R$ is uncountable, so $\R \setminus \Q$ is uncountable.
    \end{proof}
\end{thmbox}

\section{Additional Remarks}

\begin{dfnbox}{Algebraic Number, Transcandental Number}{}
    If $\alpha$ is a root of the polynomial $a_nx^n + a_{n-1} x^{n-1} + \cdots + a_1 x + a_0 = 0$ where each $a_i \in \Z$, then $\alpha$ is called an \dfntxt{algebraic number}. If $\alpha$ is not algebraic, then we call it a \dfntxt{transcendental number}.
\end{dfnbox}
 The set of all algebraic numbers is countable, so ``most'' real numbers are transcendental.

\begin{itemize}
    \item Even though $\Q$ is dense in $\R$, there are ``more'' irrational numbers than rational numbers
    \item The set $\{\sqrt[n]{m} : n,m \in \N\}$ is countable, so ``most'' real numbers are not radicals.
    \item The set of algebraic numbers is countable, so ``most'' real numbers are transcendental.
    \item $\abs{\mathcal{P}(\N)} = \abs{\R}$, but the set $\{ A \subseteq \N : A\ \text{is finite} \}$ is countable.
    \item If there exists an injection $f : A \to B$, then we say $\abs{A} \leq \abs{B}$. If there exists an injection from $A$ to $B$, but there does not exist an injection from $B$ to $A$, then we say $\abs{A} < \abs{B}$.
\end{itemize}

\begin{thmbox}{Schroeder-Bernstein}{}
    If there exists an injection $f : A \to B$ and an injection $g : B \to A$, then there exists a bijection $h : A \to B$.
\end{thmbox}

\begin{thmbox}{Continuum Hypothesis}{}
    There is no cardinality between $\abs{\N}$ and $\abs{\R}$.
\end{thmbox}

In Zermelo-Fraenkel with Choice (ZFC) set theory, the continuum hypothesis cannot be proven to be true nor false. In 1938, Godel proved that the continuum hypothesis is consistent with ZFC. In 1963, Colen proved that the negation of the continuum hypothesis is also consisten with ZFC.

\begin{notebox}
    Just because you can write a description of a set does not mean that the set exists nor makes sense.
    \begin{itemize}
        \item For example, let $A \coloneq \left\{ \bigcup \{ B : B\ \text{is a set} \} \right\}$. Then $\mathcal{P}(A) \subseteq A$, so $\abs{\mathcal{P}(A)} \leq A < \abs{\mathcal{P}(A)}$.
        \item Another example: let $B \coloneq \{\text{all sets}\}$. Let $C \coloneq \{ A : A \notin A \}$. Is $C \in C$?
    \end{itemize}
\end{notebox}


\amzindex
\end{document}
