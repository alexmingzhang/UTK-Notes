\chapter{Introduction}

TODO: Pentagon rotation and mirroring example

\section{Relations}

\begin{dfnbox}{Relation}{}
    Let $A$ and $B$ be sets.
    \begin{itemize}
        \item A \dfntxt{relation} from $A$ to $B$ is a subset of the Cartesian product $A \times B$.
        \item A \dfntxt{relation} on $A$ is a subset of the Cartesian product $A \times A$.
    \end{itemize}
    Given a relation $\rho$, we denote $(a,b) \in \rho$ as $a \mathrel{\rho} b$. If $(a,b) \notin \rho$, we write $a \mathrel{\not\rho} b$.
\end{dfnbox}

\begin{dfnbox}{Reflexive, symmetric, transitive, equivalence relation}{}
    Let $\rho$ be a relation on a set $A$.
    \begin{itemize}
        \item $\rho$ is \dfntxt{reflexive} if, for any $a \in A$, $a \mathrel{\rho} a$.
        \item $\rho$ is \dfntxt{symmetric} if $a \mathrel{\rho} b$ implies $b \mathrel{\rho} a$.
        \item $\rho$ is \dfntxt{transitive} if, whenever $a \mathrel{\rho} b$ and $b \mathrel{\rho} c$, we have $a \mathrel{\rho} c$.
    \end{itemize}
    If $\rho$ satisfies all three properties, it is called an \dfntxt{equivalence relation}. We often use $\sim$ to denote an equivalence relation.
\end{dfnbox}

\begin{dfnbox}{Equivalence class}{}
    Let $\sim$ be an equivalence relation on a set $A$, and let $a \in A$. The \dfntxt{equivalence class} of $a$ is a set defined as:
    \[ [a] \coloneq \{b \in A : a \sim b \} \]
\end{dfnbox}

\section{Functions}

\begin{dfnbox}{Function}{}
    Let $X$ and $Y$ be sets. A \dfntxt{function} from $X$ to $Y$ is a relation $f$ from $X$ to $Y$ such that, for each $x \in X$, there exists exactly one $y \in Y$ where $x \mathrel{f} y$. We write $f : X \to Y$ to mean $f$ is a function from $X$ to $Y$, and we write $f(x) = y$ to mean $x \mathrel{f} y$.
\end{dfnbox}

\begin{dfnbox}{Injective, surjective, bijective}{}
    Let $f : X \to Y$ be a function.
    \begin{itemize}
        \item $f$ is \dfntxt{injective} if, for all $x_1$ and $x_2$ where $x_1 \neq x_2$, we have $f(x_1) \neq f(x_2)$.
        \item $f$ is \dfntxt{surjective} if, for all $y \in Y$, there exists $x \in X$ such that $f(x) = y$.
        \item $f$ is \dfntxt{bijective} if it is both injective and surjective.
    \end{itemize}
\end{dfnbox}

\begin{dfnbox}{Permutation}{}
    A \dfntxt{permutation} of a set $A$ is a function from $A$ to $A$.
\end{dfnbox}

\begin{dfnbox}{Binary operation}{}
    A \dfntxt{binary operation} on a set $A$ is a function from $A \times A$ to $A$.
\end{dfnbox}

Wowzers
