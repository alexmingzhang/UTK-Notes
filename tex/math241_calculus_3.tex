\documentclass[12pt]{report}

\usepackage[margin=1in]{geometry}
\usepackage{amzmath}
\usepackage{enumitem}
\usepackage{xfrac}

\title{\textbf{Calculus III}\\
\large UT Knoxville, Spring 2023, MATH 341}
\author{Stella Thistlethwaite, Alex Zhang}

\begin{document}
\maketitle
\tableofcontents

\chapter{Introduction}
Much of our focus will be on Stoke's Theorem.

\chapter{Three-Dimensional Space}
In past math classes, we have been used to dealing in $\R^2$ where we work with two degrees of freedom: $x$ and $y$. Now, we will be working in $\R^3$ with three degrees of freedom: $x$, $y$, and $z$.

\section{Points}

\begin{dfnbox}{Point}{point}
    A \dfntxt{point} in $\R^n$ space is an $n$-tuple that specifies a location in that space.
    \tcblower
    \[ p = (p_1, \ldots, p_n) \in \R^n \]
\end{dfnbox}

\begin{dfnbox}{Distance}{}
    Given two points $a, b\in \R^n$, the \dfntxt{distance} between the two points is defined as:
    \[ d(a, b) \coloneq \sqrt{ (b_1 - a_1)^2 + \cdots + (b_n - a_n)^2} \]
\end{dfnbox}

\begin{exbox}{Distance Between Points}{}
    Find the distance between $p_1 = (-1, -1, 4)$ and $p_2 = (-1, 4, -1)$.
    \tcblower
    \begin{align*}
         d(p_1, p_2) &= \sqrt{ (-1-(-1))^2 + (4-(-1))^2 + (-1-1)^2 } \\
         &= \sqrt{ 0^2 + 5^2 + (-5)^2 } \\
         &= \sqrt{ 50 }
    \end{align*}
\end{exbox}

\begin{dfnbox}{Sphere}{sphere}
    Given a point $c = (h, k, l) \in \R^3$, a \dfntxt{sphere} is the set of all points $(x,y,z) \in \R^3$ that are a distance $r$ from the point $c = (h,k,l)$.
    \tcblower
    \[ (x-h)^2 + (y-k)^2 + (z-l)^2 = r^2 \]
\end{dfnbox}

Note that all the points of the sphere are equidistant to the center of the sphere. This means the sphere is really a hollow shell.

\begin{exbox}{Circle}{}
    Show that the following quadratic equation represents a circle by rewriting it in standard form. Find the center $c = (h,k)$ and the radius $r$.
    \[ x^2 + y^2 + x = 0 \]
    \tcblower
    To solve this, we will have to complete the square:
    \begin{align*}
        x^2 + x + y^2 &= 0 \\
        \implies x^2 + x + \frac{1}{4} + y^2 &= \frac{1}{4} \\
        \implies \left( x + \frac12 \right)^2 + y^2 &= \frac14
    \end{align*}
\end{exbox}

\begin{dfnbox}{Cylinder}{}
    Given a planar curve $c$, the surface in $\R^3$ defined by all parallel lines crossing the curve $c$ is called a \dfntxt{cylinder}.
\end{dfnbox}

Similarly, the set of all points $(x,y,z)$ such that $x$ and $y$ satisfy $(x-h)^2 + (y-k)^2 = r^2$ forms a circular cylinder. Note that our broad definition of cylinder does not require the cylinder to be circular.

Regarding the quadrants:
\begin{align*}
    Q1 &= \left\{ (x,y) \in \R^2 : x > 0, y > 0 \right\} \\
    Q4 &= \left\{ (x,y) \in \R^2 : x > 0, y < 0 \right\}
\end{align*}


\section{Vectors}

\begin{dfnbox}{Vector}{vector}
    A \dfntxt{vector} is a mathematical object that contains multiple objects of the same type.
    \tcblower
    \[ \overrightarrow{v} = \alg{v_1, \ldots, v_n} \in \R^n \]
\end{dfnbox}

As customary in most mathematics textbooks, we will always denote vectors using the little arrow thing. In the context of three-dimensional space, we will only be working with vectors with three components. In addition, we will think of vectors as having a magnitude and direction.

\begin{dfnbox}{Scalar Multiplication}{}
    Given a vector $\vec{v}$ and scalar $k$, we define \dfntxt{scalar multiplication} as:
    \[ k \cdot \vec{v} \coloneq \alg{kv_1, \ldots, kv_n} \]
\end{dfnbox}

Note that scalar multiplication is associative, commutative, and distributive.
\begin{itemize}
    \item $a(b\vec{v}) = b(a(\vec{v})) = (ab) \vec{v}$
    \item $(k_1 + k_2) \vec{v} = k_1 \vec{v} + k_2 \vec{v}$
    \item $k (\vec{v} + \vec{w}) = k{v} + k\vec{w}$
\end{itemize}

\begin{dfnbox}{Norm}{}
    A vector's \dfntxt{norm} is its magnitude or length.
    \tcblower
    \[ \norm{v} \coloneq \sqrt{v_1^2 + \cdots + v_n^2} \]
\end{dfnbox}

\begin{dfnbox}{Unit Vector}{}
    A \dfntxt{unit vector} is a vector whose magnitude is 1.
\end{dfnbox}

We will introduce shorthand notation for the three standard unit vectors:
\begin{itemize}
    \item $\hat{i} \coloneq \alg{1,0,0}$
    \item $\hat{j} \coloneq \alg{0,1,0}$
    \item $\hat{k} \coloneq \alg{0,0,1}$
\end{itemize}
These three vectors form the \dfntxt{standard basis} for $\R^3$. That is, we can express any vector in $\R^3$ as a linear combination of $\hat{i}, \hat{j}, \hat{k}$.

\begin{tecbox}{Finding a Unit Vector from a Given Vector}{}
    Given a vector $\vec{v} = \alg{x, y, z} \in \R^3$, we can find the \dfntxt{unit vector} $\vec{u}$ with the same direction by:
    \[ \vec{u} = \frac{\vec{v}}{\norm{v}} = \alg{\frac{x}{\norm{v}}, \frac{y}{\norm{v}}, \frac{z}{\norm{v}}} \]
\end{tecbox}

\begin{dfnbox}{Dot Product}{}
    Given two vectors $\vec{a}$ and $\vec{b}$ whose cardinality are both $n$, we define the \dfntxt{dot product} of $\vec{a}$ and $\vec{b}$ as:
    \[ \vec{a} \cdot \vec{b} \coloneq a_1b_1 + \cdots + a_nb_n \]
\end{dfnbox}

Like scalar multiplication, dot product is also associative, commutative, and distributive.

\begin{thmbox}{Angle Between Vectors}{}
    If $\vec{a}$ and $\vec{b}$ are vectors and $\theta$ is the angle between $\vec{a}$ and $\vec{b}$, then:
    \[ \vec{a} \cdot \vec{b} = \norm{\vec{a}} \ \norm{\vec{b}} \cdot \cos(\theta) \]
    \tcblower
    \begin{proof}
        TODO: finish proof
    \end{proof}
\end{thmbox}

\begin{dfnbox}{Parallel, Perpendicular}{}
    \begin{itemize}[noitemsep]
        \item Two vectors are \dfntxt{parallel} if the angle between the vectors is $0\deg$.
        \item Two vectors are \dfntxt{perpendicular} if the angle between the vectors is $90\deg$.
    \end{itemize}
\end{dfnbox}

\begin{dfnbox}{Orthogonal}{}
    $\vec{a}$ and $\vec{b}$ are \dfntxt{orthogonal} if $\vec{a} \cdot \vec{b} = 0$.
\end{dfnbox}

Given a vector $\vec{a} = \alg{a_1, a_2, a_3}$, we have:

\[ \frac{\vec{a}}{\norm{a}} = \alg{\cos \alpha, \cos \beta, \cos \gamma} \]
where:
\begin{itemize}
    \item $\alpha = \cos^{-1} \left( \frac{a_1}{\norm{\vec{a}}} \right) \quad$ (angle between $\vec{a}$ and the $x$-axis)
    \item $\beta = \cos^{-1} \left( \frac{a_2}{\norm{\vec{a}}} \right) \quad$ (angle between $\vec{a}$ and the $y$-axis)
    \item $\beta = \cos^{-1} \left( \frac{a_3}{\norm{\vec{a}}} \right) \quad$ (angle between $\vec{a}$ and the $z$-axis)
\end{itemize}

\begin{dfnbox}{Work}{}
    If $F$ is a force moving a particle from a point $p$ to a point $q$, the \dfntxt{work} performed by the force is given by:
    \[ W = \vec{F} \cdot \overrightarrow{PQ} \]
\end{dfnbox}

\begin{exbox}{Finding Work}{}
    Find the work done by a force $\vec{F} = \alg{3,4,5}$ in moving an object from $p = (2,1,0)$ to $q = (4,6,2)$.
    \tcblower
    First, we find $\overrightarrow{pq}$ as such:
    \begin{align*}
        \overrightarrow{pq}
        &= \alg{4-2, 6-1, 2-0} \\
        &= \alg{2,5,2}
    \end{align*}
    Then, we can find work:
    \begin{align*}
        W
        &= \vec{F} \cdot \overrightarrow{PQ} \\
        &= \alg{3,4,5} \cdot \alg{2,5,2} \\
        &= 6 + 20 + 10 \\
        &= 36
    \end{align*}
\end{exbox}

\makeamzindex
\end{document}
