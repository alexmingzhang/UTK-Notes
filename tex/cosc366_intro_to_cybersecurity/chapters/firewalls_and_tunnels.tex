\chapter{Firewalls and Tunnels}

\section{Networking}

\begin{dfnbox}{Datagram, Packet}{}
    Networked data is delivered as a series of \dfntxt{datagrams} (or \dfntxt{packets}) which are each composed of:
    \begin{itemize}
        \item a \dfntxt{header} that provides information to help route the datagram, and
        \item a \dfntxt{payload} that contains the actual data transmitted.
    \end{itemize}
\end{dfnbox}

If a datagram is too large, it can be broken into fragments.

\subsection*{Layer 1: The Physical Layer}
This is the physical connection between computers that actually sends data between computers. This is nearly always a CAT cable that uses copper wire to transmit electrical signals representing bits. Different cat cables provide different levels of speed.

\subsection*{Layer 2: Data Link}
The data link defines how data is transferred over a physical medium. This converts layer 2 datagrams into electrical signals. Most devices use the ethernet protocol where devices are identified using a MAC address.

\subsection*{Layer 3: Network Layer}

\begin{dfnbox}{Packet-Switched Networking, Hop}{}
    \dfntxt{Packet-switched networking} refers to a networking protocol where packets are sent via a series of connected devices. Each stop on the pathway between sender and receiver is called a \dfntxt{hop}.
\end{dfnbox}

This means routes are not negotiated beforehand! Every packet has to ``find'' its route to its destination.

\begin{notebox}
    The \texttt{traceroute} program can be used to see the hops between your device and the specified server.
\end{notebox}

\begin{dfnbox}{Internet Control Message Protocol (ICMP)}{}
    \dfntxt{ICMP} is a protocol used to communicate among devices on the same network about problems with data transmission
\end{dfnbox}

\subsection*{Layer 4: Transport Layer}

\begin{dfnbox}{TCP}{}
    \dfntxt{TCP} simulates a connection over a packet-switched network for bidirectional communication. It provides reliable and in-order packet delivery at the cost of some performance.
\end{dfnbox}

TCP sockets are bound ot individual connections defined by the tuple (local IP, local port, remote IP, remote port). Servers have a well-defined server-side port, but clients usually use a temporary/random port for their side of the connection. Servers accept connections using a server socket which results in the creation of a normal TCP socket for each client connection.

To establish a connection, a three-way handshake is used:
\begin{enumerate}
    \item \texttt{SYN} packet sent by the client
    \item \texttt{SYN-ACK} packet sent by the server acknowladging the first packet
    \item \texttt{ACK} packet sent by client acknowledging the second packet
\end{enumerate}

To confirm that data is received by the other side, all packets will set the ``acknowledged'' flag, denoted \texttt{ACK}. Any ``holes'' in the information are filled by the TCP protocol.

\begin{dfnbox}{UDP}{}
    \dfntxt{UDP} is a connectionless, one-directional network protocol that does not guarantee reliability or in-order packets.
\end{dfnbox}

\subsection*{Layer 5: Application}
This is where the application handles the data. Many have multiple sublayers (e.g. TLS is layered below HTTP to create HTTPS).

\subsection*{IPSec}
Initially, IP was created without security in mind. It did not guarantee confidentiality, integrity, nor authentication. \dfntxt{IPSEC} was initially designed to secure IPv6 and was later backported to IPv4.

\section{Firewall}

\begin{dfnbox}{Firewall}{}
    A \dfntxt{firewall} is some mechanism that inspects traffic entering of leaving a device in the network. It can filter \dfntxt{inbound packets} to protect against external adversaries, and it can filter \dfntxt{outbound packets} to protect against malicious insiders.
\end{dfnbox}

\subsection*{Packet Filters}
The most common type of firewall is a \dfntxt{packet filter}, which filters traffic at a packet-level based on a list of rules. The most common actions are \texttt{ALLOW}, \texttt{DROP} (don't allow the packet), or \texttt{REJECT} (drop and notify the sender). In practice, \texttt{REJECT} is rarely used.

Some limitations include:
\begin{itemize}
    \item Requires a true perimiter (can be fixed by zero-trust networking)
    \item Does not protect against lateral movement within the network
    \item Does not protect against malicious content from a benign connection (e.g. cross-site scripting attack)
    \item Can be circumvented by tunneling malicious traffic through a benign connection
\end{itemize}

\subsection*{Proxy Firewall}
A \dfntxt{proxy firewall} is a proxy that is an intermediary connection between two hosts. It can then pass all traffic to an app-level filter for inspection. It can view all traffic, including encrypted traffic. In addition to allowing or denying traffic, it can even modify traffic to possibly add or remove functionality. To any user, the proxy can be entirely transparent.

\section{Tunnels}

Tunneling is the process of transporting packets from one protocol and transmitting them using another protocol. So far, we've covered TLS and will soon explore SSH.

\subsection*{Secure Shell Protocol (SSH)}

\begin{dfnbox}{SSH}{}
    \dfntxt{SSH} is an application-layer protocol that adds confidentiality and integrity to other application-layer protocols.
\end{dfnbox}

At a high level, the remote host runs an SSH server which relays all inputs to some program (most commonly the server's shell). All traffic goes through an encrypted tunnel. A single SSH connection is \dfntxt{multiplexed}, meaning it can be used to handle multiple protocols. It also supports port forwarding (e.g. if a server hosts a website on port 8080, we can use SSH to redirect all traffic from our port 8080 to the server's port 8080).

When establishing an encrypted tunnel, we always want to authenticate the server. Each server has a unique SSH host key which identifies the system.\footnote{It's also possible to use PKI for SSH host keys, although rarely used in practice.} Most of the time, the client validates the key by trust on first use (TOFU). Then, if the host's SSH key has changed, the connection is rejected.

The server also needs to authenticate the user, most commonly by password and/or public key. Passwords are not the preferred method as they are susceptible to phishing (if connected to the wrong server) and are generally less secure. Much more secure is the public-key, which conducts a challenge-response protocol to verify ownership of the private key. After the first login, the user's public key is listed as an authorized key on the server. The user's private key should be password-protected on the client.

\subsection*{Virtual Private Networks}

Traditionally, computers are physically networked together. The boundary to a network is well-defined, and computers within that boundary are considered trusted. Incoming and outgoing traffic across that boundary is monitored in our firewall.

Obviously, the limitation is that the computers have to be physically connected. Instead, we can tunnel all computer networking through a remote network.

\begin{dfnbox}{Virtual Private Network (VPN)}{}

\end{dfnbox}

VPNs do not protect against malicious insiders nor lateral movement in the network. As such, VPNs are slowly being replaced by \dfntxt{zero-trust networking} which works using a whitelist. All connections are distrusted by default unless explicitly authenticated.
