\chapter{Blockchain}

Nowadays, we typically associate the term ``blockchain'' 

\section{Introduction}

\begin{dfnbox}{Blockchain}{}
    \dfntxt{Blockchain} is a database technology with three key properties:
    \begin{enumerate}
        \item Cryptographic append-only ledger, which stores the full history of all transactions
        \item Replication
        \item Distributed operation (i.e. decentralized)
    \end{enumerate}
\end{dfnbox}

\paragraph{Cryptographic, append-only ledger}
A blockchain's ledger uses cryptography to guarantee its integrity. It does so via an \dfntxt{Authenticated Data Structure (ADS)}, where modifications to the data can be detected (but does not necessarily prevent modification of the ADS). An ADS produces a \dfntxt{verifier} that can be used to verify the data hasn't been changed since the verifier was produced. This requires access to the ADS to verify its integrity. In some ADS systems, the verifier can also be used to create proofs of inclusion and non-inclusion. In this case, access to the ADS is not required.

\begin{dfnbox}{Hash Chain}{}
    A \dfntxt{hash chain} is an authenticated data structure based on singly linked lists where each node stores:
    \begin{itemize}
        \item its data,
        \item a pointer to the previous item, and
        \item a hash value calculated based on the value of the current item and the hash value of the previous item; the first item uses a hard-coded value, and is sometimes called the \dfntxt{genesis item}
    \end{itemize}
\end{dfnbox}

Items in the hash chain that store transactions are referred to as \dfntxt{blocks}, hence the term \dfntxt{blockchain}.

\paragraph{Replication}

Multiple entities store the blockchain ledger in its entirety as well as the verifier. If one entity modifies their ledger, the others can detect that change because their verifier values will no longer match. If the modification was malicious, the modified ledger can be restored from the replicated copies.

\paragraph{Distributed Operation}
Multiple entities operate the system, often referred to as \dfntxt{miners}. Each miner replicates the full blockchain, operating individually but verify and replicate each other.

To add a new block, miners undergo the following procedure:
\begin{enumerate}
    \item An individual miner will first add a new block to their personal copy of the blockchain. The miner generates a list of transactions to add to a blockchain. They verify the legitimacy of those transactions and then create a block with those transactions. That miner adds the block to their copy of the blockchain.
    \item Next, that miner announces the new block to the other miners. Each other miner verifies the legitimacy of the transactions in the new block. The other miners then add the new block to their copy of the blockchain.
\end{enumerate}

In public blockchains, anybody can be a miner. Each miner votes on which blocks should be added. This requires a mechanism to fairly allocate votes.

In permissioned blockchains, the identities of the miners are defined by the system. Blocks are added directly without voting. Hence, it is more efficient than open operation.

\paragraph{Miscellaneous Features}

Some blockchains also include:
\begin{itemize}
    \item Smart contracts: algorithms stored on the blockchain that are run by miners when triggered by certain transactions. These can change the state of data on the blockchain and are subject to validation like transactions. They're commonly used to automate contractual processes.
    \item Non-fungible tokens (NFTs): cryptographic tokens representing physical or digital assets
\end{itemize}

\paragraph{Challenges} Blockchain systems are severely limited in terms of:
\begin{itemize}
    \item Scalability: blockchain systems add significant overhead.
    \item On-chain correctness: it's hard to ensure that the digital blockchain fully represents the real world. What about counter-party risk, and how should bugs be handled when everything in a blockchain is permanent?
    \item Regulatory compliance: often, performance benefits of blockchain come from ignoring regulation
    \item Security and privacy: blockchain systems can have software vulnerabilities; public blockchains can be attacked by malicious entities
    \item Usability: users need to manage cryptographic keys, but key management is hard. Smart contracts need to be perfect, but development tools can't verify this.
\end{itemize}

\section{Bitcoin}

\dfntxt{Bitcoin} is a decentralized, pseudonymous cryptocurrency. A blockchain is used to store bitcoin transactions.

Each transaction stores the following information:
\begin{itemize}
    \item List of inputs: the bitcoin that will be burned
    \item List of outputs: The bitcoin that results from the transaction, which is less than or equal in value to the input bitcoin
    \item Transaction hash: generated based on the transaction's data
\end{itemize}

\paragraph{Proof of Work} To add a block, miners race to create a block with a valid hash. What if two transactions try to spend the same bitcoin? The miners get to vote, with voting power determined by computational power. This is referred to as \dfntxt{proof of work}. For public blockchains, proof-of-X protocols are used to fairly allocate votes within governance and operation. 

\section{Blockchain Myths}

\paragraph{``Blockchain is always public.''} Blockchains can be public. Openness comes with additional overhead, such as the proof of work algorithms. Permissioned blockchains limit who can participate. Membership can be static or dynamic, and the ledger itself can be public or private.

\paragraph{``Blockchain is fast and efficient.''} Blockchain has immense overhead. Ledgers store the entire history of every bitcoin transaction, and replication itself has a high storage cost. Decentralized operation is much slower than centralized operation. It's not the technology that's faster; it's the ignoring of regulation that's faster.

Systems using blockchain often claim to have increased efficiency. For example, many claim that it is faster to send money internationally using Bitcoin rather than through a traditional bank. This efficiency usually comes from those systems ignoring regulations and/or compliance. For example, Visa's network is much faster than Bitcoin for sending money, but that money won't be sent by Visa until proper regulatory compliance steps are taken, while Bitcoin payments ignore regulation.

\paragraph{``Blockchain is immutable.''} The ledger is in fact mutable. Note that miners themselves cannot change each other's copy of data, but miners can collectively agree to change the data. If only some agree, the blockchain can \dfntxt{fork} to create two blockchains with different ledgers. This is what happened with Ethereum and Ethereum Classic. These changes may be detectable by non-miners, so long as the ledger is publicly available.

\paragraph{``Blockchain is trustless.''} Although blockchain is in fact decentralized, we have only shifted our trust from a centralized entity to the bitcoin miners. This is better, but we still need to trust the software behind bitcoin, trust the connection between on-chain and off-chain assets, and trust the miners to vote fairly in transactions.

\paragraph{``Blockchain removes the need for trusted third-parties.''}
 Trusted third-parties are still needed to mediate between the real-world and on-chain ledger. For example:
 \begin{itemize}
    \item currency exchanges that convert on-chain assets to real-world money
    \item product vendors that convert on-chain assets to real-world products
    \item entities that report real-world events on the chain
    \item entities that handle dispute resolution (e.g. miners)
 \end{itemize}

\paragraph{``Smart contracts are novel and innovative.''} Smart contracts are just software attached to the blockchain database. Transactions provide smart contracts with input. Their output is used to update state stored on the blockchain. The new state is validated by the miners. Any database can have smart contract equivalents.

Admittedly, companies are using smart contracts in innovative ways. For example:
\begin{itemize}
    \item Decentralized autonamous organizations (DAOs)
    \item Automated supply chain tracking
\end{itemize}

\paragraph{``Blockchain is wasteful.''} In reality, it's only the proof-of-work algorithms which are wasteful. Public blockchains can simply use other proof-of-X protocols. Permissioned blockchains don't need to use proof-of-X protocols.

\section{Use Cases}
\paragraph{Cryptocurrency} Fiat currencies are managed by central banks. These are susceptible to inflation, seizure, and general poor management. Cryptocurrencies decentralize the management of currency, removing the need for a central bank. They have public, set-in-stone rules are in place for minting additional currency, so inflation is entirely deterministic.

\paragraph{Electronic Payments} Payments are being processed by just four corporations: Visa, mastercard, discover, and American Express. Government regulations can cause significant delays (e.g. international transactions can take days to clear). Blockchain payments should be near-instantaneous.

\paragraph{Supply Chain Management} We can track products on the blockchain, and automate contracts/payments using smart contracts. Transaction history provides a permanent record, allowing for things like root-cause analysis and taint tracking. Moreover, the permanent record supports conflict resolution. This approach also removes the need for centralized infrastructure (which is nice because many producers don't trust the man).

\paragraph{Multi-Organization Data Sharing} Blockchains can act as sharing mediators to help people find data, validate access to the data, and automate things using smart contracts.

\paragraph{Voting} This is actually a terrible idea. Physical voting is always preferred to digital voting.
