\chapter{Security Concepts and Principles}

\section{Fundamental Goals of Computer Security}

\begin{dfnbox}{Computer Security}{computer-security}
    \dfntxt{Computer security} is the practice of protecting computer-related assets from unauthorized actions, either by preventing such actions or detecting and recovering from them.
\end{dfnbox}

The goal of \nameref{dfn:computer-security} is to help users complete their desired task safely, without short or long term risks.  To do this, we support computer-based services by providing essential security properties.

\begin{dfnbox}{Confidentiality, Integrity, Availablity (CIA)}{cia}
    \begin{itemize}[noitemsep]
        \item \dfntxt{Confidentiality}: only authorized parties can access data, whether at rest or in motion (i.e. being transmitted)
        \item \dfntxt{Integrity}: data, software, or hardware remaining unchanged, except by authorized parties
        \item \dfntxt{Availability}: information, services, and computing resources are available for authorized use
    \end{itemize}
    Together, these form the \dfntxt{CIA triad}.
\end{dfnbox}

\begin{dfnbox}{Principal, Privilege}{}
    A \dfntxt{principal} is an entity with a given identity, such as a user, service, or system process. A \dfntxt{privilege} defines what a principal can do, such as read/write/execute permissions.
\end{dfnbox}

In addition to the CIA triad, we also have three security properties relating to principals.
\begin{dfnbox}{Authentication, Authorization, Auditability (Golden Principles)}{}
    The \dfntxt{Golden Principles} are three security properties regarding principals:

    \begin{itemize}[noitemsep]
        \item \dfntxt{Authentication:} assurance that a principal is who they say they are
        \item \dfntxt{Authorization:} determining whether a requested privilege or resource access should be granted to the requesting principal
        % \item \dfntxt{Authorization}: proof that an entity has the necessary privilege to take the request action; most commonly done by authenticating a principal, and lookup up its privileges
        \item \dfntxt{Auditability:} ability to identify principals responsible for past actions
    \end{itemize}

\end{dfnbox}

Auditability (or accountability) gives away two key things: who conducted the attack and the methods by which an attack was made.

\begin{dfnbox}{Trustworthy, Trusted}{}
    Something is \dfntxt{trustworthy} if it deserves our confidence.     Something is \dfntxt{trusted} if it has our confidence.

\end{dfnbox}

\begin{dfnbox}{Privacy, Confidentiality}{}
    \dfntxt{Privacy} is a sense of being in control of access that others have to ourselves. It deals exclusively with people. \dfntxt{Confidentiality} is an extension of privacy to also include personally sensitive data.
\end{dfnbox}

\section{Computer security policies and attacks}

\begin{dfnbox}{Asset}{asset}
    An \dfntxt{asset} is a resource we want to protect, such as information, software, hardware, or computing/communications services.
\end{dfnbox}

Note that asset can refer to any tangible or intangible resources.

\begin{dfnbox}{Security Policy}{}
    A \dfntxt{security policy} specifies the design intent of a system's rules and practices (i.e. what the system is supposed to do and not do).
\end{dfnbox}

\begin{dfnbox}{Adversary}{}
    An \dfntxt{adversary} is an entity who wants to violate a security policy to harm an asset. Can also be called ``threat agents'' or ``threat actors''. Some attributes of an adversary include:
    \begin{itemize}[noitemsep]
        \item \dfntxt{identity}: who are they?
        \item \dfntxt{objectives}: what assets the adversary might try to harm
        \item \dfntxt{methods}: the potential attack techniques or types of attacks
        \item \dfntxt{capabilities}: skills, knowledge, personnel, and opportunity
    \end{itemize}
\end{dfnbox}

A formal security policy should precisely define each possible system state as either authorized (secure) or unauthorized (non-secure). Non-secure states raise the potential for attacks to happen.

\begin{dfnbox}{Threat}{}
    A \dfntxt{threat} is any combination of circumstances and/or entities that allow harm to assets or cause security violations.
\end{dfnbox}

\begin{dfnbox}{Attack, Attack Vector}{attack}
    An \dfntxt{attack} is a deliberate attempt to cause a security violation. An \dfntxt{attack vector} is the specific methods and steps by which an attack was executed.

\end{dfnbox}

\begin{dfnbox}{Mitigation}{mitigation}
    \dfntxt{Mitigation} describes countermeasures to reduce the chance of a threat being actualized or lessen the cost of a successful attack.
\end{dfnbox}

\nameref{dfn:mitigation} can include operational and management processes, software controls, and other security mechanisms.

\begin{exbox}{House Security Policy}{house}
    Consider this simple security policy for a house: no one is allowed in the house unless accompanied by a family member, and only family members are authorized to take things out of the house.
    \begin{itemize}[noitemsep]
        \item The presence of someone who wants to steal an asset from our house is a \textbf{threat}.
        \item An unaccompanied stranger in the house is a \textbf{security violation}.
        \item An unlocked door is a \textbf{vulnerability}.
        \item A stranger entering through the unlocked door and stealing a television is an \textbf{attack}.
        \item Entry through the unlocked door is an \textbf{attack vector}.
    \end{itemize}
\end{exbox}

\section{Risk, risk assessment, and modeling expected losses}

\begin{dfnbox}{Risk}{risk}
    A \dfntxt{risk} is the expected loss of assets due to future attacks.
\end{dfnbox}

There are two ways we assess risk: \dfntxt{quantitative} and \dfntxt{qualitative} risk assessment.
\begin{itemize}
    \item \dfntxt{Quantitative} risk assessment computes numerical estimate of risk
    \item \dfntxt{Qualitative} risk assessment compares risks relative to each other
\end{itemize}

Quantitative risk assessment is more suited for incidents that occur regularly, with historical data and stable statistics to generate probability estimates. However, computer security incidents occur so infrequently that any estimate of probability likely isn't precise. Thus, qualitative risk assessment is usually more practical. For each asset or asset class, their relevant threats are categorized based on probability of happening and impact if it happened.

Precise estimates of risk are rarely possible in practice, so qualitative risk assessment is usually yields more informed decisions. A popular equation for modeling risk is:
\[ R = T \cdot V \cdot C \]
where:
\begin{itemize}
    \item $T$ is the probability of an attack happening,
    \item $V$ is the probability such a vulnerability exists, and
    \item $C$ is cost of a successful attack, both tangible and intangible costs
\end{itemize}

$C$ can encompass tangible losses like money or intangible losses like reputation. Whatever model we use, we can then model expected losses as such:

In risk assessment, we ask ourselves these questions:
\begin{enumerate}
    \item What assets are most valuable, and what are their values?
    \item What system vulnerabilities exist?
    \item
\end{enumerate}

In answering these questions, it becomes apparant we cannot employ strictly quantitative risk assessment. A popular model for qualitative risk assessment is the \nameref{dfn:dread} model:

\begin{dfnbox}{DREAD}{dread}
    \dfntxt{DREAD} is a method of qualitative risk assessment using a subjective scaled rating system for five attributes.

    \begin{center}\begin{tabular}{r | l | l}
        Attribute & 10 & 1 \\ \hline
        \dfntxt{Damage Potential} & data is extremely sensitive & data is worthless \\
        \dfntxt{Reproducibility} & works every time & works only once \\
        \dfntxt{Exploitability} & anyone can mount an attack & requires a nation state \\
        \dfntxt{Affected Users} & 91-100\% of users & 0\% of users \\
        \dfntxt{Discoverability} & threat is obviously apparent & threat is undetectable
    \end{tabular}\end{center}

    The final DREAD score takes the average of all five attributes.
\end{dfnbox}

A common criticism of \nameref{dfn:dread} is that rating discoverability might reward security through obscurity. People will sometimes omit discoverability, or simply assign it the maximum value all the time.


\[ ALE = \sum_{i=1}^{n} F_i \cdot C_i \]
where:
\begin{itemize}
    \item $F_i$ is the estimated frequency of events of type $i$, and
    \item $C_i$ is the average loss expected per occurence of an event of type $i$
\end{itemize}

\section{Adversary modeling and security analysis}

% Commented because we already defined adversary in a previous section
% \begin{dfnbox}{Adversary}{}
%     An \dfntxt{adversary} is an entity that wants to violate a security policy in order to harm an asset.
%     \begin{itemize}[noitemsep]
%         \item \dfntxt{identity}: who are they?
%         \item \dfntxt{objectives}: what assets the adversary might try to harm
%         \item \dfntxt{methods}: the potential attack techniques or types of attacks
%         \item \dfntxt{capabilities}: skills, knowledge, personnel, and opportunity
%         % \item funding level -- how much money an attacker has
%         % \item outsider or insider -- whether or not the adversary may have special permissions or privileges
%     \end{itemize}
% \end{dfnbox}

In designing computer security mechanisms, it is important to think like an adversary. Try to enumerate what methods might they use, and design around them. Different adversaries can have wildly different objectives, methods, and capabilities.

\begin{dfnbox}{Security Analysis}{}
    \dfntxt{Security analysis} aims to identify vulnerabilities and overlooked threats, as well as ways to improve defense against such threats
\end{dfnbox}

Types of analysis include:
\begin{itemize}[noitemsep]
    \item \dfntxt{Formal security evaluation}: standardized testing to ensure essential features and security
    \item \dfntxt{Internal vulnerability testing}: internal team trying to find vulnerabilities
    \item \dfntxt{External penetration testing}: third-party audits to find vulnerabilities; often the most effective
\end{itemize}

Security analysis is a heavily involved process that may employ a variety of methodologies. For example, manual source code review, review of design documents, and various penetration testing techniques. It's important to be aware of potential vulnerabilities as we are writing code.

\section{Threat Modeling}
A threat model will identify possible adversaries, threats, and attack vectors. We need to list a set of assumptions about the threats as well as clarify what is in and out of the scope of possibilities.

\paragraph{Diagram-Driven Threat Model}
Threat model diagrams include assets, infrastructure, and defenses/mitigations, making apparent the possible attack vectors. Popular examples include:
\begin{itemize}[noitemsep]
    \item \dfntxt{Data Flow Diagrams:} models all possible data routes
    \item \dfntxt{User Workflow Diagram:} models how users interact with the program, both frontend and backend
    \item \dfntxt{Attack Trees:} models all possible attack vectors like a flow chart; leaf nodes are initial actions
\end{itemize}

\begin{dfnbox}{STRIDE}{stride}
    \dfntxt{STRIDE} is a model for enumerating and identifying possible computer security threats.
    \begin{itemize}[noitemsep]
        \item \dfntxt{Spoofing:} attacker impersonates another user, or malicious server posing as a legitimate server
        \item \dfntxt{Tampering:} altering data without proper authorization; can occur when data is stored, processed
        \item \dfntxt{Repudiation:} lying about past actions (e.g. denying/claiming that something happened)
        \item \dfntxt{Information Disclosure:} exposure of confidential information without authorization
        \item \dfntxt{Denial of Service:} render service unusable or unreliable for users
        \item \dfntxt{Elevation of Privilege:} an unprivileged user gains privileges
    \end{itemize}
\end{dfnbox}

\section{Threat Model Gaps}
Often, wrong assumptions about risk or focus on wrong threats will lead to gaps between our threat model and what can realistically happen (e.g. if we don't account for something, but it does happen).

\paragraph{Changing Times}
New adversaries, technologies, software, and controls means we need to constantly adjust our threat model.

\section{Design Principles}
\begin{enumerate}
    \item \dfntxt{Simplicity and necessity:} complexity increases risks (KISS: keep it simple, stupid)
    \item \dfntxt{Safe Defaults:} ensure default settings are secure; users rarely change defaults
    \item \dfntxt{Open Design:} don't rely on the secrecy of our code; just assume it will leak
    \item \dfntxt{Complete Mediation:} never assume access is safe; always check access is allowed
    \item \dfntxt{Least Privilege:} don't give principals extraneous privileges; limit impact of compromise
    \item \dfntxt{Defense in depth:} multiple layers of security; don't rely on only one security control
    \item \dfntxt{Security by design:} think about security throughout development, not an afterthought
    \item \dfntxt{Design for evolution:} allow for change for whenever it's needed
\end{enumerate}

Overall, computer security is hard! \todo{explain why tho}

% \section{Review}
% CIA Triad, Golden principles, trusted vs. trustworthy, confidentialty vs. privacy.

% TODO: add more review stuff from slides here; flesh out notes to reflect review
