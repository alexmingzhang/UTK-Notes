\documentclass[12pt]{report}

\usepackage[margin=1in]{geometry}
\usepackage[math]{amznotes}
\usepackage{enumitem}
\usepackage{xfrac}


\title{\textbf{Introduction to Analysis}\\
\large UT Knoxville, Spring 2023, MATH 341}
\author{Mike Frazier, Peter Humphries, Alex Zhang}

\begin{document}
\maketitle
\tableofcontents

\addcontentsline{toc}{chapter}{Preface}
\chapter*{Preface}
These notes attempt to give a concise overview of the \textbf{Introduction to Analysis} course at the University of Tennessee (MATH 341). The contents of these notes come from Dr. Michael Frazier's lecture notes as well as Dr. Peter Humphries' lecture notes.

The first few weeks of the class are spent reviewing content from \textbf{Introduction to Abstract Mathematics} (MATH 300). As such, there will be much repeated content. Afterwards, we focus on analysis of real functions.

\chapter{Introduction}
Our goal is to understand the theory of real functions in one variable. Specifically, we will deal with functions, limits, sequences, convergence, continuity, differentiation, and integration. The same ideas, concepts, and techniques are used to study more complicated mathematics.

We will primarily focus on the idea of \textbf{convergence}. Many computational techniques and algorithms rely on iteration---successive approximations getting closer to an actual solution. In order for those algorithms to work, they need to converge towards an actual solution.

To motivate our quest to learn about convergence, let's look at some classic iterative methods.

\begin{exbox}{Newton's Method}{}
    Given $c > 0$, suppose we want to calculate $\sqrt{c}$. Start with some initial guess $x_1 > 0$.
    \begin{alignat*}{2}
        & \text{Let}\quad & x_2 &\coloneq \frac{1}{2} \left( x_1 + \frac{c}{x_1} \right) \\
        & \text{Let}\quad & x_3 &\coloneq \frac{1}{2} \left( x_2 + \frac{c}{x_2} \right) \\
        && &\vdots \\
        & \text{Let}\quad & x_{n+1} &\coloneq \frac{1}{2} \left( x_n + \frac{c}{x_n} \right)
    \end{alignat*}
    We find that $\lim_{n\to\infty} x_n = \sqrt{c}$.
\end{exbox}

Does this method work for all $c > 0$ and $x_1 > 0$? Assuming $\lim_{n\to\infty} x_n = x$ converges, then:
\begin{alignat*}{2}
    && x_{n+1} &= \frac{1}{2} \left( x_n + \frac{c}{x_n} \right) \\
    &\implies \quad &x &= \frac{1}{2} \left( x + \frac{c}{x} \right) \\
    &\implies &2x &= x + \frac{c}{x} \\
    &\implies &x &= \frac{c}{x} \\
    &\implies &x^2 &= c \\
    &\implies &x &= \sqrt{c}
\end{alignat*}

The above calculation only makes sense if we know the sequence converges. Consider the sequence $x_{n+1} = 6 - x_n$ where $x_1 = 4$. Then:
\[ x_1 = 4,\quad x_2 = 2,\quad x_3 = 4,\quad x_4 = 2,\quad \ldots \]

% TODO: add reference for monotone convergence theorem
% TODO: explain what a bounded monotone sequence even is

Since this sequence does not converge, there is no limit when $n \to \infty$! In chapter 14, we will cover the Monotone Convergence Theorem, which states that any bounded monotone sequence converges.

\begin{exbox}{Monotone Convergence Theorem}{}
    Suppose that $c > 0$ and $x_1 > 0$. Then, for $n \geq 2$, the sequence $x_{n+1} = \frac{1}{2} \left( x-N + \frac{c}{x_n} \right)$ is:
    \begin{itemize}
        \item \textbf{bounded below} because $x_n > \sqrt{c}$ when $n \geq 2$, and
        \item \textbf{decreasing} because $x_n+1 < x_n$ for $n \geq 2$.
    \end{itemize}
    Therefore, $x_n$ converges by the Monotone Convergence Theorem, and $\lim_{n\to\infty} x_n = \sqrt{c}$.
\end{exbox}

Let's look at a more complicated iterative method.

\begin{exbox}{Picard's Method}{}
    Suppose we had to solve $y\prime = f(x,y)$ where $y(x_0) = y_0$ (i.e. find a function $y$ that satisfies our two conditions). As it turns out, we can use an iterated method to solve this as well.
    \begin{itemize}
        \item Start with an initial guess $y_1(x)$
        \item Define $\displaystyle y_{n+1}(x) \coloneq y_0 + \int_{x_0}^{x} f(t, y_n)\ dt$.
    \end{itemize}
    Provided that $f$ and $y_0$ are ``well-behaving'', then the sequence of functions $y_n(x)$ converges to the solution $y(x)$.
\end{exbox}

The idea that an infinite sequence of functions can converge suggests some notion of ``distance'' between functions. We can use a number of metrics for distance, some possibilities including:
\begin{itemize}
    \item \( \displaystyle \int_a^b \abs{f(x) - g(x)}\ dx \quad \) (total area between the two functions)
    \item \( \displaystyle \sup \left\{ x : x = \abs{f(x) - g(x)} \right\} \quad \) (max possible ``vertical'' distance between the two curves)
\end{itemize}

\chapter{Logic and Proofs}

Logic is backbone of all formal mathematics. When building a logically sound model of mathematics, we start with a small collection of axioms. We then work with those axioms to deduce other logically sound statements, reaffirming what we already knew and discovering new ideas along the way.

\section{Basic Logic}
\begin{dfnbox}{Statement}{}
    A \dfntxt{statement} is a claim that is either true or false.
    \tcblower
    \[ p : \text{some claim} \]
\end{dfnbox}

We usually denote statements with a letter like $p$. For example, we can write ``$p: x > 2$'', which means $p$ represents the statement ``$x$ is greater than $2$''. Throughout this chapter, we will use $p$ and $q$ to represent arbitrary statements.

\begin{dfnbox}{Conjunction}{conjunction}
    The \dfntxt{conjunction} of two statements is itself a statement, which is true if and only if the two statements are both true.
    \tcblower
    \[ p \land q : p\ \text{is true \textbf{and}}\ q\ \text{is true} \]
\end{dfnbox}

\begin{dfnbox}{Disjunction}{disjunction}
    The \dfntxt{disjunction} of two statements is itself a statement, which is true if and only if at least one of the statements is true.
    \tcblower
    \[ p \lor q : p\ \text{is true \textbf{or}}\ q\ \text{is true} \]
\end{dfnbox}

\nameref{dfn:conjunction} and \nameref{dfn:disjunction} follow our intuition of ``and'' and inclusive ``or'', respectively. We can visualize the two logical connectives using \dfntxt{truth tables}.

\begin{exbox}{Truth Table of Conjunction}{conjunction}
    \begin{center}\begin{tabular}{c | c || c}
        $p$ & $q$ & $p \implies q$ \\ \hline
        T & T & T \\
        T & F & F \\
        F & T & F \\
        F & F & F
    \end{tabular}\end{center}
\end{exbox}

\begin{exbox}{Truth Table of Disjunction}{disjunction}
    \begin{center}\begin{tabular}{c | c || c}
        $p$ & $q$ & $p \implies q$ \\ \hline
        T & T & T \\
        T & F & T \\
        F & T & T \\
        F & F & F
    \end{tabular}\end{center}
\end{exbox}


% TODO: add truth tables

\begin{dfnbox}{Negation}{}
    The \dfntxt{negation} of a statement is a statement with opposite truth values.
    \tcblower
    \[ \neg p \]
\end{dfnbox}

\begin{dfnbox}{Implication}{}
    An \dfntxt{implication} ``$p$ implies $q$'' states ``if $p$ is true, then $q$ is true''.
    \tcblower
    \[ p \implies q \]
\end{dfnbox}

In the implication $p \implies q$, we call $p$ the \dfntxt{hypothesis} and $q$ the \dfntxt{conclusion}. If the hypothesis is false to begin with, then the implication is not really meaningful. Instead of assigning those kinds of implications no truth value, we simply consider them true by convention. These kinds of truths are called \dfntxt{vacuous truths}.

\begin{exbox}{Truth Table of Logical Implication}{}
    \begin{center}\begin{tabular}{c | c || c}
        $p$ & $q$ & $p \implies q$ \\ \hline
        T & T & T \\
        T & F & F \\
        F & T & T \\
        F & F & T
    \end{tabular}\end{center}
\end{exbox}

\begin{exbox}{Simple Statements}{}
    Let $p : x > 2$ and $q : x^2 > 1$. Consider the following statements:
    \begin{itemize}
        \item ``For all real numbers $x$, $p \implies q$''

        \textbf{True.} If $x > 2$, then $x^2 > 1$.\footnote{This is normally where we would rigorously prove such a statement, but we will omit this for now.}

        \item ``For all real numbers $x$, $q \implies p$''

        \textbf{False.} Consider $x = 1.1$. Then $x^2 = 1.21 > 1$, but $x = 1.1 < 2$.
    \end{itemize}
\end{exbox}

\begin{dfnbox}{Logical Equivalence}{equiv}
    $p$ and $q$ are \dfntxt{logically equivalent} if $p \implies q$ and $q \implies p$.
    \tcblower
    \[ p \iff q \]
\end{dfnbox}

In other words, $p \iff q$ means that $p$ and $q$ share the same truth value. Either $p$ and $q$ are \textbf{always both true}, or $p$ and $q$ are \textbf{always both false}. Logical equivalence says nothing about the truth of $p$ and $q$ themselves.

We can also say ``$p$ if and only if $q$'' or ``$p$ iff $q$'' to denote logical equivalence.

\begin{exbox}{Truth Table of Logical Equivalence}{}
    \begin{center}\begin{tabular}{c | c || c}
        $p$ & $q$ & $p \iff q$ \\ \hline
        T & T & T \\
        T & F & F \\
        F & T & F \\
        F & F & T
    \end{tabular}\end{center}
\end{exbox}

\begin{dfnbox}{Converse}{converse}
    Given the implication $p \implies q$, its \dfntxt{converse} statement is $q \implies p$.
\end{dfnbox}

It's important to note that an implication and its converse have no intrinsic equivalence.

\begin{dfnbox}{Contrapositive}{contrapositive}
    Given the implication $p \implies q$, its \dfntxt{contrapositive} statement is $\neg q \implies \neg p$.
\end{dfnbox}

Unlike the converse, an implication and its contrapositive are logically equivalent. To help our intuition, we can construct a truth table.

\begin{exbox}{Truth Table of Contrapositive}{}
    \begin{center}\begin{tabular}{c | c || c | c | c | c}
        $p$ & $q$ & $\neg p$ & $\neg q$ & $p \implies q$ & $\neg q \implies \neg p$ \\ \hline
        T & T & F & F & T & T \\
        T & F & F & T & F & F \\
        F & T & T & F & T & T \\
        F & F & T & T & T & T
    \end{tabular}\end{center}
\end{exbox}

As we can see, no matter what the truth values of the hypothesis and conclusion are, an implication and its contrapositive always have the same truth values.

\begin{notebox}
    When constructing a truth table, we must include \textbf{all} intermediate statements, not just the final statement.
\end{notebox}

% TODO: add truth table for contrapositive


\section{Proofs and Proof Techniques}
While truth tables are a quick way to gauge whether simple statements hold, they become impractical once we involve more complicated statements. Furthermore, truth tables don't really show intuition behind complicated statements whereas proofs should ultimately fuel our intuition.

Very often, we will have to prove some implication like $p \implies q$. Recall how an implication is only false if $p$ is true but $q$ is false. Therefore, we would only have to consider that case where $p$ is true but $q$ is false. We can prove an implication is true by simply showing that such a case could never happen. There are three simple proof techniques for doing so:

\begin{enumerate}
    \item \dfntxt{Direct Proof:} Assume $p$ is true, then reason that $q$ must be true as well.
    \item \dfntxt{Proof by Contradiction:} Assume both $p$ and $\neg q$ are true, then logically derive some contradiction.
    \item \dfntxt{Proof by Contrapositive:} Assume $\neg q$ is true, then reason that $\neg p$ must be true as well.
\end{enumerate}

It's hard to decide which proof technique is easiest for any given problem. Direct proofs are often more ``enlightening'', but it can be difficult to find the appropriate logic to reach the conclusion. It may be easier to try proof by contradiction or contrapositive.

\begin{tecbox}{Proof by Contradiction}{contradiction}
    To prove $p \implies q$ by contradiction, we carry out the following steps:
    \begin{enumerate}
        \item Assume $p$ is true, and suppose for the sake of contradiction $\neg q$ is true.
        \item Logically derive a statement that contradicts something we know to be true.
        \item Ultimately conclude that if $p$ is true, then $q$ must be true.
    \end{enumerate}
\end{tecbox}

In terms of logic notation, proof by contradiction follows:
\[ \left[ \left( p \land (\neg q) \right) \implies \text{Contradiction} \right] \implies \left[ p \implies q \right]\]

\begin{exbox}{Truth Table of \nameref{tec:contradiction}}{}
    \begin{center}\begin{tabular}{c | c || c | c | c | c }
        $p$ & $q$ & $p \implies q$ & $\neg q$ & $p \land (\neg q)$ & $\neg \left[ p \land (\neg q) \right]$ \\ \hline
        T & T & T & F & F & T \\
        T & F & F & T & T & F \\
        F & T & T & F & F & T \\
        F & F & T & T & F & T
    \end{tabular}\end{center}
\end{exbox}

By the above truth table, we can safely assume the following logical equivalence:
\[ (p \implies q) \iff \neg \left[ p \land (\neg q) \right] \]

\begin{tecbox}{Proof by Contrapositive}{}
    To prove $p \implies q$ by contrapositive, we carry out the following steps:
    \begin{enumerate}
        \item Assume $\neg q$ is true.
        \item Directly prove that $\neg p$ is true.
    \end{enumerate}
\end{tecbox}

In terms of logic notation, proof by contrapositive follows:
\[ (\neg q \implies \neg p) \iff (p \implies q) \]
We can actually prove this using proof by contradiction!
\begin{exbox}{Logical Equivalence of Contrapositive}{}
    Given statements $p$ and $q$, $p \implies q$ and $\neg q \implies \neg p$ are equivalent.
    \tcblower
    \begin{proof}
        Assume $p \implies q$. To prove $\neg q \implies \neg p$, we can suppose for contradiction that $\neg q$ and $p$ are both true. But $p \implies q$, so $q$ is true which contradicts $\neg q$. Hence, the assumption that $p$ is true was incorrect. Thus, $\neg q \implies \neg p$.

        Assume $\neg q \implies \neg p$. From above, we have $\neg ( \neg p ) \implies \neg (\neg q)$, so $p \implies q$.
    \end{proof}
\end{exbox}

\begin{exbox}{Proving Simple Logic Statements}{}
    Let $p$, $q$, and $r$ be arbitrary statements. Prove that $\left[ p \implies (q \lor r) \right] \iff \left[ (p \land \neg q) \implies r \right]$.
    \tcblower
    \begin{proof}
        Assume $p \implies (q \lor r)$. Suppose $p \land \neg q$. Then $p$ is true, so $q \lor r$ is true by assumption. Also, $\neg q$ is true, so $r$ must be true from $q \lor r$.

        Assume $(p \land \neg q) \implies r$. Suppose $p$ is true. There are two possibilities:
        \begin{enumerate}
            \item If $q$ is true, then $q \lor r$ is true.
            \item If $\neg q$ is true, then $p \land \neg q$ is true. Thus, $r$ is true by assumption. Hence, $q \lor r$ is true.
        \end{enumerate}
    \end{proof}
\end{exbox}

\chapter{Naive Set Theory}
Set theory is a whole other can of worms that really isn't that meaningful right now. Instead, we will take a naive approach to sets and define them informally. That way, we can avoid the chicanery and get to what really matters.

\section{Sets}
\begin{dfnbox}{Set}{set}
    A \dfntxt{set} is a collection of distinct objects.
\end{dfnbox}

For example, $\N \coloneq \{1,2,3\ldots\}$ is the set of all natural numbers, and $\Z \coloneq \{ \ldots, 1, 2, 3, \ldots\}$ is the set of all integers. It's conventional to use capital letters to denote sets and use lowercase letters to denote elements of sets. Throughout this
chapter, we will use $A$ and $B$ to represent arbitrary sets.

\begin{dfnbox}{Membership, $\in$}{}
    We write $a \in A$ to mean ``$a$ is in $A$''.
\end{dfnbox}

\begin{dfnbox}{Subset, $\subseteq$}{}
    $A$ is a \dfntxt{subset} of $B$ if everything in $A$ is also in $B$.
    \tcblower
    \[ A \subseteq B \iff \forall(x \in A)(x \in B) \]
\end{dfnbox}

\begin{dfnbox}{Set Equality, $=$}{}
    $A$ \dfntxt{equals} $B$ if $A$ is a subset of $B$ and $B$ is a subset of $A$.
    \tcblower
    \[ A = B \iff (A \subseteq B \land B \subseteq A) \]
\end{dfnbox}

\begin{dfnbox}{Proper Subset, $\subsetneq$}{}
    $A$ is a \dfntxt{proper subset} of $B$ if $A$ is a subset of $B$ but $B$ is not a subset of $A$.
    \tcblower
    \[ A \subsetneq B \iff (A \subseteq B \land B \not\subseteq A) \]
\end{dfnbox}

In other words, $A$ is a proper subset of $B$ if everything in $A$ is also in $B$, but $B$ has something that $A$ does not.

\begin{dfnbox}{Empty Set ($\emptyset$)}{}
    The \dfntxt{empty set} is the set that contains no elements.
    \tcblower
    \[ \emptyset \coloneq \{ \} \]
\end{dfnbox}

As a convention, we will assume that $\emptyset$ is a subset of any set, including itself.

% A useful way of visualizing set relations is with venn diagrams.

\begin{tecbox}{Proving a Subset Relation}{}
    To prove that $A \subseteq B$:
    \begin{enumerate}
        \item Let $x$ be an arbitrary element of $A$.
        \item Show that $x \in B$.
    \end{enumerate}
    \tcblower
    To prove that $A \not\subseteq B$, choose a specific $x \in A$ and show $x \notin B$.
\end{tecbox}

\begin{exbox}{Proving Simple Subset Relation}{}
    Suppose that $A \subseteq B$ and $B \subseteq C$. Prove that $A \subseteq C$.
    \tcblower
    \begin{proof}
        Let $x \in A$ be arbitrary. Since $A \subseteq B$, then $x \in B$. Similarly, since $B \subseteq C$, then $x \in C$. Therefore, $A \subseteq C$.
    \end{proof}
\end{exbox}

\begin{dfnbox}{Union}{}
    The \dfntxt{union} of two sets is the set of all things that are in one or the other set.
    \tcblower
    \[ A \cup B \coloneq \left\{ x : x \in A \lor x \in B \right\} \]
\end{dfnbox}

\begin{dfnbox}{Intersection}{}
    The \dfntxt{intersection} of two sets is the set of all things that are in both sets.
    \tcblower
    \[ A \cap B \coloneq \left\{ x : x \in A \land x \in B \right\} \]
\end{dfnbox}

More generally, we can apply union and intersection to an arbitrary number of sets, finite or infinite. We use a notation similar to summation using $\sum$. Let $\Lambda$ be an indexing set, and for each $\lambda \in \Lambda$, let $A_\lambda$ be a set.
\begin{align*}
    \bigcup_{\lambda \in \Lambda} A_\lambda &= \left\{ x : x \in A_\lambda\ \text{for some}\ \lambda \in \Lambda \right\} \\
    \bigcap_{\lambda \in \Lambda} A_\lambda &= \left\{ x : x \in A_\lambda\ \text{for all}\ \lambda \in \Lambda \right\}
\end{align*}
\begin{exbox}{Indexed Sets}{}
    For $n \in \N$, let $A_n = \left[ \frac{1}{n}, 1 \right] = \left\{ x \in \R : \frac{1}{n} \leq x \leq 1 \right\}$. Prove that:
    \begin{enumerate}[label=(\alph*)]
        \item $\bigcup_{n=1}^\infty = (0,1]$
        \item $\bigcap_{n=1}^\infty = \{1\}$
    \end{enumerate}
    \tcblower
    \begin{proof}[Proof of (a)]
        Suppose $x \in \bigcup_{n=1}^\infty A_n$. Then there exists $n \in \N$ such that $x \in A_n = \left[ \frac{1}{n}, 1 \right]$. That is, $0 < \frac{1}{n} \leq x \leq 1$. Therefore, $x \in (0, 1]$.

        Suppose $x \in (0, 1]$. Then $x > 0$, so there exists $n_0 \in \N$ such that $\frac{1}{n_0} < x$. Then $\frac{1}{n_0} \leq x \leq 1$, so $x \in A_{n_0}$. Therefore, $x \in \bigcup_{n=1}^\infty A_n$.
    \end{proof}

    \begin{proof}[Proof of (b)]
        Suppose $x \in \bigcap_{n=1}^\infty A_n$. Then $x \in A_1 = \{1\}$.

        Suppose $x \in \{1\}$. Then $x = 1 \in \left[ \frac{1}{n}, 1 \right]$ for all $n \in \N$. Therefore, $x \in \bigcap_{n=1}^\infty A_n$.
    \end{proof}
\end{exbox}

\begin{dfnbox}{Set Minus}{}
    The \dfntxt{set difference} of two sets is the set of all things that are in first set but not the second set.
    \tcblower
    \[ A \setminus B = \{ x \in A : x \notin B \} \]
\end{dfnbox}

\begin{dfnbox}{Complement}{}
    Let $X$ be a set called the \dfntxt{universal set}. The \dfntxt{complement} of $A$ in $X$ is defined as $X \setminus A$.
    \tcblower
    \[ A^c = X \setminus A = \{ x \in X : x \notin A \} \]
\end{dfnbox}

\begin{thmbox}{De Morgan's Laws for Sets}{}
    Suppose $X$ is a set, and for any subset $S$ of $X$, let $S^c = X \setminus S$. Suppose that $A_\lambda \subseteq X$ for every $\lambda$ belonging to some index set $\Lambda$. Prove that:
    \begin{enumerate}[label=(\alph*)]
        \item \( \left( \bigcup_{\lambda \in \Lambda} A_\lambda \right)^c = \bigcap_{\lambda \in \Lambda} A_\lambda^c \);
        \item \( \left( \bigcap_{\lambda \in \Lambda} A_\lambda \right)^c = \bigcup_{\lambda \in \Lambda}A_\lambda^c \).
    \end{enumerate}
    \tcblower
    \begin{proof}[Proof of (a)]
        First, let $a \in \left( \bigcup_{\lambda \in \Lambda} A_\lambda \right)^c$. Then, $a \in X \setminus \left( \bigcup_{\lambda \in \Lambda} A_\lambda \right)$, so $a \in X$ but $a \notin \left( \bigcup_{\lambda \in \Lambda} A_\lambda \right)$. Thus, $a \notin A_\lambda$ for any $\lambda \in \Lambda$, so $a \in X \setminus A_\lambda$ for all $\lambda \in \Lambda$. In other words, $a \in \bigcap_{\lambda \in \Lambda} A_\lambda^c$.

        Next, let $a \in \bigcap_{\lambda \in \Lambda} A_\lambda^c$. Then $a \in A_\lambda^c$ for all $\lambda \in \Lambda$, so $a \in X$ but $a \notin A_\lambda$ for all $\lambda \in \Lambda$. That is, $a \notin \left( \bigcup_{\lambda\in\Lambda} A_\lambda \right)$. In other words, $a \in \left( \bigcup_{\lambda\in\Lambda} A_\lambda \right) ^ c$.
    \end{proof}

    \begin{proof}[Proof of (b)]
        First, let $a \in \left( \bigcap_{\lambda \in \Lambda} A_\lambda \right)^c$. Then, $a \in X \setminus  \bigcap_{\lambda \in \Lambda} A_\lambda$, so $a \in X$ but $a \notin \bigcap_{\lambda \in \Lambda} A_\lambda$. That is, $a \notin A_\lambda$ for some $\lambda \in \Lambda$. Thus, $a \in X \setminus A_\lambda$ for some $\lambda \in \Lambda$. Therefore, $a \in \bigcup_{\lambda \in \Lambda} A_\lambda^c$.

        Next, let $a \in \bigcup_{\lambda \in \Lambda} A_\lambda^c$. Then $a \in A_\lambda^c$ for some $\lambda \in \Lambda$, so $a \in X$ but $a \notin A_\lambda$ for some $\lambda \in \Lambda$. That is, $a \notin \left( \bigcap_{\lambda \in \Lambda} A_\lambda \right)$. Therefore, $a \in  \left( \bigcap_{\lambda \in \Lambda} A_\lambda \right)^c$.
    \end{proof}
\end{thmbox}

\section{Functions}
We generally think of functions as a ``map'' or ``rule'' that assigns numbers to other numbers. For example, $f(x) = 2x$ maps $1 \mapsto 2$, $2 \mapsto 4$, etc. In formal mathematics, it's conventional to actually define functions in terms of sets.


\begin{dfnbox}{Cartesian Product}{}
    Let $X$ and $Y$ be sets. The \dfntxt{Cartesian product} of $X$ and $Y$ is the set of all ordered pairs $(x,y)$ where $x \in X$ and $y \in Y$.
    \tcblower
    \[ X \times Y \coloneq \left\{ (x,y) : x \in X \land y \in Y \right\} \]
\end{dfnbox}

\begin{dfnbox}{Function}{function}
    Let $X$ and $Y$ be sets. A \dfntxt{function} from $X$ to $Y$ is a subset of $X \times Y$ such that for every $x \in X$, there exists a unique $y \in Y$ where $(x,y) \in f$.
    \tcblower
    \[ f : X \to Y \]
\end{dfnbox}


Given $f : X \to Y$, we call $X$ the \dfntxt{domain} of $f$ and $Y$ the \dfntxt{codomain} of $f$. Given $x \in X$, we write $f(x)$ to denote the unique element of $Y$ such that $(x,y) \in f$.
\[ f(x) = y \iff (x,y) \in f \]

\begin{dfnbox}{Function Image}{image}
    Let $f : X \to Y$ be a function and $A \subseteq X$. The \dfntxt{image} of $A$ under $f$ is the set containing all possible function outputs from all inputs in $A$.
    \tcblower
    \[ f[A] \coloneq \{ f(a) : a \in A \} \]
\end{dfnbox}

Given $f : X \to Y$, we call $f[X]$ the \dfntxt{range} of $f$.

\begin{exbox}{Function Images}{}
    Suppose $f : X \to Y$ is a function, and $A_\lambda \subseteq X$ for each $\lambda \in \Lambda$. Then:
    \begin{enumerate}[label=(\alph*)]
        \item $f \left[ \bigcup_{\lambda \in \Lambda} A_\lambda \right] = \bigcup_{\lambda \in \Lambda} f \left[ A_\lambda \right]$
        \item $f \left[ \bigcap_{\lambda \in \Lambda} A_\lambda \right] \subseteq \bigcap_{\lambda \in \Lambda} f \left[ A_\lambda \right]$
    \end{enumerate}
    \tcblower
    In this example, we will only prove the ``forward'' direction. That is, we want to show that $f \left[ \bigcup_{\lambda \in \Lambda} A_\lambda \right] \subseteq \bigcup_{\lambda \in \Lambda} f \left[ A_\lambda \right]$.
    \begin{proof}[Proof of (a)]
        Let $y \in f \left[ \bigcup_{\lambda \in \Lambda} A_\lambda \right]$. By definition of \nameref{dfn:image}, there exists $x \in \bigcup_{\lambda \in \Lambda} A_\lambda$ such that $y = f(x)$. Thus, there exists $\lambda_0 \in \Lambda$ such that $x \in \lambda_0$. That is, $y \in f \left[ A_{\lambda_0} \right]$. Therefore, $y \in \bigcup_{\lambda \in \Lambda} f \left[ A_\lambda \right]$.
    \end{proof}
\end{exbox}

\begin{dfnbox}{Function Inverse Image}{inverse-image}
    Let $f : X \to Y$ be a function and $B \subseteq Y$. The \dfntxt{inverse image} of $B$ under $f$ is the set containing all possible function inputs whose output is in $B$.
    \tcblower
    \[ f^{-1}[B] \coloneq \{ x \in X: f(x) \in B \} \]
\end{dfnbox}

Note the following logical equivalence:
\[ x \in f^{-1} [B] \iff f(x) \in B \]

\begin{exbox}{Function Inverse Images}{}
    Suppose $f : X \to Y$ is a function, and $B_\lambda \subseteq Y$ for each $\lambda \in \Lambda$. Then:
    \[ f^{-1} \left[ \bigcup_{\lambda \in \Lambda} B_\lambda \right] = \bigcup_{\lambda \in \Lambda} f^{-1} \left[ B_\lambda \right] \]
    \tcblower
    Again, we will only prove the ``forward direction''.
    \begin{proof}
        Let $x \in f^{-1} \left[ \bigcup_{\lambda \in \Lambda} B_\lambda \right]$. Then, $f(x) \in \bigcup_{\lambda \in \Lambda} B_\lambda$. That is, $f(x) \in B_{\lambda_0}$ for some $\lambda_0 \in \Lambda$. Thus, $x \in f^{-1} \left[ B_{\lambda_0} \right]$, so $x \in \bigcup_{\lambda \in \Lambda} f^{-1} \left[ B_\lambda \right]$.
    \end{proof}
\end{exbox}

\begin{dfnbox}{Injective}{}
    A function $f : X \to Y$ is \dfntxt{injective} if no two inputs in $X$ have the same output in $Y$.
    \tcblower
    \[ \forall (x_1, x_2 \in X) \left[ x_1 \neq x_2 \implies f\left(x_1\right) \neq f\left(x_2\right) \right] \]
\end{dfnbox}

\begin{tecbox}{Proving a Function is Injective}{}
    To prove a function $f : X \to Y$ is injective:
    \begin{enumerate}
        \item Let $x_1, x_2 \in X$ where $f(x_1) = f(x_2)$.
        \item Reason that $x_1 = x_2$.
    \end{enumerate}
\end{tecbox}

\begin{exbox}{Proving Injectivity}{}
    $f(x) = -3x-7$ is injective.
    \tcblower
    \begin{proof}
        Suppose $f(x_1) = f(x_2)$. Then $-3x_1+7 = -3x_2+7$, so $-3x_1 = -3x_2$. Thus, $x_1 = x_2$, so $f$ is injective.
    \end{proof}
\end{exbox}

\begin{exbox}{Disproving Injectivity}{}
    Prove that $f(x)=x^2$ is not injective.
    \tcblower
    \begin{proof}
        $f(-1) = 1$ and $f(1) = 1$, but $-1 \neq 1$. Thus, $f$ is not injective.
    \end{proof}
\end{exbox}

\begin{dfnbox}{Surjective}{}
    A function $f : X \to Y$ is \dfntxt{surjective} if everything in $Y$ has a corresponding input in $X$.
    \tcblower
    \[ \forall (y \in Y) \left[ \exists (x \in X) (f(x) = y) \right] \]
\end{dfnbox}

Note that $f : X \to f[X]$ is \textbf{always} surjective.

\begin{tecbox}{Proving a Function is Surjective}{}
    To prove a function $f : X \to Y$ is surjective:
    \begin{enumerate}
        \item Let $y \in Y$ be arbitrary.
        \item ``Undo'' the function $f$ to obtain $x \in X$ where $f(x)=y$.
    \end{enumerate}
\end{tecbox}

\begin{exbox}{Proving Surjectivity}{}
    Prove that $f : \R \to \R$ defined by $f(x) = -3x+7$ is surjective.
    \tcblower
    \begin{proof}
        Let $y \in Y$ be arbitrary. Let $x \coloneq \frac{y-7}{-3}$. Then $x \in \R$, and:
        \begin{align*}
            f(x)
            &= -3 \left( \frac{y-7}{-3} \right) + 7 \\
            &= (y-7) + 7 \\
            &= y
        \end{align*}
        Therefore, $f$ is surjective.
    \end{proof}
\end{exbox}

\begin{dfnbox}{Bijective}{}
    A function $f : X \to Y$ is \dfntxt{bijective} if it is both injective and surjective.
\end{dfnbox}

\begin{dfnbox}{Function Composition}{}
    Let $f : X \to Y$ and $g : Y \to Z$ be functions. The \dfntxt{composition} of $f$ and $g$ is a function $g \circ f : X \to Z$ defined by:
    \[ (g \circ f) (x) \coloneq g(f(x)) \]
\end{dfnbox}

\begin{thmbox}{Composition Preserves Injectivity and Surjectivity}{}
    Suppose $f : X \to Y$ and $g : Y \to Z$ are functions.
    \begin{enumerate}[label=(\alph*)]
        \item If $f$ and $g$ are injective, then $g \circ f$ is injective.
        \item If $f$ and $g$ are surjective, then $g \circ f$ is surjective.
        \item If $f$ and $g$ are bijective, then $g \circ f$ is bijective.
    \end{enumerate}
    \tcblower
    \begin{proof}[Proof of (a)]
        Let $x_1, x_2 \in X$. Suppose that $(g \circ f)(x_1) = (g \circ f)(x_2)$. Then, $g(f(x_1)) = g(f(x_2))$. Because $g$ is injective, we have $f(x_1) = f(x_2)$. Because $f$ is injective, we have $x_1 = x_2$. Therefore, $g \circ f$ is injective.
    \end{proof}

    \begin{proof}[Proof of (b)]
        Let $z \in Z$. Because $g$ is surjective, there exists an element $y \in Y$ such that $g(y) = z$. Because $f$ is surjective, there exists an element $x \in X$ such that $f(x) = y$. Thus, $(g \circ f)(x) = g(f(x)) = g(y) = z$. Therefore, $g \circ f$ is surjective.
    \end{proof}

    \begin{proof}[Proof of (c)]
        We know that from (a) and (b) composition preserves injectivity and surjectivity. Thus, composition must also preserve bijectivity.
    \end{proof}
\end{thmbox}

\begin{dfnbox}{Inverse Function}{}
    Let $f : X \to Y$ be a bijection. The \dfntxt{inverse function} of $f$ is a function $f^{-1} : Y \to X$ defined by:
    \[ f^{-1} \coloneq \{ (y,x) \in Y \times X : (x,y) \in f \} \]
\end{dfnbox}

The notation for inverse functions conflicts with the notation for inverse images. A key distinction to make it that only bijections can have an inverse function, but we can apply the inverse image to any function. Thus, given a bijection $f : X \to Y$, we know $f^{-1}(f(x)) = x$ for all $x \in X$, and $f(f^{-1}(y)) = y$ for all $y \in Y$.

\begin{exbox}{}{}
    Let $f : X \to Y$ and $g : Y \to X$ be functions such that $(g \circ f) = x$ for all $x \in X$, and $(f \circ g)(y) = y$ for all $y \in Y$. $f^{-1} = g$.
    \tcblower
    \begin{proof}
        todo: finish proof
    \end{proof}
\end{exbox}

\chapter{Number Systems}
Our goal is to create an axiomatic basis for the real numbers $\R$. We need to establish axioms for $\R$ and then derive all further properties from the axioms. We would like these axioms to be as minimal and agreeable as possible; however, finding axioms that characterize $\R$ is not easy. Instead, we'll start from the natural numbers $\N$ and expand from there.

\section{Natural Numbers $\N$ and Induction}
How do we define the natural numbers? Listing every natural number is definitely not an option. We could try to define the natural numbers as $\N \coloneq \{ 1, 2, \ldots \}$. However, the ``$\ldots$'' is ambiguous. Instead, we can come up with some rules such that only the natural numbers could satisfy those rules.

\begin{dfnbox}{Peano Axioms for $\N$}{}
    The \dfntxt{Peano axioms} are axioms that can be used to define the natural numbers $\N$.
    \begin{enumerate}[noitemsep]
        \item $1 \in \N$
        \item Every $n \in \N$ has a successor called $n+1$.
        \item $1$ is \textbf{not} the successor of any $n \in \N$.
        \item If $n,m \in \N$ have the same successor, then $n = m$.
        \item If $1 \in S$ and every $n \in S$ has a successor, then $\N \subseteq S$.
    \end{enumerate}
\end{dfnbox}

\begin{notebox}
    Note that there is no one ``prescribed'' way to do define the natural numbers. This is just the most popular approach.
\end{notebox}

From the fifth axiom, we can derive a new proof technique for proving an arbitrary statement for all natural numbers.

\begin{thmbox}{Principle of Induction}{induction}
    Let $P(n)$ is a statement for each $n \in \N$. Suppose:
    \begin{enumerate}
        \item $P(1)$ is true, and
        \item if $P(n)$ is true, then $P(n+1)$ is true.
    \end{enumerate}
    Then $P(n)$ is true for all $n \in \N$.
    \tcblower
    \begin{proof}
        Let $P(n)$ be a statement for each $n \in \N$. Assume $P(1)$ is true and $P(n) \implies P(n+1)$ for all $n \in \N$. Let $S \coloneq \{ n \in \N : P(n) \} \subseteq \N$. Then $1 \in S$ because $P(1)$ is true. Note that if $n \in S$, then $P(n)$ is true. Hence, $P(n+1)$ is true by assumption. Thus, $n+1 \in S$. By the fifth Peano axiom, we have $S = \N$.
    \end{proof}
\end{thmbox}

A proof by induction kind of has a ``domino effect''. We set up the dominoes by proving $P(n) \implies P(n+1)$ and knock over the first domino by proving $P(1)$. The result is that all the dominoes will topple each other, leaving no domino standing.

\[ \underbracket{P_1}_{\text{by (1)}} \implies \underbracket{P_2}_{\text{by (2)}} \implies \underbracket{P_3}_{\text{by (2)}} \implies \ldots \]

\begin{tecbox}{Proof by Induction}{induction}
    To prove a statement $P(n)$ for all $n \in \N$:
    \begin{enumerate}
        \item \textbf{Base Case:} Prove $P(1)$.
        \item \textbf{Induction Step:} Assume $P(n)$ is true from some $n \in \N$, then prove $P(n) \implies P(n+1)$.
    \end{enumerate}
\end{tecbox}

It is crucial that we actually use our assumption that $P(n)$ is true in the induction step. Otherwise, our proof is most likely wrong.

\begin{exbox}{Simple Proof by Induction}{}
    Prove that $1+2+\cdots + n = \frac{n(n+1)}{2}$ for all $n \in \N$.
    \tcblower
    \begin{proof}
        Let $P(n)$ be the statement $1 + 2 + \cdots + n = \frac{n(n+1)}{2}$.

        \textbf{Base Case:} When $n=1$, $\text{LHS} = 1$ and $\text{RHS} = \frac{1(1+1)}{2} = 1$, so $P(1)$ is true.

        \textbf{Induction Step:} Assume that $P(n)$ is true for some $n \in \N$. Then:
        \begin{align*}
            1 + 2 + \cdots + n + (n+1)
            &= \frac{n(n+1)}{2} + (n+1) \\
            &= (n+1) \left( \frac{n}{2} + 1 \right) \\
            &= \frac{(n+1)(n+2)}{2}
        \end{align*}
        That is, $P(n+1)$ is true. By the \nameref{thm:induction}, $P(n)$ is true for all $n \in \N$.
    \end{proof}
\end{exbox}

\section{Integers $\Z$}
On $\N$, our idea of addition ($+$) and multiplication ($\cdot$) already satisfy the following properties:

\begin{tabular}{l l l}
    Commutativity & $n+m = m+n$ & $n \cdot m = m \cdot n$ \\
    Associativity & $(n \cdot m) \cdot k$ & $n \cdot (m \cdot k)$ \\
    Distributivity & $n \cdot (m + k) = n \cdot m + n \cdot k$ \\
    Identity & $n \cdot 1 = n$
\end{tabular}

We can expand this number system by including:
\begin{enumerate}
    \item an \dfntxt{additive identity} ($n+0 = n$ for all $n \in \N$)
    \item \dfntxt{additive inverses} (for all $n \in \N$, add $-n$ so $-n + n = 0$)
    \item \dfntxt{multiplicative inverses} (for all $n \in \Z \setminus \{0\}$, define $\sfrac{1}{n}$ such that $n \cdot \sfrac{1}{n} = 1$)
\end{enumerate}

From just 1 and 2, we then have the set of integers ($\Z$). To attain the rational numbers ($\Q$), we include 3 and define $m \cdot \sfrac{1}{n} = \sfrac{m}{n}$ when $n \neq 0$. Thus, we have:

\[ \Q \coloneq \left\{ \frac{m}{n} : m,n \in Z \land n \neq 0 \right\} \]

To ensure multiplication works as intended, we also define $\frac{m}{n} \cdot \frac{k}{l} \coloneq \frac{m \cdot k}{n \cdot l}$.

We say $\frac{m_1}{n_1} = \frac{m_2}{n_2}$ if and only if $m_1n_1 = m_2n_2$ where $n_1, n_2 \neq 0$. In other words, $\frac{m_1}{n_1} \sim \frac{m_2}{n_2} \iff m_1n_2 = m_2n_1$. Thus, $\Q$ is the set of equivalence classes for this relation.

If $n = kp$ and $m = kq$, where $k,p,q \in \Z$, $k \neq 0$, $q \neq 0$, then:
\[ \frac{n}{m} = \frac{kp}{kq} = \frac{k}{p}, \quad \text{because}\ kpq = kqp \]
If $n$ and $m$ have no common factor (except $\pm 1$), then we say that $\sfrac{n}{m} \in \Q$ is in the ``lowest terms'' or ``reduced terms''. The set $(Q, +, \cdot)$ forms a field. However, we cannot write $x = \sfrac{n}{m}$ where $x^2 = 2$.

\begin{exbox}{}{}
    for $n \in \Z$, if $n^2$ is even, then $n$ is even.
\end{exbox}

\begin{thmbox}{Root 2 is Irrational}{}
    \begin{proof}
        Suppose for contradiction that there exist $n,m \in \Z$ such that $\left( \frac{n}{m} \right)^2 = 2$. If $n = kp$ and $m = kq$, then we can ``cancel'' the common factor $k$ to write $\frac{n}{m} = \frac{p}{q}$. That is, we can assume that $n$ and $m$ have no (non-trivial) common factors. Now, $\frac{n^2}{m^2} = 2$, so $n^2 = 2m^2$. Thus, $n^2$ is an even number.

    \end{proof}
\end{thmbox}

\amzindex
\end{document}
