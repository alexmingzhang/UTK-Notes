\documentclass[12pt]{report}

\usepackage[margin=1in]{geometry}
\usepackage[math]{amznotes}
\usepackage{enumitem}
\usepackage{xfrac}
\usepackage{tikz}
\usetikzlibrary{calc, 3d}

\title{\textbf{Calculus III}\\
\large UT Knoxville, Spring 2023, MATH 341}
\author{Stella Thistlethwaite, Alex Zhang}

\begin{document}
\maketitle
\tableofcontents

\chapter{Introduction}
Much of our focus will be on Stoke's Theorem.

\chapter{Three-Dimensional Space}
In past math classes, we have been used to dealing in $\R^2$ where we work with two degrees of freedom: $x$ and $y$. Now, we will be working in $\R^3$ with three degrees of freedom: $x$, $y$, and $z$.

\section{Points}

\begin{dfnbox}{Point}{point}
    A \dfntxt{point} in $\R^n$ space is an $n$-tuple that specifies a location in that space.
    \tcblower
    \[ p = (p_1, \ldots, p_n) \in \R^n \]
\end{dfnbox}

\begin{dfnbox}{Distance}{}
    Given two points $a, b\in \R^n$, the \dfntxt{distance} between the two points is defined as:
    \[ d(a, b) \coloneq \sqrt{ (b_1 - a_1)^2 + \cdots + (b_n - a_n)^2} \]
\end{dfnbox}

\begin{exbox}{Distance Between Points}{}
    Find the distance between $p_1 = (-1, -1, 4)$ and $p_2 = (-1, 4, -1)$.
    \tcblower
    \begin{align*}
         d(p_1, p_2) &= \sqrt{ (-1-(-1))^2 + (4-(-1))^2 + (-1-1)^2 } \\
         &= \sqrt{ 0^2 + 5^2 + (-5)^2 } \\
         &= \sqrt{ 50 }
    \end{align*}
\end{exbox}

\begin{dfnbox}{Sphere}{sphere}
    Given a point $c = (h, k, l) \in \R^3$, a \dfntxt{sphere} is the set of all points $(x,y,z) \in \R^3$ that are a distance $r$ from the point $c = (h,k,l)$.
    \tcblower
    \[ (x-h)^2 + (y-k)^2 + (z-l)^2 = r^2 \]
\end{dfnbox}

Note that all the points of the sphere are equidistant to the center of the sphere. This means the sphere is really a hollow shell.

\begin{exbox}{Circle}{}
    Show that the following quadratic equation represents a circle by rewriting it in standard form. Find the center $c = (h,k)$ and the radius $r$.
    \[ x^2 + y^2 + x = 0 \]
    \tcblower
    To solve this, we will have to complete the square:
    \begin{align*}
        x^2 + x + y^2 &= 0 \\
        \implies x^2 + x + \frac{1}{4} + y^2 &= \frac{1}{4} \\
        \implies \left( x + \frac12 \right)^2 + y^2 &= \frac14
    \end{align*}
\end{exbox}

\section{Vectors}

\begin{dfnbox}{Vector}{vector}
    A \dfntxt{vector} is a mathematical object that contains multiple objects of the same type.
    \tcblower
    \[ \overrightarrow{v} = \alg{v_1, \ldots, v_n} \in \R^n \]
\end{dfnbox}

As customary in most mathematics textbooks, we will always denote vectors using the little arrow thing. In the context of three-dimensional space, we will only be working with vectors with three components. In addition, we will think of vectors as having a magnitude and direction.

\begin{dfnbox}{Scalar Multiplication}{}
    Given a vector $\vec{v}$ and scalar $k$, we define \dfntxt{scalar multiplication} as:
    \[ k \cdot \vec{v} \coloneq \alg{kv_1, \ldots, kv_n} \]
\end{dfnbox}

Note that scalar multiplication is associative, commutative, and distributive.
\begin{itemize}
    \item $a(b\vec{v}) = b(a(\vec{v})) = (ab) \vec{v}$
    \item $(k_1 + k_2) \vec{v} = k_1 \vec{v} + k_2 \vec{v}$
    \item $k (\vec{v} + \vec{w}) = k{v} + k\vec{w}$
\end{itemize}

\begin{dfnbox}{Norm}{}
    A vector's \dfntxt{norm} is its magnitude or length.
    \tcblower
    \[ \norm{v} \coloneq \sqrt{v_1^2 + \cdots + v_n^2} \]
\end{dfnbox}

\begin{dfnbox}{Unit Vector}{}
    A \dfntxt{unit vector} is a vector whose magnitude is 1.
\end{dfnbox}

We will introduce shorthand notation for the three standard unit vectors:
\begin{itemize}
    \item $\hat{i} \coloneq \alg{1,0,0}$
    \item $\hat{j} \coloneq \alg{0,1,0}$
    \item $\hat{k} \coloneq \alg{0,0,1}$
\end{itemize}
These three vectors form the \dfntxt{standard basis} for $\R^3$. That is, we can express any vector in $\R^3$ as a linear combination of $\hat{i}, \hat{j}, \hat{k}$.

\begin{tecbox}{Finding a Unit Vector from a Given Vector}{}
    Given a vector $\vec{v} = \alg{x, y, z} \in \R^3$, we can find the \dfntxt{unit vector} $\vec{u}$ with the same direction by:
    \[ \vec{u} = \frac{\vec{v}}{\norm{v}} = \alg{\frac{x}{\norm{v}}, \frac{y}{\norm{v}}, \frac{z}{\norm{v}}} \]
\end{tecbox}

\begin{dfnbox}{Dot Product}{}
    Given two vectors $\vec{a}$ and $\vec{b}$ whose cardinality are both $n$, we define the \dfntxt{dot product} of $\vec{a}$ and $\vec{b}$ as:
    \[ \vec{a} \cdot \vec{b} \coloneq a_1b_1 + \cdots + a_nb_n \]
\end{dfnbox}

Like scalar multiplication, dot product is also associative, commutative, and distributive.

\begin{thmbox}{Angle Between Vectors}{}
    If $\vec{a}$ and $\vec{b}$ are vectors and $\theta$ is the angle between $\vec{a}$ and $\vec{b}$, then:
    \[ \vec{a} \cdot \vec{b} = \norm{\vec{a}} \ \norm{\vec{b}} \cdot \cos(\theta) \]
    \tcblower
    \begin{proof}
        TODO: finish proof
    \end{proof}
\end{thmbox}

\begin{dfnbox}{Parallel, Perpendicular}{}
    \begin{itemize}[noitemsep]
        \item Two vectors are \dfntxt{parallel} if the angle between the vectors is $0\deg$.
        \item Two vectors are \dfntxt{perpendicular} if the angle between the vectors is $90\deg$.
    \end{itemize}
\end{dfnbox}

\begin{dfnbox}{Orthogonal}{}
    $\vec{a}$ and $\vec{b}$ are \dfntxt{orthogonal} if $\vec{a} \cdot \vec{b} = 0$.
\end{dfnbox}

Given a vector $\vec{a} = \alg{a_1, a_2, a_3}$, we have:

\[ \frac{\vec{a}}{\norm{a}} = \alg{\cos \alpha, \cos \beta, \cos \gamma} \]
where:
\begin{itemize}
    \item $\alpha = \cos^{-1} \left( \frac{a_1}{\norm{\vec{a}}} \right) \quad$ (angle between $\vec{a}$ and the $x$-axis)
    \item $\beta = \cos^{-1} \left( \frac{a_2}{\norm{\vec{a}}} \right) \quad$ (angle between $\vec{a}$ and the $y$-axis)
    \item $\beta = \cos^{-1} \left( \frac{a_3}{\norm{\vec{a}}} \right) \quad$ (angle between $\vec{a}$ and the $z$-axis)
\end{itemize}

\begin{dfnbox}{Work}{}
    If $F$ is a force moving a particle from a point $P$ to a point $Q$, the \dfntxt{work} performed by the force is given by:
    \[ W = \vec{F} \cdot \overrightarrow{PQ} \]
\end{dfnbox}

\begin{exbox}{Finding Work}{}
    Find the work done by a force $\vec{F} = \alg{3,4,5}$ in moving an object from $p = (2,1,0)$ to $q = (4,6,2)$.
    \tcblower
    First, we find $\overrightarrow{pq}$ as such:
    \begin{align*}
        \overrightarrow{pq}
        &= \alg{4-2, 6-1, 2-0} \\
        &= \alg{2,5,2}
    \end{align*}
    Then, we can find work:
    \begin{align*}
        W
        &= \vec{F} \cdot \overrightarrow{PQ} \\
        &= \alg{3,4,5} \cdot \alg{2,5,2} \\
        &= 6 + 20 + 10 \\
        &= 36
    \end{align*}
\end{exbox}

\section{Gradient}

\begin{dfnbox}{Gradient}{}
    Let $f : \R^n \to \R$ be a function. The \dfntxt{gradient} of $f$ is a function $\nabla f : \R^n \to \R^n$ defined by:
    \[ \nabla f(x_1, \ldots x_n) = \alg{ \frac{\partial f}{\partial x_1} , \ldots , \frac{\partial f}{\partial x_n} } \]
\end{dfnbox}

\begin{exbox}{Gradient}{}
    Let $f : \R^3 \to \R$ be a function defined by $f(T,L,\rho) = \frac{1}{2L} \sqrt{\frac{T}{\rho}}$

    The gradient of $f(T,L,P)$ is denoted
    \begin{align*}
        \nabla f(T,L,\rho)
        &= \alg{\frac{\partial f}{\partial T}, \frac{\partial f}{\partial L}, \frac{\partial f}{\partial \rho}} \\
        &= \alg{\frac{1}{4L\sqrt{T \rho}}, -\frac{1}{2L^2} \sqrt{\frac{T}{\rho}}, -\frac{1}{4L} \sqrt{\frac{T}{\rho^3}}}
    \end{align*}
    We can then calculate gradient as such:
    \begin{align*}
        \nabla f(2,1,1)
        &= \alg{ \frac{1}{4(1)\sqrt{(2)(1)}} , -\frac{1}{2(1)}\sqrt{\frac{2}{1}} , -\frac{1}{(4)(1)} \sqrt{\frac{2}{1}} } \\
        &= \alg{ \frac{1}{4\sqrt{2}} , -\frac{\sqrt{2}}{2} , - \frac{\sqrt{2}}{4} }
    \end{align*}
\end{exbox}

\begin{dfnbox}{Directional Derivative}{}
    The \dfntxt{directional derivative} of $f(x,y,z)$ in the direction of $\vec{a}$ is defined as:
    \[ \nabla f(x,y,z) \cdot \frac{\vec{a}}{\norm{\vec{a}}} \]
\end{dfnbox}

\begin{exbox}{Directional Derivative}{}
    If $f(x,y,z) = xy^2z^5$, find the directional derivative of $f(x,y,z)$ at the point $(1,0,-2)$ in the direction of the unit vector $\vec{u} = \frac{\vec{a}}{\norm{\vec{a}}}, \vec{a} = \alg{1,2,-2}$.
    \tcblower
    For this, we calculate $\nabla f(1,0,-1)$, then calculate the dot product of $\nabla f(1,0,-1)$ with the unit vector $\vec{u} = \alg{\sfrac{1}{3}, \sfrac{2}{3}, \sfrac{-2}{3}}$. Thus, the directional derivative of $f(x,y,z)$ at $(1,0,-1)$ denoted by $Df(1,0,-1)$ in the direction of $\vec{u}$ is:
    \begin{align*}
        Df(1,0,-1)
        &= \nabla f(1,2,-2) \cdot \vec{u} \\
        &= \alg{\frac{\partial f}{\partial x}, \frac{\partial f}{\partial y}, \frac{\partial f}{\partial z}} \cdot \vec{u} \\
        &= \alg{0,0,0} \cdot \alg{\frac13, \frac23, -\frac23} \\
        &= 0
    \end{align*}
\end{exbox}

\section{Projecting Vectors}
\newcommand{\vectorproj}[2][]{\textit{proj}_{\vect{#1}}\vect{#2}}

Projecting a vector onto another vector

\begin{dfnbox}{Scalar Projection}{}
    Given $\vec{a}$ and $\vec{b}$, the \dfntxt{scalar projection} of $\vec{b}$ onto $\vec{a}$ is the norm of the vector projection of $\vec{b}$ onto $\vec{a}$.
    \tcblower
    \[ \comp_{\vec{a}} \vec{b} \coloneq \frac{\vec{b} \cdot \vec{a}}{\norm{\vec{a}}} \]
\end{dfnbox}

\begin{dfnbox}{Vector Projection}{}
    Given $\vec{a}$ and $\vec{b}$ that are non-zero vectors, the \dfntxt{vector projection} of $\vec{b}$ onto the vector $\vec{a}$ is defined by:
    \[ \proj_{\vec{a}} \vec{b} \coloneq \comp_{\vec{a}} \vec{b} \frac{\vec{a}}{\norm{\vec{a}}} \]
\end{dfnbox}

\section{Cross Product}

\begin{dfnbox}{Cross Product}{}
    Given two vectors $\vec{a}, \vec{b} \in \R^3$, the \dfntxt{cross product} of $\vec{a}$ and $\vec{b}$ is a vector that is orthogonal to both $\vec{a}$ and $\vec{b}$.
    \tcblower
    \[ \vec{a} \times \vec{b} \coloneq \vec{c} \quad \text{where} \quad \vec{a} \cdot \vec{c} = 0 \quad \textbf{and} \quad \vec{b} \cdot \vec{c} = 0 \]
\end{dfnbox}

The cross product is exclusive to vectors in three dimensions.

\begin{tecbox}{Calculating Cross Product}{}
Let $\vec{a} \coloneq \alg{a_1, a_2, a_3}$ and $\vec{b} \coloneq \alg{b_1, b_2, b_3}$ To find $\vec{a} \times \vec{b}$, we:
\begin{enumerate}
    \item Create a matrix as such:
    \[ \begin{bmatrix}
        \hat{i} & \hat{j} & \hat{k} \\
        a_1 & a_2 & a_3 \\
        b_1 & b_2 & b_3
    \end{bmatrix} \]
    \item Find the determinant of the matrix by cofactor expansion on the first row.
    \begin{align*}
        \vec{a} \times \vec{b} &= \hat{i} \begin{vmatrix} a_2 & a_3 \\ b_2 & b_3 \end{vmatrix} - \hat{j} \begin{vmatrix} a_1 & a_3 \\ b_1 & b_3 \end{vmatrix} + \hat{k} \begin{vmatrix} a_1 & a_2 \\ b_1 & b_2 \end{vmatrix} \\
        &= \hat{i}(a_2b_3 - b_2a_3) - \hat{j}(a_1b_3 - b_1a_3) + \hat{k}(a_1b_2 - b_1a_2) \\
        &= \alg{ a_2b_3 - b_2a_3 ,\  -a_1b_3 + b_1a_3 ,\  a_1b_2 - b_1a_2 }
    \end{align*}
\end{enumerate}
\end{tecbox}

\begin{itemize}
    \item $\vec{a} \times \vec{b} = -\vec{b} \times \vec{a}$
    \item $\vec{a} \times (\vec{b} + \vec{c}) = \vec{a} \times \vec{b} + \vec{a} \times \vec{c}$
    \item $(\vec{a} + \vec{b}) \times \vec{c} = \vec{a} \times \vec{c} + \vec{b} \times \vec{c}$
    \item If $r \in \R$, then $(r\vec{a}) \times \vec{b} = \vec{a} \times (r\vec{b}) = r(\vec{a} \times \vec{b})$
    \item $\vec{a} \cdot (\vec{b} \times \vec{c}) = (\vec{a} \times \vec{b}) \cdot c$
    \item $\vec{a} \times (\vec{b} \times \vec{c}) = (\vec{a} \cdot \vec{c}) \vec{b} - (\vec{a} \cdot \vec{b}) \vec{c}$.
\end{itemize}

\begin{thmbox}{}{}
    If $\vec{a}$ and $\vec{b}$ are two non-zero vectors in $\R^3$ and $\theta$ is the angle between $\vec{a}$ and $\vec{b}$, then:
    \[ \norm{\vec{a} \times \vec{b}} = \norm{\vec{a}} \norm{\vec{b}} \sin{\theta} \]
\end{thmbox}

\section{Torque}

\begin{dfnbox}{Torque}{}
    If $\vec{F}$ is a force applied to an object with position vector $\vec{r}$, then the torque $\vec{\Tau}$ produced by $\vec{F}$ is given by:
    \[ \vec{\Tau} \coloneq \vec{r} \times \vec{F} \]
\end{dfnbox}

\section{Cylinders and Quadratic Surfaces}

\begin{dfnbox}{Planar Curve}{}
    A \dfntxt{planar curve} is any curve that lies on a single plane.
\end{dfnbox}

\begin{dfnbox}{Cylinder}{}
    Given a planar curve $c$, the surface in $\R^3$ defined by all parallel lines crossing the curve $c$ is called a \dfntxt{cylinder}.
\end{dfnbox}

Note that our broad definition of cylinder does not require the cylinder to be circular, nor does it require it to be straight. For example, we could have a planar curve defined by $x^2 + y^2 = 1$ and create a circular cylinder with radius $1$. We could also have a planar curve defined by $y = x^2$ and create a \dfntxt{parabolic cylinder}.

\begin{exbox}{}{}
    Consider the curve $x^2+y^2=z^2$. For every $z_0$ at $x=y=0$, we have a point, say $p \coloneq (0,0,0)$yh
\end{exbox}

\begin{dfnbox}{Cone}{}

\end{dfnbox}

\begin{dfnbox}{Conic Surface}{}
    A \dfntxt{conic surface} is a surface that is attained by taking a cross-section of a cone.
\end{dfnbox}

There are four types of conic surfaces:
\begin{enumerate}[noitemsep]
    \item The cross-section parallel to the $xy$-plane is a \dfntxt{circle}.
    \item The cross-section slightly angled from the $xy$-plane is a \dfntxt{ellipse}.
    \item The cross-section parallel to a generating line is a \dfntxt{parabola}.
    \item The cross-section parallel to the $z$ axis is a \dfntxt{hyperbola}.
\end{enumerate}

\begin{dfnbox}{Quadratic Surface}{}
    A \dfntxt{quadratic surface} in $\R^3$ is the set of points whose coordinates satisfy a quadratic polynomial in the variables $x,y,z$.
\end{dfnbox}

For example, the standard equation for a sphere is a quadratic surface.

\begin{dfnbox}{Ellipsoid}{}
    An \dfntxt{ellipsoid} is a quadratic surface in $\R^3$ defined by the equation:
    \[ \frac{(x-h)^2}{a^2} + \frac{(y-k)^2}{b^2} + \frac{(z-l)^2}{c^2} = 1 \]
\end{dfnbox}

The cross-sections of an ellipsoid with each coordinate plane ($xy$-plane, $xz$-plane, $yz$-plane) is just an ellipse. In other words:
\begin{itemize}[noitemsep]
    \item If we set $z = l$, we get an ellipse in the $xy$-plane
    \item If we set $y = k$, we get an ellipse in the $xz$-plane
    \item If we set $x = h$, we get an ellipse in the $yz$-plane
\end{itemize}

We can also have \dfntxt{hyperboloids}:
\begin{itemize}
    \item Type 1: $\frac{x^2}{a^2} + \frac{y^2}{b^2} - \frac{z^2}{c^2} = 1$
    \item Type 2: $\frac{x^2}{a^2} - \frac{y^2}{b^2} - \frac{z^2}{c^2} = 1$
\end{itemize}

For type 2 hyperboloids, its cross-section with the $xy$ and $xz$ plane is a hyperbola. Note that there is no cross-section with the $yz$ plane. This is because the equation $- \frac{y^2}{b^2} - \frac{z^2}{c^2} = 1$ has no solution in the real numbers.

The standard form equation for a parabola is:
\[ z - l = \pm c \left[ (x-l)^2 + (y-k)^2 \right] \]

TODO: more quadratic surfaces here

\chapter{Vector Functions}

\begin{dfnbox}{Vector Function}{}
    A \dfntxt{vector function} $\vec{r}(t)$ is a function that maps each $t \in \R$ to a corresponding vector in $\R^n$. In other words:
    \[ \vec{r} : \R \to \R^n \]
\end{dfnbox}

For example, we can have $\vec{r}(t) = \alg{2-3t, 5-7t, t}$. Then $\vec{r}(t)$ is a vector function that parameterizes a line that passes through $(2,5,0)$ and with vector direction $\vec{v} = \alg{-3, -7, 1}$.

\begin{exbox}{Circle Vector Function}{}
    Let $\vec{r}_1(t) = \alg{ \cos t, \sin t}$ where $0 \leq t < 2\pi$. This function's graph is a circle of radius $1$. We can think of this as $\cos^2 t + \sin^2 t = 1$.

    Similarly, consider $\vec{r}_2(t) = \alg{ 5 \cos t, 8 \sin t }$. Thus, $\sfrac{x}{5} = \cos t$ and $\sfrac{y}{8} = \sin{t}$. Using the equation $cos^2 t + sin^2 t = 1$, we now have $(\sfrac{x}{5})^2 + (\sfrac{y}{8})^2 = 1$. It's an ellipse.
\end{exbox}

If a vector function is given by $\vec{r}(t) = \alg{f(t), g(t), h(t)}$, then the domain of $\vec{r}(t)$ is the intersection of the domains of $f,g,h$. This is denoted as $\text{Dom}(\vec{r}(t))$.

\section{Limits and Continuity}
If $\vec{r}(t) = \alg{f(t), g(t), h(t)}$ is a vector function, we say that:
\[ \lim_{t \to a} \vec{r}(t) = \alg{ \lim_{t \to a} f(t), \lim_{t \to a} g(t), \lim_{t \to a} h(t) } \]

\begin{notebox}
    Recall L'Hopital's rule:
    \[ \lim_{x \to c} \frac{f(x)}{g(x)} = \lim_{x \to c} \frac{f\prime(x)}{g\prime(x)} \]
    Only applies if the left-hand side is an indeterminate form (i.e. a fraction whose denominator is $0$).
\end{notebox}

\begin{dfnbox}{Continuity}{}
    We say a function $f : X \to Y$ is \dfntxt{continuous} at some value $a$ if:
    \begin{enumerate}[noitemsep]
        \item $a \in X$,
        \item $\lim_{x \to a} f(x)$ exists, and
        \item $f(a) = \lim_{x \to a} f(x)$
    \end{enumerate}
\end{dfnbox}

Similarly, a vector function is continuous if it satisfies the above conditions.

Recall that for a function $f : \R \to \R$, the derivative at some value $x$ can be generalized by:
\[ f\prime(x) = \lim_{h \to 0} \frac{f(x + h) - f(x)}{h} \]
Similarly, the derivative of a vector function at some value $x$ can be generalized as:
\[ \vec{r}(t) = \lim_{h \to 0} \frac{\vec{r}(x+h) - \vec{r}(x)}{h} \]
Thus, if $\vec{r}(t) = \alg{f(t), g(t), h(t)}$, we have:
\[ \vec{r}\prime(t) = \alg{f\prime(t), g\prime(t), h\prime(t)} \]

\begin{thmbox}{Properties of the differentials of vector functions}{}
    Let $\vec{u}(t) = \alg{f_1(t), g_1(t), h_1(t)}$ and $\vec{v}(t) = \alg{f_2(t), g_2(t), h_2(t) }$ where $f_1, g_1, h_1, f_2, g_2, h_2$ are differentiable. Let $c$ be a scalar. Then:
    \begin{itemize}
        \item $\frac{d}{dt} \vec{u}(t) = \alg{f_1\prime(t), g_1\prime(t), h_1\prime(t)}$
        \item $\frac{d}{dt} \left( \vec{u}(t) + \vec{v}(t) \right) = \frac{d}{dt} \vec{u}(t) + \frac{d}{dt} \vec{v}(t)$
        \item $\frac{d}{dt} c \vec{v}(t) = c \frac{d}{dt} \vec{v}(t)$
        \item $\frac{d}{dt} f(t) \vec{v}(t) = f\prime(t) \vec{v}(t) + f(t) \vec{v}\prime(t)$
        \item $\frac{d}{dt} \vec{u}(t) \cdot \vec{v}(t) = \vec{u}\prime(t) \cdot \vec{v}(t) + \vec{u}(t) \cdot \vec{v}\prime(t)$
        \item $\frac{d}{dt} \vec{u}(t) \times \vec{v}(t) = \vec{u}\prime(t) \times \vec{v}(t) + \vec{u}(t) \times \vec{v}\prime(t) \quad$ (note that order here is sensitive)
        \item $\frac{d}{dt} \vec{u}(f(t)) = \vec{u}\prime(f(t)) f\prime(t)$
    \end{itemize}
\end{thmbox}

If we consider the curve $C$ formed by all the terminal points of the graph of a continuous vector function $\vec{r}(t) = \alg{x(t), y(t), z(t)}$, we call $\vec{r}(t)$ a \dfntxt{parameterization} of the curve $C$, and we call $\vec{r}(t) = \alg{x(t), y(t), z(t)}$ the \dfntxt{displacement vector}.

\begin{exbox}{}{}
    Find the parametric equations of the tangent line to the helix with parameterization $\vec{r}(t) \coloneq \alg{2 \cos t, \sin t, t}$ at the point $(0,1,\sfrac{\pi}{2})$
    \tcblower
    \begin{align*}
        \vec{r}\prime \left( \frac{\pi}{2} \right)
        &= \alg{-2 \sin \frac{\pi}{2}, \cos \frac{\pi}{2}, 1}
    \end{align*}
    Then the parameterized line is:
    \[ \alg{0,1,\frac{\pi}{2}} + t\alg{-2,0,1} \]
\end{exbox}

\amzindex
\end{document}
