\chapter{Multiple Integrals}

% If $z = f(x,y)$ is defined on a rectangle $R = [a,b] \times [c,d]$, then recall that we can

Recall the intuition behind integrals: we are chopping up a function's graph into little bits and summing the area of each of those bits. The same intuition can be applied to multiple integrals. Instead of our bits being slivers of the two-dimensional graph, we will instead have our bits be very skinny rectangular prisms of the three-dimensional graph.

\section{Double Integrals}

\begin{dfnbox}{Double Integral}{}
    The \dfntxt{double integral} of $f$ over a rectangle $R$ is:
    \[ \iint\limits_R f(x,y) dA =  \]
    \todo[inline]{formal definition}
\end{dfnbox}

\begin{thmbox}{}{}
    If $f(x,y)$ is a continuous function defined on a rectangle $R \coloneq [a,b] \times [c,d]$, then the limit always exists, and
    \[ \iint\limits_R f(x,y) dA = \int_c^d \left( \int_a^b f(x,y)\ dx \right)\ dy \]
\end{thmbox}
