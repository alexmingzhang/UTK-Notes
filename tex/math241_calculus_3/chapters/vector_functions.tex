\chapter{Vector Functions}

\begin{dfnbox}{Vector Function}{}
    A \dfntxt{vector function} $\vec{r}(t)$ is a function that maps each $t \in \R$ to a corresponding vector in $\R^n$. In other words:
    \[ \vec{r} : \R \to \R^n \]
\end{dfnbox}

For example, we can have $\vec{r}(t) = \alg{2-3t, 5-7t, t}$. Then $\vec{r}(t)$ is a vector function that parameterizes a line that passes through $(2,5,0)$ and with vector direction $\vec{v} = \alg{-3, -7, 1}$.

\begin{exbox}{Circle Vector Function}{}
    Let $\vec{r}_1(t) = \alg{ \cos t, \sin t}$ where $0 \leq t < 2\pi$. This function's graph is a circle of radius $1$. We can think of this as $\cos^2 t + \sin^2 t = 1$.

    Similarly, consider $\vec{r}_2(t) = \alg{ 5 \cos t, 8 \sin t }$. Thus, $\sfrac{x}{5} = \cos t$ and $\sfrac{y}{8} = \sin{t}$. Using the equation $cos^2 t + sin^2 t = 1$, we now have $(\sfrac{x}{5})^2 + (\sfrac{y}{8})^2 = 1$. It's an ellipse.
\end{exbox}

If a vector function is given by $\vec{r}(t) = \alg{f(t), g(t), h(t)}$, then the domain of $\vec{r}(t)$ is the intersection of the domains of $f,g,h$. This is denoted as $\text{Dom}(\vec{r}(t))$.

\section{Limits and Continuity}
If $\vec{r}(t) = \alg{f(t), g(t), h(t)}$ is a vector function, we say that:
\[ \lim_{t \to a} \vec{r}(t) = \alg{ \lim_{t \to a} f(t), \lim_{t \to a} g(t), \lim_{t \to a} h(t) } \]

\begin{notebox}
    Recall L'Hopital's rule:
    \[ \lim_{x \to c} \frac{f(x)}{g(x)} = \lim_{x \to c} \frac{f\prime(x)}{g\prime(x)} \]
    Only applies if the left-hand side is an indeterminate form (i.e. a fraction whose denominator is $0$).
\end{notebox}

\begin{dfnbox}{Continuity}{}
    We say a function $f : X \to Y$ is \dfntxt{continuous} at some value $a$ if:
    \begin{enumerate}[noitemsep]
        \item $a \in X$,
        \item $\lim_{x \to a} f(x)$ exists, and
        \item $f(a) = \lim_{x \to a} f(x)$
    \end{enumerate}
\end{dfnbox}

Similarly, a vector function is continuous if it satisfies the above conditions.

Recall that for a function $f : \R \to \R$, the derivative at some value $x$ can be generalized by:
\[ f\prime(x) = \lim_{h \to 0} \frac{f(x + h) - f(x)}{h} \]
Similarly, the derivative of a vector function at some value $x$ can be generalized as:
\[ \vec{r}(t) = \lim_{h \to 0} \frac{\vec{r}(x+h) - \vec{r}(x)}{h} \]
Thus, if $\vec{r}(t) = \alg{f(t), g(t), h(t)}$, we have:
\[ \vec{r}\prime(t) = \alg{f\prime(t), g\prime(t), h\prime(t)} \]

\begin{thmbox}{Properties of the differentials of vector functions}{}
    Let $\vec{u}(t) = \alg{f_1(t), g_1(t), h_1(t)}$ and $\vec{v}(t) = \alg{f_2(t), g_2(t), h_2(t) }$ where $f_1, g_1, h_1, f_2, g_2, h_2$ are differentiable. Let $c$ be a scalar. Then:
    \begin{itemize}
        \item $\frac{d}{dt} \vec{u}(t) = \alg{f_1\prime(t), g_1\prime(t), h_1\prime(t)}$
        \item $\frac{d}{dt} \left( \vec{u}(t) + \vec{v}(t) \right) = \frac{d}{dt} \vec{u}(t) + \frac{d}{dt} \vec{v}(t)$
        \item $\frac{d}{dt} c \vec{v}(t) = c \frac{d}{dt} \vec{v}(t)$
        \item $\frac{d}{dt} f(t) \vec{v}(t) = f\prime(t) \vec{v}(t) + f(t) \vec{v}\prime(t)$
        \item $\frac{d}{dt} \vec{u}(t) \cdot \vec{v}(t) = \vec{u}\prime(t) \cdot \vec{v}(t) + \vec{u}(t) \cdot \vec{v}\prime(t)$
        \item $\frac{d}{dt} \vec{u}(t) \times \vec{v}(t) = \vec{u}\prime(t) \times \vec{v}(t) + \vec{u}(t) \times \vec{v}\prime(t) \quad$ (note that order here is sensitive)
        \item $\frac{d}{dt} \vec{u}(f(t)) = \vec{u}\prime(f(t)) f\prime(t)$
    \end{itemize}
\end{thmbox}

If we consider the curve $C$ formed by all the terminal points of the graph of a continuous vector function $\vec{r}(t) = \alg{x(t), y(t), z(t)}$, we call $\vec{r}(t)$ a \dfntxt{parameterization} of the curve $C$, and we call $\vec{r}(t) = \alg{x(t), y(t), z(t)}$ the \dfntxt{displacement vector}.

\begin{exbox}{}{}
    Find the parametric equations of the tangent line to the helix with parameterization $\vec{r}(t) \coloneq \alg{2 \cos t, \sin t, t}$ at the point $(0,1,\sfrac{\pi}{2})$
    \tcblower
    \begin{align*}
        \vec{r}\prime \left( \frac{\pi}{2} \right)
        &= \alg{-2 \sin \frac{\pi}{2}, \cos \frac{\pi}{2}, 1}
    \end{align*}
    Then the parameterized line is:
    \[ \alg{0,1,\frac{\pi}{2}} + t\alg{-2,0,1} \]
\end{exbox}

\section{Integrability}

If $\vec{r}(t) = \alg{x(t), y(t), z(t)}$ for $t \in \R$, and $x(t)$, $y(t)$, and $z(t)$ are integrable on the interval $[a,b]$, then:
\begin{align*}
     \int_a^b \vec{r}(t)\ dt
     &= \int_a^b \alg{x(t), y(t), z(t)}\ dt \\
     &= \alg{ \int_a^b x(t)\ dt, \int_a^b y(t)\ dt, \int_a^b z(t)\ dt}
\end{align*}

If $\vec{r}(t) = \alg{x(t), y(t), z(t)}$ where $t \in [a,b]$, and if $\vec{r}$ is a parameterization of a curve $C$, then $\vec{v}(t) = \vec{r}\prime(t)$ is the \dfntxt{velocity} vector, and $\vec{a}(t) = \vec{r}\dprime (t)$ is the \dfntxt{acceleration vector}. Also, $\norm{ \vec{r}\prime(t) }$ is the \dfntxt{speed}, and $\norm{ \vec{r}\dprime(t) }$ is the \dfntxt{acceleration}.

\begin{dfnbox}{Unit Tangent Vector}{}
    Given $\vec{r}(t) = \alg{x(t), y(t), z(t)}$, the \dfntxt{unit tangent vector} is a unit vector that is tangent to the curve defined by $\vec{r}(t)$.
    \tcblower
    \[ T(t) \coloneq \frac{\vec{r}\prime(t)}{\norm{ \vec{r}\prime(t) }} \]
\end{dfnbox}

\begin{dfnbox}{Normal Vector}{}
    Given $\vec{r}(t) = \alg{x(t), y(t), z(t)}$, the \dfntxt{normal vector} is a unit vector that
    \todo[inline]{finish definition}
    \tcblower
    \[ N(t) \coloneq  \frac{\vec{r}\dprime(t)}{\norm{ \vec{r}\dprime(t)} } \]
\end{dfnbox}

\begin{dfnbox}{Osculating Plane, Binormal Vector}{}
    At each point $t$, $T(t)$ and $N(t)$ determine a plane called the \dfntxt{osculating plane}. An equation of the osculating plane to $C$ at $P_0 = \vec{r}(t)$ has a normal direction $B(t) \coloneq T(t) \times N(t)$ called the \dfntxt{binormal vector} to the curve $C$ at $p_0$. The plane determined by $N(t)$ and $B(t)$ is called the \dfntxt{normal plane} to the curve $C$ at terminal point of $\vec{r}(t) = p_0$, and it has normal direction given by $T(t)$.
\end{dfnbox}

\section{Arc Length and Curvature}
If a curve in the $xy$-plane is parameterized by $x=f(t)$ and $y=g(t)$ for $t \in [a,b]$, and $f$ and $g$ are differentiable on $[a,b]$, then the total arc length of the curve is given by:
\[ \int_a^b \sqrt{f\prime(t)^2 + g\prime(t)^2}\ dt\]
Similarly, if $C$ is a curve in $\R^3$ parameterized by a vector function $\vec{r}(t) = \alg{f(t), g(t), h(t)}$ for $t \in [a,b]$. Then the total arc length of the curve is given by:
\begin{genbox}{Arc Length of a Curve in $\R^3$}
\[ \int_a^b \sqrt{f\prime(t)^2 + g\prime(t)^2 + h\prime(t)^2}\ dt \]
\end{genbox}


\begin{dfnbox}{Curvature, Osculating Circle}{}
    The \dfntxt{curvature} of a curve at some point is how much it curves at that point. More specifically, it's the inverse of the radius of the \dfntxt{osculating circle}, tangent to the point that most closely ``matches'' the curve. This circle is called the \dfntxt{osculation circle}, and it lies in the osculating plane.
    \tcblower
    Formally, if $C$ is a curve parameterized by $\vec{r}(t) = \alg{f(t), g(t), h(t)}$ for $t \in [a,b]$, and $f,g,h$ are differentiable twice on $[a,b]$, then the \dfntxt{curvature} of $C$ is given by:
    \[ k(t) \coloneq \norm{\frac{d}{ds} \vec{T}(t)} = \frac{\norm{\vec{T}\prime(t)}}{\norm{\vec{r}\prime(t)}} \]
\end{dfnbox}

A lower curvature at some point actually means that it curves more.

\begin{thmbox}{}{}
    The curvature given by a vector function $\vec{r}(t) : \R \to \R^3$ is given by:
    \[ k(t) = \frac{\norm{\vec{r}\prime(t) \times \vec{r}\dprime(t)}}{\norm{\vec{r}\prime(t)}^3} \]
    If $C$ is a planar curve given by $y = f(x)$, then $k(t) = \frac{\norm{f\dprime(t)}}{\left( 1+[f\prime(t)]^2 \right)^{\sfrac{3}{2}}}$
\end{thmbox}

\begin{dfnbox}{Torsion}{}
    The \dfntxt{torsion} of a curve $C$ at a point $\bar{r}(t)$ (the endpoint of $\vec{r}(t)$) is a measure of how much $C$ departs from the osculating plane.
    \[ \tau \coloneq \frac{d}{ds} \vec{B}(t) \cdot \vec{N}(t) \]
\end{dfnbox}

\begin{thmbox}{}{}
    If $C$ is a ``smooth'' curve parameterized by $\vec{r}(t)$, then the torsion is given by:
    \[ \tau \coloneq \frac{ \left[ \vec{r}\prime(t) \times \vec{r}\dprime(t) \right] \cdot \vec{r}\prime\prime\prime(t)}{\norm{\vec{r}\prime(t) \times \vec{r}\dprime(t)}} \]
\end{thmbox}

Tangential component of $\vec{a}$ is given by:
\[ a_T = \frac{\vec{r}\prime(t) \cdot \vec{r}\dprime(t)}{\norm{\vec{r}\prime(t)}} \]

\section{Projectile Motion}

Suppose we launch some projectile from an initial position $(0,0)$, initial angle $\theta$, and initial velocity $\vec{v}_0$. Disregarding air resistance, the only force acting on this projectile is gravity. Thus, the force vector is given by $\vec{F} = \alg{0, -g, 0}$ where $g$ is a gravitational constant. Then, the velocity at any time $t$ is given by:
\[ \vec{t} = Q1RFSDA \]
